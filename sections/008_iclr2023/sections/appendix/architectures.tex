\subsection{Model details} 

\textbf{Core architecture.} We use the same feature extractor for both the DUMs architecture. The list of core architectures used across the experiments are: \textit{ResNet18 / ResNet50 / EfficientNet / Swin} \citep{he2016resnet, tan2021effcientnet, liu2021swin} from the torchvision repository \footnote{\url{https://pytorch.org/vision/stable/models.html}} and \textit{Wide-ResNet-28-10} \citep{zagoruyko2016wide} from the original implementation of DUE. Except for the experiment on architecture type and size where ResNet18 has output channels for the residual blocks with size $[64,128,256,512]$,  ResNet18 has output channels for the residual blocks with size $[32, 64, 128, 256]$ which causes small differences in final accuracy.

\textbf{Uncertainty head.} For DUE we use the original implementation \footnote{\url{https://github.com/y0ast/DUE}} with by default we use the RBF kernel function. For NatPN we use the original implementation \footnote{\url{https://github.com/borchero/natural-posterior-network}} but change the uncertainty head with a more expressive density estimator. As seen in Table \ref{tab:normalizing_flow}, we found that a more expressive normalizing flow with resampled base \citep{durkan2019nsf, stimper2022resampled-nf} improves significantly the results over a simpler radial normalizing flow \citep{rezende2015nf} across all the metrics. For all the experiments (except toys where we use radial flow) we use NSF-R with 16 layers.

\begin{table}[!htp]\centering
\caption{\textbf{Normalizing flow expressivity comparison.} Using more expressive normalizing flow significantely improves all the results for NatPN.}
\label{tab:normalizing_flow}
\tiny

\resizebox{0.9\textwidth}{!}{%
\begin{tabular}{lccccccccc}\toprule
&\multicolumn{4}{c}{\textbf{CIFAR100}} &\multicolumn{4}{c}{\textbf{Camelyon}} \\
\cmidrule(lr){2-5}\cmidrule(lr){6-9}
\textbf{Head} &\textbf{Accuracy ($\uparrow$)} &\textbf{Brier Score ($\downarrow$)} &\textbf{OOD Pred. ($\uparrow$)} &\textbf{OOD Epis. ($\uparrow$)} &\textbf{Accuracy ($\uparrow$)} &\textbf{Brier Score ($\downarrow$)} &\textbf{OOD Pred. ($\uparrow$)} &\textbf{OOD Epis. ($\uparrow$)} \\
\midrule
Radial &71.09 $\pm$ 0.21 &52.27 $\pm$ 0.28 &72.84 $\pm$ 1.82 &50.95 $\pm$ 2.16 &83.14 $\pm$ 0.93 &24.55 $\pm$ 1.91 &60.27 $\pm$ 5.29 &69.16 $\pm$ 7.94 \\
NSF-R &\textbf{71.61 $\pm$ 0.07} &\textbf{43.44 $\pm$ 0.11} &\textbf{73.54 $\pm$ 1.69} &\textbf{72.85 $\pm$ 1.25} &\textbf{89.84 $\pm$ 7.93} &\textbf{12.52 $\pm$ 6.17} &\textbf{64.14 $\pm$ 10.42} &\textbf{81.33 $\pm$ 8.78} \\
\bottomrule
\end{tabular}
}
\end{table}