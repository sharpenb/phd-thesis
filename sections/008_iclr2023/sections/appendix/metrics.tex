\subsection{Metric details}
\label{sec:metric_details}

% We measure the results using:

\textbf{Accuracy.} The standard accuracy $\frac{1}{N}\sum_i \mathbbm{1}[y^{*,(i)} = y^{(i)}]$ is used, where $y^{*,(i)}$ is the true label and $y^{(i)}$ is the predicted label. 

\textbf{Calibration.} The Brier score $\frac{1}{C}\sum_i^N||\vp^{(i)} - \vy^{*, (i)}||$ is used, where $\vp^{(i)}$ is the predicted softmax probability and $\vy^{*, (i)}$ is the one-hot encoded true label. Lower calibration indicates a better calibrated model. Note that in constrast with the Expected Calibration Error (ECE), the Brier score is a strictly proper scoring rule which makes it a particularly good evaluation metric for calibration \cite{scoring-rules}.

\textbf{OOD Generalization.} We apply accuracy and calibration to the distribution shifted OOD dataset and compare the results with the ID dataset to estimate the model's ability for generalization.

\textbf{OOD Detection.} We treat this task as a binary classification, where we assign class 1 to ID data and class 0 to OOD data using the aleatoric, epistemic, and predictive uncertainty estimates as scores for OOD data. This allows to compute the final scores using the area under the receiver operating characteristic curve (AUC-ROC) to measure the model's ability to detect OOD data. 
