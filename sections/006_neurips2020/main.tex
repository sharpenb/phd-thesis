\documentclass{article}

% if you need to pass options to natbib, use, e.g.:
%     \PassOptionsToPackage{numbers, compress}{natbib}
% before loading neurips_2020

% ready for submission
% \usepackage{neurips_2020}

% to compile a preprint version, e.g., for submission to arXiv, add add the
% [preprint] option:
    %  \usepackage[preprint]{neurips_2020}

% to compile a camera-ready version, add the [final] option, e.g.:
    \usepackage[final, nonatbib]{neurips_2020}

% to avoid loading the natbib package, add option nonatbib:
%\usepackage[]{neurips_2020}

\usepackage[utf8]{inputenc} % allow utf-8 input
\usepackage[T1]{fontenc}    % use 8-bit T1 fonts
\usepackage{hyperref}       % hyperlinks
\usepackage{url}            % simple URL typesetting
\usepackage{booktabs}       % professional-quality tables
\usepackage{amsfonts}       % blackboard math symbols
\usepackage{nicefrac}       % compact symbols for 1/2, etc.
\usepackage{microtype}      % microtypography

\usepackage{graphicx}
\usepackage{subcaption}
\usepackage{booktabs} % for professional tables
\usepackage{makecell}

\usepackage{bm}
\usepackage{amsmath}
\usepackage{float}
\usepackage[inline]{enumitem}
\usepackage{amsmath, amssymb}
\usepackage{bbold}
\usepackage{wrapfig}
\usepackage{tikz}
\usepackage{amsthm}
\usepackage{xspace}
\usepackage{stmaryrd}
\usepackage{caption} 
\captionsetup[table]{skip=6pt}
\usetikzlibrary{shapes,snakes, arrows}
\usepackage[table]{colortbl}


% hyperref makes hyperlinks in the resulting PDF.
% If your build breaks (sometimes temporarily if a hyperlink spans a page)
% please comment out the following usepackage line and replace
% \usepackage{icml2020} with \usepackage[nohyperref]{icml2020} above.
\usepackage{hyperref}
\usepackage{xspace}

\definecolor{Gray}{gray}{0.85}
\newcommand{\theHalgorithm}{\arabic{algorithm}}
\newcommand{\E}{\mathbb{E}}
\newcommand{\prob}{\mathbb{P}}
\newcommand{\Dir}{\mathrm{Dir}}
\newcommand{\Ber}{\mathrm{Ber}}
\newcommand{\Cat}{\mathrm{Cat}}
\newcommand{\Var}{\mathrm{Var}}
\newcommand{\Cov}{\mathrm{Cov}}
\newcommand\UCE{Uncertain Cross Entropy loss\xspace}
\newcommand\UCEacro{UCE\xspace}
\newcommand\ours{Posterior Network\xspace}
\newcommand\oursacro{PostNet\xspace}
\newcommand\SeqBn{Seq-Bn\xspace}
\newcommand\SeqNoBn{Seq-No-Bn\xspace}
\newcommand\NoFlow{No-Flow\xspace}
\newcommand\NoUCE{No-Bayes-Loss\xspace}
\newcommand\idata{i\xspace}
\newcommand\dataix{^{(\idata)}}
\newcommand\iclass{c\xspace}
\newcommand\nclass{C\xspace}
\newcommand\latent{\bold{z}}
\newcommand\vect[1]{\mathbf{#1}}
\newcommand{\latentdim}{H}
\newtheorem{property}{Property}
\newtheorem{proposition}{Proposition}
\newcommand\dz[1]{\textcolor{violet}{(DZ: #1)}}
\newcommand\bc[1]{\textcolor{blue}{(BC: #1)}}
\newcommand\sg[1]{\textcolor{green}{(SG: #1)}}

\title{\ours: Uncertainty Estimation without OOD Samples via Density-Based Pseudo-Counts}

% The \author macro works with any number of authors. There are two commands
% used to separate the names and addresses of multiple authors: \And and \AND.
%
% Using \And between authors leaves it to LaTeX to determine where to break the
% lines. Using \AND forces a line break at that point. So, if LaTeX puts 3 of 4
% authors names on the first line, and the last on the second line, try using
% \AND instead of \And before the third author name.

\author{%
  Bertrand Charpentier, Daniel Z\"ugner, Stephan G\"unnemann\\
  Technical University of Munich, Germany\\
  \texttt{\{charpent, zuegnerd, guennemann\}@in.tum.de} \\
  % examples of more authors
  % \And
  % Coauthor \\
  % Affiliation \\
  % Address \\
  % \texttt{email} \\
  % \AND
  % Coauthor \\
  % Affiliation \\
  % Address \\
  % \texttt{email} \\
  % \And
  % Coauthor \\
  % Affiliation \\
  % Address \\
  % \texttt{email} \\
  % \And
  % Coauthor \\
  % Affiliation \\
  % Address \\
  % \texttt{email} \\
}

\begin{document}

\maketitle
\begin{abstract}
Accurate estimation of aleatoric and epistemic uncertainty is crucial to build safe and reliable systems. Traditional approaches, such as dropout and ensemble methods, estimate uncertainty by sampling probability predictions from different submodels, which leads to slow uncertainty estimation at inference time. Recent works address this drawback by directly predicting parameters of prior distributions over the probability predictions with a neural network. While this approach has demonstrated accurate uncertainty estimation, it requires defining arbitrary target parameters for in-distribution data and makes the unrealistic assumption that out-of-distribution (OOD) data is known at training time. 

In this work we propose the Posterior Network (PostNet), which uses Normalizing Flows to predict an individual closed-form posterior distribution over predicted probabilites for any input sample. The posterior distributions learned by PostNet accurately reflect uncertainty for in- and out-of-distribution data -- without requiring access to OOD data at training time. PostNet achieves state-of-the art results in OOD detection and in uncertainty calibration under dataset shifts.
\end{abstract}

\section{Introduction}
%\label{sec:introduction_008}

% \looseness=-1
% Safety is critical to the adoption of deep learning in domains such as autonomous driving, medical diagnosis, or financial trading systems. A solution for this problem is to create reliable models capable to estimate the uncertainty of its own predictions. 
% Different uncertainty types are divided in \textit{aleatoric} uncertainty quantified by the inherited noise in the data, thus irreducible; \textit{epistemic} uncertainty quantified by the modeling choice or lack of data, thus reducible; \textit{predictive} uncertainty, a combination of aleatoric and epistemic \citep{gal2016uncertainty}. 
% In practice, high quality uncertainty estimates must be calibrated and able to detect Out-Of-Distribution (OOD) data like anomalies while preserving good Out-Of-Distribution (OOD) generalization performances like on dataset shifts.

While we have focused in the previous sections on proposing new Bayesian models for efficient uncertainty estimation on independent data, we now turn our attention on the practical considerations when using efficient uncertainty estimation methods. 

Recently, a family of methods for uncertainty estimation named Deterministic Uncertainty Methods (DUMs) have emerged \citep{postels2022practicalitydum}. 
Contrary to uncertainty methods such as Ensembles \citep{ensembles}, MC Dropout \citep{dropout} or other Bayesian neural networks on weights \citep{bayesian-networks}, which require multiple forward passes to make predictions, DUMs only require a single forward pass, thus making them significantly more computationally efficient. 
%These models can make predictions in only a single forward pass,  %thus being computationally efficient. 
Generally, DUMs are composed of three components with high potential impact on their performances: the \emph{training} procedure which is supposed to optimize the model toward high predictive and uncertainty performances,  the core \emph{architecture} which is supposed to define informative embeddings used to make predictions, and the \emph{prior} which is supposed to define the default uncertain predictions. In this work, we investigate the role of these three components on the quality of DUMs uncertainty estimates by evaluating calibration performances, OOD detection, and OOD generalization. Our main contributions are:
\vspace{-2mm}
\begin{itemize}
%\setlength\itemsep{-1mm}
    \item \textbf{Training}: We show that \emph{decoupling the learning rates} of the core architecture and uncertainty heads of DUMs, \emph{jointly training} the core architecture and the uncertainty head of DUMs, and \emph{pretraining} with \emph{more data} and \emph{higher data quality} improve uncertainty performances. 
    \item \textbf{Architecture}: We demonstrate that the expressiveness of the core architecture defined by the \emph{architecture type}, \emph{architecture size}, and \emph{dimension of the latent space} is crucial for \emph{both} predictive and uncertainty performances. Further, we show that applying additional regularization constraints to avoid \emph{feature collapse} does not find better trade-off between OOD detection and generalization, even sometimes degrading performances.
    \item \textbf{Prior}: In contrast to Bayesian neural networks on weights where the choice of prior is critical \citep{bayesposterior2020wenzel, fortuin2022prior, noci2021prior_cpe, kapoor2022prior_cpe}, we empirically show that the choice of prior defined in the training loss or in the uncertainty head of DUMs has a relatively small effect on the final uncertainty performances.
\end{itemize}

\section{\PostNet}
\label{sec:model_006}

In classification, we can distinguish between two types of uncertainty for a given input $\vect{x}\dataix$: the uncertainty on the class prediction \smash{$y\dataix \in \{1, \ldots, \nclass\}$} (i.e.\ aleatoric unceratinty), and the uncertainty on the categorical distribution prediction \smash{$\vect{p}\dataix = [p\dataix_1, \ldots, p\dataix_\nclass]$} (i.e.\ epistemic uncertainty). A convenient way to model both is to describe the \emph{epistemic distribution} \smash{$q\dataix$} of the categorical distribution prediction \smash{$\vect{p}\dataix$}, i.e.\ \smash{$\vect{p}\dataix \sim q\dataix$}. From the epistemic distribution follows naturally an estimate of the \emph{aleatoric distribution} of the class prediction \smash{$y\dataix \sim \text{Cat}(\bar{\vect{p}}\dataix)$} where \smash{$\E_{q\dataix}[\vect{p}\dataix] = \bar{\vect{p}}\dataix$}.

Approaches like ensembles \cite{ensemble_simple} and dropout \cite{drop_out} model $q\dataix$ implicitly, which only allows them to estimate statistics at the cost of $S$ samples (e.g. \smash{$\E_{q\dataix}[\vect{p}\dataix] \approx \frac{1}{S}\sum_{s=1}^{S} \tilde{\vect{p}}\dataix$} where \smash{$\tilde{\vect{p}}\dataix$} is sampled from \smash{$q\dataix$}). 
%
Another class of models \cite{prior_net, rev_kl_prior_net, uceloss, evidential_uncertainty} explicitly parametrizes the epistemic distribution with a Dirichlet distribution (i.e.\ \smash{$q^{(\idata)} = \text{Dir}(\bm{\alpha}\dataix)$} where \smash{$f_{\theta}(x\dataix) = \bm{\alpha} \dataix \in \mathbb{R}_+^\nclass$}), which is the natural prior for categorical distributions. This parametrization is convenient since it requires only one pass to compute epistemic distribution, aleatoric distribution and class prediction:
\begin{equation}
\begin{aligned}
q^{(\idata)} &= \text{Dir}(\bm{\alpha}\dataix),\hspace{25pt} 
\bar{p}_\iclass\dataix &= \frac{\alpha_\iclass}{\alpha_0} \text{ with } \alpha_0 = \sum^{\nclass}_{\iclass=1} \alpha_\iclass,\hspace{25pt}
y^{(\idata)} &= \arg \max \;[\bar{p}_1, ..., \bar{p}_\nclass]
\end{aligned}
\end{equation}
The concentration parameters \smash{$\alpha\dataix_\iclass$} can be interpreted as the number of observed samples of class $\iclass$ and, thus, are a good indicator of epistemic uncertainty for non-degenerate Dirichlet distributions (i.e.\ \smash{$\alpha\dataix_\iclass \geq 1$}). To learn these parameters, Prior Networks \cite{prior_net, rev_kl_prior_net} use OOD samples for training and define different target values for ID and OOD data. For ID data, \smash{$\alpha\dataix_\iclass$} is set to an arbitrary, large number if $\iclass$ is the correct class and $1$ otherwise. For OOD data, \smash{$\alpha\dataix_\iclass$} is set to $1$ for all classes. 

This approach has four issues:
%\begin{itemize}
    %\item 
    \textbf{(1)} The knowledge of OOD data for training is unrealistic. In practice, we might not have these data, since OOD samples are by definition not likely to be observed.
    %\item 
    \textbf{(2)} Discriminating in- from out-of-distribution data by providing an explicit set of OOD samples is hopeless. Since any data not from the data distribution is OOD, it is therefore impossible to characterize the infinitely large OOD distribution with an explicit data set.
    %\item 
    \textbf{(3)} The predicted Dirichlet parameters can take any value, especially for new OOD samples which were not seen during training. In the same way, the sum of the total fictitious prior observations over the full input domain $\int \alpha_0(\vect{x}) d\vect{x}$  is not bounded and in particular can be much larger than the number of ground-truth observations $N$. This can result in undesired behavior and assign arbitrarily high epistemic certainty for OOD data not seen during training.
    %\item 
    \textbf{(4)} Besides producing such arbitrarily high prior confidence, PNs can also produce degenerate concentration parameters (i.e.\ $\alpha_\iclass$ < 1). While \cite{rev_kl_prior_net} tried to fix this issue by using a different loss, nothing intrinsically prevents Prior Networks from predicting degenerate prior distributions.
%\end{itemize}
In the following section we describe how \PostNet solves these drawbacks.

\subsection{An input-dependent Bayesian posterior}

First, recall the Bayesian update of a single categorical distribution \smash{$y \sim \text{Cat}(\vect{p})$}. It consists in (1) introducing a prior Dirichlet distribution over its parameters i.e. \smash{$\prob(\vect{p}) = \text{Dir}(\bm{\beta^\text{prior}})$} where \smash{$\bm{\beta^\text{prior}} \in \mathbb{R}_+^\nclass$}, and (2) using $N$ given observations \smash{$y^{(1)},..., y^{(N)}$} to form the posterior distribution $\prob(\vect{p}|\{y^{(j)}\}_{j =1}^N)= \text{Dir}(\bm{\beta^\text{prior}} + \bm{\beta^\text{data}})$ where \smash{$\beta^\text{data}_\iclass = \sum_j \mathbb{1}_{y^{(j)}=\iclass}$} are the class counts. That is, the Bayesian update is
\begin{equation}
    \begin{aligned}
    \prob(\vect{p}|\{y^{(j)}\}_{j =1}^n) \propto \prob(\{y^{(j)}\}_{j =1}^n|\vect{p}) \times \prob(\vect{p}).
    \end{aligned}
\end{equation}
Observing no data (i.e. $\beta_\iclass^\text{data} \rightarrow 0$) would lead to flat categorical distribution (i.e. \smash{$p_\iclass = {\beta_\iclass^{\text{prior}}}\cdot ({\sum_{\iclass'}\beta_{\iclass'}^{\text{prior}}})^{-1}$}), while observing many samples (i.e. \smash{$\beta_\iclass^\text{data}$} is large) would converge to the true data distribution (i.e. $p_\iclass \approx \frac{\beta_\iclass}{\sum_i\beta_i}$).
Furthermore, we remark that $N$ behaves like a certainty budget distributed over all classes i.e. \smash{$N = \sum_\iclass \beta^\text{data}_\iclass$}.

Classification is more complex. Generally, we predict the class label \smash{$y\dataix$} from a different categorical distribution \smash{$\text{Cat}(\vect{p}\dataix)$} for each input \smash{$\vect{x}\dataix$}. \PostNetacro extends the Bayesian treatment of a single categorical distribution to classification by predicting an individual posterior update for any possible input. To this end, it distinguishes between a fixed prior parameter \smash{$ \bm{\beta}^{\text{prior}}$} and the additional learned (pseudo) counts \smash{$\bm{\beta}\dataix$} to form the parameters of the posterior Dirichlet distribution \smash{$\bm{\alpha}\dataix = \bm{\beta}^{\text{prior}} + \bm{\beta}\dataix$}. 
%\bc{Remark that directly using the single label \smash{$y\dataix$} to learn independently the posterior for input \smash{$\vect{x}\dataix$ (i.e. $\beta_\iclass\dataix = 1$} for the true class and \smash{$\beta_\iclass\dataix = 0$} otherwise) would lead to a poor uncertainty estimate and more importantly not generalize to new inputs}. 
Hence, \PostNetacro's posterior update is equivalent to predicting a set of pseudo observations \smash{$\{\tilde{y}^{(j)}\}_j\dataix$} per input \smash{$\vect{x}\dataix$}, accordingly \smash{$\beta_\iclass\dataix = \sum_j  \mathbb{1}_{\tilde{y}^{(j)}=\iclass}$} and
\begin{equation}
    \begin{aligned}
    \prob(\vect{p}\dataix|\{\tilde{y}^{(j)}\}_j\dataix) \propto \prob(\{\tilde{y}^{(j)}\}_j\dataix|\vect{p}\dataix) \times \prob(\vect{p}\dataix).
    \end{aligned}
\end{equation}
In practice, we set \smash{$\bm{\beta}^{\text{prior}}=\vect{1}$} leading to a flat equiprobable prior when the model brings no additional evidence, i.e.\ when \smash{$\bm{\beta}\dataix = \vect{0}$}. 

The parametrization of $\beta_\iclass\dataix$ is crucial and based on two main components. The first component is an encoder neural network, $f_{\theta}$ that maps a data point $\vect{x}\dataix$ onto a low-dimensional latent vector $\latent\dataix=f_{\theta}(\vect{x}\dataix) \in \mathbb{R}^{\latentdim}$. The second component is to learn a \textit{normalized} probability density $\prob(\latent | \iclass; \phi)$ per class on this latent space; intuitively acting as class conditionals in the latent space. Given these and the number of ground-truth observations $N_\iclass$ in class  $\iclass$, we define:
\begin{equation}
\begin{aligned}
\label{eq:additional_observations}
	\beta_\iclass\dataix &= N_\iclass \cdot \prob(\latent\dataix | c; \phi) = N \cdot \prob(\latent\dataix | \iclass; \phi) \cdot \prob(\iclass),
\end{aligned}
\end{equation}
which corresponds to the number of (pseudo) observations of class $\iclass$ at $\latent \dataix$. Note that it is crucial that $\prob(\latent | \iclass; \phi)$ corresponds to a proper normalized density function since this will ensure the model's epistemic uncertainty to increase outside the known distribution. Indeed, the core idea of our approach is to parameterize these distributions by a flexible, yet tractable family: normalizing flows (e.g. radial flow \cite{vi_flow} or IAF \cite{iaf_flow}). Note that normalizing flows are theoretically capable of modeling any continuous distribution given an expressive and deep enough model \cite{neural_flow, iaf_flow}.

In practice, we observed that various architectures can be used for the encoder. Also, similarly to GAN training \cite{GAN_batch_norm}, we observed that adding a batch normalization after the encoder made the training more stable. It facilitates the match between the latent positions output by the encoder and non-zero density regions learned by the normalizing flows. Remark that we can theoretically use any density estimator on the latent space. We experimentally compared Mixtures of Gaussians (MoG), radial flow \cite{vi_flow} and IAF \cite{iaf_flow}. While all density types performed reasonably well (see Tab.~\ref{fig:unc_sensorless_drive} and app.), we observed a better performance of flow-based density estimation in general. We decided to use radial flow for its good trade-off between flexibility, stability, and compactness (only few parameters).

\textbf{Model discussion.} Equation \ref{eq:additional_observations} exhibits a set of interesting properties which ensure reasonable uncertainty estimates for ID and OOD samples. To highlight the properties of the Dirichlet distributions learned by \PostNetacro, we assume in this paragraph that we have a fixed encoder $f_\theta$ and normalizing flow model parameterized by $\phi$. Writing the mean of the Dirichlet distribution parametrized by \eqref{eq:additional_observations} and using Bayes' theorem gives:
\begin{equation}
\E_{\vect{p}\sim \Dir(\boldsymbol{\alpha}\dataix)}[p_\iclass] = \frac{\beta_\iclass^\text{prior} + N \cdot \prob(\iclass | \latent\dataix; \phi) \cdot \prob(\latent\dataix; \phi)}{\sum_\iclass\beta_\iclass^\text{prior} + N \cdot \prob(\latent\dataix; \phi)}
\end{equation}
For very likely in-distribution data (i.e. $\prob(\latent\dataix ; \phi) \rightarrow \infty$), the aleatoric distribution estimate \smash{$\bar{\vect{p}}\dataix = \E_{\vect{p}\sim \Dir(\boldsymbol{\alpha}\dataix)}[p_\iclass]$} converges to the true categorical distribution \smash{$\prob(\iclass|\latent \dataix; \phi)$}. Thus, predictions are more accurate and calibrated for likely samples. Conversely, for out-of-distribution samples (i.e. \smash{$\prob(\latent\dataix ; \phi) \rightarrow 0$}), the aleatoric distribution estimate \smash{$\bar{\vect{p}}\dataix = \E_{\vect{p}\sim \Dir(\boldsymbol{\alpha}\dataix)}[p_\iclass]$} converges to the flat prior distribution (e.g. \smash{$p_\iclass=\frac{1}{\nclass}$} if \smash{$\bm{\beta}^{prior}=\vect{1}$}). In the same way, we show in the appendix that the covariance of the epistemic distribution converges to $\vect{0}$ for very likely in-distribution data, meaning no epistemic uncertainty. Thus, uncertainty for in-distribution data is reduced to the (inherent) aleatoric uncertainty and zero epistemic uncertainty.
%
Similarly to the single categorical distribution case, increasing the training dataset size (i.e. $N \rightarrow \infty$) also leads to the mean prediction $\E_{\vect{p}\sim \Dir(\boldsymbol{\alpha})}[p_\iclass]$ converging to the class posterior \smash{$\prob(c | \latent\dataix; \phi)$}. On the contrary, no observed data (i.e $N=0$) again leads to the model reverting to a flat prior.

\PostNet also handles limited certainty budgets at different levels. At the sample level, the certainty budget \smash{$\alpha_0\dataix = \sum_\iclass \alpha_\iclass\dataix$} is distributed over classes. At the class level, the certainty budget \smash{$N_\iclass = \int N_\iclass \; \prob(\latent | c; \phi) \; d\latent = N_\iclass  \int \prob(\latent | c; \phi) \; d\latent$} is distributed over samples. At the dataset level, the certainty budget \smash{$N = \sum_\iclass \int N_\iclass \; \prob(\latent | c; \phi) \; d\latent$} is distributed over classes and samples. Regions of latent space with many training examples are assigned high density $\prob(\latent | c; \phi)$, forcing low density elsewhere to fulfill the integration constraint. Consequently, density estimation using normalizing flows enables \PostNetacro to learn out-of-distribution uncertainty by observing only in-distribution data.

\begin{wrapfigure}[16]{r}{0.505\textwidth}
    \vspace{-.3cm}
	\resizebox{.505 \columnwidth}{!}{
	\includegraphics{sections/006_neurips2020/diagram/diagram_new-crop.pdf}
	}
	\caption{Overview of \PostNet.}
	\label{fig:overview}
	\vspace{-.5cm}
\end{wrapfigure}

\textbf{Overview.} In Figure~\ref{fig:overview} we provide an overview of \PostNet. We have three example inputs, \smash{$\vect{x}^{(1)}$,} \smash{$\vect{x}^{(2)}$}, and \smash{$\vect{x}^{(3)}$}, which are mapped onto their respective latent space coordinates $\latent\dataix$ by the encoding neural network $f_\theta$. The normalizing flow component learns flexible (normalized) density functions $\prob(\latent|c;\phi)$, for which we evaluate their densities at the positions of the latent vectors \smash{$\latent\dataix$}. These densities are used to parameterize a Dirichlet distribution for each data point, as seen on the right hand side. Higher densities correspond to higher confidence in the Dirichlet distributions -- we can observe that the out-of-distribution sample \smash{$\vect{x}^{(3)}$} is mapped to a point with (almost) no density, and hence its predicted Dirichlet distribution has very high epistemic uncertainty. On the other hand, \smash{$\vect{x}^{(2)}$} is an ambiguous example that could depict either the digit 0 or 6. This is reflected in its corresponding Dirichlet distribution, which has high aleatoric uncertainty (as the sample is ambiguous), but low epistemic uncertainty (since it is from the distribution of hand-drawn digits). The unambiguous sample \smash{$\vect{x}^{(1)}$} has low overall uncertainty.

Lastly, since both the encoder network $f_\theta$ and the normalizing flow parameterized by $\phi$ are fully differentiable, we can learn their parameters jointly in an end-to-end fashion. We do this via a novel loss  defined in Sec.\ref{sec:uncertainty_loss_006} which emerges from Bayesian learning principles \cite{PAC_bayesian_estimator} and is related to \UCEacro \cite{uceloss}.

\subsection{Density estimation for OOD detection}

Normalized densities, as used by \PostNetacro, are well suited to discriminate between ID data (with high likelihoood) and OOD data (with low likelihood). While it is also possible to learn a normalizing flow model on the input domain directly (e.g., \cite{NIPS2017_6828, NIPS2018_8224, grathwohl2018scalable}), this is very computationally demanding and might not be necessary for a discriminative model. Futhermore, density estimation is prone to the curse of dimensionality in high dimensional spaces \cite{typicality_OOD_generative, waic_robust_anomaly_detection}. For example, unsupervised deep generative models like \cite{glow} or \cite{pixel_cnn} have been shown to be unable to distinguish between ID and OOD samples in some situations when working on all features directly \cite{deep_generative_do_not_know, energy_based_classifier}. 

To circumvent these issues, \PostNetacro leverages two techniques. First, it uses the full class label information. Thus, \PostNetacro assigns a density per class and regularizes the epistemic uncertainty with the training class counts $N_\iclass$. Second, \PostNetacro performs density estimation on a low dimensional latent space describing relevant features for the classification task (Fig.~\ref{fig:mnist_2D_latent_space}; Fig.~\ref{fig:overview}).
Hence, \PostNetacro does not suffer from the curse of dimensionality and still enjoys the benefits of a properly normalized density function. Using the inductive bias of a discriminative task and low dimensional latent representations improved OOD detection in  \cite{nf_fail_ood} as well.

\section{Uncertainty-Aware Loss Computation}
\label{sec:uncertainty_loss_006}

A crucial design choice for neural network learning is the loss function. \PostNetacro estimates both aleatoric and epistemic uncertainty by learning a distribution $q\dataix$ for data point $\idata$ which is close to the true posterior of the categorical distribution $\vect{p}\dataix$ given the training data $\vect{x}\dataix$:
%\begin{equation}
    $q(\vect{p}\dataix) \simeq \prob(\vect{p}\dataix | \vect{x}\dataix)$.
%\end{equation}
One way to approximate the posterior is the following Bayesian loss function \cite{update-belief-propagation,PAC-bayesian_estimator,opt_info_processing_bayes}, which has the nice property of being optimal when $q\dataix$ is equal to the true posterior distribution:
\begin{equation}
\label{general_bayesian_loss}
    q^* = \underset{q\dataix \in \mathcal{P}}{\arg \min}\; \E_{\psi\dataix \sim q(\psi\dataix)}[l(\psi\dataix, \vect{x}\dataix)] - H(q\dataix),
\end{equation}
where $l$ is a generic loss over \smash{$\psi\dataix$} satisfying \smash{$ 0 < \int \exp(-l(\psi, x)) d\psi < \infty$, $\mathcal{P}$} is the family of distributions we consider and \smash{$H(q\dataix)$} denotes the entropy of $q\dataix$.

Applied in our case, \PostNet learns a distribution $q$ from the family of the Dirichlet distributions \smash{$\text{Dir}(\bm{\alpha}\dataix) = \mathcal{P}$} over the parameters \smash{$\vect{p}\dataix = \psi\dataix$}. Instantiating the loss $l$ with the cross-entropy loss (CE) we obtain the following optimization objective
\begin{equation}
\label{dirichlet_bayesian_loss}
       \min_{\theta,\phi} \mathcal{L} = \min_{\theta, \phi} \frac{1}{N} \sum_{\idata}^{N} \underbrace{\E_{q(\mathbf{p}\dataix)}[\text{CE}(\mathbf{p}\dataix, \vect{y}\dataix)]}_{\text{(1)}} - \underbrace{H(q\dataix)}_{\text{(2)}}
\end{equation}
where $\mathbf{y}\dataix$ corresponds to the one-hot encoded ground-truth class of data point $i$.
Optimizing this loss approximates the true posterior distribution for the the categorical distribution $\vect{p}\dataix$. The first term (1) corresponds the \UCE (\UCEacro) introduced by \cite{uceloss}, which is known to increase confidence for observed data. The second term (2) is an entropy regularizer, which emerges naturally and favors smooth distributions $q\dataix$. Here we are optimizing jointly over the neural network parameters $\theta$ and $\phi$. We also experimented with sequential training i.e. optimizing over the normalizing flow component only with a pre-trained model (see \cref{fig:ablation_sensorless_drive} and \cref{sec:app_additional_results}). Once trained, \PostNetacro can predict uncertainty-aware Dirichlet distributions for unseen data points.

Observe that \cref{dirichlet_bayesian_loss} is equivalent to the ELBO loss used in variational inference when using a uniform Dirichlet prior (i.e. $(1)=-\E_{q(\mathbf{p}\dataix)}[\log \prob(\vect{y}\dataix|\mathbf{p}\dataix)]$ and $(2)=\text{KL}(q\dataix||\prob(\mathbf{p}\dataix))$ where $\prob(\mathbf{p}\dataix) = \text{Dir}(\mathbf{1})$). The more general \cref{general_bayesian_loss}, however, is not necessarily equal to an ELBO loss.

Another interesting special case is to consider the family of Dirac distributions as \smash{$\mathcal{P}$} instead of the family of Dirichlet distributions. In this case we find back the traditional cross-entropy loss, which performs a simple point estimate for the distribution $\vect{p}\dataix$. CE is therefore not suited to learn a distribution with non-zero variance, as explained in \cite{uceloss}.

Other approaches, such as dropout, approximate the expectation in \UCEacro by sampling from \smash{$q\dataix$}. Our approach has the advantage of using closed-form expressions both for \UCEacro \cite{uceloss} and the entropy term, thus being efficient and exact. The weight of the entropy regularization is a hyperparameter; experiments have shown \PostNetacro to be fairly insensitive to it, so in our experiments we set it to \smash{$10^{-5}$}.

\section{Experimental Evaluation}
\label{sec:experiments_006}

In this section we compare our model to previous methods on a rich set of experiments. The code and further supplementary material is available online (\url{www.daml.in.tum.de/postnet}).

\paragraph{Baselines.} We have special focus on comparing with other models parametrizing Dirichlet distributions. We use Prior Networks (PN) trained with KL divergence (\textbf{KL-PN}) \cite{prior_net} and Reverse KL divergence (\textbf{RKL-PN}) \cite{rev_kl_prior_net}. These methods assume the knowledge of in- and out-of-distribution samples. For fair evaluation, the actual OOD test data cannot be used; instead, we used uniform noise on the valid domain as OOD training data. Additionally we trained RKL-PN with FashionMNIST as OOD data for MNIST (\textbf{RKL-PN w/ F.}). We also compare to Distribution Distillation (\textbf{Distill.}) \cite{distribution_distillation}, which learns Dirichlet distributions with maximum likelihood by using soft-labels from an ensemble of networks. 
As further baselines, we compare to dropout models (\textbf{Dropout Net}) \cite{drop_out} and ensemble methods (\textbf{Ensemble Net}) \cite{ensemble_simple}, which are state of the art in many tasks involving uncertainty estimation \cite{uncertainty_survey}. Empirical estimates of the mean and variance of \smash{$q\dataix$} are computed based on the neuron drop probability $p_{\text{drop}}$, and $m$ individually trained networks for ensemble.

All models share the same core architecture using 3 dense layers for tabular data, and 3 conv. + 3 dense layers for image data. Similarly to \cite{prior_net, rev_kl_prior_net}, we also used the VGG16 architecture \cite{vgg} on CIFAR10. We performed a grid search on $p_{\text{drop}}$, $m$, learning rate and hidden dimensions, and report results for the best configurations. Results are obtained from 5 trained models with different initializations.
Moreoever, for all experiments, we split the data into train, validation and test set  ($60\%$, $20\%$, $20\%$) and train/evaluate all models on $5$ different splits. Besides the mean we also report the standard error of the mean. Further details are given in the appendix.

\paragraph{Datasets.} We evaluate on the following real-world datasets: \textbf{Segment} \cite{uci_datasets}, \textbf{Sensorless Drive} \cite{uci_datasets}, \textbf{MNIST} \cite{mnist} and \textbf{CIFAR10} \cite{cifar10}. The former two datasets (Segment and Sensorless Drive) are tabular datasets with %unbounded input domain of 
dimensionality 18 and 49 and with 7 and 11 classes, respectively. We rescale all inputs between $[0, 1]$ by using the $\min$ and $\max$ value of each dimension from the training set. Additionally, we compare all models on 2D synthetic data composed of three Gaussians each. Datasets are presented in  more detail in the appendix.

\paragraph{Uncertainty Metrics.} We follow the method proposed in \cite{uncertainty_survey} and evaluate the coherence of confidence, uncertainty calibration and OOD detection. Note that our goal is not to improve accuracy; still we report the numbers in the experiments.

\underline{Confidence calibration:} We aim to answer `\textit{Are more confident (i.e. less uncertain) predictions more likely to be correct?}'. We use the area under the precision-recall curve (AUC-PR) to measure confidence calibration. For aleatoric confidence calibration (\textbf{Alea. Conf.}) we use \smash{$\underset{\iclass}{\max} \; \bar{\vect{p}}_\iclass\dataix$} as the scores with labels 1 for correct and 0 for incorrect predictions. For epistemic confidence calibration (\textbf{Epist. Conf.}), we distinguish Dirichlet-based models, and dropout and ensemble models. For the former we use \smash{$\underset{\iclass}{\max}\; \bm{\alpha}_\iclass\dataix$} as scores, and for the latter we use the (inverse) empirical variance \smash{$\tilde{\vect{p}}_c\dataix$} of the predicted class, estimated from 10 samples.

\underline{Uncertainty calibration:} We used Brier score (\textbf{Brier}), which is computed as $\frac{1}{N}\sum_{i}^N \|\bar{\vect{p}}\dataix - \vect{y}\dataix\|_2$, where $\vect{y}\dataix$ is the one-hot encoded ground-truth class of data point $i$. For Brier score, lower is better.

\underline{OOD detection:} Our main focus lies on the models' ability to detect out-of-distribution samples. We used AUC-PR to measure performance. For aleatoric OOD detection (\textbf{Alea. OOD}), the scores are \smash{$\underset{\iclass}{\max} \; \bar{\vect{p}}_\iclass\dataix$} with labels 1 for ID data and 0 for OOD data. Fo epistemic OOD detection (\textbf{Epist. OOD}), the scores for Dirichlet-based models are given by \smash{$\alpha_0\dataix = \sum_\iclass \alpha_\iclass\dataix$}, while we use the (inverse) empirical variance \smash{$\tilde{\vect{p}}\dataix$} for ensemble and dropout models. To provide a comprehensive overview of OOD detection results we use different types of OOD data as described in the following: 
\begin{itemize}
    \item \textit{Unseen datasets.} We use data from other datasets as OOD data for the image-based models. We use data from FashionMNIST \cite{fashionmnist} and K-MNIST \cite{kmnist}  as OOD data for models trained on MNIST, and data from SVHN \cite{svhn} as OOD for CIFAR10.
    \item \textit{Left-out classes.} For the tabular datasets (Segment and Sensorless Drive) there are no other datasets that are from the same domain. To simulate OOD data we remove one or more classes from the training data and instead consider them as OOD data. We removed one class (class sky) from the Segment dataset and two classes from Sensorless Drive (class 10 and 11).
    \item \textit{Out-of-domain.} In this novel evaluation we consider an extreme case of OOD data for which the data comes from different value ranges (\textbf{OODom}). E.g., for images we feed unscaled versions in the range $[0, 255]$ instead of scaled versions in $[0,1]$. We argue that models should easily be able to detect data that is extremely far from the data distribution. However, as it turns out, this is surprisingly difficult for many baseline models.
    \item \textit{Dataset shifts.} Finally, for CIFAR10, we use 15 different image corruptions at 5 different severity levels \cite{benchmarking_corruptions}. This setting evaluates the models' ability to detect low-quality data (Fig.~\ref{cifar_shifts}b,c). 
\end{itemize}

\begin{table}
    \centering
    \resizebox{.9 \textwidth}{!}{%
\begin{tabular}{lllllll}
\toprule
{} &  \textbf{Acc.} & \textbf{Alea. Conf.} & \textbf{Epist. Conf.} & \textbf{Brier} & \textbf{OOD Alea.} & \textbf{OOD Epist.} \\
\midrule
\midrule
\textbf{Drop Out  } &  89.32$\pm$0.2 &        98.21$\pm$0.1 &         95.24$\pm$0.2 &  28.86$\pm$0.4 &      35.41$\pm$0.4 &       40.61$\pm$0.7 \\
\textbf{Ensemble  } &  99.37$\pm$0.0 &        99.99$\pm$0.0 &         *99.98$\pm$0.0 &   2.47$\pm$0.1 &      50.01$\pm$0.0 &       50.62$\pm$0.1 \\
\midrule
\textbf{Distill.  } &  93.66$\pm$1.5 &        98.29$\pm$0.5 &         98.15$\pm$0.5 &  44.94$\pm$1.4 &       32.1$\pm$0.6 &       31.17$\pm$0.2 \\
\textbf{KL-PN     } &  94.77$\pm$0.9 &        99.52$\pm$0.1 &         99.47$\pm$0.1 &  21.47$\pm$1.9 &      35.48$\pm$0.8 &        33.2$\pm$0.6 \\
\textbf{RKL-PN    } &  99.42$\pm$0.0 &        99.96$\pm$0.0 &         99.89$\pm$0.0 &   9.07$\pm$0.1 &      45.89$\pm$1.6 &       38.14$\pm$0.8 \\
\textbf{PostN Rad.} &  98.02$\pm$0.1 &        99.89$\pm$0.0 &         99.47$\pm$0.0 &   5.51$\pm$0.2 &      72.89$\pm$0.8 &       \textbf{*88.73$\pm$0.5} \\
\textbf{PostN IAF } &  \textbf{*99.52$\pm$0.0} &        \textbf{*100.0$\pm$0.0} &         \textbf{99.92$\pm$0.0}&   \textbf{*1.43$\pm$0.1} &      \textbf{*82.96$\pm$0.8} &       88.65$\pm$0.4 \\
\bottomrule
\end{tabular}

    }
    \caption{Results on Sensorless Drive dataset. Bold numbers indicate best score among Dirichlet parametrized models and starred numbers indicate best scores among all models.}
    \label{fig:unc_sensorless_drive}
%\vspace{-.0cm}
\end{table}

\textbf{Results.} 
Results for the Sensorles Drive dataset are shown in Tab.~\ref{fig:unc_sensorless_drive}. Tables for other datasets are in the appendix. Even without requiring expensive sampling, \PostNetacro performs on par for accuracy and confidence scores with other models, brings a significant improvement for calibration within the Dirichlet-based models, and outperforms all other models by a large margin (more than $+30\%$ abs. improvement) for OOD detection. Radial flow and IAF variants both achieve strong performance for all datasets (see app.). We use the smaller model (i.e. Radial flow) for comparison in the following. In our experiments, note that using one Radial flow per class represents a small overhead of only $80$ parameters per class, which is negligible compared to the encoder architectures (e.g. VGG16 has 138M parameters).

\begin{figure}
    \centering
    \begin{subfigure}[t]{0.33 \textwidth}
        \centering
        \includegraphics[width=0.5 \textwidth]{sections/006_neurips2020/figures/three_gaussians_no_flow-crop.pdf}
        \caption{\PostNetacro: \NoFlow}
    \end{subfigure}%   
    \begin{subfigure}[t]{0.33 \textwidth}
        \centering
        \includegraphics[width=0.5 \textwidth]{sections/006_neurips2020/figures/three_gaussians_no_UCE-crop.pdf}
        \caption{\PostNetacro: \NoUCE}
    \end{subfigure}%   
    \begin{subfigure}[t]{0.33 \textwidth}
        \centering
        \includegraphics[width=0.5 \textwidth]{sections/006_neurips2020/figures/three_gaussians_normal-crop.pdf}
        \caption{\PostNetacro: Complete}
    \end{subfigure}%

    \caption{Uncertainty visualization for a 2D 3-Gaussians dataset. Red dots indicate the Gaussians means. Darker regions indicate high epistemic uncertainty for a class prediction. Ablated models fail even a simple dataset while \PostNetacro shows high certainty around gaussians means only.}
    \label{fig:toy_ablation}
	%\vspace{-.0cm}
\end{figure}

\begin{table}[H]
\resizebox{1 \textwidth}{!}{%
\begin{tabular}{lllllll}
\toprule
{} &  \textbf{Acc.} & \textbf{Alea. Conf.} & \textbf{Epist. Conf.}  & \textbf{Brier} & \textbf{OOD Alea.} & \textbf{OOD Epist.} \\
\midrule
\midrule
\textbf{PostN: No-Flow        } &  \cellcolor{Gray} 55.38$\pm$0.7 & \cellcolor{Gray} 85.46$\pm$0.3 & \cellcolor{Gray} 82.58$\pm$0.6 & \cellcolor{Gray} 64.4$\pm$0.6 & \cellcolor{Gray} 29.59$\pm$0.1 & \cellcolor{Gray} 31.15$\pm$0.4 \\
\textbf{PostN: No-Bayes-Loss} &   96.6$\pm$0.2 & 99.74$\pm$0.0 & 98.68$\pm$0.1 & \cellcolor{Gray} 8.85$\pm$0.4 & \cellcolor{Gray} 62.39$\pm$1.5 & \cellcolor{Gray} 82.63$\pm$1.4 \\
\textbf{PostN: Seq-No-Bn   } &  \cellcolor{Gray} 15.09$\pm$1.0 & \cellcolor{Gray} 39.88$\pm$7.2 &  \cellcolor{Gray}  39.86$\pm$7.2 &  \cellcolor{Gray} 89.88$\pm$1.3 & \cellcolor{Gray} 57.19$\pm$2.5 & \cellcolor{Gray} 56.74$\pm$2.4 \\
\textbf{PostN: Seq-Bn    } &  98.42$\pm$0.1 & 99.92$\pm$0.0 & 98.76$\pm$0.1 &   5.41$\pm$0.1 & \cellcolor{Gray} 52.35$\pm$0.7 &  \cellcolor{Gray} 71.75$\pm$1.9 \\
\bottomrule
\end{tabular}

    }
    \caption{Ablation study results on Sensorless Drive dataset. Gray cells indicate significant drops in scores compared to complete \PostNetacro Rad. in Tab.~\ref{fig:unc_sensorless_drive}.}
        \label{fig:ablation_sensorless_drive}
    % \vspace{-.5cm}
\end{table}

We performed an ablation study on each component of \PostNetacro to evaluate their individual contributions. We were especially interested in comparing stability and uncertainty estimates. Thus, we removed independently the normalizing flow component (\NoFlow) and the novel Bayesian loss (\NoUCE) replaced by the classic cross-entropy loss. Furthermore, we used pre-trained models and subsequently only trained the normalizing flow component, with or without a batch normalization layer (\SeqBn and \SeqNoBn). We report results in Tab.~\ref{fig:ablation_sensorless_drive}. \NoFlow has a significant drop in OOD detection scores similarly to Prior Networks; not surprising since they mainly differ by their loss. This underlines the importance of using normalized density estimation to differentiate ID and OOD data. The lower performance of \NoUCE compared to the original model indicates the benefit of using our Bayesian loss.
 \SeqBn obtains good performance for some of the metrics, which as a by-product, allows to estimate uncertainty on pre-trained models. Though, we noticed better performance for joint training in general. As shown by \SeqNoBn scores, the batch normalization layer brings stability. It intuitively facilitates predicted latent positions to lie on non-zero density regions. Similar conclusions can be drawn on the toy dataset (see Fig.~\ref{fig:toy_ablation}) and the Segment dataset (see app.). We further compare various density types and latent dimensions in appendix. We noticed that a too high latent dimension leads to a performance decrease. We also observed that flow-based density estimation generally achieves better scores.

\begin{table}[H]
    \resizebox{1 \textwidth}{!}{%
\begin{tabular}{lllllllll}
\toprule
{} & \textbf{OOD K.} & \textbf{OOD K.} & \textbf{OOD F.} & \textbf{OOD F.} & \textbf{OODom K.} & \textbf{OODom K.} & \textbf{OODom F.} & \textbf{OODom F.} \\
{} & \textbf{Alea.} & \textbf{Epist.} & \textbf{Alea.} & \textbf{Epist.} & \textbf{Alea.} & \textbf{Epist.} & \textbf{Alea.} & \textbf{Epist.} \\
\midrule
\midrule
\textbf{RKL-PN      } &         60.76$\pm$2.9 &          53.76$\pm$3.4 &         78.45$\pm$3.1 &          72.18$\pm$3.6 &            9.35$\pm$0.1 &             8.94$\pm$0.0 &            9.53$\pm$0.1 &             8.96$\pm$0.0 \\
\textbf{RKL-PN w/ F.} &         81.34$\pm$4.5 &          78.07$\pm$4.8 &         \textbf{100.0$\pm$0.0} &          \textbf{100.0$\pm$0.0} &            9.24$\pm$0.1 &             9.08$\pm$0.1 &           88.96$\pm$4.4 &            87.49$\pm$5.0 \\
\textbf{PostN       } &         \textbf{95.75$\pm$0.2} &          \textbf{94.59$\pm$0.3} &         97.78$\pm$0.2 &          97.24$\pm$0.3 &           \textbf{100.0$\pm$0.0} &            \textbf{100.0$\pm$0.0} &           \textbf{100.0$\pm$0.0} &            \textbf{100.0$\pm$0.0} \\
\bottomrule
\end{tabular}

    }
    \caption{Results on MNIST for OOD detection against KMNIST (K.) and FashionMNIST (F.). We trained Rev. KL divergence PriorNets with uniform noise (RKL-PN) and Fashion MNIST (RKL-PN w/ F.) as OOD. \PostNetacro requires no OOD data. Larger numbers are better.}
    \label{fig:unc_MNIST}
    % \vspace{-.5cm}
\end{table}

Results of the comparison between RKL-PN, RKL-PN w/ F and \PostNetacro for OOD detection on MNIST are shown in Tab.~ \ref{fig:unc_MNIST}. Not surprisingly, the usage of FashionMNIST as OOD data for training helped RKL-PN to detect other FashionMNIST data. Except for FashionMNIST OOD, \PostNetacro still outperforms RKL-PN w/ F. in OOD detection for other datasets. We noticed that tabular datasets, defined on an unbounded input domain, are more difficult for baselines. One explanation is that due to the $\min$/$\max$ normalization it can happen that test samples lie outside the interval $[0,1]$ observed during training. For images, the input domain is compact, which allows to define a valid distribution for OOD data (e.g. uniform) which makes OODom data challenging (see OOD vs OODom in Tab.~\ref{fig:unc_MNIST}).

\begin{table}[H]
    \resizebox{1 \textwidth}{!}{%
\begin{tabular}{lllllllll}
\toprule
{} &  \textbf{Acc.} & \textbf{Alea. Conf.} & \textbf{Epist. Conf.}  & \textbf{Brier} & \textbf{OOD Alea.} & \textbf{OOD Epist.} & \textbf{OODom Alea.} & \textbf{OODom Epist.} \\
\midrule
\midrule
\textbf{Drop Out C.} &  71.73$\pm$0.2 &        92.18$\pm$0.1 &          84.38$\pm$0.3 &  49.76$\pm$0.2 &      \textbf{72.94$\pm$0.3} &       41.68$\pm$0.5 &         28.3$\pm$1.8 &          47.1$\pm$3.3 \\
\textbf{KL-PN C.   } &  48.84$\pm$0.5 &        78.01$\pm$0.6 &          77.99$\pm$0.7 &  83.11$\pm$0.6 &      59.32$\pm$1.1 &       58.03$\pm$0.8 &        17.79$\pm$0.0 &         20.25$\pm$0.2 \\
\textbf{RKL-PN C.  } &  62.91$\pm$0.3 &        85.62$\pm$0.2 &          81.73$\pm$0.2 &  58.12$\pm$0.4 &      67.07$\pm$0.4 &       56.64$\pm$0.8 &        17.83$\pm$0.0 &         17.76$\pm$0.0 \\
\textbf{PostN C.   } &  \textbf{76.46$\pm$0.3} &        \textbf{94.75$\pm$0.1} &          \textbf{94.34$\pm$0.1} &  \textbf{37.39$\pm$0.4} &      72.83$\pm$0.6 &       \textbf{72.82$\pm$0.7} &        \textbf{100.0$\pm$0.0} &         \textbf{100.0$\pm$0.0} \\
\midrule
\textbf{Drop Out V.} &  82.84$\pm$0.1 &        97.15$\pm$0.0 &           96.6$\pm$0.0 &  27.15$\pm$0.2 &      51.39$\pm$0.1 &       53.64$\pm$0.1 &        51.38$\pm$0.1 &         53.66$\pm$0.1 \\
\textbf{KL-PN V.   } &  27.46$\pm$1.7 &        50.61$\pm$4.0 &          52.49$\pm$4.2 &  87.28$\pm$1.0 &      43.96$\pm$1.9 &       43.23$\pm$2.3 &        18.14$\pm$0.1 &         19.12$\pm$0.4 \\
\textbf{RKL-PN V.  } &  64.76$\pm$0.3 &        86.11$\pm$0.4 &          85.59$\pm$0.3 &  54.73$\pm$0.4 &      53.61$\pm$1.1 &       49.37$\pm$0.8 &        29.07$\pm$2.1 &         24.84$\pm$1.3 \\
\textbf{PostN V.   } &  \textbf{84.85$\pm$0.0} &        \textbf{97.76$\pm$0.0} &          \textbf{97.25$\pm$0.0} &  \textbf{22.84$\pm$0.0} &      \textbf{80.21$\pm$0.2} &       \textbf{77.71$\pm$0.3} &        \textbf{91.35$\pm$0.5} &         \textbf{99.25$\pm$0.1} \\
\bottomrule
\end{tabular}

    }
    \caption{Results on CIFAR10 with simple convolutional architectures (C.) and VGG16 (V.). Bold numbers indicate best score among one architecture type.}
    \label{fig:unc_CIFAR10}
    % \vspace{-.5cm}
\end{table}

Uncertainty estimation should be good regardless of the model accuracy. It is even more important for less accurate models since they actually \emph{do not know} (i.e.\ they do more mistakes). Thus, we compared the models that use a single network for training (using a convolutional architecture and VGG16) in Tab.~\ref{fig:unc_CIFAR10}. Without the knowledge of true OOD data (SVHN) during training, Prior Networks struggle to achieve good performance. In contrast, \PostNetacro outputs high quality uncertainty estimates regardless of the architecture used for the encoder. We report additional results for PostNet using other encoder architectures (convolutional architecture, AlexNet \cite{alexnet}, VGG \cite{vgg} and ResNet \cite{resnet}) in  Table~\ref{tab:architecture_CIFAR10}. Deep generative models as Glow \cite{glow} using density estimation on input space are unable to distinguish between CIFAR10 and SVHN \cite{deep_generative_do_not_know}. In contrast, \PostNetacro clearly distinguishes between in-distribution data (CIFAR10) with low entropy, out-of-distribution (OOD SVHN) with high entropy, and close to the maximum possible entropy for out-of-domain data (OODom SVHN) (see Fig.~\ref{cifar_shifts}a). Similar conclusions hold for MNIST and FashionMNIST (see app.). Furthermore, results for the image perturbations on CIFAR10 introduced by \cite{benchmarking_corruptions} are presented in Fig.~\ref{cifar_shifts}.  We define the average change in confidence as the ratio between the average confidence \smash{$\frac{1}{N}\sum_i^{N}\alpha_0\dataix$} at severity 1 vs other severity levels. As larger shifts correspond to larger differences in the underlying distributions, we expect uncertainty-aware models to become less certain for more severe perturbations. \PostNet exhibits, as desired, the largest decrease in confidence with stronger corruptions (see Fig.~\ref{cifar_shifts}b) while maintaining a high accuracy (see Fig.~\ref{cifar_shifts}c).

\begin{table}[ht]
    \resizebox{1 \textwidth}{!}{%
\begin{tabular}{lllllllll}
\toprule
{} &  \textbf{Acc.} & \textbf{Alea. Conf.} & \textbf{Epist. Conf.}  & \textbf{Brier} & \textbf{OOD Alea.} & \textbf{OOD Epist.} & \textbf{OODom Alea.} & \textbf{OODom Epist.} \\
\midrule
\textbf{\PostNetacro: Conv.  } &  78.58$\pm$0.1 &        95.45$\pm$0.0 &          93.36$\pm$0.0 &  33.84$\pm$0.2 &      72.21$\pm$0.1 &       57.72$\pm$0.7 &        100.0$\pm$0.0 &         100.0$\pm$0.0 \\
\textbf{\PostNetacro: Alexnet} &  80.81$\pm$0.2 &        96.33$\pm$0.1 &          95.35$\pm$0.1 &  29.99$\pm$0.3 &       73.4$\pm$0.7 &       67.05$\pm$0.6 &        97.64$\pm$0.4 &         99.64$\pm$0.1 \\
\textbf{\PostNetacro: VGG    } &  84.85$\pm$0.0 &        97.76$\pm$0.0 &          97.25$\pm$0.0 & 22.84$\pm$0.0 &      80.21$\pm$0.2 &       77.71$\pm$0.3 &        91.35$\pm$0.5 &         99.25$\pm$0.1 \\
\textbf{\PostNetacro: Resnet } &  87.86$\pm$0.2 &        98.35$\pm$0.0 &          97.13$\pm$0.0 &  19.33$\pm$0.3 &      79.92$\pm$0.4 &       72.25$\pm$0.6 &        99.94$\pm$0.0 &         99.94$\pm$0.0 \\
\bottomrule
\end{tabular}

    }
    \caption{Results of \PostNetacro with different encoder architectures. It shows good uncertainty estimation regardless of the architecture complexity.}
    \label{tab:architecture_CIFAR10}
\end{table}

\begin{figure}[H]
    % \vspace{-.5cm}
    \centering
    \begin{subfigure}[t]{0.33 \textwidth}
        \centering
        \includegraphics[width=.98 \textwidth]{sections/006_neurips2020/figures/entropy_CIFAR10.png}
        \caption{ID/OOD/OODom entropy}
    \end{subfigure}%  
    \begin{subfigure}[t]{0.33 \textwidth}
        \centering
        \includegraphics[width=.98 \textwidth]{sections/006_neurips2020/figures/shifts_CIFAR10_conf.png}
        \caption{Confidence under data shifts}
    \end{subfigure}%   
    \begin{subfigure}[t]{0.33 \textwidth}
        \centering
        \includegraphics[width=.98 \textwidth]{sections/006_neurips2020/figures/shifts_CIFAR10_acc.png}
        \caption{Accuracy under data shifts}
    \end{subfigure}%   
    \caption{(a) shows entropy of the aleatoric distributions predicted by \PostNetacro on CIFAR10 (ID) and SVHN (OOD, OODom). The value $2.3026^*$ denotes the highest achievable entropy for 10 classes. \PostNetacro can easily distinguish between the three data types. (b) and (c) present averaged confidence and accuracy under 15 dataset shifts introduced by \cite{benchmarking_corruptions} on CIFAR10 with conv. architecture. On more severe perturbations (i.e. data further away from data distribution), \PostNetacro assigns higher epistemic uncertainty as desired. Baselines keeps same confidence even for less accurate predictions.}
    \label{cifar_shifts}
    %\vspace{-.5cm}
\end{figure}
\section{Conclusion}
\label{sec:conclusion_008}

This chapter analyzes robustness of uncertainty estimation by DBU models and answers multiple questions in this context. Our results show: (1) While uncertainty estimates are a good indicator to identify correctly classified samples on unperturbed data, performance decrease drastically on perturbed data-points. (2) None of the Dirichlet-based uncertainty models is able to detect PGD label attacks against the class prediction by uncertainty estimation, regardless of the used uncertainty measure. (3) Detecting OOD samples and distinguishing between ID-data and OOD-data is not robust. (4) Applying median smoothing to  uncertainty estimates increases robustness of DBU models w.r.t. all analyzed tasks, while adversarial training based on label or uncertainty attacks resulted in minor improvements. 






\section*{Broader Impact}
\label{ethical_impact}

Traditional classification models without uncertainty estimate can be dangerous when used without domain expertise. They might have indeed unexpected behavior in new anomalous situations/input and are unaware of the underlying risk of their predictions. Uncertainty aware models like \oursacro try to mitigate the risk of such autonomous predictions by attaching a confidence score to their predictions. On one hand, uncertainty aware predictions could be particularly beneficial in domains with potential critical consequences and prone to automation (e.g. finance, autonomous driving or medicine). When applied, these models are able to refrain from predicting if the data is out of their domain of expertise. \ours makes a significant step further in this direction by even not requiring to observe similar anomalous situations during training. Anomalous data are typically not known in advance since they are rare by definition. Thus \ours significantly increases the applicability of uncertainty estimation across application domains. On the other hand, high-quality of uncertainty estimation might also give a false sense of security. A potential risk is that an excessive trust in the model behavior leads to a lack of human supervision. For example in medicine, predictions wrongly deemed safe could have dramatic repercussions without human control.
\begin{ack}
This research was supported by the BMW AG. The authors would like to thank Bernhard Schlegel for helpful discussion and comments. 
\end{ack}

\bibliography{main}
\bibliographystyle{plain}
\clearpage
\section{Appendix}
\label{sec:appendix}

%\subsection{Dirichlet-based uncertainty models}
%
%In this section, we provide details on the losses used by each DBU model. \PostNet uses a Bayesian loss which can be expressed as follows:
%
%\begin{equation}
%\begin{aligned}
%    L_{\mathrm{\PostNet}} &= \frac{1}{N} \sum_i \E_{q(p\dataix)}  [\mathrm{CE} (p\dataix, y\dataix)] - H(q\dataix)
%\end{aligned}
%\end{equation}
%where $\mathrm{CE}$ denotes the cross-entropy. Both the expectation term (i.e. $\E_{q(p\dataix)}  [\mathrm{CE} (p\dataix, y\dataix)]$) and the entropy term (i.e. $H(q\dataix)$) can be computed in closed-form \citep{charpentier2020}. \PriorNet uses a loss composed of two KL divergence terms for ID and OOD data: 
%\begin{equation}
%\begin{aligned}
%    L_{\mathrm{\PriorNet}} &= \frac{1}{N} \left[\sum_{\vx\dataix \in \text{ID data}}  [\mathrm{KL} [\mathrm{Dir} (\alpha^{\mathrm{ID}}) || q\dataix]]  \right. \\
%                           &+ \left.\sum_{\vx\dataix \in OOD data} [\mathrm{KL} [\mathrm{Dir} (\alpha^{\mathrm{OOD}}) || q\dataix]]\right]. \\
%\end{aligned}
%\end{equation}
%Both KL divergences terms can be computed in closed-form \citep{malinin2019}. The precision $\alpha^{\mathrm{ID}}$ and $\alpha^{\mathrm{OOD}}$ are hyper-parameters. The precision $\alpha^{\mathrm{ID}}$ is usually set to $1e^{1}$ for the correct class and $1$ otherwise. The precision $\alpha^{\mathrm{OOD}}$ is usually set to $\mathbf{1}$. \DDNet uses use the Dirichlet likelihood of soft labels produce by an ensemble of $M$ neural networks:
%\begin{equation}
%\begin{aligned}
%    L_{\mathrm{\DDNet}} &= - \frac{1}{N}  \sum_i \sum_{m=1}^{M} [\ln q\dataix(\pi^{im})] \\
%\end{aligned}
%\end{equation}
%where $\pi^{im}$ denotes the soft-label of $m$th neural network. The Dirichlet likelihood can be computed in closed-form \citep{malinin2019ensemble}. \EvNet uses the expected mean square error between the one-hot encoded label and the predicted categorical distribution:
%
%\begin{equation}
%\begin{aligned}
%    L_{\mathrm{\EvNet}} &= \frac{1}{N} \sum_i \E_{\vp\dataix \sim \text{Dir}(\bm{\alpha}\dataix)}||\vy*\dataix - \vp\dataix||^2 \\
%\end{aligned}
%\end{equation}
%where $\vy*\dataix$ denotes the one-hot encoded label. The expected MSE loss can also be computed in closed form \citep{sensoy2018}.
%For more details please have a look at the original paper on \PriorNet \citep{malini2018}, \PostNet \citep{charpentier2020}, \DDNet \citep{malinin2019} and \EvNet \citep{sensoy2018}. 


\subsection{Closed-form computation of uncertainty measures \& Uncertainty attacks}
\label{subsec:appendix_measurecomp}

Dirichlet-based uncertainty models allow to compute several uncertainty measures in closed form (see \citep{malini2018} for a derivation). As proposed by \cite{malini2018}, we use precision~$m_{\alpha_0}$, differential entropy~$m_{\mathrm{diffE}}$ and mutual information~$m_{\mathrm{MI}}$ to estimate uncertainty on predictions.

The differential entropy $m_{\mathrm{diffE}}$ of a DBU model reaches its maximum value for equally probable categorical distributions and thus, a on flat Dirichlet distribution. It is a measure for distributional uncertainty and expected to be low on ID data, but high on OOD data. 
%
\begin{equation}
\begin{aligned}
	%m_{\mathrm{diffE}}  &= \sum_c^K \ln \Gamma (\alpha_c) - \ln \Gamma (\alpha_0) - \sum_c^K (\alpha_c -1) \cdot (\Psi (\alpha_c) - \Psi (\alpha_0))
	m_{\mathrm{diffE}}  = &\sum_c^K \ln \Gamma (\alpha_c) - \ln \Gamma (\alpha_0) \\
	&- \sum_c^K (\alpha_c -1) \cdot (\Psi (\alpha_c) - \Psi (\alpha_0))
\end{aligned}
\end{equation}
%
where $\alpha$ are the parameters of the Dirichlet-distribution, $\Gamma$ is the Gamma function and $\Psi$ is the Digamma function. 


The mutual information $m_{\mathrm{MI}}$ is the difference between the total uncertainty (entropy of the expected distribution) and the expected uncertainty on the data (expected entropy of the distribution). This uncertainty is expected to be low on ID data and high on OOD data. 
%
\begin{equation}
\begin{aligned}
	m_{\mathrm{MI}}  &= - \sum_{c=1}^{K} \frac{\alpha_c}{\alpha_0} \left( \ln \frac{\alpha_c}{\alpha_0} - \Psi(\alpha_c +1) + \Psi (\alpha_0 +1) \right)
\end{aligned}
\end{equation}

Furthermore, we use the precision~$\alpha_0$ to measure uncertainty, which is expected to be high on ID data and low on OOD data.
%
\begin{equation}
\begin{aligned}
	m_{\alpha_0}        &= \alpha_0 = \sum_{c=1}^{K} \alpha_c 
\end{aligned}
\end{equation}


As these uncertainty measures are computed in closed form and it is possible to obtain their gradients, we use them (i.e. $m_{\mathrm{diffE}}$, $m_{\mathrm{MI}}$, $m_{\alpha_0}$) are target function of our uncertainty attacks. Changing the attacked target function allows us to use a wide range of gradient-based attacks such as FGSM attacks, PGD attacks, but also more sophisticated attacks such as Carlini-Wagner attacks. 



\subsection{Details of the Experimental setup}
\label{subsec:exp_setup}

\textbf{Models.} We trained all models with a similar based architecture. We used namely 3 linear layers for vector data sets, 3 convolutional layers with size of 5 + 3 linear layers for MNIST and the VGG16 \cite{vgg} architecture with batch normalization for CIFAR10. All the implementation are performed using Pytorch \citep{pytorch}. We optimized all models using Adam optimizer. We performed early stopping by checking for loss improvement every 2 epochs and a patience of 10. The models were trained on GPUs (1 TB SSD).

We performed a grid-search for hyper-parameters for all models. The learning rate grid search was done in $[1e^{-5}, 1e^{-3}]$. For \PostNet, we used Radial Flows with a depth of 6 and a latent space equal to 6. Further, we performed a grid search for the regularizing factor in $[1e^{-7}, 1e^{-4}]$. For \PriorNet, we performed a grid search for the OOD loss weight in $[1, 10]$. For \DDNet, we distilled the knowledge of $5$ neural networks after a grid search in $[2, 5, 10, 20]$ neural networks. Note that it already implied a significant overhead at training compare to other models.

\textbf{Metrics.} For all experiments, we focused on using AUC-PR scores since it is well suited to imbalance tasks \citep{imbalance_apr} while bringing theoretically similar information than AUC-ROC scores \citep{apr_auroc}. We scaled all scores from $[0, 1]$ to $[0, 100]$. All results are average over 5 training runs using the best hyper-parameters found after the grid search.

\textbf{Data sets.} For vector data sets, we use 5 different random splits to train all models. We split the data in training, validation and test sets ($60\%$, $20\%$, $20\%$). 

We use the segment vector data set \cite{uci_datasets}, where the goal is to classify areas of images into $7$ classes (window, foliage, grass, brickface, path, cement, sky). We remove class window from ID training data to provide OOD training data to \PriorNet. Further, We remove the class 'sky' from training and instead use it as the OOD data set for OOD detection experiments. Each input is composed of $18$ attributes describing the image area. The data set contains $2,310$ samples in total.

We further use the Sensorless Drive vector data set \cite{uci_datasets}, where the goal is to classify extracted motor current measurements into $11$ different classes. We remove class 9 from ID training data to provide OOD training data to \PriorNet. We remove classes 10 and 11 from training and use them as the OOD dataset for OOD detection experiments. Each input is composed of $49$ attributes describing motor behaviour. The data set contains $58,509$ samples in total.

Additionally, we use the MNIST image data set \cite{mnist} where the goal is to classify pictures of hand-drawn digits into $10$ classes (from digit $0$ to digit $9$). Each input is composed of a $1 \times 28 \times 28$ tensor. The data set contains $70,000$ samples. For OOD detection experiments, we use FashionMNIST \cite{fashionmnist} and KMNIST \cite{kmnist} containing images of Japanese characters and images of clothes, respectively. FashionMNIST was used as training OOD for \PriorNet while KMNIST is used as OOD at test time.

Finally, we use the CIFAR10 image data set \cite{cifar10} where the goal is to classify a picture of objects into $10$ classes (airplane, automobile, bird, cat, deer, dog, frog, horse, ship, truck). Each input is a $3 \times 32 \times 32$ tensor. The data set contains $60,000$ samples. For OOD detection experiments, we use street view house numbers (SVHN) \cite{svhn}  and CIFAR100 \citep{cifar10} containing images of numbers and objects respectively. CIFAR100 was used as training OOD for \PriorNet while SVHN is used as OOD at test time.
 
\textbf{Perturbations.} For all label and uncertainty attacks, we used Fast Gradient Sign Methods and Project Gradient Descent. We tried 6 different attack radii $[0.0, 0.1, 0.2, 0.5, 1.0, 2.0, 4.0]$. These radii operate on the input space after data normalization. We bound perturbations by~$L_{\infty}$-norm or by~$L_2$-norm, with 
%
\begin{equation}
\begin{aligned}
	L_{\infty} (x) = \max_{i=1,\dots, D} \left|x_i\right| \mathrm{~~~~and~~~~}
	L_2 (x)        = (\sum_{i=1}^{D} x_i^2)^{0.5}.
\end{aligned}
\end{equation}
%
For $L_{\infty}$-norm it is obvious how to relate perturbation size~$\varepsilon$ with perturbed input images, because all inputs are standardized such that the values of their features are between~$0$ and~$1$.
A perturbation of size~$\varepsilon=0$ corresponds to the original input, while a perturbation of size~$\varepsilon=1$ corresponds to the whole input space and allows to change all features to any value. 

For~$L_2$-norm the relation between perturbation size~$\varepsilon$ and perturbed input images is less obvious. To justify our choice for~$\varepsilon$ w.r.t. this norm, we relate perturbations size~$\varepsilon_2$ corresponding to $L_2$-norm with perturbations size~$\varepsilon_{\infty}$ corresponding to $L_{\infty}$-norm. 
First, we compute~$\varepsilon_2$, such that the $L_2$-norm is the smallest super-set of the $L_{\infty}$-norm. Let us consider a perturbation of~$\varepsilon_{\infty}$. The largest~$L_2$-norm would be obtained if each feature is perturbed by~$\varepsilon_{\infty}$. Thus, perturbation~$\varepsilon_2$, such that $L_2$ encloses~$L_{\infty}$ is $\varepsilon_2 = (\sum_{i=1}^{D} \varepsilon_{\infty}^2)^{0.5} = \sqrt{D} \varepsilon_{\infty}$. For the MNIST-data set, with $D=28 \times 28$ input features $L_2$-norm with $\varepsilon_2=28$ encloses $L_{\infty}$-norm with~$\varepsilon_{\infty}=1$. 

Alternatively, $\varepsilon_2$ can be computes such that the volume spanned by~$L_2$-norm is equivalent to the one spanned by~$L_{\infty}$-norm. Using that the volume spanned by $L_{\infty}$-norm is $\varepsilon_{\infty}^D$ and the volume spanned by $L_2$-norm is 
$\frac{\pi^{0.5 D} \varepsilon_2^D}{\Gamma(0.5 D +1)}$ (where $\Gamma$ is the Gamma-function), we obtain volume equivalence if 
$\varepsilon_2 = \Gamma(0.5 D +1)^{\frac{1}{D}} \sqrt{\pi} \varepsilon_{\infty}$. For the MNIST-data set, with $D=28 \times 28$ input features $L_2$-norm with $\varepsilon_2 \approx 21.39$ is volume equivalent to $L_{\infty}$-norm with~$\varepsilon_{\infty}=1$.











\newpage 
\subsection{Additional Experiments}



Table~\ref{tab:acc_label_attack} and~\ref{tab:acc_label_attack_fgsm} illustrate that no DBU model maintains high accuracy under gradient-based label attacks. Accuracy under PGD attacks decreases more than under FGSM attacks, since PGD is stronger.  Interestingly Noise attacks achieve also good performances with increasing Noise standard deviation. Note that the attack is not constraint to be with a given radius for Noise attacks.



\begin{table*}[htbp!]
 	\centering
 	\caption{Accuracy under PGD label attacks.}
 	\begin{small}
 		\begin{tabular}{@{}rrrrrrrrc|crrrrrrr@{}}
 			\toprule
  			Att. Rad. & 0.0 & 0.1 & 0.2 & 0.5 & 1.0 & 2.0 & 4.0 & & & 0.0 & 0.1 & 0.2 & 0.5 & 1.0 & 2.0 & 4.0 \\
 			\midrule
 			& \multicolumn{7}{c}{MNIST} & & & \multicolumn{7}{c}{CIFAR10} \\
 			\PostNet  &  \bf{99.4} &  \bf{99.2} &  \bf{98.8} &  96.8 &  89.6 &  53.8 &  13.0 & &
 			          &  89.5 &  73.5 &  51.7 &  13.2 &   2.2 &   0.8 &  0.3 \\
 			\PriorNet &  99.3 &  99.1 &  \bf{98.8} &  97.4 &  \bf{93.9} &  \bf{75.3} &   4.8 & &
 			          &  88.2 &  \bf{77.8} &  \bf{68.4} &  \bf{54.0} &  \bf{37.9} &  \bf{17.5} &  \bf{5.1} \\
 		    \DDNet    &  \bf{99.4} &  99.1 &  \bf{98.8} &  \bf{97.5} &  91.6 &  48.8 &   0.2 & &
 		              &  86.1 &  73.9 &  59.1 &  20.5 &   1.5 &   0.0 &  0.0 \\
 		    \EvNet    &  99.2 &  98.9 &  98.4 &  96.8 &  92.4 &  73.1 &  \bf{40.9} & &
 		              &  \bf{89.8} &  71.7 &  48.8 &  11.5 &   2.7 &   1.5 &  0.4 \\
 		    \midrule
 		 & \multicolumn{7}{c}{Sensorless} & & & \multicolumn{7}{c}{Segment} \\
 			\PostNet  &  98.3 &  13.1 &   6.4 &   4.0 &  \bf{7.0} &  \bf{9.8} &  \bf{11.3} & &
 			          &  98.9 &  82.8 &  \bf{50.1} &  \bf{19.2} &  \bf{8.8} &  \bf{5.1} &  \bf{8.6}   \\
 			\PriorNet &  \bf{99.3} &  16.5 &   5.6 &   1.2 &  0.4 &  0.2 &   1.6 & &
 			          &  \bf{99.5} &  90.7 &  47.6 &   7.8 &  0.2 &  0.0 &  0.4 \\
 		    \DDNet    &  \bf{99.3} &  12.4 &   2.4 &   0.6 &  0.3 &  0.1 &   0.1 & &
 		              &  99.2 &  \bf{90.8} &  45.7 &   6.9 &  0.0 &  0.0 &  0.0 \\
 		    \EvNet    &  99.0 &  \bf{35.3} &  \bf{22.3} &  \bf{11.2} &  \bf{7.0} &  5.2 &   4.0 & &
 		              &  99.3 &  91.8 &  54.0 &  10.3 &  0.8 &  0.5 &  0.6 \\
 			\bottomrule
 		\end{tabular}
 	\end{small}
 	\label{tab:acc_label_attack}
\end{table*}



\begin{table*}[htbp!]
 	\centering
 	\caption{Accuracy under FGSM label attacks.}
 	\begin{small}
 		\begin{tabular}{@{}rrrrrrrrc|crrrrrrr@{}}
 			\toprule
 			Att. Rad. & 0.0 & 0.1 & 0.2 & 0.5 & 1.0 & 2.0 & 4.0 & & & 0.0 & 0.1 & 0.2 & 0.5 & 1.0 & 2.0 & 4.0 \\
 			\midrule
 			& \multicolumn{7}{c}{MNIST} & & & \multicolumn{7}{c}{CIFAR10} \\
 			\PostNet  & \bf{99.4} &  \bf{99.2} &  \bf{98.9} &  97.7 &  95.2 &  \bf{90.1} &  \bf{79.2} & &
 			          & 89.5 &  72.3 &  54.9 &  31.2 &  21.0 &  16.8 &  15.6 \\
 			\PriorNet & 99.3 &  99.1 &  \bf{98.9} &  97.7 &  \bf{95.8} &  93.2 &  76.7 & &
 			          & 88.2 &  \bf{77.3} &  \bf{70.1} &  \bf{59.4} &  \bf{52.3} &  \bf{48.5} &  \bf{46.8} \\
 		    \DDNet    & \bf{99.4} &  \bf{99.2} &  \bf{98.9} &  \bf{97.8} &  94.7 &  79.2 &  25.2 & &
 			          & 86.1 &  73.0 &  60.2 &  32.5 &  14.6 &   7.1 &   6.0 \\
 		    \EvNet    & 99.2 &  98.9 &  98.6 &  97.6 &  \bf{95.8} &  \bf{90.1} &  74.4 & &
 			          & \bf{89.8} &  71.4 &  54.5 &  29.6 &  18.1 &  14.4 &  13.4 \\
 		    \midrule
 		     & \multicolumn{7}{c}{Sensorless} & & & \multicolumn{7}{c}{Segment} \\
 			\PostNet  & 98.3 &  19.6 &  10.9 &  10.9 &  11.9 &  12.4 &  12.5 & &
 			          & 98.9 &  79.6 &  \bf{57.3} &  \bf{31.5} &  \bf{18.4} &  \bf{20.6} &  \bf{19.9} \\
 			\PriorNet & \bf{99.3} &  24.7 &  11.8 &   8.6 &   8.5 &   8.1 &   8.3 & &
 			          & \bf{99.5} &  85.5 &  40.5 &   8.9 &   0.4 &   0.3 &   0.2 \\
 		    \DDNet    & \bf{99.3} &  18.0 &   8.2 &   6.5 &   5.4 &   6.7 &   7.8 & &
 			          & 99.2 &  86.4 &  36.2 &  11.9 &   0.9 &   0.0 &   0.0 \\
 		    \EvNet    & 99.0 &  \bf{42.0} &  \bf{28.0} &  \bf{17.5} &  \bf{13.7} &  \bf{13.6} &  \bf{14.9} & &
 			          & 99.3 &  \bf{90.6} &  55.2 &  14.2 &   2.4 &   0.5 &   0.1 \\
 			\bottomrule
 		\end{tabular}
 	\end{small}
 	\label{tab:acc_label_attack_fgsm}
\end{table*}


\begin{table*}[htbp!]
 	\centering
 	\caption{Accuracy under Noise label attacks.}
 	\begin{small}
 		\begin{tabular}{@{}rrrrrrrrc|crrrrrrr@{}}
 			\toprule
 			%& \multicolumn{7}{c}{MNIST} & & & \multicolumn{7}{c}{CIFAR10} \\
 			%\cmidrule{2-8}  \cmidrule{11-16}
 			Noise Std & 0.0 & 0.1 & 0.2 & 0.5 & 1.0 & 2.0 & 4.0 & & & 0.0 & 0.1 & 0.2 & 0.5 & 1.0 & 2.0 & 4.0 \\
 			\midrule
 			& \multicolumn{7}{c}{MNIST} & & & \multicolumn{7}{c}{CIFAR10} \\
 			\PostNet  & \bf{99.4} &  \bf{98.6} &  91.8 &  \bf{14.9} &  \bf{1.3} &  \bf{0.1} &  0.0 & &
 			          & \bf{91.7} &  21.5 &  10.1 &   0.1 &   1.2 &  0.0 &  1.9 \\
 			\PriorNet & 99.3 &  98.5 &  \bf{95.7} &  14.4 &  0.0 &  0.0 &  0.0 & &
 			          & 87.7 &  \bf{28.1} &  \bf{11.2} &   9.7 &   5.0 &  \bf{8.5} &  \bf{9.0}\\
 		    \DDNet    & \bf{99.4} &  \bf{98.6} &  92.4 &  13.3 &  0.7 &  0.0 &  0.0 & &
 			          & 81.7 &  23.0 &  \bf{11.2} &  \bf{11.2} &  \bf{11.0} &  7.8 &  6.7 \\
 		    \EvNet    & 99.3 &  96.9 &  81.6 &  11.7 &  0.5 &  0.0 &  0.0 & &
 			          & 89.5 &  20.7 &  11.1 &   5.2 &   0.5 &  2.3 &  3.9 \\
 		    \midrule
 		     & \multicolumn{7}{c}{Sensorless} & & & \multicolumn{7}{c}{Segment} \\
 			\PostNet  & 98.1 &  0.1 &  \bf{3.7} &  \bf{11.7} &  \bf{11.7} &  \bf{11.7} &  \bf{11.7} & &
 			          & 98.5 &  39.4 &   3.9 &  \bf{1.8} &  \bf{12.1} &  \bf{20.3} &  \bf{22.1} \\
 			\PriorNet & \bf{99.3} &  0.2 &  0.0 &   0.0 &   0.0 &   0.3 &   2.4 & &
 			          & \bf{99.4} &  47.9 &   8.8 &  0.0 &   0.0 &   0.0 &   0.0 \\
 		    \DDNet    & 99.0 &  \bf{0.4} &  0.1 &   0.0 &   0.0 &   0.0 &   0.0 & &
 			          & 99.1 &  50.0 &  \bf{10.3} &  0.0 &   0.0 &   0.3 &   0.0 \\
 		    \EvNet    & 98.6 &  0.2 &  0.0 &   0.1 &   1.4 &   4.6 &   8.8 & &
 			          & 99.1 &  \bf{50.3} &  \bf{10.3} &  1.2 &   0.3 &   0.0 &   1.5 \\
 			\bottomrule
 		\end{tabular}
 	\end{small}
 	\label{tab:acc_label_attack_noise_attack}
\end{table*}

\clearpage
\subsubsection{Uncertainty estimation under label attacks}

\textbf{Is low uncertainty a reliable indicator of correct predictions?}

On non-perturbed data uncertainty estimates are an indicator of correctly classified samples, but if the input data is perturbed none of the DBU models maintains its high performance. Thus, uncertainty estimates are not a robust indicator of correctly labeled inputs. 

\begin{table*}[htbp!]
 	\centering
 	\caption{Distinguishing between correctly and wrongly predicted labels based on the differential entropy under PGD label attacks (AUC-PR).}
 	\begin{small}
 		\begin{tabular}{@{}rrrrrrrrc|crrrrrrr@{}}
 			\toprule
 			& \multicolumn{7}{c}{MNIST} & & & \multicolumn{7}{c}{Segment} \\
 			\cmidrule{2-8}  \cmidrule{11-16}
 			Att. Rad. & 0.0 & 0.1 & 0.2 & 0.5 & 1.0 & 2.0 & 4.0 & & & 0.0 & 0.1 & 0.2 & 0.5 & 1.0 & 2.0 & 4.0 \\
 			\midrule
 			%& \multicolumn{7}{c}{MNIST} & & & \multicolumn{7}{c}{Segment} \\
 			\PostNet  &  99.9 &   99.9 &  99.8 &  98.7 &  89.5 &  43.5 &   9.0 & & 
 			          &  99.9 &  77.6 &  31.6 &  \bf{11.1} &  \bf{5.3} &  \bf{4.4} &   8.7 \\
 			\PriorNet &  99.9 &   99.8 &  99.6 &  97.7 &  90.5 &  \bf{69.1} &   6.4 & & 
 			          &  \bf{100.0} &  \bf{96.8} &  44.5 &   4.5 &  0.4 &  0.0 &  \bf{15.2} \\
 		    \DDNet    &  \bf{100.0} &  \bf{100.0} &  \bf{99.9} &  \bf{99.7} &  \bf{97.6} &  50.2 &   0.1 & &
 		              &  \bf{100.0} &  \bf{96.8} &  \bf{54.0} &   4.3 &  0.0 &  0.0 &   0.0 \\
 		    \EvNet    &  99.6 &   99.3 &  98.7 &  96.1 &  88.8 &  63.1 &  \bf{31.7} & &
 		              &  \bf{100.0} &  95.9 &  44.3 &   5.9 &  0.8 &  0.6 &   0.7 \\
 			\bottomrule
 		\end{tabular}
 	\end{small}
 	\label{tab:conf_label_attack_2}
\end{table*}



\begin{table*}[htbp!]
 	\centering
 	\caption{Distinguishing between correctly and wrongly predicted labels based on the precision~$\alpha_0$ under PGD label attacks (AUC-PR).}
 	\begin{small}
 		\begin{tabular}{@{}rrrrrrrrc|crrrrrrr@{}}
 			\toprule
 			%& \multicolumn{7}{c}{MNIST} & & & \multicolumn{7}{c}{CIFAR10} \\
 			%\cmidrule{2-8}  \cmidrule{11-16}
 			Att. Rad. & 0.0 & 0.1 & 0.2 & 0.5 & 1.0 & 2.0 & 4.0 & & & 0.0 & 0.1 & 0.2 & 0.5 & 1.0 & 2.0 & 4.0 \\
 			\midrule
 			& \multicolumn{7}{c}{MNIST} & & & \multicolumn{7}{c}{CIFAR10} \\
            \PostNet  & \bf{100.0} &   99.9 &   99.7 &  98.2 &  87.9 &  39.1 &   6.9 & &
                      & \bf{98.7} &  88.6 &  56.2 &   7.8 &   1.2 &   0.4 &  0.3  \\
            \PriorNet &  99.9 &   99.8 &   99.6 &  97.7 &  90.4 & \bf{69.1} &   6.6  & &
                      &  92.9 &  77.7 &  60.5 & \bf{37.6} & \bf{24.9} & \bf{11.3} & \bf{3.0} \\
            \DDNet    & \bf{100.0} & \bf{100.0} & \bf{100.0} & \bf{99.8} & \bf{98.2} &  51.1 &   0.1  & &
                      &  97.6 & \bf{91.8} & \bf{78.3} &  18.1 &   0.8 &   0.0 &  0.0  \\
            \EvNet    &  99.6 &   99.2 &   98.6 &  95.7 &  88.6 &  63.6 & \bf{32.6} & &
                                  &  97.9 &  85.9 &  57.2 &  10.2 &   4.0 &   2.4 &  0.3  \\
 		    \midrule
 		     & \multicolumn{7}{c}{Sensorless} & & & \multicolumn{7}{c}{Segment} \\
            \PostNet  &  99.6 &   7.0 &   3.3 &  3.1 & \bf{6.9} & \bf{9.8} & \bf{11.3} & &
                      &  99.9 &  74.2 &  31.6 & \bf{11.1} & \bf{5.0} & \bf{4.2} & \bf{8.6} \\
            \PriorNet &  99.8 &  10.5 &   3.2 &  0.6 &  0.2 &  0.2 &   1.8 & &
                     & \bf{100.0} &  96.9 & \bf{45.2} &   4.4 &  0.4 &  0.0 &  1.2  \\
            \DDNet    &  99.8 &   8.7 &   1.3 &  0.3 &  0.2 &  0.1 &   0.2 & &
                    & \bf{100.0} & \bf{97.1} &  45.0 &   4.1 &  0.0 &  0.0 &  0.0 \\
            \EvNet    & \bf{99.9} & \bf{23.2} & \bf{13.2} & \bf{6.0} &  3.7 &  2.7 &   2.1 & &
                    & \bf{100.0} &  95.7 &  44.5 &   5.9 &  0.8 &  0.6 &  0.7 \\
 			\bottomrule
 		\end{tabular}
 	\end{small}
 	\label{tab:conf_label_attack_alpha}
\end{table*}



\begin{table*}[htbp!]
 	\centering
 	\caption{Distinguishing between correctly and wrongly predicted labels based on the mutual information under PGD label attacks (AUC-PR).}
 	\begin{small}
 		\begin{tabular}{@{}rrrrrrrrc|crrrrrrr@{}}
 			\toprule
 			%& \multicolumn{7}{c}{MNIST} & & & \multicolumn{7}{c}{CIFAR10} \\
 			%\cmidrule{2-8}  \cmidrule{11-16}
 			Att. Rad. & 0.0 & 0.1 & 0.2 & 0.5 & 1.0 & 2.0 & 4.0 & & & 0.0 & 0.1 & 0.2 & 0.5 & 1.0 & 2.0 & 4.0 \\
 			\midrule
 			& \multicolumn{7}{c}{MNIST} & & & \multicolumn{7}{c}{CIFAR10} \\
            \PostNet  & 99.7 &  99.7 &  99.6 &  99.2 &  92.4 &  40.0 &   6.9 & &
                      & \bf{97.3} &  84.5 &  56.2 &  12.2 &   2.4 &   0.7 &  0.3  \\
            \PriorNet &  99.9 &  99.8 &  99.6 &  97.7 &  90.3 & \bf{68.9} &   6.4  & &
                      &  82.7 &  65.6 &  51.4 & \bf{35.5} & \bf{24.4} & \bf{11.0} & \bf{2.9} \\
            \DDNet    & \bf{100.0} & \bf{99.9} & \bf{99.9} & \bf{99.7} & \bf{97.4} &  50.2 &   0.1  & &
                      &  96.9 & \bf{90.8} & \bf{77.2} &  18.8 &   0.8 &   0.0 &  0.0  \\
            \EvNet    &  97.8 &  97.0 &  95.7 &  92.6 &  86.1 &  62.3 & \bf{28.9} & &
                      &  91.3 &  72.4 &  47.9 &  11.4 &   1.6 &   0.9 &  1.6  \\
 		    \midrule
 		     & \multicolumn{7}{c}{Sensorless} & & & \multicolumn{7}{c}{Segment} \\
            \PostNet  &  99.3 &   7.0 &   3.3 &  3.3 & \bf{7.0} & \bf{9.8} &  11.3 & &
                      &  99.9 &  73.2 &  31.5 & \bf{11.1} & \bf{5.0} & \bf{4.3} & \bf{8.7} \\
            \PriorNet & \bf{99.8} &  10.5 &   3.2 &  0.6 &  0.2 &  0.1 & \bf{11.8} & &
                      & \bf{100.0} & \bf{96.6} & \bf{45.2} &   4.5 &  0.4 &  0.0 &  1.1  \\
            \DDNet    &  99.6 &   8.6 &   1.3 &  0.3 &  0.2 &  0.1 &   0.1 & &
                      & \bf{100.0} &  96.5 &  42.4 &   4.1 &  0.0 &  0.0 &  0.0 \\
            \EvNet    &  99.1 & \bf{22.0} & \bf{12.6} & \bf{5.9} &  3.7 &  2.7 &   2.2 & &
                      & \bf{100.0} &  90.5 &  41.0 &   5.9 &  0.8 &  0.6 &  0.7 \\
 			\bottomrule
 		\end{tabular}
 	\end{small}
 	\label{tab:conf_label_attack_mi}
\end{table*}


\begin{table*}[htbp!]
 	\centering
 	\caption{Distinguishing between correctly and wrongly predicted labels based on the differential entropy under FGSM label attacks (AUC-PR).}
 	\begin{small}
 		\begin{tabular}{@{}rrrrrrrrc|crrrrrrr@{}}
 			\toprule
 			%& \multicolumn{7}{c}{MNIST} & & & \multicolumn{7}{c}{CIFAR10} \\
 			%\cmidrule{2-8}  \cmidrule{11-16}
 			Att. Rad. & 0.0 & 0.1 & 0.2 & 0.5 & 1.0 & 2.0 & 4.0 & & & 0.0 & 0.1 & 0.2 & 0.5 & 1.0 & 2.0 & 4.0 \\
 			\midrule
 			& \multicolumn{7}{c}{MNIST} & & & \multicolumn{7}{c}{CIFAR10} \\
             \PostNet  & 99.9 &   99.9 &  99.8 &  99.4 &  97.8 & \bf{92.1} & \bf{83.2} & &
                      & \bf{98.5} &  88.7 &  68.9 &  31.0 &  18.6 &  15.5 &  16.7 \\
            \PriorNet & 99.9 &   99.9 &  99.7 &  98.3 &  94.1 &  88.5 &  78.6  & &
                      & 90.1 &  73.6 &  61.6 & \bf{46.1} & \bf{38.5} & \bf{35.6} & \bf{37.3} \\
            \DDNet    & \bf{100.0} & \bf{100.0} & \bf{99.9} & \bf{99.8} & \bf{98.7} &  86.4 &  23.0 & &
                      & 97.3 & \bf{90.6} & \bf{78.7} &  39.4 &  13.7 &   6.0 &   5.1 \\
            \EvNet    & 99.6 &   99.4 &  99.1 &  97.8 &  95.8 &  90.4 &  76.8 & &
                      & 98.0 &  86.2 &  67.4 &  32.7 &  19.9 &  18.2 &  19.7 \\		
 		    \midrule
 		     & \multicolumn{7}{c}{Sensorless} & & & \multicolumn{7}{c}{Segment} \\
             \PostNet  & 99.7 &  11.7 &   7.3 &   9.3 &  11.8 &  12.5 &  12.5 & &
                      & 99.9 &  73.6 &  40.6 & \bf{23.7} & \bf{17.2} & \bf{19.8} & \bf{20.2} \\
            \PriorNet & 99.8 &  21.4 &  10.4 &   8.5 &   9.0 &   9.2 &  10.3 & &
                      & \bf{100.0} &  93.7 &  37.7 &   5.8 &   1.1 &   0.9 &   0.8 \\
            \DDNet    & 99.7 &  18.5 &   5.4 &   4.3 &   4.2 &   5.7 &   7.9 & &
                      & \bf{100.0} & \bf{94.1} &  42.9 &   7.2 &   1.0 &   0.0 &   0.0 \\
            \EvNet    & \bf{99.9} & \bf{44.8} & \bf{29.2} & \bf{18.2} & \bf{15.1} & \bf{14.9} & \bf{15.5} & &
                      & \bf{100.0} &  93.7 & \bf{48.7} &   8.7 &   2.4 &   1.6 &   0.5 \\
 			\bottomrule
 		\end{tabular}
 	\end{small}
 	\label{tab:conf_label_attack_fgsm}
\end{table*}

\begin{table*}[htbp!]
 	\centering
 	\caption{Distinguishing between correctly and wrongly predicted labels based on the differential entropy under Noise label attacks (AUC-PR).}
 	\begin{small}
 		\begin{tabular}{@{}rrrrrrrrc|crrrrrrr@{}}
 			\toprule
 			%& \multicolumn{7}{c}{MNIST} & & & \multicolumn{7}{c}{CIFAR10} \\
 			%\cmidrule{2-8}  \cmidrule{11-16}
 			Noise Std & 0.0 & 0.1 & 0.2 & 0.5 & 1.0 & 2.0 & 4.0 & & & 0.0 & 0.1 & 0.2 & 0.5 & 1.0 & 2.0 & 4.0 \\
 			\midrule
 			& \multicolumn{7}{c}{MNIST} & & & \multicolumn{7}{c}{CIFAR10} \\
             \PostNet  & 99.9 &  99.8 &  99.6 &  \bf{74.2} &  \bf{7.4} &  \bf{0.2} &  0.0 & &
                      & \bf{98.}7 &  \bf{76.}3 &  24.3 &   0.4 &   4.9 &  0.0 &  1.7 \\
            \PriorNet & 99.9 &  99.9 &  \bf{99.8} &  73.4 &  0.0 &  0.0 &  0.0  & &
                      & 85.0 &  27.8 &  15.9 &  \bf{20.}4 &   7.0 &  \bf{7.}7 &  \bf{8.3} \\
            \DDNet    & \bf{100.0} &  \bf{99.9} &  99.4 &  51.1 &  0.6 &  0.1 &  0.0 & &
                      & 96.1 &  61.0 &  \bf{39.}8 &  14.2 &  \bf{11.}3 &  6.9 &  6.9 \\
            \EvNet    & 99.5 &  98.4 &  88.5 &  20.2 &  0.9 &  0.0 &  0.0  & &
                      & 97.5 &  66.1 &  21.4 &   7.7 &   2.3 &  3.0 &  3.8 \\		
 		    \midrule
 		     & \multicolumn{7}{c}{Sensorless} & & & \multicolumn{7}{c}{Segment} \\
             \PostNet  & 99.7 &  0.3 &  \bf{3.2} &  \bf{13.3} &  \bf{12.0} &  \bf{11.7} &  \bf{11.7} & &
                      & 99.9 &  53.9 &   4.8 &  1.8 &  \bf{11.2} &  \bf{21.7} &  \bf{21.6} \\
            \PriorNet & \bf{100.0} &  0.3 &  0.0 &   0.0 &   0.0 &  7.8 &  11.5 & &
                      & \bf{100.0} &  \bf{84.5} &  15.6 &  0.0 &   0.0 &   0.0 &   0.0 \\
            \DDNet    & 99.7 &  \bf{0.9} &  0.6 &   0.0 &   0.0 &   0.0 &   0.0 & &
                      & \bf{100.0} &  82.7 &  \bf{23.9} &  0.0 &   0.0 &   0.6 &   0.0 \\
            \EvNet    & 99.8 &  0.3 &  0.0 &   0.1 &   1.7 &   5.5 &  10.0 & &
                      & \bf{100.0} &  78.3 &  19.0 &  \bf{3.5} &   0.5 &   0.0 &   1.7 \\
 			\bottomrule
 		\end{tabular}
 	\end{small}
 	\label{tab:conf_label_attack_noise_attack}
\end{table*}





Table~\ref{tab:conf_label_attack}, \ref{tab:conf_label_attack_2}, \ref{tab:conf_label_attack_alpha}, and~\ref{tab:conf_label_attack_mi} illustrate that neither differential entropy nor precision, nor mutual information are a reliable indicator of correct predictions under PGD attacks. 
DBU-models achieve significantly better results when they are attacked by FGSM-attacks (Table~\ref{tab:conf_label_attack_fgsm}), but as FGSM attacks provide much weaker adversarial examples than PGD attacks, this cannot be seen as real advantage. 





\clearpage
\textbf{Can we use uncertainty estimates to detect attacks against the class prediction?}

PGD attacks do not explicitly consider uncertainty during the computation of adversarial examples, but they seem to provide perturbed inputs with similar uncertainty as the original input. 



\begin{table*}[htbp!]
 	\centering
 	\caption{Attack-Detection based on differential entropy under PGD label attacks (AUC-PR).}
 	\begin{small}
 		\begin{tabular}{@{}rrrrrrrc|crrrrrr@{}}
 			\toprule
 		     & \multicolumn{6}{c}{MNIST} & & & \multicolumn{6}{c}{Segment} \\
 			\cmidrule{2-7}  \cmidrule{10-15}
 			Att. Rad. & 0.1 & 0.2 & 0.5 & 1.0 & 2.0 & 4.0 & & & 0.1 & 0.2 & 0.5 & 1.0 & 2.0 & 4.0 \\
 			\midrule
            \PostNet  &  57.7 &  66.3 &  83.4 &  90.5 &  79.0 &  50.1 & & 
                      & \bf{95.6} &  73.5 & \bf{47.0} & \bf{42.3} & \bf{53.4} & \bf{82.7} \\
            \PriorNet & \bf{67.7} & \bf{83.2} & \bf{97.1} & \bf{96.7} &  92.1 &  82.9 & & 
                      &  86.7 &  83.3 &  38.0 &  31.3 &  30.8 &  31.5 \\
            \DDNet    &  53.4 &  57.1 &  68.5 &  83.9 & \bf{96.0} & \bf{86.3} & & 
                      &  76.1 & \bf{83.5} &  45.4 &  32.4 &  30.8 &  30.8 \\
            \EvNet    &  54.8 &  59.0 &  68.5 &  75.9 &  72.6 &  59.8 & & 
                      &  94.9 &  80.9 &  41.5 &  32.5 &  31.1 &  31.1 \\
 			\bottomrule 			
 		\end{tabular}
 	\end{small}
 	\label{tab:label_attack_detect_2}
\end{table*}




\begin{table*}[htbp!]
 	\centering
 	\caption{Attack-Detection based on precision $\alpha_0$ under PGD label attacks (AUC-PR).}
 	\begin{small}
 		\begin{tabular}{@{}rrrrrrrc|crrrrrr@{}}
 			\toprule
 			Att. Rad. & 0.1 & 0.2 & 0.5 & 1.0 & 2.0 & 4.0 & & & 0.1 & 0.2 & 0.5 & 1.0 & 2.0 & 4.0 \\
 			\midrule
 		     & \multicolumn{6}{c}{MNIST} & & & \multicolumn{6}{c}{CIFAR10} \\
            \PostNet  &  63.3 &  75.7 &  92.6 &  95.1 &  75.3 &  39.5 & & 
                      & \bf{63.4} & \bf{66.9} &  42.1 &  32.9 &  31.6 &  31.2 \\
            \PriorNet & \bf{67.6} & \bf{83.2} & \bf{97.1} & \bf{96.9} & \bf{92.7} & \bf{84.7} & & 
                      &  53.3 &  56.0 &  55.6 & \bf{49.2} &  42.2 &  35.4 \\
            \DDNet    &  52.7 &  55.7 &  64.7 &  78.4 &  91.9 &  80.9 & & 
                      &  55.8 &  60.5 & \bf{57.3} &  38.7 &  32.3 &  31.4  \\
            \EvNet    &  49.1 &  48.0 &  45.1 &  42.7 &  41.8 &  39.2 & & 
                      &  48.4 &  46.9 &  46.3 &  46.3 & \bf{44.5} & \bf{42.5} \\
 		    \midrule
 		  	& \multicolumn{6}{c}{Sensorless} & & & \multicolumn{6}{c}{Segment} \\
            \PostNet  &  39.8 &  35.8 &  35.4 & \bf{52.0} & \bf{88.2} & \bf{99.0} & & 
                      & \bf{94.6} &  70.3 & \bf{46.3} & \bf{42.6} & \bf{54.9} & \bf{84.0} \\
            \PriorNet &  40.9 &  35.1 &  32.0 &  31.1 &  30.7 &  30.7 & & 
                      &  82.7 &  82.6 &  39.4 &  31.6 &  30.8 &  30.8 \\
            \DDNet    & \bf{47.7} & \bf{40.3} &  35.3 &  32.8 &  31.3 &  30.8 & & 
                      &  80.0 & \bf{86.0} &  43.3 &  33.6 &  31.0 &  30.8 \\
            \EvNet    &  45.4 &  39.7 & \bf{36.1} &  34.8 &  34.7 &  36.0 & & 
                      &  90.9 &  72.4 &  40.4 &  32.4 &  31.1 &  31.1 \\
             			\bottomrule
 		\end{tabular}
 	\end{small}
 	\label{tab:label_attack_detect_auroc_1}
\end{table*}

\begin{table*}[htbp!]
 	\centering
 	\caption{Attack-Detection based on mutual information under PGD label attacks (AUC-PR).}
 	\begin{small}
 		\begin{tabular}{@{}rrrrrrrc|crrrrrr@{}}
 			\toprule
 			Att. Rad. & 0.1 & 0.2 & 0.5 & 1.0 & 2.0 & 4.0 & & & 0.1 & 0.2 & 0.5 & 1.0 & 2.0 & 4.0 \\
 			\midrule
 		     & \multicolumn{6}{c}{MNIST} & & & \multicolumn{6}{c}{CIFAR10} \\
 		    \PostNet  &   42.2 &  37.5 &  36.7 &  54.5 &  70.5 &  70.3 & & 
                        &   52.2 &  52.1 &  50.0 & \bf{65.9} & \bf{76.3} & \bf{80.7} \\
            \PriorNet & \bf{67.7} & \bf{83.3} & \bf{97.1} & \bf{96.9} &  92.6 & \bf{84.5} & & 
                      &   54.0 &  56.9 &  56.3 &  49.7 &  42.4 &  35.5  \\
            \DDNet    &   53.1 &  56.3 &  66.5 &  81.0 & \bf{94.0} &  82.9 & & 
                      & \bf{56.0} & \bf{60.8} & \bf{57.4} &  38.2 &  32.1 &  31.3  \\
            \EvNet    &   49.1 &  48.0 &  45.2 &  42.9 &  41.9 &  39.3 & & 
                      &   48.7 &  47.3 &  46.3 &  46.0 &  44.1 &  42.2 \\
 		    \midrule
 		  	& \multicolumn{6}{c}{Sensorless} & & & \multicolumn{6}{c}{Segment} \\
            \PostNet  & \bf{75.3} & \bf{76.6} & \bf{66.5} & \bf{57.7} & \bf{85.6} & \bf{98.7} & & 
                      & \bf{94.8} &  73.5 & \bf{55.9} & \bf{47.9} & \bf{58.0} & \bf{84.0} \\
            \PriorNet &  40.7 &  35.0 &  32.0 &  31.0 &  30.7 &  30.7 & & 
                      &  83.5 &  82.7 &  39.2 &  31.6 &  30.8 &  30.8 \\
            \DDNet    &  48.0 &  40.0 &  35.2 &  32.6 &  31.2 &  30.8 & & 
                      &  82.4 & \bf{88.1} &  43.4 &  33.4 &  30.9 &  30.8 \\
            \EvNet    &  45.5 &  39.7 &  36.1 &  34.8 &  34.7 &  36.0 & & 
                      &  91.7 &  72.9 &  40.5 &  32.4 &  31.1 &  31.1  \\ 			
 			\bottomrule
 		\end{tabular}
 	\end{small}
 	\label{tab:label_attack_detect_auroc_2}
\end{table*}

FGSM and Noise attacks are easier to detect, but also weaker thand PGD attacks. This suggests that DBU models are capable of detecting weak attacks by using uncertainty estimation.

\begin{table*}[htbp!]
 	\centering
 	\caption{Attack-Detection based on differential entropy under FGSM label attacks (AUC-PR).}
 	\begin{small}
 		\begin{tabular}{@{}rrrrrrrc|crrrrrr@{}}
 			\toprule
 			Att. Rad. & 0.1 & 0.2 & 0.5 & 1.0 & 2.0 & 4.0 & & & 0.1 & 0.2 & 0.5 & 1.0 & 2.0 & 4.0 \\
 			\midrule
 		     & \multicolumn{6}{c}{MNIST} & & & \multicolumn{6}{c}{CIFAR10} \\
            \PostNet  &   55.9 &  61.8 &  74.8 &  84.0 &  88.9 &  89.9 & & 
                      & \bf{62.1} & \bf{67.2} &  65.7 &  63.1 &  65.4 &  73.8  \\
            \PriorNet & \bf{67.4} & \bf{82.4} & \bf{96.9} & \bf{98.3} & \bf{98.9} & \bf{99.6} & & 
                      &   58.4 &  63.1 &  68.5 & \bf{70.1} &  68.5 &  62.5  \\
            \DDNet    &   53.6 &  57.3 &  68.3 &  82.6 &  95.6 &  98.7 & & 
                      &   57.2 &  62.9 & \bf{69.1} &  68.7 & \bf{69.7} & \bf{76.5} \\
            \EvNet    &   54.1 &  57.4 &  63.8 &  67.6 &  68.6 &  69.9 & & 
                      &   57.8 &  61.7 &  63.3 &  62.9 &  65.7 &  72.5 \\ 		    
 		    \midrule
 		  	& \multicolumn{6}{c}{Sensorless} & & & \multicolumn{6}{c}{Segment} \\
            \PostNet  & \bf{98.4} & \bf{99.8} & \bf{99.9} & \bf{99.9} & \bf{99.9} & \bf{99.9} & & 
                      & \bf{96.9} & \bf{93.9} & \bf{99.5} & \bf{99.9} & \bf{100.0} & \bf{100.0} \\
            \PriorNet &  48.7 &  38.6 &  32.7 &  32.9 &  38.6 &  44.3 & & 
                      &  89.0 &  80.8 &  46.7 &  37.2 &   33.7 &   32.4 \\
            \DDNet    &  61.5 &  47.8 &  37.1 &  33.1 &  32.4 &  33.2 & & 
                      &  79.6 &  86.2 &  60.2 &  47.5 &   36.6 &   31.6 \\
            \EvNet    &  67.3 &  65.5 &  72.3 &  73.4 &  75.3 &  79.1 & & 
                      &  95.7 &  87.2 &  59.3 &  51.7 &   51.1 &   53.5 \\
 			\bottomrule
 		\end{tabular}
 	\end{small}
 	\label{tab:label_attack_detect_auroc_3}
\end{table*}

\begin{table*}[htbp!]
 	\centering
 	\caption{Attack-Detection based on differential entropy under Noise label attacks (AUC-PR).}
 	\begin{small}
 		\begin{tabular}{@{}rrrrrrrc|crrrrrr@{}}
 			\toprule
 			Noise Std. & 0.1 & 0.2 & 0.5 & 1.0 & 2.0 & 4.0 & & & 0.1 & 0.2 & 0.5 & 1.0 & 2.0 & 4.0 \\
 			\midrule
 		     & \multicolumn{6}{c}{MNIST} & & & \multicolumn{6}{c}{CIFAR10} \\
            \PostNet  &   51.3 &  65.3 &  93.8 &   95.1 &  95.2 &  95.2 & & 
                      & \bf{80.8} &  \bf{84.5} &  \bf{97.6} &  \bf{99.5} &  99.3 &   98.2  \\
            \PriorNet & 32.5 &  36.8 &  88.9 &   99.6 &  99.7 &  92.7 & & 
                      &   34.7 &  32.3 &  34.3 &  60.3 &  95.5 &  \bf{100.0}  \\
            \DDNet    &   \bf{60.7} &  \bf{87.6} &  \bf{99.8} &  \bf{100.0} &  \bf{99.9} &  \bf{99.8} & & 
                      &   59.1 &  62.6 &  81.5 &  98.6 &  \bf{99.8} &   98.7 \\
            \EvNet    &   51.2 &  55.7 &  66.9 &   70.3 &  68.0 &  67.1 & & 
                      &   75.7 &  78.6 &  88.2 &  97.8 &  96.4 &   95.6 \\ 		    
 		    \midrule
 		  	& \multicolumn{6}{c}{Sensorless} & & & \multicolumn{6}{c}{Segment} \\
            \PostNet  & \bf{99.8} &  \bf{100.0} &  \bf{100.0} &  \bf{100.0} &  \bf{100.0} &  \bf{100.0} & & 
                      &  \bf{95.6} &  \bf{99.4} &  \bf{100.0} &  \bf{100.0} &  \bf{100.0} &  \bf{100.0} \\
            \PriorNet & 42.0 &   33.8 &   31.5 &   34.7 &   43.7 &   47.0 & & 
                      & 56.7 &  56.7 &   39.8 &   33.7 &   31.9 &   33.7 \\
            \DDNet    &  53.4 &   43.5 &   34.3 &   31.6 &   32.5 &   36.1 & & 
                      &  57.0 &  58.9 &   43.1 &   33.7 &   31.5 &   31.3 \\
            \EvNet    &  67.1 &   78.8 &   88.3 &   95.4 &   96.9 &   97.8 & & 
                      &  60.8 &  63.5 &   61.2 &   64.8 &   73.7 &   85.2 \\
 			\bottomrule
 		\end{tabular}
 	\end{small}
 	\label{tab:label_attack_detect_auroc_4}
\end{table*}


\clearpage
\subsubsection{Attacking uncertainty estimation}

\textbf{Are uncertainty estimates a robust feature for OOD detection?}

Using uncertainty estimation to distinguish between ID and OOD data is not robust as shown in the following tables.

 \begin{table*}[htbp!]
 	\centering
 	\caption{OOD detection based on differential entropy under PGD uncertainty attacks against differential entropy on ID data and OOD data (AUC-PR).}
 	\begin{small}
 		\begin{tabular}{@{}rrrrrrrrc|crrrrrrr@{}}
 			\toprule
 			& \multicolumn{7}{c}{ID-Attack (non-attacked OOD)} &  & &  \multicolumn{7}{c}{OOD-Attack (non-attacked ID)} \\
 			\cmidrule{2-8}  \cmidrule{11-17}
 			Att. Rad. & 0.0 & 0.1 & 0.2 & 0.5 & 1.0 & 2.0 & 4.0 & & &
 			            0.0 & 0.1 & 0.2 & 0.5 & 1.0 & 2.0 & 4.0 \\
 			\midrule
 			& \multicolumn{16}{c}{\textbf{MNIST -- KMNIST}} \\
            \PostNet  & 94.5 &  94.1 &  93.9 &  91.1 &  77.1 &  44.0 &  31.9 & &
                      & 94.5 &  93.1 &  91.4 &  82.1 &  62.2 &  50.7 &  48.8 \\
            \PriorNet & \bf{99.6} & \bf{99.4} & \bf{99.1} & \bf{97.8} & \bf{93.8} & \bf{77.6} &  \bf{32.0} & &
                      & \bf{99.6} & \bf{99.4} & \bf{99.1} &  98.0 &  94.6 &  85.5 &  \bf{73.9} \\
            \DDNet    & 99.3 &  99.1 &  98.9 & \bf{97.8} &  93.5 &  63.3 &  30.7 & &
                      & 99.3 &  99.1 &  99.0 & \bf{98.3} & \bf{96.7} & \bf{91.3} &  73.8  \\
            \EvNet    & 69.0 &  67.1 &  65.6 &  61.8 &  57.4 &  50.9 &  43.6 & &
                      & 69.0 &  55.8 &  48.0 &  39.4 &  36.2 &  34.9 &  34.4  \\
 			\midrule
 			& \multicolumn{16}{c}{\textbf{Seg. -- Seg. class sky}} \\
            \PostNet  & \bf{99.0} & \bf{80.7} & \bf{53.5} & \bf{38.0} & \bf{34.0} & \bf{41.6} &  \bf{49.5} & &
                      & \bf{99.0} & \bf{88.4} &  69.2 &  45.1 & \bf{36.4} & \bf{42.6} &  \bf{75.4} \\
            \PriorNet & 34.8 &  31.4 &  30.9 &  30.8 &  30.8 &  30.8 &  30.8 & &
                      & 34.8 &  31.8 &  31.0 &  30.8 &  30.8 &  30.8 &  32.1 \\
            \DDNet    & 31.5 &  30.9 &  30.8 &  30.8 &  30.8 &  30.8 &  30.8 & &
                      & 31.5 &  31.0 &  30.8 &  30.8 &  30.8 &  30.8 &  30.8  \\
            \EvNet    & 92.5 &  67.2 &  43.2 &  31.6 &  30.9 &  30.9 &  31.2 & &
                      & 92.5 &  86.1 & \bf{82.7} & \bf{48.9} &  32.7 &  30.9 &  30.9  \\
 			\bottomrule
 		\end{tabular}
 	\end{small}
 	\label{tab:id_ood_attacks_part2}
\end{table*}

 \begin{table*}[htbp!]
 	\centering
 	\caption{OOD detection under PGD uncertainty attacks against differential entropy on ID data and OOD data (AUC-ROC).}
 	\begin{small}
 		\begin{tabular}{@{}rrrrrrrrc|crrrrrrr@{}}
 			\toprule
 			& \multicolumn{7}{c}{ID-Attack (non-attacked OOD)} &  & &  \multicolumn{7}{c}{OOD-Attack (non-attacked ID)} \\
 			\cmidrule{2-8}  \cmidrule{11-17}
 			Att. Rad. & 0.0 & 0.1 & 0.2 & 0.5 & 1.0 & 2.0 & 4.0 & & &
 			            0.0 & 0.1 & 0.2 & 0.5 & 1.0 & 2.0 & 4.0 \\
 			\midrule
 			& \multicolumn{16}{c}{\textbf{MNIST -- KMNIST}} \\
            \PostNet  & 91.6 &  91.3 &  91.9 &  91.5 &  80.2 &  38.8 &   9.2 & &
                      & 91.6 &  90.4 &  89.0 &  81.6 &  62.6 &  45.0 &  43.1 \\
            \PriorNet & \bf{99.8} & \bf{99.7} & \bf{99.5} & \bf{99.0} & \bf{97.1} & \bf{81.1} &   8.7 & &
                      & \bf{99.8} & \bf{99.7} & \bf{99.6} & \bf{99.1} & \bf{97.7} & \bf{93.0} &  \bf{84.9} \\
            \DDNet    & 99.2 &  98.9 &  98.6 &  97.3 &  92.1 &  58.2 &   1.2 & &
                      & 99.2 &  99.0 &  98.8 &  97.9 &  95.8 &  89.1 &  69.3 \\
            \EvNet    & 81.2 &  79.6 &  78.2 &  74.6 &  69.5 &  58.7 &  \bf{43.0} & &
                      & 81.2 &  67.2 &  54.8 &  35.4 &  25.5 &  20.7 &  18.5 \\
 			\midrule
 			& \multicolumn{16}{c}{\textbf{CIFAR10 -- SVHN}} \\
            \PostNet  & 87.0 &  71.9 &  56.3 & \bf{30.2} & \bf{20.2} & \bf{15.0} &  \bf{9.7} & &
                      & 87.0 &  71.0 &  54.3 &  33.5 &  30.3 &  26.2 &  19.4 \\
            \PriorNet & 62.4 &  48.2 &  35.9 &  13.8 &   3.6 &   0.9 &  0.3 & &
                      & 62.4 &  48.0 &  35.6 &  14.8 &   6.6 &   3.4 &   1.6 \\
            \DDNet    & 87.0 & \bf{76.0} & \bf{63.6} &  29.3 &   6.1 &   1.1 &  0.4 & &
                      & 87.0 & \bf{78.1} & \bf{66.1} &  26.2 &   5.1 &   0.7 &   0.1 \\
            \EvNet    & \bf{88.0} &  69.1 &  51.7 &  24.6 &  15.5 &   9.5 &  4.2 & &
                      & \bf{88.0} &  72.0 &  60.7 & \bf{47.9} & \bf{42.1} & \bf{33.3} &  \bf{24.0} \\
 			\midrule
 			& \multicolumn{16}{c}{\textbf{Sens. -- Sens. class 10, 11}} \\
            \PostNet  & \bf{85.3} & \bf{49.1} & \bf{38.1} & \bf{7.8} & \bf{8.2} &  8.2 &   8.2 & &
                      & \bf{85.3} & \bf{57.2} & \bf{54.0} & \bf{27.3} & \bf{31.5} & \bf{86.7} &  \bf{99.5} \\
            \PriorNet & 28.1 &   0.8 &   0.3 &  0.4 &  1.6 & \bf{8.4} &  \bf{26.8} & &
                      & 28.1 &   2.5 &   0.7 &   0.2 &   2.3 &  18.9 &  41.0 \\
            \DDNet    & 21.0 &   3.0 &   0.9 &  0.4 &  0.6 &  2.1 &   7.3 & &
                      & 21.0 &   4.4 &   2.1 &   1.9 &   2.2 &   2.2 &   4.1 \\
            \EvNet    & 74.2 &  21.4 &  12.2 &  4.3 &  1.4 &  0.6 &   0.3 & &
                      & 74.2 &  45.3 &  38.5 &  19.6 &   9.6 &  12.1 &  26.0 \\
 			\midrule
 			& \multicolumn{16}{c}{\textbf{Seg. -- Seg. class sky}} \\
            \PostNet  & \bf{99.2} & \bf{84.7} & \bf{55.5} & \bf{23.0} & \bf{9.7} & \bf{4.4} &  \bf{4.7} & &
                      & \bf{99.2} & \bf{92.1} & \bf{77.1} &  41.5 & \bf{24.9} & \bf{41.0} &  \bf{80.8} \\
            \PriorNet & 17.1 &   4.4 &   1.3 &   0.0 &  0.0 &  0.0 &  0.1 & &
                      & 17.1 &   5.9 &   1.5 &   0.1 &   0.0 &   0.1 &   5.8 \\
            \DDNet    & 4.1 &   1.1 &   0.0 &   0.0 &  0.0 &  0.0 &  0.0 & &
                      &  4.1 &   1.8 &   0.4 &   0.0 &   0.0 &   0.0 &   0.0 \\
            \EvNet    & 91.2 &  54.5 &  23.3 &   3.9 &  0.9 &  0.4 &  0.2 & &
                      & 91.2 &  82.9 &  76.4 & \bf{42.2} &   9.7 &   0.8 &   0.6 \\
 			\bottomrule
 		\end{tabular}
 	\end{small}
 	\label{tab:id_ood_attacks_measure_diffe_auroc}
\end{table*}




 \begin{table*}[htbp!]
 	\centering
 	\caption{OOD detection (AU-PR) under PGD uncertainty attacks against precision~$\alpha_0$ on ID data and OOD data.}
 	\begin{small}
 		\begin{tabular}{@{}rrrrrrrrc|crrrrrrr@{}}
 			\toprule
 			& \multicolumn{7}{c}{ID-Attack (non-attacked OOD)} &  & &  \multicolumn{7}{c}{OOD-Attack (non-attacked ID)} \\
 			\cmidrule{2-8}  \cmidrule{11-17}
 			Att. Rad. & 0.0 & 0.1 & 0.2 & 0.5 & 1.0 & 2.0 & 4.0 & & &
 			            0.0 & 0.1 & 0.2 & 0.5 & 1.0 & 2.0 & 4.0 \\
 			\midrule
 			& \multicolumn{16}{c}{\textbf{MNIST -- KMNIST}} \\
            \PostNet  & 98.4 &  97.4 &  96.0 &  88.8 &  70.9 &  39.3 &  31.3 & &
                      & 98.4 &  97.2 &  95.2 &  82.8 &  52.6 &  34.3 &  32.1 \\
            \PriorNet & \bf{99.6} & \bf{99.5} & \bf{99.2} & \bf{98.0} & \bf{94.1} & \bf{76.0} &  31.1 & &
                      & \bf{99.6} & \bf{99.5} & \bf{99.2} & \bf{98.2} & \bf{95.3} & \bf{87.5} & \bf{75.6} \\
            \DDNet    & 97.2 &  96.7 &  96.1 &  93.8 &  86.4 &  53.2 &  31.0 & &
                      & 97.2 &  96.7 &  96.2 &  94.5 &  91.1 &  82.9 &  64.6 \\
            \EvNet    & 39.8 &  39.2 &  38.8 &  37.9 &  37.1 &  36.3 & \bf{35.4} & &
                      & 39.8 &  34.5 &  32.5 &  31.2 &  31.0 &  30.9 &  31.0 \\
 			\midrule
 			& \multicolumn{16}{c}{\textbf{CIFAR10 -- SVHN}} \\
            \PostNet  & \bf{82.4} &  63.8 &  46.1 &  22.3 &  17.4 &  16.7 &  16.4 & &
                      & \bf{82.4} &  61.8 &  41.5 &  21.8 & \bf{19.8} & \bf{17.5} & \bf{15.8} \\
            \PriorNet & 37.9 &  25.0 &  19.2 &  15.8 &  15.4 &  15.4 &  15.4 & &
                      & 37.9 &  25.9 &  19.4 &  15.6 &  15.4 &  15.4 &  15.4 \\
            \DDNet    & 81.1 & \bf{70.1} & \bf{58.4} & \bf{30.0} &  16.7 &  15.5 &  15.4 & &
                      & 81.1 & \bf{71.2} & \bf{59.9} & \bf{27.8} &  16.5 &  15.5 &  15.4 \\
            \EvNet    & 34.7 &  27.4 &  25.4 &  22.0 & \bf{19.7} & \bf{18.1} & \bf{17.1} & &
                      & 34.7 &  19.4 &  18.1 &  17.1 &  16.8 &  16.2 &  15.7 \\
 			\midrule
 			& \multicolumn{16}{c}{\textbf{Sens. -- Sens. class 10, 11}} \\
            \PostNet  & \bf{77.4} & \bf{39.6} & \bf{35.9} & \bf{31.7} & \bf{44.4} & \bf{44.4} & \bf{44.4} & &
                      & \bf{77.4} &  40.3 & \bf{38.6} &  29.5 & \bf{34.0} & \bf{79.4} & \bf{97.4} \\
            \PriorNet & 35.9 &  27.0 &  26.8 &  26.8 &  26.8 &  27.5 &  36.2 & &
                      & 35.9 &  27.7 &  27.0 &  26.7 &  26.6 &  26.5 &  26.5 \\
            \DDNet    & 55.6 &  34.4 &  31.7 &  30.4 &  29.5 &  30.2 &  33.4 & &
                      & 55.6 & \bf{40.9} &  34.1 &  28.0 &  26.9 &  26.6 &  26.5 \\
            \EvNet    & 66.3 &  33.3 &  29.7 &  27.0 &  27.1 &  29.2 &  33.9 & &
                      & 66.3 &  39.3 &  37.1 & \bf{31.3} &  28.3 &  28.4 &  29.7 \\
 			\midrule
 			& \multicolumn{16}{c}{\textbf{Seg. -- Seg. class sky}} \\
            \PostNet  & \bf{98.4} &  74.8 &  51.0 & \bf{37.2} & \bf{32.8} & \bf{43.5} & \bf{49.9} & &
                      & \bf{98.4} &  84.7 &  66.1 &  42.4 &  34.8 & \bf{40.9} & \bf{71.2} \\
            \PriorNet & 32.1 &  30.9 &  30.8 &  30.8 &  30.8 &  30.8 &  30.8 & &
                      & 32.1 &  31.0 &  30.8 &  30.8 &  30.8 &  30.8 &  30.8 \\
            \DDNet    & 31.0 &  30.8 &  30.8 &  30.8 &  30.8 &  30.8 &  30.8 & &
                      & 31.0 &  30.8 &  30.8 &  30.8 &  30.8 &  30.8 &  30.8 \\
            \EvNet    & 98.3 & \bf{83.0} & \bf{60.5} &  34.0 &  31.0 &  30.8 &  30.8 & &
                      & 98.3 & \bf{94.4} & \bf{88.8} & \bf{65.6} & \bf{37.0} &  31.4 &  30.9 \\
 			\bottomrule
 		\end{tabular}
 	\end{small}
 	\label{tab:id_ood_attacks_measure_alpha0_aupr}
\end{table*}



 \begin{table*}[htbp!]
 	\centering
 	\caption{OOD detection (AUC-ROC) under PGD uncertainty attacks against precision~$\alpha_0$ on ID data and OOD data.}
 	\begin{small}
 		\begin{tabular}{@{}rrrrrrrrc|crrrrrrr@{}}
 			\toprule
 			& \multicolumn{7}{c}{ID-Attack (non-attacked OOD)} &  & &  \multicolumn{7}{c}{OOD-Attack (non-attacked ID)} \\
 			\cmidrule{2-8}  \cmidrule{11-17}
 			Att. Rad. & 0.0 & 0.1 & 0.2 & 0.5 & 1.0 & 2.0 & 4.0 & & &
 			            0.0 & 0.1 & 0.2 & 0.5 & 1.0 & 2.0 & 4.0 \\
 			\midrule
 			& \multicolumn{16}{c}{\textbf{MNIST -- KMNIST}} \\
            \PostNet  & 98.4 &  97.6 &  96.4 &  90.9 &  74.0 &  28.9 &   6.3 & &
                      & 98.4 &  97.6 &  96.3 &  89.0 &  61.3 &  19.6 &   9.7 \\
            \PriorNet & \bf{99.8} & \bf{99.7} & \bf{99.6} & \bf{99.1} & \bf{97.2} & \bf{79.4} &   4.4 & &
                      & \bf{99.8} & \bf{99.7} & \bf{99.6} & \bf{99.2} & \bf{98.0} & \bf{93.9} & \bf{85.8} \\
            \DDNet    & 96.5 &  95.9 &  95.1 &  92.0 &  82.6 &  44.3 &   3.5 & &
                      & 96.5 &  95.9 &  95.2 &  92.9 &  88.6 &  78.7 &  59.4 \\
            \EvNet    & 35.9 &  34.1 &  32.8 &  30.1 &  27.4 &  24.6 & \bf{21.4} & &
                      & 35.9 &  18.7 &  10.4 &   3.7 &   2.0 &   1.7 &   2.0 \\
 			\midrule
 			& \multicolumn{16}{c}{\textbf{CIFAR10 -- SVHN}} \\
            \PostNet  & \bf{87.4} &  71.2 &  54.8 &  29.2 &  19.0 &  14.0 &   9.4 & &
                      & \bf{87.4} &  71.4 &  54.1 &  30.1 & \bf{25.8} & \bf{17.5} & \bf{5.8} \\
            \PriorNet & 45.6 &  31.1 &  20.4 &   6.3 &   1.4 &   0.3 &   0.1 & &
                      & 45.6 &  32.2 &  21.7 &   5.4 &   1.0 &   0.3 &  0.1 \\
            \DDNet    & 84.9 & \bf{73.8} & \bf{61.8} &  30.2 &   9.3 &   3.0 &   0.8 & &
                      & 84.9 & \bf{76.6} & \bf{66.2} & \bf{34.6} &  10.4 &   2.3 &  0.3 \\
            \EvNet    & 61.2 &  49.4 &  45.2 & \bf{37.6} & \bf{30.5} & \bf{23.4} & \bf{17.0} & &
                      & 61.2 &  29.4 &  23.0 &  16.8 &  14.2 &  10.2 &  5.5 \\
 			\midrule
 			& \multicolumn{16}{c}{\textbf{Sens. -- Sens. class 10, 11}} \\
            \PostNet  & \bf{87.2} & \bf{48.8} & \bf{37.3} &   4.1 &   0.7 &   0.7 &   0.7 & &
                      & \bf{87.2} & \bf{50.0} & \bf{45.4} &  16.5 & \bf{27.6} & \bf{81.9} & \bf{98.0} \\
            \PriorNet & 37.3 &   3.5 &   2.4 &   2.2 &   2.9 &   6.3 & \bf{19.2} & &
                      & 37.3 &   8.0 &   3.6 &   1.4 &   0.6 &   0.1 &   0.0 \\
            \DDNet    & 55.2 &  23.7 &  17.7 & \bf{14.1} & \bf{12.5} & \bf{12.7} &  15.7 & &
                      & 55.2 &  37.1 &  27.7 &   9.4 &   2.5 &   0.6 &   0.1 \\
            \EvNet    & 75.5 &  30.8 &  18.2 &   5.8 &   1.6 &   0.6 &   0.2 & &
                      & 75.5 &  47.8 &  41.9 & \bf{24.1} &  10.2 &  10.2 &  15.6 \\
             \midrule
 			& \multicolumn{16}{c}{\textbf{Seg. -- Seg. class sky}} \\
            \PostNet  & \bf{98.6} &  77.7 & \bf{50.8} & \bf{20.3} & \bf{8.2} & \bf{1.3} & \bf{0.5} & &
                      & \bf{98.6} &  88.9 &  73.4 &  36.2 &  19.4 & \bf{36.7} & \bf{75.2} \\
            \PriorNet & 8.5 &   1.3 &   0.2 &   0.0 &  0.0 &  0.0 &  0.1 & &
                      & 8.5 &   2.0 &   0.4 &   0.0 &   0.0 &   0.0 &   0.0 \\
            \DDNet    & 2.2 &   0.3 &   0.0 &   0.0 &  0.0 &  0.0 &  0.0 & &
                      & 2.2 &   0.5 &   0.1 &   0.0 &   0.0 &   0.0 &   0.0 \\
            \EvNet    & 97.7 & \bf{78.4} &  47.7 &   9.9 &  1.2 &  0.2 &  0.1 & &
                      & 97.7 & \bf{93.5} & \bf{86.9} & \bf{62.2} & \bf{21.5} &   3.7 &   1.0 \\
 			\bottomrule
 		\end{tabular}
 	\end{small}
 	\label{tab:id_ood_attacks_measure_alpha0_auroc}
\end{table*}



 \begin{table*}[htbp!]
 	\centering
 	\caption{OOD detection (AU-PR) under PGD uncertainty attacks against distributional uncertainty on ID data and OOD data.}
 	\begin{small}
 		\begin{tabular}{@{}rrrrrrrrc|crrrrrrr@{}}
 			\toprule
 			& \multicolumn{7}{c}{ID-Attack (non-attacked OOD)} &  & &  \multicolumn{7}{c}{OOD-Attack (non-attacked ID)} \\
 			\cmidrule{2-8}  \cmidrule{11-17}
 			Att. Rad. & 0.0 & 0.1 & 0.2 & 0.5 & 1.0 & 2.0 & 4.0 & & &
 			            0.0 & 0.1 & 0.2 & 0.5 & 1.0 & 2.0 & 4.0 \\
 			\midrule
 			& \multicolumn{16}{c}{\textbf{MNIST -- KMNIST}} \\
            \PostNet  & 80.5 &  76.2 &  73.4 &  69.1 &  66.6 &  65.4 & \bf{60.2} & &
                      & 80.5 &  72.1 &  63.9 &  43.9 &  33.0 &  30.9 &  30.8 \\
            \PriorNet & \bf{99.6} & \bf{99.4} & \bf{99.2} & \bf{98.0} & \bf{94.1} & \bf{76.3} &  31.2 & &
                      & \bf{99.6} & \bf{99.4} & \bf{99.2} & \bf{98.2} & \bf{95.2} & \bf{87.2} & \bf{75.2} \\
            \DDNet    & 98.4 &  98.1 &  97.7 &  95.8 &  89.5 &  56.2 &  30.9 & &
                      & 98.4 &  98.1 &  97.8 &  96.5 &  93.8 &  86.3 &  67.7 \\
            \EvNet    & 40.1 &  39.5 &  39.1 &  38.2 &  37.3 &  36.5 &  35.6 & &
                      & 40.1 &  34.6 &  32.6 &  31.3 &  31.0 &  31.0 &  31.1 \\
 			\midrule
 			& \multicolumn{16}{c}{\textbf{CIFAR10 -- SVHN}} \\
            \PostNet  & 64.2 &  44.7 &  37.5 & \bf{31.1} & \bf{28.5} & \bf{25.0} & \bf{19.3} & &
                      & 64.2 &  31.0 &  19.5 &  16.3 &  16.4 & \bf{16.5} & \bf{16.3} \\
            \PriorNet & 40.8 &  27.4 &  20.4 &  15.9 &  15.4 &  15.4 &  15.4  & &
                      & 40.8 &  28.3 &  21.1 &  15.9 &  15.4 &  15.4 &  15.4 \\
            \DDNet    & \bf{82.0} & \bf{71.0} & \bf{59.1} &  29.9 &  16.6 &  15.5 &  15.4 & &
                      & \bf{82.0} & \bf{72.2} & \bf{60.3} & \bf{26.3} &  16.2 &  15.4 &  15.4 \\
            \EvNet    & 36.4 &  28.7 &  26.5 &  22.8 &  20.2 &  18.4 &  17.2 & &
                      & 36.4 &  19.8 &  18.3 &  17.2 & \bf{16.9} &  16.2 &  15.7 \\
 			\midrule
 			& \multicolumn{16}{c}{\textbf{Sens. -- Sens. class 10, 11}} \\
            \PostNet  & \bf{79.1} & \bf{40.3} & \bf{35.9} & \bf{33.0} & \bf{45.5} & \bf{45.5} &  45.5 & &
                      & \bf{79.1} & \bf{47.3} & \bf{43.7} & \bf{36.5} & \bf{37.9} & \bf{74.6} & \bf{96.5} \\
            \PriorNet & 35.5 &  26.8 &  26.7 &  26.9 &  29.6 &  43.7 & \bf{68.7} & &
                      & 35.5 &  27.5 &  26.9 &  26.7 &  26.6 &  26.5 &  26.5 \\
            \DDNet    & 52.9 &  31.7 &  29.8 &  29.1 &  28.4 &  30.1 &  37.6 & &
                      & 52.9 &  38.4 &  31.5 &  27.5 &  26.8 &  26.6 &  26.5 \\
            \EvNet    & 66.3 &  33.3 &  29.6 &  27.0 &  27.2 &  29.3 &  35.2 & &
                      & 66.3 &  39.3 &  37.1 &  31.3 &  28.3 &  28.4 &  29.7 \\
 			\midrule
 			& \multicolumn{16}{c}{\textbf{Seg. -- Seg. class sky}} \\
            \PostNet  & 98.0 &  76.3 &  53.1 & \bf{37.4} & \bf{32.9} & \bf{44.6} & \bf{50.2} & &
                      & 98.0 &  83.5 &  64.8 &  41.8 &  35.4 & \bf{43.1} & \bf{71.3} \\
            \PriorNet & 32.3 &  30.9 &  30.8 &  30.8 &  30.8 &  32.5 &  45.0 & &
                      & 32.3 &  31.0 &  30.8 &  30.8 &  30.8 &  30.8 &  30.8 \\
            \DDNet    & 30.9 &  30.8 &  30.8 &  30.8 &  30.8 &  30.8 &  30.8 & &
                      & 30.9 &  30.8 &  30.8 &  30.8 &  30.8 &  30.8 &  30.8 \\
            \EvNet    & \bf{98.1} & \bf{82.1} & \bf{59.1} &  33.8 &  31.0 &  30.8 &  30.8 & &
                      & \bf{98.1} & \bf{93.8} & \bf{88.2} & \bf{64.5} & \bf{36.4} &  31.3 &  31.0 \\
 			\bottomrule
 		\end{tabular}
 	\end{small}
 	\label{tab:id_ood_attacks_measure_distU_aupr}
\end{table*}





 \begin{table*}[htbp!]
 	\centering
 	\caption{OOD detection (AUC-ROC) under PGD uncertainty attacks against distributional uncertainty on ID data and OOD data.}
 	\begin{small}
 		\begin{tabular}{@{}rrrrrrrrc|crrrrrrr@{}}
 			\toprule
 			& \multicolumn{7}{c}{ID-Attack (non-attacked OOD)} &  & &  \multicolumn{7}{c}{OOD-Attack (non-attacked ID)} \\
 			\cmidrule{2-8}  \cmidrule{11-17}
 			Att. Rad. & 0.0 & 0.1 & 0.2 & 0.5 & 1.0 & 2.0 & 4.0 & & &
 			            0.0 & 0.1 & 0.2 & 0.5 & 1.0 & 2.0 & 4.0 \\
 			\midrule
 			& \multicolumn{16}{c}{\textbf{MNIST -- KMNIST}} \\
            \PostNet  & 90.1 &  88.0 &  86.2 &  82.2 &  79.0 &  77.1 & \bf{66.1} & &
                      & 90.1 &  84.5 &  77.2 &  46.4 &  12.9 &   2.7 &   2.4 \\
            \PriorNet & \bf{99.8} & \bf{99.7} & \bf{99.6} & \bf{99.1} & \bf{97.2} & \bf{79.7} &   4.7 & &
                      & \bf{99.8} & \bf{99.7} & \bf{99.6} & \bf{99.2} & \bf{97.9} & \bf{93.7} & \bf{85.6} \\
            \DDNet    & 98.1 &  97.7 &  97.2 &  94.8 &  87.0 &  48.7 &   3.0 & &
                      & 98.1 &  97.8 &  97.3 &  95.8 &  92.3 &  83.3 &  63.3 \\
            \EvNet    & 36.8 &  35.0 &  33.7 &  30.9 &  28.2 &  25.3 &  22.1 & &
                      & 36.8 &  19.3 &  10.7 &   3.9 &   2.1 &   1.8 &   2.2 \\
 			\midrule
 			& \multicolumn{16}{c}{\textbf{CIFAR10 -- SVHN}} \\
            \PostNet  & 82.9 &  67.7 &  59.2 & \bf{51.3} & \bf{47.7} & \bf{40.1} & \bf{24.2} & &
                    & 82.9 &  51.9 &  26.2 &   8.9 &   9.5 & \bf{11.1} & \bf{9.9} \\
            \PriorNet & 48.0 &  33.6 &  22.5 &   7.1 &   1.6 &   0.3 &   0.1 & &
                      & 48.0 &  34.8 &  24.0 &   6.7 &   1.6 &   0.6 &  0.2 \\
            \DDNet    & \bf{85.9} & \bf{74.9} & \bf{62.7} &  30.1 &   8.3 &   2.3 &   0.6 & &
                    & \bf{85.9} & \bf{77.6} & \bf{66.9} & \bf{32.1} &   8.0 &   1.5 &  0.2 \\
            \EvNet    & 63.3 &  51.4 &  47.1 &  39.3 &  32.1 &  24.9 &  17.9 & &
                    & 63.3 &  31.1 &  24.4 &  17.7 & \bf{15.0} &  10.7 &  5.7 \\
 			\midrule
 			& \multicolumn{16}{c}{\textbf{Sens. -- Sens. class 10, 11}} \\
            \PostNet  & \bf{87.1} & \bf{50.9} & \bf{37.8} &   5.5 &  4.5 &   4.5 &   4.5 & &
                      & \bf{87.1} & \bf{55.3} & \bf{51.1} & \bf{34.4} & \bf{38.9} & \bf{79.7} & \bf{97.9} \\
            \PriorNet & 36.5 &   2.9 &   1.8 &   1.8 &  5.2 & \bf{21.5} & \bf{52.8} & &
                      & 36.5 &   7.3 &   3.0 &   1.3 &   0.5 &   0.1 &   0.0 \\
            \DDNet    & 52.3 &  18.7 &  13.1 & \bf{10.3} & \bf{9.3} &  10.8 &  18.4 & &
                      & 52.3 &  33.1 &  22.0 &   6.7 &   2.2 &   0.6 &   0.1 \\
            \EvNet    & 75.5 &  30.7 &  18.1 &   5.8 &  1.6 &   0.6 &   0.8 & &
                      & 75.5 &  47.7 &  41.8 &  23.8 &  10.3 &  10.2 &  15.8 \\
 			\midrule
 			& \multicolumn{16}{c}{\textbf{Seg. -- Seg. class sky}} \\
            \PostNet  & \bf{98.6} & \bf{78.3} & \bf{51.9} & \bf{20.5} & \bf{8.3} & \bf{2.1} &   1.7 & &
                      & \bf{98.6} &  88.8 &  73.1 &  35.9 & \bf{21.4} & \bf{39.9} & \bf{75.9} \\
            \PriorNet & 9.4 &   1.6 &   0.3 &   0.0 &  0.0 &  1.8 & \bf{15.4} & &
                      & 9.4 &   2.4 &   0.4 &   0.0 &   0.0 &   0.0 &   0.0 \\
            \DDNet    & 1.3 &   0.2 &   0.0 &   0.0 &  0.0 &  0.0 &   0.0 & &
                      & 1.3 &   0.2 &   0.0 &   0.0 &   0.0 &   0.0 &   0.0 \\
            \EvNet    & 97.4 &  77.1 &  45.9 &   9.4 &  1.3 &  0.2 &   0.1 & &
                      & 97.4 & \bf{92.9} & \bf{86.1} & \bf{60.9} &  20.4 &   3.0 &   1.2 \\
 			\bottomrule
 		\end{tabular}
 	\end{small}
 	\label{tab:id_ood_attacks_measure_distU_auroc}
\end{table*}

 \begin{table*}[htbp!]
 	\centering
 	\caption{OOD detection (AU-PR) under FGSM uncertainty attacks against differential entropy on ID data and OOD data.}
 	\begin{small}
 		\begin{tabular}{@{}rrrrrrrrc|crrrrrrr@{}}
 			\toprule
 			& \multicolumn{7}{c}{ID-Attack (non-attacked OOD)} &  & &  \multicolumn{7}{c}{OOD-Attack (non-attacked ID)} \\
 			\cmidrule{2-8}  \cmidrule{11-17}
 			Att. Rad. & 0.0 & 0.1 & 0.2 & 0.5 & 1.0 & 2.0 & 4.0 & & &
 			            0.0 & 0.1 & 0.2 & 0.5 & 1.0 & 2.0 & 4.0 \\
 			\midrule
 			& \multicolumn{16}{c}{\textbf{MNIST -- KMNIST}} \\
            \PostNet  & 94.5 &  94.2 &  94.1 &  93.5 &  89.9 &  81.2 &  \bf{71.6} & &
                      & 94.5 &  93.3 &  92.0 &  87.6 &  81.1 &  75.7 &  75.7 \\
            \PriorNet & \bf{99.6} &  \bf{99.4} &  \bf{99.2} &  \bf{98.1} &  \bf{95.6} &  \bf{90.0} &  65.3 & &
                      & \bf{99.6} &  \bf{99.4} &  \bf{99.2} &  \bf{98.6} &  97.5 &  \bf{95.9} &  \bf{94.4} \\
            \DDNet    & 99.3 &  99.1 &  98.9 &  98.0 &  95.4 &  80.9 &  48.2 & &
                      & 99.3 &  99.2 &  99.0 &  98.5 &  \bf{97.6} &  95.5 &  92.0 \\
            \EvNet    & 69.0 &  67.4 &  66.2 &  64.0 &  61.9 &  59.8 &  56.70 & &
                      & 9.0 &  60.1 &  56.5 &  53.4 &  52.7 &  52.9 &  53.5 \\
 			\midrule
 			& \multicolumn{16}{c}{\textbf{CIFAR10 -- SVHN}} \\
            \PostNet  & 81.8 &  66.2 &  61.6 &  \bf{64.2} &  \bf{65.7} &  61.3 &  48.4 & &
                    & 81.8 &  63.1 &  51.9 &  43.4 &  46.6 &  \bf{61.7} &  \bf{77.0} \\
            \PriorNet & 54.4 &  40.6 &  33.8 &  27.0 &  25.5 &  27.2 &  35.5 & &
                      & 54.4 &  42.3 &  36.8 &  30.6 &  28.3 &  29.5 &  32.1 \\
            \DDNet    & \bf{82.8} &  \bf{71.9} &  \bf{64.6} &  53.8 &  50.2 &  47.8 &  41.0 & &
                    & \bf{82.8} &  \bf{71.5} &  \bf{60.5} &  39.1 &  31.4 &  41.2 &  66.6 \\
            \EvNet    & 80.3 &  67.8 &  64.0 &  61.9 &  61.6 &  57.4 &  \bf{49.6} & &
                    & 80.3 &  59.2 &  51.5 &  \bf{46.7} &  \bf{49.0} &  56.3 &  64.6 \\
 			\midrule
 			& \multicolumn{16}{c}{\textbf{Sens. -- Sens. class 10, 11}} \\
            \PostNet  & \bf{74.5} &  40.6 &  37.2 &  31.4 &  38.1 &  44.9 &  45.9 & &
                      & \bf{74.5} &  \bf{99.6} &  \bf{99.8} &  \bf{99.9} &  \bf{99.9} &  \bf{99.9} &  \bf{99.9} \\
            \PriorNet & 32.3 &  35.7 &  \bf{57.6} &  \bf{83.1} &  \bf{88.8} &  79.7 &  70.0 & &
                      & 32.3 &  28.3 &  28.1 &  27.6 &  28.0 &  32.7 &  38.5 \\
            \DDNet    & 31.7 &  31.3 &  44.4 &  70.3 &  87.9 &  \bf{92.5} &  \bf{91.9} & &
                      & 31.7 &  28.8 &  29.3 &  29.1 &  27.7 &  27.9 &  28.01 \\
            \EvNet    & 66.5 &  \bf{45.7} &  46.8 &  42.3 &  42.0 &  41.4 &  41.8 & &
                      & 66.5 &  54.7 &  66.5 &  76.2 &  71.1 &  75.3 &  75.8 \\
 			\midrule
 			& \multicolumn{16}{c}{\textbf{Seg. -- Seg. class sky}} \\
            \PostNet  & \bf{99.0} &  \bf{80.8} &  \bf{66.4} &  43.6 &  37.0 &  35.5 &  43.0 & &
                      & \bf{99.0} &  \bf{94.8} &  \bf{92.0} &  \bf{98.5} &  \bf{99.7} &  \bf{100.0} &  \bf{100.0} \\
            \PriorNet & 34.8 &  31.2 &  31.4 &  46.3 &  \bf{74.0} &  \bf{88.8} &  \bf{94.5} & &
                      & 34.8 &  31.6 &  31.0 &  31.2 &  30.9 &   30.8 &   30.8 \\
            \DDNet    & 31.5 &  30.8 &  30.8 &  30.9 &  37.9 &  56.2 &  84.3 & &
                      & 31.5 &  30.9 &  30.8 &  30.8 &  30.8 &   30.8 &   30.8 \\
            \EvNet    & 92.5 &  64.9 &  54.6 &  \bf{66.6} &  69.5 &  69.6 &  64.6 & &
                      & 92.5 &  85.9 &  83.0 &  66.3 &  66.1 &   61.1 &   56.8  \\
 			\bottomrule
 		\end{tabular}
 	\end{small}
 	\label{tab:id_ood_attacks_measure_diffE_aupr_fgsm}
\end{table*}


 \begin{table*}[htbp!]
 	\centering
 	\caption{OOD detection (AU-PR) under Noise uncertainty attacks against differential entropy on ID data and OOD data.}
 	\begin{small}
 		\begin{tabular}{@{}rrrrrrrrc|crrrrrrr@{}}
 			\toprule
 			& \multicolumn{7}{c}{ID-Attack (non-attacked OOD)} &  & &  \multicolumn{7}{c}{OOD-Attack (non-attacked ID)} \\
 			\cmidrule{2-8}  \cmidrule{11-17}
 			Noise Std & 0.0 & 0.1 & 0.2 & 0.5 & 1.0 & 2.0 & 4.0 & & &
 			            0.0 & 0.1 & 0.2 & 0.5 & 1.0 & 2.0 & 4.0 \\
 			\midrule
 			& \multicolumn{16}{c}{\textbf{MNIST -- KMNIST}} \\
            \PostNet  & 93.0 &  94.2 &  82.3 &  34.4 &  31.6 &  31.0 &  30.9 & &
                      & 92.2 &  91.8 &  91.5 &  92.3 &   92.7 &  93.2 &  93.5 \\
            \PriorNet & \bf{99.7} &  \bf{99.6} &  \bf{96.7} &  \bf{40.0} &  \bf{40.6} &  \bf{45.7} &  \bf{55.6} & &
                      & \bf{99.5} &  97.3 &  96.5 &  99.4 &  \bf{100.0} &  99.5 &  72.4 \\
            \DDNet    & 99.1 &  97.5 &  81.2 &  31.3 &  31.0 &  30.9 &  31.2 & &
                      & 99.0 &  \bf{98.8} &  \bf{99.2} &  \bf{99.8} &   99.9 &  \bf{99.8} &  \bf{99.1} \\
            \EvNet    & 65.5 &  60.5 &  51.4 &  35.3 &  34.5 &  35.5 &  35.0 & &
                      & 62.5 &  47.2 &  40.9 &  35.1 &   34.6 &  33.5 &  34.9 \\
 			\midrule
 			& \multicolumn{16}{c}{\textbf{CIFAR10 -- SVHN}} \\
            \PostNet  & 88.5 &  41.4 &  39.8 &  31.0 &  30.7 &  31.6 &  33.9 & &
                    & 88.5 &  \bf{86.6} &  \bf{81.9} & \bf{ 93.}0 &  \bf{98.5} &  98.6 &   97.3 \\
            \PriorNet & 73.3 &  88.3 &  \bf{95.3} &  \bf{92.4} &  \bf{70.4} &  30.9 &  30.8 & &
                      & 73.3 &  31.6 &  30.9 &  31.7 &  51.8 &  94.3 &  \bf{100.0} \\
            \DDNet    & 87.3 &  69.3 &  78.4 &  55.2 &  31.6 &  30.7 &  31.4 & &
                    & 87.3 &  55.8 &  57.9 &  73.9 &  97.3 &  \bf{99.5} &   97.2 \\
            \EvNet    & \bf{92.4} &  \bf{56.8} &  53.8 &  33.4 &  30.9 &  \bf{32.9} &  \bf{36.6} & &
                    & \bf{92.4} &  73.7 &  73.5 &  77.7 &  93.7 &  92.5 &   92.1 \\
 			\midrule
 			& \multicolumn{16}{c}{\textbf{Sens. -- Sens. class 10, 11}} \\
            \PostNet  & \bf{85.3} &  \bf{30.8} &  \bf{39.4} &  50.0 &  50.0 &  50.0 &  50.0 & &
                      & \bf{85.3} &  \bf{98.9} &  \bf{100.0} &  \bf{100.0} &  \bf{100.0} &  \bf{100.0} &  \bf{100.0} \\
            \PriorNet & 32.3 &  \bf{30.8} &  34.9 &  \bf{83.7} &  \bf{77.7} &  49.8 &  \bf{80.3} & &
                      & 32.3 &  30.7 &   30.7 &   32.5 &   40.1 &   49.9 &   47.6 \\
            \DDNet    & 31.1 &  30.7 &  30.7 &  32.4 &  58.8 &  \bf{88.1} &  74.3 & &
                      & 31.1 &  30.7 &   30.7 &   30.7 &   30.8 &   31.6 &   39.1 \\
            \EvNet    & 80.3 &  \bf{30.8} &  31.2 &  37.9 &  46.3 &  50.0 &  50.0 & &
                      & 80.3 &  34.6 &   38.4 &   53.9 &   69.3 &   78.8 &   81.5 \\
 			\midrule
 			& \multicolumn{16}{c}{\textbf{Seg. -- Seg. class sky}} \\
            \PostNet  & \bf{99.9} &  \bf{41.8} &  30.8 &  \bf{34.5} &  \bf{49.1} &  50.0 &  50.0 & &
                      & \bf{99.9} &  \bf{97.4} &  \bf{96.6} &  \bf{99.5} &  \bf{100.0} &  \bf{100.0} &  \bf{100.0} \\
            \PriorNet & 31.0 &  30.8 &  30.8 &  30.8 &  32.7 &  \bf{69.0} &  78.3 & &
                      & 31.0 &  30.8 &  30.8 &  30.8 &   30.9 &   31.1 &   32.4 \\
            \DDNet    & 30.8 &  30.8 &  30.8 &  30.8 &  30.8 &  58.2 &  \bf{91.3} & &
                      & 30.8 &  30.8 &  30.8 &  30.8 &   30.8 &   30.8 &   31.9 \\
            \EvNet    & 99.1 &  38.1 &  \bf{32.2} &  30.8 &  30.8 &  32.2 &  37.5 & &
                      & 99.1 &  95.6 &  87.6 &  58.0 &   44.9 &   46.6 &   53.8  \\
 			\bottomrule
 		\end{tabular}
 	\end{small}
 	\label{tab:id_ood_attacks_measure_diffE_aupr_noise}
\end{table*}








\clearpage
\subsection{How to make DBU models more robust}

To improve robustness of DBU models we perform median smoothing and adversarial training. Smoothing computes the smooth median, worst case and best case performance of DBU models for three tasks:  distinguishing between correct and wrong predictions, attack detection, distinguishing between ID data and OOD data under label attacks and under uncertainty attacks. 

%%%% Confidence %%%%

\begin{table*}[ht!]
	\centering
	\caption{Distinguishing between correctly and wrongly labeled inputs based on differential entropy under PGD label attacks. Smoothed DBU models on CIFAR10. Column format: guaranteed lowest performance $\cdot$ empirical performance $\cdot$ guaranteed highest performance (blue: normally/adversarially trained smooth classifier is more robust than the base model).}
	\label{tab:cifar10_smooth_confidence}
	%\begin{tiny}
	\resizebox{\textwidth}{!}{
    \begin{tabular}{llcccccc}
    \toprule
    & \textbf{Att. Rad.} & 0.0 &   0.1 &  0.2 &  0.5 &  1.0 &  2.0 \\
    \midrule
    %& \multicolumn{6}{c}{\textbf{Smoothed models}} \\
      \multirow{4}{1.2cm}{Smoothed models} & \textbf{PostNet} &  $80.5\cdot\bm{{\color{black}91.5}}\cdot94.5$ &  
  $52.8\cdot\bm{{\color{black}71.6}}\cdot95.2$ &  
  $31.9\cdot\bm{{\color{black}51.0}}\cdot96.8$ &  
  $\phantom{0}5.6\cdot\bm{{\color{blue}11.7}}\cdot100.0$ &  
  $\phantom{0}0.3\cdot\bm{{\color{black}\phantom{0}0.6}}\cdot100.0$ &  
  $\phantom{0}0.0\cdot\bm{{\color{black}\phantom{0}0.0}}\cdot100.0$ \\
 & \textbf{PriorNet} &  
 $81.9\cdot\bm{{\color{black}86.8}}\cdot88.0$ &     
 $69.6\cdot\bm{{\color{blue}78.0}}\cdot90.1$ &     
 $50.9\cdot\bm{{\color{blue}65.8}}\cdot89.4$ & 
 $36.5\cdot\bm{{\color{blue}59.9}}\cdot\phantom{0}97.0$ &   
 $24.3\cdot\bm{{\color{blue}39.3}}\cdot100.0$ &    
 $\phantom{0}9.2\cdot\bm{{\color{blue}17.9}}\cdot100.0$ \\
  &   \textbf{DDNet} &  
  $65.9\cdot\bm{{\color{black}81.2}}\cdot83.0$ &  
  $55.8\cdot\bm{{\color{black}70.5}}\cdot87.2$ &  
  $37.8\cdot\bm{{\color{black}56.8}}\cdot88.1$ &  
  $10.1\cdot\bm{{\color{blue}21.9}}\cdot\phantom{0}94.3$ &      
  $\phantom{0}0.9\cdot\phantom{0}\bm{{\color{blue}1.6}}\cdot\phantom{0}99.6$ &                   
  $\phantom{0}0.0\cdot\phantom{0}\bm{0.0}\cdot100.0$ \\
  &   \textbf{EvNet} &  
  $76.3\cdot\bm{{\color{black}90.2}}\cdot91.7$ &  
  $54.7\cdot\bm{{\color{black}74.3}}\cdot95.7$ &  
  $31.6\cdot\bm{{\color{black}51.5}}\cdot94.5$ &   
  $\phantom{0}5.8\cdot\bm{{\color{blue}11.9}}\cdot\phantom{0}86.9$ &     
  $\phantom{0}1.9\cdot\bm{{\color{blue}\phantom{0}7.0}}\cdot100.0$ &     
  $\phantom{0}1.1\cdot\bm{{\color{blue}\phantom{0}4.0}}\cdot100.0$ \\

    \midrule
    %& \multicolumn{6}{c}{\textbf{Smoothed models + adversarial training using label attacks}} \\
      \multirow{4}{1.2cm}{Smoothed + adv. label attacks} &  
  \textbf{PostNet} &  - &  
  $52.1\cdot\bm{{\color{black}71.8}}\cdot95.6$ &  
  $31.2\cdot\bm{{\color{black}47.9}}\cdot96.1$ &  
  $\phantom{0}7.8\cdot\bm{{\color{blue}14.7}}\cdot\phantom{0}98.6$ &    
  $\phantom{0}1.8\cdot\phantom{0}\bm{{\color{blue}4.4}}\cdot100.0$ &  
  $\phantom{0}0.3\cdot\phantom{0}\bm{{\color{black}0.5}}\cdot100.0$ \\
 & \textbf{PriorNet} &  - &  
 $57.6\cdot\bm{{\color{black}71.7}}\cdot88.9$ &     
 $46.1\cdot\bm{{\color{blue}64.5}}\cdot90.1$ &  
 $38.1\cdot\bm{{\color{blue}59.3}}\cdot\phantom{0}99.5$ &  
 $32.3\cdot\bm{{\color{blue}51.7}}\cdot100.0$ &    
 $22.1\cdot\bm{{\color{blue}41.6}}\cdot\phantom{0}97.4$ \\
   & \textbf{DDNet} &  - &  
   $58.6\cdot\bm{{\color{black}78.4}}\cdot92.2$ &  
   $49.4\cdot\bm{{\color{black}66.0}}\cdot90.5$ &  
   $12.0\cdot\bm{{\color{blue}21.4}}\cdot\phantom{0}98.1$ &     
   $\phantom{0}0.8\cdot\phantom{0}\bm{{\color{blue}1.0}}\cdot\phantom{0}96.6$ &                   
   $\phantom{0}0.0\cdot\phantom{0}\bm{0.0}\cdot100.0$ \\
    & \textbf{EvNet} &  - &  
    $24.3\cdot\bm{{\color{black}34.2}}\cdot51.8$ &  
    $32.6\cdot\bm{{\color{black}49.5}}\cdot95.5$ &  
    $\phantom{0}5.9\cdot\bm{{\color{blue}13.0}}\cdot100.0$ &  
    $\phantom{0}2.6\cdot\phantom{0}\bm{{\color{black}5.2}}\cdot\phantom{0}99.9$ &     
    $\phantom{0}2.9\cdot\phantom{0}\bm{{\color{blue}5.9}}\cdot100.0$ \\

    \midrule
    %& \multicolumn{6}{c}{\textbf{Smoothed models + adversarial training using uncertainty attacks}} \\
    \multirow{4}{1.3cm}{Smoothed + adv. uncert. attacks} &   
\textbf{PostNet} &  - &  
$52.8\cdot\bm{{\color{black}74.2}}\cdot94.6$ &  
$33.0\cdot\bm{{\color{black}49.4}}\cdot87.5$ &   
$\phantom{0}7.7\cdot\bm{{\color{blue}14.2}}\cdot\phantom{0}99.0$ &  
$\phantom{0}0.6\cdot\phantom{0}\bm{{\color{black}1.2}}\cdot100.0$ &    
$\phantom{0}0.7\cdot\phantom{0}\bm{{\color{blue}1.1}}\cdot100.0$ \\
 & \textbf{PriorNet} &  - &  
 $50.6\cdot\bm{{\color{black}68.1}}\cdot88.6$ &     
 $44.4\cdot\bm{{\color{blue}66.1}}\cdot96.0$ &  
 $35.1\cdot\bm{{\color{blue}57.4}}\cdot\phantom{0}98.4$ &   
 $18.4\cdot\bm{{\color{blue}32.2}}\cdot100.0$ &  
 $15.2\cdot\bm{{\color{blue}29.3}}\cdot100.0$ \\
   & \textbf{DDNet} &  - &  
   $68.8\cdot\bm{{\color{black}84.4}}\cdot93.2$ &  
   $45.1\cdot\bm{{\color{black}60.8}}\cdot86.8$ &  
   $12.3\cdot\bm{{\color{blue}22.0}}\cdot\phantom{0}91.0$ &      
   $\phantom{0}0.8\cdot\phantom{0}\bm{{\color{blue}1.7}}\cdot\phantom{0}87.0$ &                  
   $\phantom{0}0.0\cdot\phantom{0}\bm{0.0}\cdot100.0$ \\
&    \textbf{EvNet} &  - &  
$54.2\cdot\bm{{\color{black}73.7}}\cdot96.1$ &  
$30.5\cdot\bm{{\color{black}50.0}}\cdot99.5$ &  
$\phantom{0}7.1\cdot\bm{{\color{blue}13.9}}\cdot100.0$ &      
$\phantom{0}3.7\cdot\phantom{0}\bm{{\color{blue}8.7}}\cdot\phantom{0}75.2$ &    
$\phantom{0}3.3\cdot\phantom{0}\bm{{\color{blue}5.8}}\cdot100.0$ \\

    \bottomrule
    \end{tabular}}
	%\end{tiny}
\end{table*}


\begin{table*}[ht!]
	\centering
	\caption{Distinguishing between correctly and wrongly labeled inputs based on differential entropy under PGD label attacks. Smoothed DBU models on MNIST. Column format: guaranteed lowest performance $\cdot$ empirical performance $\cdot$ guaranteed highest performance (blue: normally/adversarially trained smooth classifier is more robust than the base model).}
	\label{tab:mnist_smooth_confidence}
	%\begin{tiny}
	\resizebox{\textwidth}{!}{
		\begin{tabular}{llccccccc}
			\toprule
			& \textbf{Att. Rad.} & 0.0 &   0.1 &  0.2 &  0.5 &  1.0 &  2.0 \\
			\midrule
			%& \multicolumn{6}{c}{\textbf{Smoothed models}} \\
              \multirow{4}{1.2cm}{Smoothed models} &   
  \textbf{PostNet} &  
  $97.2\cdot\bm{{\color{black}99.4}}\cdot100.0$ & 
  $95.9\cdot\bm{{\color{black}99.1}}\cdot99.9$ &  
  $94.7\cdot\bm{{\color{black}98.9}}\cdot99.9$ & 
  $89.3\cdot\bm{{\color{black}96.8}}\cdot\phantom{0}99.9$ & 
  $75.5\cdot\bm{{\color{blue}90.2}}\cdot100.0$ & 
  $35.5\cdot\bm{{\color{blue}56.7}}\cdot100.0$ \\
& \textbf{PriorNet} &  
$96.8\cdot\bm{{\color{black}99.2}}\cdot\phantom{0}99.3$ & 
$95.5\cdot\bm{{\color{black}99.1}}\cdot99.7$ & 
$94.6\cdot\bm{{\color{black}98.8}}\cdot99.7$ & 
$90.2\cdot\bm{{\color{black}97.2}}\cdot\phantom{0}99.9$ & 
$81.1\cdot\bm{{\color{blue}93.4}}\cdot\phantom{0}99.9$ &  
$53.9\cdot\bm{{\color{blue}75.2}}\cdot100.0$ \\
 &   \textbf{DDNet} &  
 $97.6\cdot\bm{{\color{black}99.4}}\cdot\phantom{0}99.5$ &  
 $96.8\cdot\bm{{\color{black}99.2}}\cdot99.4$ &
 $95.5\cdot\bm{{\color{black}98.8}}\cdot99.4$ & 
 $90.4\cdot\bm{{\color{black}97.2}}\cdot\phantom{0}99.8$ &  
 $77.0\cdot\bm{{\color{black}91.3}}\cdot100.0$ &
 $29.2\cdot\bm{{\color{black}48.6}}\cdot100.0$ \\
  &  \textbf{EvNet} &   
  $97.3\cdot\bm{{\color{black}99.4}}\cdot\phantom{0}99.4$ & 
  $95.4\cdot\bm{{\color{black}98.8}}\cdot99.6$ &            
  $93.9\cdot\bm{98.7}\cdot99.9$ & 
  $89.0\cdot\bm{{\color{blue}96.5}}\cdot100.0$ &  
  $78.9\cdot\bm{{\color{blue}92.9}}\cdot100.0$ & 
  $52.2\cdot\bm{{\color{blue}73.2}}\cdot100.0$ \\

			\midrule
			%& \multicolumn{6}{c}{\textbf{Adversarially trained models using label attacks}} \\
              \multirow{4}{1.2cm}{Smoothed + adv. w. label attacks} &   
  \textbf{PostNet} &  - &  
  $94.4\cdot\bm{{\color{black}98.6}}\cdot99.5$ &  
  $90.6\cdot\bm{{\color{black}97.9}}\cdot99.9$ &  
  $83.4\cdot\bm{{\color{black}93.1}}\cdot\phantom{0}99.9$ &   
  $72.1\cdot\bm{{\color{blue}91.2}}\cdot100.0$ &   
  $41.8\cdot\bm{{\color{blue}65.0}}\cdot100.0$ \\
& \textbf{PriorNet} &  - & 
$94.4\cdot\bm{{\color{black}98.5}}\cdot99.5$ &
$93.6\cdot\bm{{\color{black}98.8}}\cdot99.8$ &  
$89.1\cdot\bm{{\color{black}96.6}}\cdot\phantom{0}99.8$ &  
$81.5\cdot\bm{{\color{blue}94.5}}\cdot100.0$ &  
$71.6\cdot\bm{{\color{blue}88.4}}\cdot100.0$ \\
 &   \textbf{DDNet} &  - &
 $94.9\cdot\bm{{\color{black}98.3}}\cdot98.7$ & 
 $94.6\cdot\bm{{\color{black}97.9}}\cdot98.9$ & 
 $88.2\cdot\bm{{\color{black}97.4}}\cdot\phantom{0}99.8$ &  
 $72.1\cdot\bm{{\color{black}89.3}}\cdot100.0$ &
 $28.1\cdot\bm{{\color{black}49.3}}\cdot100.0$ \\
  &  \textbf{EvNet} &  - & 
  $88.8\cdot\bm{{\color{black}95.3}}\cdot97.9$ & 
  $91.5\cdot\bm{{\color{black}97.1}}\cdot99.4$ & 
  $85.2\cdot\bm{{\color{black}94.9}}\cdot100.0$ &  
  $78.1\cdot\bm{{\color{blue}91.4}}\cdot100.0$ &    
  $54.3\cdot\bm{{\color{blue}75.3}}\cdot100.0$ \\

            \midrule
			%& \multicolumn{6}{c}{\textbf{Smoothed models + adversarial training using uncertainty attacks}} \\
              \multirow{4}{1.2cm}{Smoothed + adv. w. uncert. attacks} &   
  \textbf{PostNet} &  - & 
  $92.8\cdot\bm{{\color{black}98.3}}\cdot99.8$ &
  $92.5\cdot\bm{{\color{black}98.3}}\cdot99.9$ & 
  $86.2\cdot\bm{{\color{black}94.8}}\cdot\phantom{0}99.8$ & 
  $71.0\cdot\bm{89.5}\cdot100.0$ &
  $34.6\cdot\bm{{\color{blue}54.2}}\cdot100.0$ \\
& \textbf{PriorNet} &  - & 
$95.1\cdot\bm{{\color{black}98.6}}\cdot99.6$ &
$94.1\cdot\bm{{\color{black}98.0}}\cdot99.4$ & 
$87.7\cdot\bm{{\color{black}97.2}}\cdot\phantom{0}99.9$ &  
$80.2\cdot\bm{{\color{blue}93.4}}\cdot100.0$ & 
$68.5\cdot\bm{{\color{blue}87.8}}\cdot100.0$ \\
 &   \textbf{DDNet} &  - & 
 $96.0\cdot\bm{{\color{black}98.4}}\cdot98.8$ &
 $95.0\cdot\bm{{\color{black}97.6}}\cdot98.7$ & 
 $87.6\cdot\bm{{\color{black}95.3}}\cdot\phantom{0}99.7$ &  
 $73.9\cdot\bm{{\color{black}90.2}}\cdot100.0$ & 
 $32.8\cdot\bm{{\color{blue}54.4}}\cdot100.0$ \\
  &  \textbf{EvNet} &  - & 
  $93.3\cdot\bm{{\color{black}98.6}}\cdot99.5$ &  
  $89.8\cdot\bm{{\color{black}97.2}}\cdot99.2$ & 
  $86.2\cdot\bm{{\color{black}95.4}}\cdot100.0$ & 
  $82.1\cdot\bm{{\color{blue}93.7}}\cdot100.0$ & 
  $52.4\cdot\bm{{\color{blue}73.3}}\cdot100.0$ \\

			\bottomrule
		\end{tabular}}
	%\end{tiny}
\end{table*}





\begin{table*}[ht!]
	\centering
	\caption{Distinguishing between correctly and wrongly labeled inputs based on differential entropy under PGD label attacks. Smoothed DBU models on Sensorless. Column format: guaranteed lowest performance $\cdot$ empirical performance $\cdot$ guaranteed highest performance (blue: normally/adversarially trained smooth classifier is more robust than the base model).}
	\label{tab:sensorless_smooth_confidence}
	%\begin{tiny}
	\resizebox{\textwidth}{!}{
		\begin{tabular}{llccccccc}
			\toprule
			& \textbf{Att. Rad.} & 0.0 &   0.1 &  0.2 &  0.5 &  1.0 &  2.0 \\
			\midrule
			%& \multicolumn{6}{c}{\textbf{Smoothed models}} \\
              \multirow{4}{1.2cm}{Smoothed models} &  
  \textbf{PostNet} &  
  $93.5\cdot\bm{{\color{black}98.4}}\cdot100.0$ & 
  $\phantom{0}6.7\cdot\bm{{\color{blue}12.4}}\cdot100.0$ &  
  $\phantom{0}2.9\cdot\phantom{0}\bm{{\color{blue}5.3}}\cdot100.0$ &  
  $\phantom{0}4.1\cdot\phantom{0}\bm{{\color{blue}4.1}}\cdot\phantom{0}49.1$ & 
  $\phantom{0}6.4\cdot\phantom{0}\bm{{\color{black}6.4}}\cdot\phantom{00}6.4$ &
  $10.6\cdot\bm{{\color{blue}10.6}}\cdot\phantom{0}10.6$ \\
& \textbf{PriorNet} & 
$97.1\cdot\bm{{\color{black}99.3}}\cdot100.0$ &   
$\phantom{0}8.6\cdot\bm{{\color{blue}17.6}}\cdot100.0$ & 
$\phantom{0}3.3\cdot\phantom{0}\bm{{\color{blue}7.7}}\cdot100.0$ &  
$\phantom{0}0.7\cdot\phantom{0}\bm{{\color{blue}1.5}}\cdot100.0$ & 
$\phantom{0}0.4\cdot\phantom{0}\bm{{\color{blue}0.7}}\cdot100.0$ &  
$\phantom{0}0.1\cdot\phantom{0}\bm{0.2}\cdot100.0$ \\
 &   \textbf{DDNet} &  
 $95.9\cdot\bm{{\color{black}98.9}}\cdot\phantom{0}99.7$ & 
 $\phantom{0}7.0\cdot\bm{{\color{blue}14.0}}\cdot100.0$ & 
 $\phantom{0}0.8\cdot\phantom{0}\bm{{\color{black}1.3}}\cdot100.0$ &   
 $\phantom{0}0.2\cdot\phantom{0}\bm{0.4}\cdot100.0$ &           
 $\phantom{0}0.2\cdot\phantom{0}\bm{0.2}\cdot100.0$ &   
 $\phantom{0}0.2\cdot\phantom{0}\bm{{\color{blue}0.4}}\cdot100.0$ \\
  &  \textbf{EvNet} &   
  $94.0\cdot\bm{{\color{black}99.0}}\cdot\phantom{0}99.9$ & 
  $18.1\cdot\bm{{\color{blue}34.2}}\cdot100.0$ & 
  $\phantom{0}9.6\cdot\bm{{\color{blue}17.1}}\cdot100.0$ &
  $\phantom{0}4.1\cdot\phantom{0}\bm{{\color{blue}6.8}}\cdot100.0$ &  
  $\phantom{0}2.7\cdot\phantom{0}\bm{{\color{blue}4.9}}\cdot100.0$ &
  $\phantom{0}2.4\cdot\phantom{0}\bm{{\color{blue}4.3}}\cdot100.0$ \\

			\midrule
			%& \multicolumn{6}{c}{\textbf{Smoothed models + adversarial training using label attacks}} \\
              \multirow{4}{1.2cm}{Smoothed + adv. w. label attacks} &
  \textbf{PostNet} &  - &  
  $\phantom{0}7.9\cdot\bm{{\color{blue}14.9}}\cdot100.0$ & 
  $\phantom{0}2.9\cdot\phantom{0}\bm{{\color{blue}6.3}}\cdot100.0$ &     
  $\phantom{0}6.6\cdot\phantom{0}\bm{{\color{blue}6.6}}\cdot\phantom{10}6.6$ &     
  $\phantom{0}7.2\cdot\phantom{0}\bm{{\color{blue}7.2}}\cdot\phantom{10}7.2$ &   
  $\phantom{0}9.6\cdot\phantom{0}\bm{{\color{black}9.6}}\cdot\phantom{10}9.6$ \\
 & \textbf{PriorNet} &  - &
 $18.1\cdot\bm{{\color{blue}32.1}}\cdot100.0$ & 
 $\phantom{0}8.7\cdot\bm{{\color{blue}16.7}}\cdot100.0$ &  
 $\phantom{0}0.1\cdot\phantom{0}\bm{{\color{black}0.2}}\cdot100.0$ &  
 $\phantom{0}0.0\cdot\phantom{0}\bm{{\color{black}0.0}}\cdot100.0$ &  
 $\phantom{0}0.8\cdot\phantom{0}\bm{{\color{blue}1.0}}\cdot100.0$ \\
   & \textbf{DDNet} &  - &  
   $\phantom{0}6.9\cdot\bm{{\color{blue}13.4}}\cdot100.0$ &  
   $\phantom{0}4.3\cdot\phantom{0}\bm{{\color{blue}9.0}}\cdot100.0$ &  
   $\phantom{0}0.2\cdot\phantom{0}\bm{{\color{black}0.3}}\cdot100.0$ &  
   $\phantom{0}0.2\cdot\phantom{0}\bm{{\color{blue}0.4}}\cdot100.0$ &  
   $\phantom{0}0.2\cdot\phantom{0}\bm{{\color{blue}0.8}}\cdot100.0$ \\
&    \textbf{EvNet} &  - & 
$19.7\cdot\bm{{\color{blue}35.7}}\cdot100.0$ & 
$\phantom{0}9.4\cdot\bm{{\color{blue}16.2}}\cdot100.0$ & 
$\phantom{0}1.6\cdot\phantom{0}\bm{{\color{black}3.0}}\cdot100.0$ & 
$\phantom{0}2.5\cdot\phantom{0}\bm{{\color{blue}5.6}}\cdot100.0$ & 
$\phantom{0}1.0\cdot\phantom{0}\bm{{\color{black}1.8}}\cdot100.0$ \\

			\midrule
			%& \multicolumn{6}{c}{\textbf{Smoothed models + adversarial training using uncertainty attacks}} \\
              \multirow{4}{1.2cm}{Smoothed + adv. uncert. attacks} &  
  \textbf{PostNet} &  - &    
  $\phantom{0}7.9\cdot\bm{{\color{blue}14.4}}\cdot100.0$ &   
  $\phantom{0}4.8\cdot\phantom{0}\bm{{\color{blue}9.3}}\cdot100.0$ &   
  $\phantom{0}6.6\cdot\phantom{0}\bm{{\color{blue}6.6}}\cdot\phantom{00}6.6$ &   
  $\phantom{0}6.7\cdot\phantom{0}\bm{{\color{black}6.7}}\cdot\phantom{00}6.7$ &    
  $10.6\cdot\bm{{\color{blue}10.6}}\cdot\phantom{0}10.6$ \\
 & \textbf{PriorNet} &  - &   
 $19.1\cdot\bm{{\color{blue}32.7}}\cdot100.0$ &   
 $\phantom{0}6.9\cdot\bm{{\color{blue}13.7}}\cdot100.0$ &  
 $\phantom{0}0.7\cdot\phantom{0}\bm{{\color{blue}1.7}}\cdot100.0$ & 
 $\phantom{0}0.0\cdot\phantom{0}\bm{{\color{black}0.0}}\cdot100.0$ &
 $\phantom{0}0.0\cdot\phantom{0}\bm{{\color{black}0.0}}\cdot100.0$ \\
   & \textbf{DDNet} &  - & 
   $\phantom{0}5.4\cdot\bm{{\color{black}10.2}}\cdot100.0$ &
   $\phantom{0}0.7\cdot\phantom{0}\bm{{\color{blue}1.8}}\cdot100.0$ & 
   $\phantom{0}0.5\cdot\phantom{0}\bm{{\color{blue}0.9}}\cdot100.0$ &  
   $\phantom{0}0.3\cdot\phantom{0}\bm{{\color{blue}1.2}}\cdot100.0$ &
   $\phantom{0}0.2\cdot\phantom{0}\bm{{\color{blue}0.6}}\cdot100.0$ \\
&    \textbf{EvNet} &  - &   
$22.3\cdot\bm{{\color{blue}38.4}}\cdot100.0$ & 
$11.7\cdot\bm{{\color{blue}22.4}}\cdot100.0$ &  
$\phantom{0}7.1\cdot\bm{{\color{blue}13.1}}\cdot100.0$ & 
$\phantom{0}1.8\cdot\phantom{0}\bm{{\color{black}3.4}}\cdot100.0$ & 
$\phantom{0}0.6\cdot\phantom{0}\bm{{\color{black}1.0}}\cdot100.0$ \\

			\bottomrule
		\end{tabular}}
	%\end{tiny}
\end{table*}


\begin{table*}[ht!]
	\centering
	\caption{Distinguishing between correctly and wrongly labeled inputs based on differential entropy under PGD label attacks. Smoothed DBU models on Segment. Column format: guaranteed lowest performance $\cdot$ empirical performance $\cdot$ guaranteed highest performance (blue: normally/adversarially trained smooth classifier is more robust than the base model)..}
	\label{tab:segment_smooth_confidence}
	%\begin{tiny}
	\resizebox{\textwidth}{!}{
		\begin{tabular}{llccccccc}
			\toprule
			& \textbf{Att. Rad.} & 0.0 &   0.1 &  0.2 &  0.5 &  1.0 &  2.0 \\
			\midrule
			%& \multicolumn{6}{c}{\textbf{Smoothed models}} \\
               \multirow{4}{1.2cm}{Smoothed models} & 
   \textbf{PostNet} &  
   $94.0\cdot\bm{{\color{black}99.1}}\cdot99.8$ & 
   $63.5\cdot\bm{{\color{blue}84.7}}\cdot100.0$ & 
   $33.2\cdot\bm{{\color{blue}56.1}}\cdot100.0$ & 
   $10.2\cdot\bm{{\color{blue}16.9}}\cdot100.0$ &  
   $\phantom{0}5.2\cdot\bm{{\color{blue}10.3}}\cdot100.0$ &  
   $\phantom{0}0.3\cdot\phantom{0}\bm{{\color{black}0.3}}\cdot\phantom{00}0.3$ \\
& \textbf{PriorNet} &  
$97.0\cdot\bm{{\color{black}99.8}}\cdot99.9$ & 
$75.6\cdot\bm{{\color{black}90.8}}\cdot100.0$ &
$31.1\cdot\bm{{\color{blue}50.8}}\cdot100.0$ & 
$\phantom{0}2.6\cdot\phantom{0}\bm{{\color{blue}4.7}}\cdot100.0$ &  
$\phantom{0}0.0\cdot\phantom{0}\bm{{\color{black}0.0}}\cdot100.0$ &      
$\phantom{0}0.0\cdot\phantom{0}\bm{0.0}\cdot100.0$ \\
 &   \textbf{DDNet} & 
 $96.2\cdot\bm{{\color{black}99.5}}\cdot99.7$ & 
 $75.7\cdot\bm{{\color{black}89.8}}\cdot\phantom{0}99.9$ &
 $28.5\cdot\bm{{\color{black}51.6}}\cdot100.0$ & 
 $3.7\cdot\phantom{0}\bm{{\color{blue}8.2}}\cdot100.0$ &        
 $\phantom{0}0.0\cdot\phantom{0}\bm{0.0}\cdot100.0$ &              
 $\phantom{0}0.0\cdot\phantom{0}\bm{0.0}\cdot100.0$ \\
  &  \textbf{EvNet} &  
  $95.8\cdot\bm{{\color{black}99.6}}\cdot99.9$ & 
  $80.2\cdot\bm{{\color{black}93.7}}\cdot100.0$ &
  $35.2\cdot\bm{{\color{blue}57.2}}\cdot100.0$ &
  $\phantom{0}6.8\cdot\bm{{\color{blue}12.0}}\cdot100.0$ &
  $\phantom{0}1.2\cdot\phantom{0}\bm{{\color{blue}2.1}}\cdot100.0$ & 
  $\phantom{0}1.1\cdot\phantom{0}\bm{{\color{blue}2.0}}\cdot100.0$ \\

			\midrule
			%& \multicolumn{6}{c}{\textbf{Smoothed models + adversarial training using label attacks}} \\
              \multirow{4}{1.2cm}{Smoothed + adv. w. label attacks} &  
  \textbf{PostNet} &  - &  
  $66.0\cdot\bm{{\color{blue}85.5}}\cdot100.0$ & 
  $22.5\cdot\bm{{\color{blue}41.0}}\cdot100.0$ &  
  $\phantom{0}9.0\cdot\bm{{\color{blue}16.3}}\cdot100.0$ &   
  $\phantom{0}5.2\cdot\phantom{0}\bm{{\color{blue}9.7}}\cdot100.0$ & 
  $\phantom{0}0.6\cdot\phantom{0}\bm{{\color{black}0.6}}\cdot\phantom{00}0.6$ \\
& \textbf{PriorNet} &  - & 
$79.0\cdot\bm{{\color{black}92.4}}\cdot100.0$ & 
$45.2\cdot\bm{{\color{blue}68.8}}\cdot100.0$ &   
$\phantom{0}9.2\cdot\bm{{\color{blue}13.9}}\cdot100.0$ &  
$\phantom{0}0.0\cdot\phantom{0}\bm{{\color{black}0.0}}\cdot100.0$ &         
$\phantom{0}0.0\cdot\phantom{0}\bm{0.0}\cdot100.0$ \\
 &   \textbf{DDNet} &  - &  
 $76.2\cdot\bm{{\color{black}91.0}}\cdot\phantom{0}99.6$ &  
 $27.2\cdot\bm{{\color{black}45.3}}\cdot100.0$ &  
 $\phantom{0}2.3\cdot\phantom{0}\bm{4.3}\cdot100.0$ &                  
 $\phantom{0}0.0\cdot\phantom{0}\bm{0.0}\cdot100.0$ &               
 $\phantom{0}0.0\cdot\phantom{0}\bm{0.0}\cdot100.0$ \\
  &  \textbf{EvNet} &  - &
  $82.7\cdot\bm{{\color{black}95.2}}\cdot100.0$ &   
  $34.0\cdot\bm{{\color{blue}53.8}}\cdot100.0$ &  
  $10.9\cdot\bm{{\color{blue}23.2}}\cdot100.0$ &
  $\phantom{0}0.5\cdot\phantom{0}\bm{{\color{blue}4.2}}\cdot100.0$ & 
  $\phantom{0}2.1\cdot\phantom{0}\bm{{\color{blue}5.1}}\cdot100.0$ \\

			\midrule
			%& \multicolumn{6}{c}{\textbf{Smoothed models + adversarial training using uncertainty attacks}} \\
                \multirow{4}{1.2cm}{Smoothed + adv. w. uncert. attacks} &
    \textbf{PostNet} &  - &    
    $71.5\cdot\bm{{\color{blue}87.6}}\cdot100.0$ &  
    $33.5\cdot\bm{{\color{blue}54.5}}\cdot100.0$ &  
    $12.8\cdot\bm{{\color{blue}25.6}}\cdot100.0$ & 
    $\phantom{0}6.5\cdot\bm{{\color{blue}10.3}}\cdot\phantom{0}87.2$ & 
    $\phantom{0}0.0\cdot\phantom{0}\bm{{\color{black}0.0}}\cdot100.0$ \\
 & \textbf{PriorNet} &  - & 
 $82.1\cdot\bm{{\color{black}96.5}}\cdot100.0$ & 
 $44.1\cdot\bm{{\color{blue}65.4}}\cdot100.0$ & 
 $\phantom{0}9.0\cdot\bm{{\color{blue}15.7}}\cdot100.0$ & 
 $\phantom{0}0.0\cdot\phantom{0}\bm{{\color{black}0.0}}\cdot100.0$ & 
 $\phantom{0}0.0\cdot\phantom{0}\bm{0.0}\cdot100.0$ \\
   & \textbf{DDNet} &  - &   
   $77.4\cdot\bm{{\color{black}91.4}}\cdot\phantom{0}99.9$ &
   $29.4\cdot\bm{{\color{black}50.3}}\cdot100.0$ & 
   $\phantom{0}4.0\cdot\phantom{0}\bm{{\color{blue}6.5}}\cdot100.0$ &    
   $\phantom{0}0.0\cdot\phantom{0}\bm{0.0}\cdot100.0$ &          
   $\phantom{0}0.0\cdot\phantom{0}\bm{0.0}\cdot100.0$ \\
&    \textbf{EvNet} &  - & 
$76.2\cdot\bm{{\color{black}90.7}}\cdot100.0$ &   
$35.7\cdot\bm{{\color{blue}55.4}}\cdot100.0$ &  
$\phantom{0}4.2\cdot\phantom{0}\bm{{\color{blue}6.4}}\cdot100.0$ &    
$\phantom{0}0.8\cdot\phantom{0}\bm{{\color{blue}1.4}}\cdot100.0$ &
$\phantom{0}0.0\cdot\phantom{0}\bm{{\color{black}0.0}}\cdot100.0$ \\

			\bottomrule
		\end{tabular}}
	%\end{tiny}
\end{table*}






\begin{table*}[ht!]
	\centering
	\caption{Distinguishing between correctly and wrongly labeled inputs based on differential entropy under FGSM label attacks. Smoothed DBU models on CIFAR10. Column format: guaranteed lowest performance $\cdot$ empirical performance $\cdot$ guaranteed highest performance (blue: normally/adversarially trained smooth classifier is more robust than the base model).}
	\label{tab:cifar10_smooth_confidence_fgsm}
	%\begin{tiny}
	\resizebox{\textwidth}{!}{
		\begin{tabular}{llccccccc}
			\toprule
			& \textbf{Att. Rad.} & 0.0 &   0.1 &  0.2 &  0.5 &  1.0 &  2.0 \\
			\midrule
			%& \multicolumn{6}{c}{\textbf{Smoothed models}} \\
              \multirow{4}{1.2cm}{Smoothed models} &  
  \textbf{PostNet} & 
  $80.5\cdot\bm{{\color{black}91.4}}\cdot94.4$ &  
  $52.3\cdot\bm{{\color{black}73.2}}\cdot95.4$ &
  $35.8\cdot\bm{{\color{black}57.2}}\cdot97.5$ & 
  $17.0\cdot\bm{{\color{black}29.0}}\cdot100.0$ &  
  $10.2\cdot\bm{{\color{blue}18.7}}\cdot100.0$ &  
  $\phantom{0}8.1\cdot\bm{{\color{black}14.7}}\cdot100.0$ \\
& \textbf{PriorNet} & 
$81.9\cdot\bm{{\color{black}87.7}}\cdot88.8$ &  
$69.6\cdot\bm{{\color{blue}78.4}}\cdot90.3$ &   
$53.3\cdot\bm{{\color{blue}70.5}}\cdot91.7$ &  
$42.1\cdot\bm{{\color{blue}62.6}}\cdot\phantom{0}97.2$ &  
$37.5\cdot\bm{{\color{blue}55.7}}\cdot100.0$ &  
$36.0\cdot\bm{{\color{blue}59.5}}\cdot100.0$ \\
 &   \textbf{DDNet} & 
 $65.9\cdot\bm{{\color{black}84.1}}\cdot85.6$ &  
 $55.3\cdot\bm{{\color{black}69.6}}\cdot87.0$ &  
 $38.6\cdot\bm{{\color{black}55.8}}\cdot87.2$ &  
 $16.3\cdot\bm{{\color{black}28.5}}\cdot\phantom{0}94.6$ &  
 $\phantom{0}6.4\cdot\bm{{\color{black}12.0}}\cdot\phantom{0}99.9$ &     
 $\phantom{0}3.6\cdot\phantom{0}\bm{{\color{blue}7.2}}\cdot100.0$ \\
  &  \textbf{EvNet} & 
  $76.3\cdot\bm{{\color{black}90.4}}\cdot91.7$ & 
  $54.1\cdot\bm{{\color{black}74.5}}\cdot95.5$ & 
  $35.5\cdot\bm{{\color{black}54.7}}\cdot95.1$ &
  $14.6\cdot\bm{{\color{black}29.3}}\cdot\phantom{0}95.6$ & 
  $\phantom{0}8.6\cdot\bm{{\color{black}16.1}}\cdot100.0$ & 
  $\phantom{0}7.2\cdot\bm{{\color{black}13.0}}\cdot100.0$ \\

			\midrule
			%& \multicolumn{6}{c}{\textbf{Smoothed models + adversarial training using label attacks}} \\
              \multirow{4}{1.2cm}{Smoothed + adv. label attacks} &  
  \textbf{PostNet} &  - &  
  $52.3\cdot\bm{{\color{black}71.6}}\cdot95.1$ &  
  $34.7\cdot\bm{{\color{black}54.8}}\cdot96.6$ &      
  $18.9\cdot\bm{{\color{blue}32.1}}\cdot\phantom{0}99.4$ &   
  $10.9\cdot\bm{{\color{blue}19.2}}\cdot100.0$ &   
  $\phantom{0}8.5\cdot\bm{{\color{blue}16.2}}\cdot100.0$ \\
 & \textbf{PriorNet} &  - &  
 $58.1\cdot\bm{{\color{black}69.6}}\cdot87.6$ &     
 $47.1\cdot\bm{{\color{blue}65.7}}\cdot90.3$ &      
 $40.2\cdot\bm{{\color{blue}59.5}}\cdot\phantom{0}99.3$ &  
 $36.2\cdot\bm{{\color{blue}59.5}}\cdot100.0$ &   
 $25.1\cdot\bm{{\color{blue}42.1}}\cdot\phantom{0}97.7$ \\
   & \textbf{DDNet} &  - &  
   $57.1\cdot\bm{{\color{black}75.2}}\cdot91.0$ &  
   $49.3\cdot\bm{{\color{black}65.3}}\cdot90.5$ &   
   $18.4\cdot\bm{{\color{black}33.6}}\cdot\phantom{0}98.5$ &  
   $\phantom{0}7.6\cdot\bm{{\color{black}13.5}}\cdot\phantom{0}99.9$ &    
   $\phantom{0}3.3\cdot\phantom{0}\bm{{\color{blue}9.6}}\cdot100.0$ \\
&    \textbf{EvNet} &  - &  
$24.1\cdot\bm{{\color{black}36.5}}\cdot54.2$ &  
$37.1\cdot\bm{{\color{black}56.7}}\cdot96.7$ &  
$16.2\cdot\bm{{\color{black}29.9}}\cdot100.0$ &   
$11.4\cdot\bm{{\color{blue}21.8}}\cdot100.0$ & 
$13.0\cdot\bm{{\color{blue}26.1}}\cdot100.0$ \\
			\midrule
			%& \multicolumn{6}{c}{\textbf{Smoothed models + adversarial training using uncertainty attacks}} \\
              \multirow{4}{1.2cm}{Smoothed + adv. w. uncert. attacks} &  
  \textbf{PostNet} &  - &  
  $52.0\cdot\bm{{\color{black}71.8}}\cdot94.5$ & 
  $35.8\cdot\bm{{\color{black}54.6}}\cdot89.9$ &   
  $18.4\cdot\bm{{\color{blue}33.6}}\cdot\phantom{0}99.8$ &   
  $10.2\cdot\bm{{\color{blue}19.1}}\cdot100.0$ & 
  $12.2\cdot\bm{{\color{blue}23.0}}\cdot100.0$ \\
 & \textbf{PriorNet} &  - & 
 $50.6\cdot\bm{{\color{black}67.3}}\cdot88.5$ &  
 $46.2\cdot\bm{{\color{blue}64.3}}\cdot95.1$ &   
 $39.9\cdot\bm{{\color{blue}60.8}}\cdot\phantom{0}98.5$ & 
 $27.7\cdot\bm{{\color{blue}46.2}}\cdot100.0$ &
 $28.5\cdot\bm{{\color{blue}48.6}}\cdot100.0$ \\
   & \textbf{DDNet} &  - & 
   $67.7\cdot\bm{{\color{black}82.2}}\cdot92.4$ &
   $45.7\cdot\bm{{\color{black}64.7}}\cdot88.8$ &  
   $20.5\cdot\bm{{\color{black}34.8}}\cdot\phantom{0}93.6$ &  
   $\phantom{0}6.1\cdot\bm{{\color{black}13.1}}\cdot\phantom{0}91.8$ &   
   $\phantom{0}4.1\cdot\phantom{0}\bm{{\color{blue}8.4}}\cdot100.0$ \\
&    \textbf{EvNet} &  - &  
$53.9\cdot\bm{{\color{black}73.6}}\cdot96.3$ & 
$34.2\cdot\bm{{\color{black}55.3}}\cdot99.7$ & 
$16.1\cdot\bm{{\color{black}31.2}}\cdot100.0$ &  
$\phantom{0}6.1\cdot\bm{{\color{black}13.5}}\cdot\phantom{0}86.1$ &  
$18.1\cdot\bm{{\color{blue}34.0}}\cdot100.0$ \\

			\bottomrule
		\end{tabular}}
	%\end{tiny}
\end{table*}

\begin{table*}[ht!]
	\centering
	\caption{Distinguishing between correctly and wrongly labeled inputs based on differential entropy under FGSM label attacks. Smoothed DBU models on MNIST. Column format: guaranteed lowest performance $\cdot$ empirical performance $\cdot$ guaranteed highest performance (blue: normally/adversarially trained smooth classifier is more robust than the base model).}
	\label{tab:mnist_smooth_confidence_fgsm}
	%\begin{tiny}
	\resizebox{\textwidth}{!}{
		\begin{tabular}{llccccccc}
			\toprule
			& \textbf{Att. Rad.} & 0.0 &   0.1 &  0.2 &  0.5 &  1.0 &  2.0 \\
			\midrule
			%& \multicolumn{6}{c}{\textbf{Smoothed models}} \\
              \multirow{4}{1.2cm}{Smoothed models} & 
  \textbf{PostNet} & 
  $97.2\cdot\bm{{\color{black}99.3}}\cdot99.9$ &
  $96.1\cdot\bm{{\color{black}99.2}}\cdot99.9$ & 
  $95.2\cdot\bm{{\color{black}98.9}}\cdot99.9$ & 
  $91.7\cdot\bm{{\color{black}98.0}}\cdot\phantom{0}99.9$ &  
  $86.1\cdot\bm{{\color{black}95.9}}\cdot100.0$ & 
  $75.7\cdot\bm{{\color{black}91.1}}\cdot100.0$ \\
&  \textbf{PriorNet} & 
$96.8\cdot\bm{{\color{black}99.2}}\cdot99.3$ & 
$95.5\cdot\bm{{\color{black}99.0}}\cdot99.6$ & 
$94.7\cdot\bm{{\color{black}98.7}}\cdot99.6$ & 
$91.3\cdot\bm{{\color{black}97.6}}\cdot\phantom{0}99.9$ &  
$85.5\cdot\bm{{\color{blue}95.6}}\cdot100.0$ & 
$78.7\cdot\bm{{\color{blue}92.4}}\cdot100.0$ \\
  &  \textbf{DDNet} & 
  $97.6\cdot\bm{{\color{black}99.3}}\cdot99.4$ & 
  $96.8\cdot\bm{{\color{black}99.2}}\cdot99.5$ & 
  $95.6\cdot\bm{{\color{black}98.7}}\cdot99.4$ & 
  $91.7\cdot\bm{{\color{black}97.7}}\cdot\phantom{0}99.9$ &  
  $83.4\cdot\bm{{\color{black}95.2}}\cdot100.0$ &
  $58.3\cdot\bm{{\color{black}79.6}}\cdot100.0$ \\
   & \textbf{EvNet} & 
   $97.3\cdot\bm{{\color{black}99.3}}\cdot99.4$ &  
   $95.5\cdot\bm{{\color{black}99.0}}\cdot99.6$ & 
   $94.3\cdot\bm{{\color{black}98.9}}\cdot99.9$ & 
   $92.0\cdot\bm{{\color{black}97.7}}\cdot100.0$ &
   $87.4\cdot\bm{{\color{blue}96.3}}\cdot100.0$ &   
   $78.8\cdot\bm{{\color{blue}92.4}}\cdot100.0$ \\

			\midrule
			%& \multicolumn{6}{c}{\textbf{Smoothed models + adversarial training using label attacks}} \\
               \multirow{4}{1.2cm}{Smoothed + adv. label attacks} & 
   \textbf{PostNet} &  - &  
   $95.1\cdot\bm{{\color{black}98.9}}\cdot99.8$ & 
   $91.2\cdot\bm{{\color{black}97.2}}\cdot99.6$ &  
   $87.6\cdot\bm{{\color{black}96.3}}\cdot\phantom{0}99.9$ &  
   $81.0\cdot\bm{{\color{black}93.3}}\cdot100.0$ &
   $69.9\cdot\bm{{\color{black}87.2}}\cdot100.0$ \\
 & \textbf{PriorNet} &  - &  
 $94.4\cdot\bm{{\color{black}98.7}}\cdot99.7$ & 
 $93.6\cdot\bm{{\color{black}98.2}}\cdot99.3$ & 
 $89.4\cdot\bm{{\color{black}96.3}}\cdot\phantom{0}99.8$ &  
 $84.5\cdot\bm{{\color{blue}95.1}}\cdot100.0$ & 
 $81.7\cdot\bm{{\color{blue}92.5}}\cdot100.0$ \\
   & \textbf{DDNet} &  - & 
   $95.5\cdot\bm{{\color{black}98.6}}\cdot99.0$ & 
   $94.6\cdot\bm{{\color{black}98.7}}\cdot99.4$ &  
   $89.7\cdot\bm{{\color{black}97.1}}\cdot\phantom{0}99.8$ & 
   $80.0\cdot\bm{{\color{black}93.6}}\cdot100.0$ &  
   $54.4\cdot\bm{{\color{black}74.5}}\cdot100.0$ \\
&    \textbf{EvNet} &  - &  
$88.9\cdot\bm{{\color{black}94.8}}\cdot98.1$ &  
$91.5\cdot\bm{{\color{black}98.4}}\cdot99.8$ &  
$89.2\cdot\bm{{\color{black}97.0}}\cdot100.0$ & 
$83.6\cdot\bm{{\color{black}94.7}}\cdot100.0$ & 
$72.3\cdot\bm{{\color{black}88.0}}\cdot100.0$ \\

			\midrule
			%& \multicolumn{6}{c}{\textbf{Smoothed models + adversarial training using uncertainty attacks}} \\
               \multirow{4}{1.2cm}{Smoothed + adv. uncert. attacks} & 
   \textbf{PostNet} &  - & 
   $92.8\cdot\bm{{\color{black}98.5}}\cdot99.9$ &
   $92.8\cdot\bm{{\color{black}98.7}}\cdot99.9$ & 
   $89.0\cdot\bm{{\color{black}96.3}}\cdot\phantom{0}99.8$ & 
   $80.8\cdot\bm{{\color{black}93.4}}\cdot100.0$ &
   $71.6\cdot\bm{{\color{black}86.9}}\cdot100.0$ \\
 & \textbf{PriorNet} &  - &
 $95.1\cdot\bm{{\color{black}98.1}}\cdot98.9$ &
 $94.3\cdot\bm{{\color{black}97.7}}\cdot99.1$ & 
 $88.5\cdot\bm{{\color{black}96.8}}\cdot\phantom{0}99.9$ &  
 $83.4\cdot\bm{{\color{blue}94.5}}\cdot100.0$ &   
 $78.9\cdot\bm{{\color{blue}92.2}}\cdot100.0$ \\
   & \textbf{DDNet} &  - &  
   $96.0\cdot\bm{{\color{black}98.7}}\cdot99.0$ & 
   $95.5\cdot\bm{{\color{black}98.6}}\cdot99.3$ &  
   $89.5\cdot\bm{{\color{black}95.6}}\cdot\phantom{0}99.7$ & 
   $79.6\cdot\bm{{\color{black}93.1}}\cdot100.0$ & 
   $55.9\cdot\bm{{\color{black}77.1}}\cdot100.0$ \\
&    \textbf{EvNet} &  - &  
$93.3\cdot\bm{{\color{black}98.9}}\cdot99.4$ & 
$90.1\cdot\bm{{\color{black}97.9}}\cdot99.4$ & 
$87.9\cdot\bm{{\color{black}96.3}}\cdot100.0$ & 
$84.1\cdot\bm{{\color{black}94.2}}\cdot100.0$ & 
$69.2\cdot\bm{{\color{black}86.9}}\cdot100.0$ \\

			\bottomrule
		\end{tabular}}
	%\end{tiny}
\end{table*}


\begin{table*}[ht!]
	\centering
	\caption{Distinguishing between correctly and wrongly labeled inputs based on differential entropy under FGSM label attacks. Smoothed DBU models on Sensorless. Column format: guaranteed lowest performance $\cdot$ empirical performance $\cdot$ guaranteed highest performance (blue: normally/adversarially trained smooth classifier is more robust than the base model).}
	\label{tab:sensorless_smooth_confidence_fgsm}
	%\begin{tiny}
	\resizebox{\textwidth}{!}{
		\begin{tabular}{llccccccc}
			\toprule
			& \textbf{Att. Rad.} & 0.0 &   0.1 &  0.2 &  0.5 &  1.0 &  2.0 \\
			\midrule
			%& \multicolumn{6}{c}{\textbf{Smoothed models}} \\
               \multirow{4}{1.2cm}{Smoothed models} &
   \textbf{PostNet} & 
   $94.5\cdot\bm{{\color{black}98.1}}\cdot100.0$ &  
   $10.3\cdot\bm{{\color{blue}19.6}}\cdot100.0$ &   
   $\phantom{0}5.1\cdot\bm{{\color{blue}11.0}}\cdot100.0$ &   
   $\phantom{0}6.4\cdot\phantom{0}\bm{{\color{black}6.4}}\cdot\phantom{00}6.4$ &  
   $10.4\cdot\bm{{\color{black}10.4}}\cdot\phantom{0}10.4$ & 
   $11.4\cdot\bm{{\color{black}11.4}}\cdot\phantom{0}11.4$ \\
& \textbf{PriorNet} & 
$97.1\cdot\bm{{\color{black}99.5}}\cdot100.0$ &  
$13.6\cdot\bm{{\color{blue}27.3}}\cdot100.0$ & 
$\phantom{0}6.3\cdot\bm{{\color{blue}12.4}}\cdot100.0$ &  
$\phantom{0}2.6\cdot\phantom{0}\bm{{\color{black}6.8}}\cdot100.0$ &  
$\phantom{0}3.1\cdot\phantom{0}\bm{{\color{black}7.3}}\cdot100.0$ & 
$\phantom{0}3.2\cdot\phantom{0}\bm{{\color{black}6.7}}\cdot100.0$ \\
 &   \textbf{DDNet} &   
 $95.9\cdot\bm{{\color{black}99.4}}\cdot\phantom{0}99.8$ & 
 $\phantom{0}8.6\cdot\bm{{\color{black}14.9}}\cdot100.0$ & 
 $\phantom{0}1.6\cdot\phantom{0}\bm{{\color{black}3.8}}\cdot100.0$ &  
 $\phantom{0}2.6\cdot\phantom{0}\bm{{\color{blue}4.5}}\cdot100.0$ &   
 $\phantom{0}3.4\cdot\phantom{0}\bm{{\color{blue}6.9}}\cdot100.0$ &  
 $\phantom{0}3.3\cdot\phantom{0}\bm{{\color{blue}6.4}}\cdot100.0$ \\
  &  \textbf{EvNet} & 
  $94.0\cdot\bm{{\color{black}98.5}}\cdot\phantom{0}99.7$ & 
  $26.0\cdot\bm{{\color{black}43.2}}\cdot100.0$ &
  $15.8\cdot\bm{{\color{blue}30.8}}\cdot100.0$ & 
  $11.7\cdot\bm{{\color{blue}20.2}}\cdot100.0$ & 
  $\phantom{0}8.1\cdot\bm{{\color{black}15.0}}\cdot100.0$ &
  $\phantom{0}7.6\cdot\bm{{\color{black}12.7}}\cdot100.0$ \\

			\midrule
			%& \multicolumn{6}{c}{\textbf{Smoothed models + adversarial training using label attacks}} \\
              \multirow{4}{1.2cm}{Smoothed + adv. label attacks} &  
  \textbf{PostNet} &  - &    
  $13.1\cdot\bm{{\color{blue}24.3}}\cdot100.0$ &  
  $\phantom{0}5.7\cdot\bm{{\color{blue}11.9}}\cdot100.0$ &  
  $\phantom{0}9.4\cdot\phantom{0}\bm{{\color{blue}9.4}}\cdot\phantom{0}9.4$ &  
  $11.2\cdot\bm{{\color{black}11.2}}\cdot\phantom{0}11.2$ & 
  $11.8\cdot\bm{{\color{black}11.8}}\cdot\phantom{0}11.8$ \\
 & \textbf{PriorNet} &  - &    
 $22.4\cdot\bm{{\color{blue}38.2}}\cdot100.0$ &
 $11.8\cdot\bm{{\color{blue}22.1}}\cdot100.0$ &  
 $\phantom{0}0.2\cdot\phantom{0}\bm{{\color{black}0.6}}\cdot100.0$ & 
 $\phantom{0}0.0\cdot\phantom{0}\bm{{\color{black}0.0}}\cdot100.0$ &  
 $\phantom{0}0.1\cdot\phantom{0}\bm{{\color{black}0.1}}\cdot100.0$ \\
   & \textbf{DDNet} &  - &   
   $\phantom{0}7.3\cdot\bm{{\color{black}13.2}}\cdot100.0$ &  
   $\phantom{0}8.5\cdot\bm{{\color{blue}17.2}}\cdot100.0$ &   
   $\phantom{0}3.6\cdot\phantom{0}\bm{{\color{blue}7.9}}\cdot100.0$ &    
   $\phantom{0}3.8\cdot\phantom{0}\bm{{\color{blue}7.6}}\cdot100.0$ &  
   $\phantom{0}0.8\cdot\phantom{0}\bm{{\color{black}1.2}}\cdot100.0$ \\
&    \textbf{EvNet} &  - & 
$25.5\cdot\bm{{\color{black}42.0}}\cdot100.0$ & 
$15.6\cdot\bm{{\color{blue}30.2}}\cdot100.0$ &  
$10.4\cdot\bm{{\color{blue}19.5}}\cdot100.0$ & 
$\phantom{0}8.6\cdot\bm{{\color{blue}16.4}}\cdot100.0$ &
$\phantom{0}7.8\cdot\bm{{\color{black}14.7}}\cdot100.0$ \\

			\midrule
			%& \multicolumn{6}{c}{\textbf{Smoothed models + adversarial training using uncertainty attacks}} \\
              \multirow{4}{1.2cm}{Smoothed + adv. w. uncert. attacks} & 
  \textbf{PostNet} &  - &   
  $10.6\cdot\bm{{\color{blue}20.3}}\cdot100.0$ &   
  $\phantom{0}5.2\cdot\phantom{0}\bm{{\color{blue}9.9}}\cdot100.0$ &   
  $10.9\cdot\bm{{\color{blue}10.9}}\cdot10.9$ & 
  $11.6\cdot\bm{{\color{black}11.6}}\cdot\phantom{0}11.6$ & 
  $11.7\cdot\bm{{\color{black}11.7}}\cdot\phantom{0}11.7$ \\
& \textbf{PriorNet} &  - & 
$25.7\cdot\bm{{\color{blue}45.0}}\cdot100.0$ & 
$12.0\cdot\bm{{\color{blue}20.5}}\cdot100.0$ &
$\phantom{0}1.1\cdot\phantom{0}\bm{{\color{black}3.7}}\cdot100.0$ &  
$\phantom{0}0.0\cdot\phantom{0}\bm{{\color{black}0.0}}\cdot100.0$ & 
$\phantom{0}0.0\cdot\phantom{0}\bm{{\color{black}0.0}}\cdot100.0$ \\
 &   \textbf{DDNet} &  - & 
 $\phantom{0}7.9\cdot\bm{{\color{black}16.4}}\cdot100.0$ &
 $\phantom{0}1.2\cdot\phantom{0}\bm{{\color{black}3.8}}\cdot100.0$ & 
 $\phantom{0}3.4\cdot\phantom{0}\bm{{\color{blue}6.3}}\cdot100.0$ &  
 $\phantom{0}3.9\cdot\phantom{0}\bm{{\color{blue}7.9}}\cdot100.0$ & 
 $\phantom{0}3.3\cdot\phantom{0}\bm{{\color{blue}8.0}}\cdot100.0$ \\
  &  \textbf{EvNet} &  - & 
  $27.9\cdot\bm{{\color{blue}49.2}}\cdot100.0$ &  
  $18.4\cdot\bm{{\color{blue}32.9}}\cdot100.0$ & 
  $16.4\cdot\bm{{\color{blue}29.3}}\cdot100.0$ &
  $\phantom{0}5.9\cdot\bm{{\color{black}10.8}}\cdot100.0$ &  
  $\phantom{0}8.5\cdot\bm{{\color{blue}16.1}}\cdot100.0$ \\

			\bottomrule
		\end{tabular}}
	%\end{tiny}
\end{table*}


\begin{table*}[ht!]
	\centering
	\caption{Distinguishing between correctly and wrongly labeled inputs based on differential entropy under FGSM label attacks. Smoothed DBU models on Segment. Column format: guaranteed lowest performance $\cdot$ empirical performance $\cdot$ guaranteed highest performance (blue: normally/adversarially trained smooth classifier is more robust than the base model).}
	\label{tab:segments_smooth_confidence_fgsm}
	%\begin{tiny}
	\resizebox{\textwidth}{!}{
		\begin{tabular}{llccccccc}
			\toprule
			& \textbf{Att. Rad.} & 0.0 &   0.1 &  0.2 &  0.5 &  1.0 &  2.0 \\
			\midrule
			%& \multicolumn{6}{c}{\textbf{Smoothed models}} \\
                \multirow{4}{1.2cm}{Smoothed models} &
    \textbf{PostNet} & 
    $94.0\cdot\bm{{\color{black}99.2}}\cdot99.8$ &    
    $55.2\cdot\bm{{\color{blue}78.3}}\cdot100.0$ &   
    $40.1\cdot\bm{{\color{blue}61.4}}\cdot100.0$ &  
    $17.9\cdot\bm{{\color{blue}31.7}}\cdot100.0$ & 
    $\phantom{0}6.8\cdot\bm{{\color{black}12.7}}\cdot100.0$ &
    $17.6\cdot\bm{{\color{black}17.9}}\cdot\phantom{0}18.0$ \\
 & \textbf{PriorNet} & 
 $97.0\cdot\bm{{\color{black}99.8}}\cdot99.9$ &
 $69.2\cdot\bm{{\color{black}89.7}}\cdot100.0$ &  
 $29.7\cdot\bm{{\color{blue}45.5}}\cdot100.0$ & 
 $\phantom{0}1.7\cdot\bm{{\color{black}4.1}}\cdot100.0$ &  
 $\phantom{0}0.0\cdot\phantom{0}\bm{{\color{black}0.0}}\cdot100.0$ & 
 $\phantom{0}0.0\cdot\phantom{0}\bm{{\color{black}0.0}}\cdot100.0$ \\
   & \textbf{DDNet} & 
   $96.2\cdot\bm{{\color{black}99.5}}\cdot99.6$ &
   $70.6\cdot\bm{{\color{black}86.3}}\cdot\phantom{0}99.8$ & 
   $22.3\cdot\bm{{\color{black}38.8}}\cdot100.0$ &
   $\phantom{0}6.3\cdot\bm{{\color{blue}13.3}}\cdot100.0$ &    
   $\phantom{0}1.1\cdot\phantom{0}\bm{{\color{blue}3.0}}\cdot100.0$ &      
   $\phantom{0}0.0\cdot\phantom{0}\bm{0.0}\cdot100.0$ \\
&    \textbf{EvNet} & 
$95.8\cdot\bm{{\color{black}99.1}}\cdot99.8$ &
$78.4\cdot\bm{{\color{black}92.5}}\cdot100.0$ &  
$40.7\cdot\bm{{\color{blue}62.1}}\cdot100.0$ &  
$\phantom{0}9.8\cdot\bm{{\color{blue}17.6}}\cdot100.0$ &   
$\phantom{0}0.0\cdot\phantom{0}\bm{{\color{black}0.0}}\cdot100.0$ & 
$\phantom{0}0.0\cdot\phantom{0}\bm{{\color{black}0.0}}\cdot100.0$ \\

			\midrule
			%& \multicolumn{6}{c}{\textbf{Smoothed models + adversarial training using label attacks}} \\
                \multirow{4}{1.2cm}{Smoothed + adv. label attacks} & 
    \textbf{PostNet} &  - & 
    $66.0\cdot\bm{{\color{blue}83.5}}\cdot100.0$ &   
    $28.8\cdot\bm{{\color{blue}44.9}}\cdot100.0$ & 
    $12.3\cdot\bm{{\color{blue}24.3}}\cdot100.0$ &  
    $\phantom{0}9.3\cdot\bm{{\color{blue}17.3}}\cdot100.0$ &    
    $24.8\cdot\bm{{\color{blue}24.8}}\cdot\phantom{0}24.8$ \\
& \textbf{PriorNet} &  - & 
$75.1\cdot\bm{{\color{black}91.5}}\cdot\phantom{0}99.9$ &   
$34.0\cdot\bm{{\color{blue}60.3}}\cdot100.0$ &
$11.1\cdot\bm{{\color{blue}24.6}}\cdot100.0$ & 
$\phantom{0}0.0\cdot\phantom{0}\bm{{\color{black}0.0}}\cdot100.0$ &  
$\phantom{0}0.0\cdot\phantom{0}\bm{{\color{black}0.0}}\cdot100.0$ \\
 &   \textbf{DDNet} &  - &  
 $65.4\cdot\bm{{\color{black}82.8}}\cdot\phantom{0}99.5$ &  
 $23.1\cdot\bm{{\color{black}35.3}}\cdot100.0$ &
 $4.8\cdot\bm{{\color{blue}10.4}}\cdot100.0$ & 
 $\phantom{0}0.0\cdot\phantom{0}\bm{{\color{black}0.0}}\cdot100.0$ &     
 $\phantom{0}0.0\cdot\phantom{0}\bm{0.0}\cdot100.0$ \\
  &  \textbf{EvNet} &  - &    
  $83.4\cdot\bm{{\color{blue}95.3}}\cdot100.0$ &   
  $42.1\cdot\bm{{\color{blue}63.3}}\cdot100.0$ & 
  $15.0\cdot\bm{{\color{blue}33.6}}\cdot100.0$ & 
  $\phantom{0}0.0\cdot\phantom{0}\bm{{\color{black}0.0}}\cdot100.0$ &
  $\phantom{0}0.0\cdot\phantom{0}\bm{{\color{black}0.0}}\cdot100.0$ \\

            \midrule
			%& \multicolumn{6}{c}{\textbf{Smoothed models + adversarial training using uncertainty attacks}} \\
              \multirow{4}{1.2cm}{Smoothed + adv. uncert. attacks} &  
  \textbf{PostNet} &  - &   
  $67.8\cdot\bm{{\color{blue}86.5}}\cdot100.0$ &    
  $34.0\cdot\bm{{\color{blue}52.5}}\cdot100.0$ & 
  $16.2\cdot\bm{{\color{blue}32.8}}\cdot100.0$ &   
  $14.4\cdot\bm{{\color{blue}25.2}}\cdot\phantom{0}92.2$ &   
  $\phantom{0}7.3\cdot\phantom{0}\bm{{\color{black}7.3}}\cdot\phantom{00}7.3$ \\
 & \textbf{PriorNet} &  - & 
 $77.3\cdot\bm{{\color{black}91.2}}\cdot\phantom{0}99.9$ &  
 $39.3\cdot\bm{{\color{blue}62.7}}\cdot100.0$ &  
 $\phantom{0}9.0\cdot\bm{{\color{blue}17.8}}\cdot100.0$ & 
 $\phantom{0}0.0\cdot\phantom{0}\bm{{\color{black}0.0}}\cdot100.0$ & 
 $\phantom{0}0.0\cdot\phantom{0}\bm{{\color{black}0.0}}\cdot100.0$ \\
   & \textbf{DDNet} &  - & 
   $68.8\cdot\bm{{\color{black}88.3}}\cdot\phantom{0}99.9$ &  
   $20.4\cdot\bm{{\color{black}35.2}}\cdot100.0$ &  
   $\phantom{0}7.5\cdot\bm{{\color{blue}12.6}}\cdot100.0$ & 
   $\phantom{0}0.3\cdot\phantom{0}\bm{{\color{black}0.9}}\cdot100.0$ &          
   $\phantom{0}0.0\cdot\phantom{0}\bm{0.0}\cdot100.0$ \\
&    \textbf{EvNet} &  - & 
$74.0\cdot\bm{{\color{black}92.9}}\cdot100.0$ &   
$44.1\cdot\bm{{\color{blue}61.8}}\cdot100.0$ &   
$\phantom{0}5.3\cdot\bm{{\color{blue}13.0}}\cdot100.0$ &  
$\phantom{0}3.9\cdot\phantom{0}\bm{{\color{blue}8.2}}\cdot100.0$ &   
$\phantom{0}0.5\cdot\phantom{0}\bm{{\color{blue}4.2}}\cdot100.0$ \\

            \bottomrule
		\end{tabular}}
	%\end{tiny}
\end{table*}




%%% Attack detection %%%%

\begin{table*}[ht!]
	\centering
	\caption{Attack detection (PGD label attacks) based on differential entropy. Smoothed DBU models on CIFAR10. Column format: guaranteed lowest performance $\cdot$ empirical performance $\cdot$ guaranteed highest performance (blue: normally/adversarially trained smooth classifier is more robust than the base model).}
	\label{tab:cifar10_smooth_attackdetection}
	%\begin{tiny}
	\resizebox{\textwidth}{!}{
		\begin{tabular}{llccccc}
			\toprule
			& \textbf{Att. Rad.} &   0.1 &  0.2 &  0.5 &  1.0 &  2.0 \\
			\midrule
			%& \multicolumn{6}{c}{\textbf{Smoothed models}} \\
              \multirow{4}{1.2cm}{Smoothed models}  & \textbf{PostNet} &  %$41.3\cdot\bm{{\color{blue}50.1}}\cdot61.7$ & 
  $33.1\cdot\bm{{\color{black}50.4}}\cdot89.9$ &  
  $31.0\cdot\bm{{\color{black}50.2}}\cdot96.9$ &    
  $30.7\cdot\bm{{\color{blue}50.2}}\cdot100.0$ &    
  $30.7\cdot\bm{{\color{blue}50.0}}\cdot100.0$ &  
  $30.7\cdot\bm{{\color{blue}50.2}}\cdot100.0$ \\
 & \textbf{PriorNet} &  
 %$47.4\cdot\bm{{\color{blue}50.1}}\cdot53.0$ & 
 $35.9\cdot\bm{{\color{black}50.6}}\cdot74.5$ &  
 $33.0\cdot\bm{{\color{black}50.3}}\cdot82.8$ &  
 $31.2\cdot\bm{{\color{black}50.0}}\cdot\phantom{0}95.7$ &  
 $30.7\cdot\bm{{\color{black}50.4}}\cdot\phantom{0}99.9$ &  
 $30.7\cdot\bm{{\color{blue}50.4}}\cdot100.0$ \\
   & \textbf{DDNet} &  
   %$47.3\cdot\bm{{\color{blue}50.1}}\cdot53.3$ & 
   $36.3\cdot\bm{{\color{black}50.3}}\cdot76.4$ &  
   $32.8\cdot\bm{{\color{black}49.9}}\cdot84.6$ &  
   $30.8\cdot\bm{{\color{black}50.1}}\cdot\phantom{0}98.0$ &    
   $30.7\cdot\bm{{\color{blue}50.2}}\cdot100.0$ &  
   $30.7\cdot\bm{{\color{blue}50.2}}\cdot100.0$ \\
&    \textbf{EvNet} &  
%$46.0\cdot\bm{{\color{blue}50.1}}\cdot55.6$ & 
$32.9\cdot\bm{{\color{black}50.4}}\cdot89.8$ &  
$31.4\cdot\bm{{\color{black}50.1}}\cdot94.0$ &     
$30.8\cdot\bm{{\color{blue}50.0}}\cdot\phantom{0}98.0$ &    
$30.7\cdot\bm{{\color{blue}50.3}}\cdot100.0$ &  
$30.7\cdot\bm{{\color{blue}49.6}}\cdot100.0$ \\

			\midrule
			%& \multicolumn{6}{c}{\textbf{Smoothed models + adversarial training using label attacks}} \\
            \multirow{4}{1.3cm}{Smoothed + adv. label attacks}  & 
\textbf{PostNet} &   
$32.7\cdot\bm{{\color{black}50.1}}\cdot90.4$ &  
$31.1\cdot\bm{{\color{black}50.2}}\cdot96.5$ &     
$30.7\cdot\bm{{\color{blue}50.2}}\cdot\phantom{0}99.7$ &     
$30.7\cdot\bm{{\color{blue}50.3}}\cdot100.0$ &  
$30.7\cdot\bm{{\color{blue}50.2}}\cdot100.0$ \\
 & \textbf{PriorNet} & % - &  
 $35.2\cdot\bm{{\color{black}51.8}}\cdot78.6$ &  
 $32.8\cdot\bm{{\color{black}51.1}}\cdot84.4$ &  
 $30.8\cdot\bm{{\color{black}50.2}}\cdot\phantom{0}98.7$ &  
 $30.7\cdot\bm{{\color{black}50.5}}\cdot100.0$ &   
 $30.8\cdot\bm{{\color{blue}50.1}}\cdot\phantom{0}98.2$ \\
   & \textbf{DDNet} &  %- & 
   $35.5\cdot\bm{{\color{black}50.6}}\cdot79.2$ &  
   $33.4\cdot\bm{{\color{black}50.3}}\cdot84.1$ &  
   $30.8\cdot\bm{{\color{black}50.1}}\cdot\phantom{0}99.2$ &     
   $30.7\cdot\bm{{\color{blue}50.0}}\cdot100.0$ & 
   $30.7\cdot\bm{{\color{blue}50.5}}\cdot100.0$ \\
&    \textbf{EvNet} &  %- & 
$40.3\cdot\bm{{\color{black}50.4}}\cdot66.8$ &  
$31.4\cdot\bm{{\color{black}50.3}}\cdot95.8$ &    
$30.7\cdot\bm{{\color{blue}50.3}}\cdot100.0$ &     
$30.7\cdot\bm{{\color{blue}50.1}}\cdot100.0$ &  
$30.7\cdot\bm{{\color{blue}50.0}}\cdot100.0$ \\
			\midrule
			%& \multicolumn{6}{c}{\textbf{Adversarially training models using uncertainty attacks}} \\
            \multirow{4}{1.3cm}{Smoothed + adv. uncert. attacks} & 
\textbf{PostNet} &  %- &  
$33.3\cdot\bm{{\color{black}50.6}}\cdot88.7$ &  
$32.5\cdot\bm{{\color{black}50.1}}\cdot87.9$ &     
$30.7\cdot\bm{{\color{blue}49.9}}\cdot\phantom{0}99.8$ &     
$30.7\cdot\bm{{\color{blue}50.1}}\cdot100.0$ &  
$30.7\cdot\bm{{\color{blue}50.0}}\cdot100.0$ \\
& \textbf{PriorNet} &  %- & 
$34.5\cdot\bm{{\color{black}51.0}}\cdot80.1$ &  
$31.4\cdot\bm{{\color{black}50.6}}\cdot92.8$ &  
$30.9\cdot\bm{{\color{black}50.0}}\cdot\phantom{0}97.7$ &  
$30.7\cdot\bm{{\color{black}50.1}}\cdot100.0$ &  
$30.7\cdot\bm{{\color{blue}50.0}}\cdot100.0$ \\
 &   \textbf{DDNet} &  %- & 
 $37.4\cdot\bm{{\color{black}50.8}}\cdot74.5$ &  
 $33.4\cdot\bm{{\color{black}50.2}}\cdot83.0$ &  
 $30.9\cdot\bm{{\color{black}50.1}}\cdot\phantom{0}96.8$ &      
 $30.8\cdot\bm{{\color{blue}49.9}}\cdot\phantom{0}98.1$ &  
 $30.7\cdot\bm{{\color{blue}49.9}}\cdot100.0$ \\
  &  \textbf{EvNet} &  %- & 
  $32.8\cdot\bm{{\color{black}50.1}}\cdot92.0$ &  
  $30.8\cdot\bm{{\color{black}50.0}}\cdot99.6$ &    
  $30.7\cdot\bm{{\color{blue}50.1}}\cdot100.0$ &      
  $31.2\cdot\bm{{\color{blue}50.2}}\cdot\phantom{0}96.1$ &  
  $31.0\cdot\bm{{\color{blue}50.0}}\cdot100.0$ \\

			\bottomrule
		\end{tabular}}
	%\end{tiny}
\end{table*}

\begin{table*}[ht!]
	\centering
	\caption{Attack detection (PGD label attacks) based on differential entropy. Smoothed DBU models on MNIST. Column format: guaranteed lowest performance $\cdot$ empirical performance $\cdot$ guaranteed highest performance (blue: normally/adversarially trained smooth classifier is more robust than the base model).}
	\label{tab:mnist_smooth_attackdetection}
	%\begin{tiny}
	\resizebox{\textwidth}{!}{
		\begin{tabular}{llcccccc}
			\toprule
			& \textbf{Att. Rad.} &   0.1 &  0.2 &  0.5 &  1.0 &  2.0 \\
			\midrule
			%& \multicolumn{6}{c}{\textbf{Smoothed models}} \\
              \multirow{4}{1.2cm}{Smoothed models} &  
  \textbf{PostNet} &  
  %$31.0\cdot\bm{50.0}\cdot94.0$ &  
  $30.9\cdot\bm{{\color{black}52.5}}\cdot95.6$ & 
  $31.5\cdot\bm{{\color{black}51.5}}\cdot90.9$ & 
  $31.1\cdot\bm{{\color{black}49.9}}\cdot97.1$ & 
  $30.7\cdot\bm{{\color{black}47.6}}\cdot100.0$ & 
  $30.7\cdot\bm{{\color{black}45.0}}\cdot100.0$ \\
 & \textbf{PriorNet} &  
 %$47.0\cdot\bm{50.0}\cdot53.2$ &  
 $38.2\cdot\bm{{\color{black}57.8}}\cdot80.9$ &  
 $36.0\cdot\bm{{\color{black}57.2}}\cdot84.3$ &  
 $31.6\cdot\bm{{\color{black}63.4}}\cdot98.4$ &
 $30.8\cdot\bm{{\color{black}61.0}}\cdot\phantom{0}99.3$ & 
 $30.7\cdot\bm{{\color{black}66.8}}\cdot100.0$ \\
   & \textbf{DDNet} &  
  % $49.2\cdot\bm{50.0}\cdot50.8$ &  
   $44.6\cdot\bm{{\color{black}51.9}}\cdot60.7$ &
   $39.3\cdot\bm{{\color{black}52.7}}\cdot72.2$ &
   $31.6\cdot\bm{{\color{black}50.9}}\cdot95.2$ & 
   $30.7\cdot\bm{{\color{black}47.3}}\cdot100.0$ &
   $30.7\cdot\bm{{\color{black}45.9}}\cdot100.0$ \\
&    \textbf{EvNet} &  
%$47.8\cdot\bm{50.0}\cdot52.4$ &  
$36.5\cdot\bm{{\color{black}51.8}}\cdot76.1$ & 
$31.5\cdot\bm{{\color{black}51.1}}\cdot93.2$ & 
$30.7\cdot\bm{{\color{black}51.1}}\cdot99.9$ &  
$30.7\cdot\bm{{\color{black}48.7}}\cdot100.0$ &
$30.7\cdot\bm{{\color{black}43.8}}\cdot100.0$ \\

			\midrule
			%& \multicolumn{6}{c}{\textbf{Smoothed models + adversarial training using label attacks}} \\
              \multirow{4}{1.2cm}{Smoothed + adv. label attacks} &  
  \textbf{PostNet} &  %- &  
  $33.6\cdot\bm{{\color{black}52.8}}\cdot82.3$ &  
  $31.4\cdot\bm{{\color{black}51.2}}\cdot91.6$ &  
  $30.9\cdot\bm{{\color{black}49.4}}\cdot\phantom{0}99.1$ &  
  $30.7\cdot\bm{{\color{black}49.3}}\cdot100.0$ & 
  $30.7\cdot\bm{{\color{black}56.0}}\cdot100.0$ \\
& \textbf{PriorNet} &  %- &  
$37.3\cdot\bm{{\color{black}60.5}}\cdot84.3$ & 
$34.3\cdot\bm{{\color{black}59.9}}\cdot87.9$ &  
$32.1\cdot\bm{{\color{black}61.0}}\cdot\phantom{0}97.0$ & 
$30.7\cdot\bm{{\color{black}69.3}}\cdot100.0$ & 
$30.7\cdot\bm{{\color{black}68.0}}\cdot100.0$ \\
 &   \textbf{DDNet} &  %- &  
 $44.8\cdot\bm{{\color{black}52.2}}\cdot61.0$ & 
 $40.2\cdot\bm{{\color{black}52.6}}\cdot70.0$ &  
 $32.5\cdot\bm{{\color{black}52.4}}\cdot\phantom{0}94.6$ &  
 $30.7\cdot\bm{{\color{black}50.3}}\cdot100.0$ & 
 $30.7\cdot\bm{{\color{black}54.6}}\cdot100.0$ \\
  &  \textbf{EvNet} &  %- &  
  $35.8\cdot\bm{{\color{black}51.2}}\cdot76.7$ & 
  $32.9\cdot\bm{{\color{black}51.0}}\cdot88.5$ & 
  $30.7\cdot\bm{{\color{black}49.5}}\cdot100.0$ &
  $30.7\cdot\bm{{\color{black}48.5}}\cdot100.0$ &  
  $30.7\cdot\bm{{\color{black}47.7}}\cdot100.0$ \\

			\midrule
			%& \multicolumn{6}{c}{\textbf{Adversarially training models using uncertainty attacks}} \\
              \multirow{4}{1.2cm}{Smoothed + adv. uncert. attacks} &  
  \textbf{PostNet} &  %- &  
  $31.2\cdot\bm{{\color{black}52.7}}\cdot92.8$ &  
  $31.3\cdot\bm{{\color{black}51.7}}\cdot92.4$ &
  $31.3\cdot\bm{{\color{black}47.3}}\cdot96.8$ &  
  $30.7\cdot\bm{{\color{black}48.9}}\cdot100.0$ &
  $30.7\cdot\bm{{\color{black}46.3}}\cdot100.0$ \\
& \textbf{PriorNet} &  %- &  
$38.3\cdot\bm{{\color{black}58.2}}\cdot81.5$ & 
$36.9\cdot\bm{{\color{black}55.5}}\cdot79.9$ & 
$31.3\cdot\bm{{\color{black}63.5}}\cdot98.9$ &  
$30.7\cdot\bm{{\color{black}68.6}}\cdot100.0$ &
$30.7\cdot\bm{{\color{black}74.6}}\cdot100.0$ \\
 &   \textbf{DDNet} &  %- &  
 $44.9\cdot\bm{{\color{black}52.2}}\cdot60.7$ & 
 $39.6\cdot\bm{{\color{black}53.3}}\cdot72.1$ &  
 $31.8\cdot\bm{{\color{black}51.7}}\cdot95.4$ & 
 $30.7\cdot\bm{{\color{black}46.1}}\cdot100.0$ & 
 $30.7\cdot\bm{{\color{black}46.0}}\cdot100.0$ \\
  &  \textbf{EvNet} &  %- &  
  $38.8\cdot\bm{{\color{black}51.9}}\cdot70.9$ &
  $34.5\cdot\bm{{\color{black}52.3}}\cdot82.9$ & 
  $30.8\cdot\bm{{\color{black}49.9}}\cdot99.6$ & 
  $30.7\cdot\bm{{\color{black}47.7}}\cdot100.0$ & 
  $30.8\cdot\bm{{\color{black}49.4}}\cdot100.0$ \\

			\bottomrule
		\end{tabular}}
	%\end{tiny}
\end{table*}



\begin{table*}[ht!]
	\centering
	\caption{Attack detection (PGD label attacks) based on differential entropy. Smoothed DBU models on Sensorless. Column format: guaranteed lowest performance $\cdot$ empirical performance $\cdot$ guaranteed highest performance (blue: normally/adversarially trained smooth classifier is more robust than the base model).}
	\label{tab:sensorless_smooth_attackdetection}
	%\begin{tiny}
	\resizebox{\textwidth}{!}{
		\begin{tabular}{llcccccc}
			\toprule
			& \textbf{Att. Rad.} &   0.1 &  0.2 &  0.5 &  1.0 &  2.0 \\
			\midrule
			%& \multicolumn{6}{c}{\textbf{Smoothed models}} \\
               \multirow{4}{1.2cm}{Smoothed models} & 
   \textbf{PostNet} &  %$30.8\cdot\bm{{\color{blue}50.1}}\cdot98.1$ & 
   $30.7\cdot\bm{{\color{blue}61.9}}\cdot100.0$ & 
   $30.7\cdot\bm{{\color{blue}60.1}}\cdot100.0$ & 
   $46.5\cdot\bm{{\color{blue}50.0}}\cdot\phantom{0}75.5$ &  
   $50.0\cdot\bm{{\color{blue}50.0}}\cdot\phantom{0}50.0$ & 
   $50.0\cdot\bm{{\color{black}50.0}}\cdot\phantom{0}50.0$ \\
 & \textbf{PriorNet} &  %$31.0\cdot\bm{{\color{blue}50.2}}\cdot95.0$ & 
 $30.7\cdot\bm{{\color{blue}50.1}}\cdot100.0$ &
 $30.7\cdot\bm{{\color{blue}46.5}}\cdot100.0$ &  
 $30.7\cdot\bm{{\color{blue}42.3}}\cdot100.0$ &  
 $30.7\cdot\bm{{\color{blue}66.7}}\cdot100.0$ &  
 $30.9\cdot\bm{{\color{blue}79.2}}\cdot100.0$ \\
   & \textbf{DDNet} &  %$36.2\cdot\bm{{\color{blue}50.2}}\cdot78.9$ & 
   $30.7\cdot\bm{{\color{blue}57.5}}\cdot100.0$ &
   $30.7\cdot\bm{{\color{blue}49.9}}\cdot100.0$ & 
   $30.7\cdot\bm{{\color{blue}45.5}}\cdot100.0$ & 
   $30.7\cdot\bm{{\color{blue}50.0}}\cdot100.0$ &
   $30.7\cdot\bm{{\color{blue}59.3}}\cdot100.0$ \\
&    \textbf{EvNet} &  %$32.3\cdot\bm{{\color{blue}50.1}}\cdot88.5$ & 
$30.7\cdot\bm{{\color{blue}62.0}}\cdot100.0$ & 
$30.7\cdot\bm{{\color{blue}59.6}}\cdot100.0$ &  
$30.7\cdot\bm{{\color{blue}55.8}}\cdot100.0$ &
$30.7\cdot\bm{{\color{blue}48.3}}\cdot100.0$ & 
$31.8\cdot\bm{{\color{blue}50.0}}\cdot100.0$ \\

			\midrule
			%& \multicolumn{6}{c}{\textbf{Smoothed models + adversarial training using label attacks}} \\
               \multirow{4}{1.2cm}{Smoothed + adv. w. label attacks} & 
   \textbf{PostNet} &  %- &  
   $30.7\cdot\bm{{\color{blue}58.8}}\cdot100.0$ &  
   $30.7\cdot\bm{{\color{blue}58.2}}\cdot100.0$ &   
   $50.0\cdot\bm{{\color{blue}50.0}}\cdot\phantom{0}50.0$ &   
   $50.0\cdot\bm{{\color{blue}50.0}}\cdot\phantom{0}50.0$ &  
   $50.0\cdot\bm{{\color{black}50.0}}\cdot\phantom{0}50.0$ \\
& \textbf{PriorNet} &  %- &  
$30.7\cdot\bm{{\color{blue}60.2}}\cdot100.0$ &  
$30.7\cdot\bm{{\color{blue}54.6}}\cdot100.0$ &  
$30.7\cdot\bm{{\color{blue}45.0}}\cdot100.0$ & 
$30.7\cdot\bm{{\color{blue}38.0}}\cdot100.0$ &  
$33.9\cdot\bm{{\color{blue}49.9}}\cdot100.0$ \\
 &   \textbf{DDNet} &  %- &  
 $30.7\cdot\bm{{\color{blue}55.4}}\cdot100.0$ &  
 $30.7\cdot\bm{{\color{blue}53.7}}\cdot100.0$ &  
 $30.7\cdot\bm{{\color{blue}44.6}}\cdot100.0$ &  
 $30.7\cdot\bm{{\color{blue}38.8}}\cdot100.0$ &  
 $30.7\cdot\bm{{\color{blue}51.9}}\cdot100.0$ \\
  &  \textbf{EvNet} &  %- &  
  $30.7\cdot\bm{{\color{blue}62.1}}\cdot100.0$ & 
  $30.7\cdot\bm{{\color{blue}54.3}}\cdot100.0$ & 
  $30.7\cdot\bm{{\color{blue}59.9}}\cdot100.0$ &  
  $30.7\cdot\bm{{\color{blue}62.1}}\cdot100.0$ & 
  $30.7\cdot\bm{{\color{blue}50.0}}\cdot100.0$ \\

			\midrule
			%& \multicolumn{6}{c}{\textbf{Adversarially training models using uncertainty attacks}} \\
              \multirow{4}{1.2cm}{Smoothed + adv. w. uncert. attacks} &  
  \textbf{PostNet} &  %- &  
  $30.7\cdot\bm{{\color{blue}63.0}}\cdot100.0$ & 
  $30.7\cdot\bm{{\color{blue}54.0}}\cdot100.0$ &  
  $50.0\cdot\bm{{\color{blue}50.0}}\cdot\phantom{0}50.0$ &   
  $50.0\cdot\bm{{\color{blue}50.0}}\cdot\phantom{0}50.0$ &  
  $50.0\cdot\bm{{\color{black}50.0}}\cdot\phantom{0}50.0$ \\
 & \textbf{PriorNet} & % - &  
 $30.7\cdot\bm{{\color{blue}58.0}}\cdot100.0$ &  
 $30.7\cdot\bm{{\color{blue}55.6}}\cdot100.0$ &
 $30.7\cdot\bm{{\color{blue}44.2}}\cdot100.0$ & 
 $30.7\cdot\bm{{\color{blue}53.5}}\cdot100.0$ &   
 $30.7\cdot\bm{{\color{blue}78.5}}\cdot100.0$ \\
   & \textbf{DDNet} &  %- &  
   $30.7\cdot\bm{{\color{blue}55.1}}\cdot100.0$ &
   $30.7\cdot\bm{{\color{blue}48.2}}\cdot100.0$ & 
   $30.7\cdot\bm{{\color{blue}50.1}}\cdot100.0$ &  
   $30.7\cdot\bm{{\color{blue}52.6}}\cdot100.0$ &
   $30.7\cdot\bm{{\color{blue}57.0}}\cdot100.0$ \\
&    \textbf{EvNet} &  %- &  
$30.7\cdot\bm{{\color{blue}63.5}}\cdot100.0$ & 
$30.7\cdot\bm{{\color{blue}54.3}}\cdot100.0$ & 
$30.7\cdot\bm{{\color{blue}54.2}}\cdot100.0$ & 
$30.7\cdot\bm{{\color{blue}45.0}}\cdot100.0$ &  
$30.7\cdot\bm{{\color{blue}50.0}}\cdot100.0$ \\

			\bottomrule
		\end{tabular}}
	%\end{tiny}
\end{table*}







\begin{table*}[ht!]
	\centering
	\caption{Attack detection (PGD label attacks) based on differential entropy. Smoothed DBU models on Segment. Column format: guaranteed lowest performance $\cdot$ empirical performance $\cdot$ guaranteed highest performance (blue: normally/adversarially trained smooth classifier is more robust than the base model).}
	\label{tab:segment_smooth_attackdetection}
	%\begin{tiny}
	\resizebox{\textwidth}{!}{
		\begin{tabular}{llcccccc}
			\toprule
			& \textbf{Att. Rad.} &   0.1 &  0.2 &  0.5 &  1.0 &  2.0 \\
			\midrule
			%& \multicolumn{6}{c}{\textbf{Smoothed models}} \\
                \multirow{4}{1.2cm}{Smoothed models} & 
    \textbf{PostNet} &  %$36.7\cdot\bm{{\color{blue}50.2}}\cdot71.0$ &  
    $30.8\cdot\bm{{\color{black}73.5}}\cdot100.0$ &  
    $30.8\cdot\bm{{\color{black}59.9}}\cdot100.0$ & 
    $30.8\cdot\bm{{\color{blue}60.3}}\cdot100.0$ &  
    $30.8\cdot\bm{{\color{blue}50.2}}\cdot100.0$ &  
    $49.5\cdot\bm{{\color{black}50.0}}\cdot\phantom{0}50.0$ \\
&  \textbf{PriorNet} &  %$41.8\cdot\bm{{\color{blue}50.3}}\cdot61.1$ &  
$30.9\cdot\bm{{\color{black}77.1}}\cdot\phantom{0}99.9$ &  
$30.8\cdot\bm{{\color{black}78.1}}\cdot100.0$ &
$30.8\cdot\bm{{\color{blue}39.5}}\cdot100.0$ & 
$30.8\cdot\bm{{\color{blue}35.2}}\cdot100.0$ &  
$30.8\cdot\bm{{\color{blue}41.4}}\cdot100.0$ \\
  &  \textbf{DDNet} &  %$44.0\cdot\bm{{\color{blue}50.2}}\cdot58.0$ &  
  $31.4\cdot\bm{{\color{black}69.6}}\cdot\phantom{0}99.5$ &  
  $30.8\cdot\bm{{\color{black}71.2}}\cdot100.0$ & 
  $30.8\cdot\bm{{\color{blue}54.3}}\cdot100.0$ &  
  $30.8\cdot\bm{{\color{blue}35.5}}\cdot100.0$ & 
  $30.8\cdot\bm{{\color{blue}35.7}}\cdot100.0$ \\
   & \textbf{EvNet} &  %$35.1\cdot\bm{{\color{blue}50.2}}\cdot76.2$ & 
   $30.8\cdot\bm{{\color{black}86.2}}\cdot100.0$ &  
   $30.8\cdot\bm{{\color{black}80.3}}\cdot100.0$ & 
   $30.8\cdot\bm{{\color{blue}54.0}}\cdot100.0$ &  
   $30.8\cdot\bm{{\color{blue}43.3}}\cdot100.0$ &
   $30.8\cdot\bm{{\color{blue}40.5}}\cdot100.0$ \\

			\midrule
			%& \multicolumn{6}{c}{\textbf{Smoothed models + adversarial training using label attacks}} \\
               \multirow{4}{1.2cm}{Smoothed + adv. w. label attacks} & 
   \textbf{PostNet} &  %- &  
   $30.8\cdot\bm{{\color{black}75.6}}\cdot100.0$ &  
   $30.8\cdot\bm{{\color{black}69.7}}\cdot100.0$ & 
   $30.8\cdot\bm{{\color{blue}66.5}}\cdot100.0$ &   
   $30.8\cdot\bm{{\color{blue}50.0}}\cdot100.0$ & 
   $50.0\cdot\bm{{\color{black}50.0}}\cdot\phantom{0}50.0$ \\
 & \textbf{PriorNet} &  %- &   
 $31.0\cdot\bm{{\color{black}74.4}}\cdot\phantom{0}99.2$ &  
 $30.8\cdot\bm{{\color{black}74.0}}\cdot100.0$ & 
 $30.8\cdot\bm{{\color{blue}59.8}}\cdot100.0$ &  
 $30.8\cdot\bm{{\color{blue}56.0}}\cdot100.0$ &  
 $30.8\cdot\bm{{\color{blue}38.8}}\cdot100.0$ \\
   & \textbf{DDNet} &  %- &   
   $31.6\cdot\bm{{\color{black}68.9}}\cdot\phantom{0}99.0$ & 
   $30.8\cdot\bm{{\color{black}72.9}}\cdot100.0$ & 
   $30.8\cdot\bm{{\color{blue}47.5}}\cdot100.0$ & 
   $30.8\cdot\bm{{\color{black}32.2}}\cdot100.0$ & 
   $30.8\cdot\bm{{\color{blue}31.8}}\cdot100.0$ \\
&    \textbf{EvNet} &  %- &  
$30.8\cdot\bm{{\color{black}83.4}}\cdot100.0$ &    
$30.8\cdot\bm{{\color{blue}87.0}}\cdot100.0$ &  
$30.8\cdot\bm{{\color{blue}61.9}}\cdot100.0$ &  
$30.8\cdot\bm{{\color{blue}39.2}}\cdot100.0$ &  
$30.8\cdot\bm{{\color{blue}41.0}}\cdot100.0$ \\

			\midrule
			%& \multicolumn{6}{c}{\textbf{Adversarially training models using uncertainty attacks}} \\
              \multirow{4}{1.2cm}{Smoothed + adv. w. uncert. attacks} &  
  \textbf{PostNet} &  %- &  
  $30.8\cdot\bm{{\color{black}73.9}}\cdot100.0$ &  
  $30.8\cdot\bm{{\color{black}64.5}}\cdot100.0$ &  
  $30.8\cdot\bm{{\color{blue}68.3}}\cdot100.0$ &    
  $33.0\cdot\bm{{\color{blue}50.0}}\cdot100.0$ & 
  $50.0\cdot\bm{{\color{black}50.0}}\cdot\phantom{0}50.0$ \\
 & \textbf{PriorNet} & % - &   
 $31.0\cdot\bm{{\color{black}73.7}}\cdot\phantom{0}99.6$ & 
 $30.8\cdot\bm{{\color{black}73.1}}\cdot100.0$ &
 $30.8\cdot\bm{{\color{blue}57.8}}\cdot100.0$ & 
 $30.8\cdot\bm{{\color{blue}44.8}}\cdot100.0$ &  
 $30.8\cdot\bm{{\color{blue}49.1}}\cdot100.0$ \\
   & \textbf{DDNet} &  %- &   
   $31.0\cdot\bm{{\color{black}70.7}}\cdot\phantom{0}99.7$ &
   $30.8\cdot\bm{{\color{black}70.6}}\cdot100.0$ &
   $30.8\cdot\bm{{\color{blue}48.6}}\cdot100.0$ &  
   $30.8\cdot\bm{{\color{black}31.6}}\cdot100.0$ & 
   $30.8\cdot\bm{{\color{blue}30.9}}\cdot100.0$ \\
&    \textbf{EvNet} &  %- &  
$30.8\cdot\bm{{\color{black}85.8}}\cdot100.0$ &    
$30.8\cdot\bm{{\color{blue}86.7}}\cdot100.0$ & 
$30.8\cdot\bm{{\color{blue}54.4}}\cdot100.0$ &  
$30.8\cdot\bm{{\color{blue}45.1}}\cdot100.0$ & 
$30.8\cdot\bm{{\color{blue}34.8}}\cdot100.0$ \\

			\bottomrule
		\end{tabular}}
	%\end{tiny}
\end{table*}






\begin{table*}[ht!]
	\centering
	\caption{Attack detection (FGSM label attacks) based on differential entropy. Smoothed DBU models on CIFAR10. Column format: guaranteed lowest performance $\cdot$ empirical performance $\cdot$ guaranteed highest performance (blue: normally/adversarially trained smooth classifier is more robust than the base model).}
	\label{tab:cifar10_smooth_attackdetection_fgsm}
	%\begin{tiny}
	\resizebox{\textwidth}{!}{
		\begin{tabular}{llcccccc}
			\toprule
			& \textbf{Att. Rad.} &   0.1 &  0.2 &  0.5 &  1.0 &  2.0 \\
			\midrule
			%& \multicolumn{6}{c}{\textbf{Smoothed models}} \\
            \input{sections/008_icml2021/tables_v2/normal-CIFAR10-in-FGSM_L2-crossentropy-attack_detection}
			\midrule
			%& \multicolumn{6}{c}{\textbf{Smoothed models + adversarial training using label attacks}} \\
              \multirow{4}{1.2cm}{Smoothed + adv. w. label attacks} &  
  \textbf{PostNet} &  %- &  
  $32.7\cdot\bm{{\color{black}50.1}}\cdot90.3$ &  
  $31.1\cdot\bm{{\color{black}50.3}}\cdot96.4$ &   
  $30.7\cdot\bm{{\color{black}50.1}}\cdot\phantom{0}99.7$ &  
  $30.7\cdot\bm{{\color{black}49.8}}\cdot100.0$ &  
  $30.7\cdot\bm{{\color{black}50.5}}\cdot100.0$ \\
 & \textbf{PriorNet} &  %- &  
 $35.4\cdot\bm{{\color{black}52.3}}\cdot78.9$ &  
 $32.9\cdot\bm{{\color{black}51.3}}\cdot84.5$ &   
 $30.7\cdot\bm{{\color{black}50.3}}\cdot\phantom{0}98.7$ &  
 $30.7\cdot\bm{{\color{black}50.7}}\cdot100.0$ &   
 $30.8\cdot\bm{{\color{black}50.2}}\cdot\phantom{0}98.2$ \\
   & \textbf{DDNet} &  %- &  
   $35.5\cdot\bm{{\color{black}50.6}}\cdot79.3$ &  
   $33.4\cdot\bm{{\color{black}50.3}}\cdot84.2$ &  
   $30.8\cdot\bm{{\color{black}50.1}}\cdot\phantom{0}99.2$ & 
   $30.7\cdot\bm{{\color{black}49.9}}\cdot100.0$ &  
   $30.7\cdot\bm{{\color{black}50.1}}\cdot100.0$ \\
&    \textbf{EvNet} &  %- &  
$40.3\cdot\bm{{\color{black}50.4}}\cdot66.8$ & 
$31.4\cdot\bm{{\color{black}50.3}}\cdot95.9$ & 
$30.7\cdot\bm{{\color{black}50.2}}\cdot100.0$ & 
$30.7\cdot\bm{{\color{black}50.1}}\cdot100.0$ &  
$30.7\cdot\bm{{\color{black}49.6}}\cdot100.0$ \\
			\midrule
			%& \multicolumn{6}{c}{\textbf{Adversarially training models using uncertainty attacks}} \\
              \multirow{4}{1.2cm}{Smoothed + adv. w. uncert. attacks} &  
  \textbf{PostNet} &  %- &  
  $33.3\cdot\bm{{\color{black}50.7}}\cdot88.7$ & 
  $32.5\cdot\bm{{\color{black}50.1}}\cdot87.8$ &   
  $30.7\cdot\bm{{\color{black}50.1}}\cdot\phantom{0}99.8$ &  
  $30.7\cdot\bm{{\color{black}50.5}}\cdot100.0$ & 
  $30.7\cdot\bm{{\color{black}50.2}}\cdot100.0$ \\
 & \textbf{PriorNet} &  %- &  
 $34.6\cdot\bm{{\color{black}51.2}}\cdot80.3$ & 
 $31.4\cdot\bm{{\color{black}50.7}}\cdot92.8$ &  
 $30.9\cdot\bm{{\color{black}50.2}}\cdot\phantom{0}97.7$ & 
 $30.7\cdot\bm{{\color{black}50.0}}\cdot100.0$ &  
 $30.7\cdot\bm{{\color{black}50.1}}\cdot100.0$ \\
   & \textbf{DDNet} & % - &  
   $37.4\cdot\bm{{\color{black}51.0}}\cdot74.7$ & 
   $33.4\cdot\bm{{\color{black}50.2}}\cdot83.0$ & 
   $30.9\cdot\bm{{\color{black}50.1}}\cdot\phantom{0}96.9$ & 
   $30.8\cdot\bm{{\color{black}50.1}}\cdot\phantom{0}98.1$ & 
   $30.7\cdot\bm{{\color{black}49.9}}\cdot100.0$ \\
&    \textbf{EvNet} &  %- &  
$32.8\cdot\bm{{\color{black}50.1}}\cdot92.0$ &  
$30.8\cdot\bm{{\color{black}50.2}}\cdot99.6$ &  
$30.7\cdot\bm{{\color{black}50.4}}\cdot100.0$ &
$31.2\cdot\bm{{\color{black}50.2}}\cdot\phantom{0}96.0$ &  
$31.0\cdot\bm{{\color{black}50.0}}\cdot100.0$ \\

			\bottomrule
		\end{tabular}}
	%\end{tiny}
\end{table*}




\begin{table*}[ht!]
	\centering
	\caption{Attack detection (FGSM label attacks) based on differential entropy. Smoothed DBU models on MNIST. Column format: guaranteed lowest performance $\cdot$ empirical performance $\cdot$ guaranteed highest performance (blue: normally/adversarially trained smooth classifier is more robust than the base model).}
	\label{tab:mnist_smooth_attackdetection_fgsm}
	%\begin{tiny}
	\resizebox{\textwidth}{!}{
		\begin{tabular}{llcccccc}
			\toprule
			& \textbf{Att. Rad.} &   0.1 &  0.2 &  0.5 &  1.0 &  2.0 \\
			\midrule
			%& \multicolumn{6}{c}{\textbf{Smoothed models}} \\
               \multirow{4}{1.2cm}{Smoothed models} & 
   \textbf{PostNet} & 
   %$31.0\cdot\bm{50.0}\cdot94.0$ &  
   $30.9\cdot\bm{{\color{black}52.3}}\cdot95.6$ &
   $31.5\cdot\bm{{\color{black}51.2}}\cdot90.8$ &  
   $31.1\cdot\bm{{\color{black}49.8}}\cdot97.0$ &  
   $30.7\cdot\bm{{\color{black}48.3}}\cdot100.0$ & 
   $30.7\cdot\bm{{\color{black}46.5}}\cdot100.0$ \\
& \textbf{PriorNet} & 
%$47.0\cdot\bm{50.0}\cdot53.2$ &  
$38.1\cdot\bm{{\color{black}57.7}}\cdot80.8$ & 
$35.8\cdot\bm{{\color{black}56.6}}\cdot84.0$ &  
$31.5\cdot\bm{{\color{black}61.7}}\cdot98.3$ & 
$30.8\cdot\bm{{\color{black}58.9}}\cdot\phantom{0}99.2$ &  
$30.7\cdot\bm{{\color{black}62.3}}\cdot100.0$ \\
 &   \textbf{DDNet} &  
 %$49.2\cdot\bm{50.0}\cdot50.8$ &  
 $44.7\cdot\bm{{\color{black}52.0}}\cdot60.9$ &
 $39.4\cdot\bm{{\color{black}52.9}}\cdot72.5$ &
 $31.6\cdot\bm{{\color{black}50.8}}\cdot95.2$ & 
 $30.7\cdot\bm{{\color{black}47.5}}\cdot100.0$ &
 $30.7\cdot\bm{{\color{black}46.8}}\cdot100.0$ \\
  &  \textbf{EvNet} &  
  %$47.8\cdot\bm{50.0}\cdot52.4$ &  
  $36.5\cdot\bm{{\color{black}51.7}}\cdot76.0$ & 
  $31.5\cdot\bm{{\color{black}51.1}}\cdot93.2$ & 
  $30.7\cdot\bm{{\color{black}50.9}}\cdot99.9$ & 
  $30.7\cdot\bm{{\color{black}48.9}}\cdot100.0$ & 
  $30.7\cdot\bm{{\color{black}46.2}}\cdot100.0$ \\

			\midrule
			%& \multicolumn{6}{c}{\textbf{Smoothed models + adversarial training using label attacks}} \\
              \multirow{4}{1.2cm}{Smoothed + adv. w. label attacks} &  
  \textbf{PostNet} &  %- &  
  $33.5\cdot\bm{{\color{black}52.6}}\cdot82.2$ &  
  $31.4\cdot\bm{{\color{black}51.0}}\cdot91.5$ &  
  $30.9\cdot\bm{{\color{black}49.8}}\cdot\phantom{0}99.0$ &  
  $30.7\cdot\bm{{\color{black}50.1}}\cdot100.0$ & 
  $30.7\cdot\bm{{\color{black}54.4}}\cdot100.0$ \\
 & \textbf{PriorNet} &  %- &  
 $37.3\cdot\bm{{\color{black}60.6}}\cdot84.3$ &  
 $34.2\cdot\bm{{\color{black}59.5}}\cdot87.8$ &  
 $32.1\cdot\bm{{\color{black}60.0}}\cdot\phantom{0}96.9$ & 
 $30.7\cdot\bm{{\color{black}66.3}}\cdot100.0$ & 
 $30.7\cdot\bm{{\color{black}63.3}}\cdot100.0$ \\
   & \textbf{DDNet} &  %- &  
   $44.9\cdot\bm{{\color{black}52.3}}\cdot61.0$ &  
   $40.3\cdot\bm{{\color{black}52.8}}\cdot70.2$ &  
   $32.5\cdot\bm{{\color{black}52.4}}\cdot\phantom{0}94.6$ &  
   $30.7\cdot\bm{{\color{black}50.0}}\cdot100.0$ &  
   $30.7\cdot\bm{{\color{black}57.1}}\cdot100.0$ \\
&    \textbf{EvNet} &  %- &  
$35.8\cdot\bm{{\color{black}51.5}}\cdot76.7$ & 
$32.9\cdot\bm{{\color{black}50.9}}\cdot88.5$ & 
$30.7\cdot\bm{{\color{black}50.0}}\cdot100.0$ &
$30.7\cdot\bm{{\color{black}48.9}}\cdot100.0$ & 
$30.7\cdot\bm{{\color{black}48.6}}\cdot100.0$ \\

			\midrule
			%& \multicolumn{6}{c}{\textbf{Adversarially training models using uncertainty attacks}} \\
              \multirow{4}{1.2cm}{Smoothed + adv. uncert. attacks} &  
  \textbf{PostNet} & % - &  
  $31.2\cdot\bm{{\color{black}52.6}}\cdot92.9$ & 
  $31.3\cdot\bm{{\color{black}51.5}}\cdot92.3$ & 
  $31.3\cdot\bm{{\color{black}48.0}}\cdot96.9$ & 
  $30.7\cdot\bm{{\color{black}49.4}}\cdot100.0$ & 
  $30.7\cdot\bm{{\color{black}48.1}}\cdot100.0$ \\
 & \textbf{PriorNet} & % - &  
 $38.3\cdot\bm{{\color{black}58.3}}\cdot81.4$ & 
 $36.8\cdot\bm{{\color{black}55.2}}\cdot79.8$ &  
 $31.3\cdot\bm{{\color{black}62.5}}\cdot98.9$ &  
 $30.7\cdot\bm{{\color{black}64.5}}\cdot100.0$ & 
 $30.7\cdot\bm{{\color{black}68.7}}\cdot100.0$ \\
   & \textbf{DDNet} &  %- &  
   $45.0\cdot\bm{{\color{black}52.3}}\cdot60.9$ & 
   $39.7\cdot\bm{{\color{black}53.5}}\cdot72.4$ &  
   $31.8\cdot\bm{{\color{black}51.7}}\cdot95.4$ & 
   $30.7\cdot\bm{{\color{black}46.6}}\cdot100.0$ &
   $30.7\cdot\bm{{\color{black}44.4}}\cdot100.0$ \\
&    \textbf{EvNet} &  %- &  
$38.8\cdot\bm{{\color{black}51.8}}\cdot70.8$ &  
$34.5\cdot\bm{{\color{black}52.0}}\cdot82.7$ & 
$30.8\cdot\bm{{\color{black}50.0}}\cdot99.6$ &  
$30.7\cdot\bm{{\color{black}49.3}}\cdot100.0$ & 
$30.8\cdot\bm{{\color{black}50.3}}\cdot100.0$ \\

			\bottomrule
		\end{tabular}}
	%\end{tiny}
\end{table*}




\begin{table*}[ht!]
	\centering
	\caption{Attack detection (FGSM label attacks) based on differential entropy. Smoothed DBU models on Sensorless. Column format: guaranteed lowest performance $\cdot$ empirical performance $\cdot$ guaranteed highest performance (blue: normally/adversarially trained smooth classifier is more robust than the base model).}
	\label{tab:sensorless_smooth_attackdetection_fgsm}
	%\begin{tiny}
	\resizebox{\textwidth}{!}{
		\begin{tabular}{llcccccc}
			\toprule
			& \textbf{Att. Rad.} &    0.1 &  0.2 &  0.5 &  1.0 &  2.0 \\
			\midrule
			%& \multicolumn{6}{c}{\textbf{Smoothed models}} \\
               \multirow{4}{1.2cm}{Smoothed models} & 
   \textbf{PostNet} &  %$30.7\cdot\bm{{\color{blue}50.1}}\cdot98.2$ & 
   $30.7\cdot\bm{{\color{black}82.0}}\cdot100.0$ & 
   $30.7\cdot\bm{{\color{black}88.6}}\cdot100.0$ & 
   $50.0\cdot\bm{{\color{black}50.0}}\cdot\phantom{0}50.1$ & 
   $50.0\cdot\bm{{\color{black}50.0}}\cdot\phantom{0}50.0$ &
   $50.0\cdot\bm{{\color{black}50.0}}\cdot\phantom{0}50.0$ \\
& \textbf{PriorNet} &  %$31.0\cdot\bm{{\color{blue}50.1}}\cdot95.0$ &    
$30.7\cdot\bm{{\color{blue}51.7}}\cdot100.0$ & 
$30.7\cdot\bm{{\color{blue}48.2}}\cdot100.0$ & 
$30.7\cdot\bm{{\color{blue}48.6}}\cdot100.0$ & 
$30.7\cdot\bm{{\color{blue}68.6}}\cdot100.0$ & 
$31.4\cdot\bm{{\color{blue}63.7}}\cdot100.0$ \\
 &   \textbf{DDNet} &  %$36.2\cdot\bm{{\color{blue}50.1}}\cdot79.0$ &    
 $30.7\cdot\bm{{\color{blue}67.1}}\cdot100.0$ &    
 $30.7\cdot\bm{{\color{blue}58.2}}\cdot100.0$ &   
 $30.7\cdot\bm{{\color{blue}51.7}}\cdot100.0$ &  
 $30.7\cdot\bm{{\color{blue}69.8}}\cdot100.0$ & 
 $30.7\cdot\bm{{\color{blue}73.7}}\cdot100.0$ \\
  &  \textbf{EvNet} &  %$32.3\cdot\bm{{\color{blue}50.1}}\cdot88.5$ &   
  $30.7\cdot\bm{{\color{blue}77.9}}\cdot100.0$ &    
  $30.7\cdot\bm{{\color{blue}85.3}}\cdot100.0$ &   
  $30.7\cdot\bm{{\color{blue}90.5}}\cdot100.0$ &  
  $30.8\cdot\bm{{\color{blue}84.3}}\cdot100.0$ & 
  $34.0\cdot\bm{{\color{black}50.0}}\cdot100.0$ \\

			\midrule
			%& \multicolumn{6}{c}{\textbf{Smoothed models + adversarial training using label attacks}} \\
              \multirow{4}{1.2cm}{Smoothed + adv. label attacks} &  
  \textbf{PostNet} &  %- &  
  $30.7\cdot\bm{{\color{black}76.7}}\cdot100.0$ & 
  $30.7\cdot\bm{{\color{black}78.7}}\cdot100.0$ & 
  $50.0\cdot\bm{{\color{black}50.0}}\cdot\phantom{0}50.0$ & 
  $50.0\cdot\bm{{\color{black}50.0}}\cdot\phantom{0}50.0$ & 
  $50.0\cdot\bm{{\color{black}50.0}}\cdot\phantom{0}50.0$ \\
 & \textbf{PriorNet} & % - &     
 $30.7\cdot\bm{{\color{blue}63.9}}\cdot100.0$ &   
 $30.7\cdot\bm{{\color{blue}58.4}}\cdot100.0$ &   
 $30.7\cdot\bm{{\color{blue}45.2}}\cdot100.0$ &
 $30.7\cdot\bm{{\color{blue}43.3}}\cdot100.0$ &
 $32.9\cdot\bm{{\color{black}35.5}}\cdot100.0$ \\
   & \textbf{DDNet} &  %- &  
   $30.7\cdot\bm{{\color{black}58.5}}\cdot100.0$ &    
   $30.7\cdot\bm{{\color{blue}75.8}}\cdot100.0$ &    
   $30.7\cdot\bm{{\color{blue}72.6}}\cdot100.0$ &  
   $30.7\cdot\bm{{\color{blue}35.6}}\cdot100.0$ & 
   $30.7\cdot\bm{{\color{blue}71.5}}\cdot100.0$ \\
&    \textbf{EvNet} &  %- &     
$30.7\cdot\bm{{\color{blue}80.4}}\cdot100.0$ &  
$30.7\cdot\bm{{\color{blue}71.5}}\cdot100.0$ &  
$30.7\cdot\bm{{\color{blue}75.3}}\cdot100.0$ &
$30.7\cdot\bm{{\color{blue}78.5}}\cdot100.0$ & 
$30.7\cdot\bm{{\color{black}50.0}}\cdot100.0$ \\

			\midrule
			%& \multicolumn{6}{c}{\textbf{Adversarially training models using uncertainty attacks}} \\
               \multirow{4}{1.2cm}{Smoothed + adv. uncert. attacks} & 
   \textbf{PostNet} &  %- &  
   $30.7\cdot\bm{{\color{black}77.4}}\cdot100.0$ & 
   $30.7\cdot\bm{{\color{black}68.0}}\cdot100.0$ & 
   $50.0\cdot\bm{{\color{black}50.0}}\cdot50.0$ &  
   $50.0\cdot\bm{{\color{black}50.0}}\cdot\phantom{0}50.0$ & 
   $50.0\cdot\bm{{\color{black}50.0}}\cdot\phantom{0}50.0$ \\
 & \textbf{PriorNet} &  %- &     
 $30.7\cdot\bm{{\color{blue}63.8}}\cdot100.0$ &  
 $30.7\cdot\bm{{\color{blue}64.1}}\cdot100.0$ & 
 $30.7\cdot\bm{{\color{blue}46.9}}\cdot100.0$ &  
 $30.7\cdot\bm{{\color{blue}48.9}}\cdot100.0$ & 
 $30.7\cdot\bm{{\color{blue}78.0}}\cdot100.0$ \\
   & \textbf{DDNet} &  %- &  
   $30.7\cdot\bm{{\color{black}56.5}}\cdot100.0$ &   
   $30.7\cdot\bm{{\color{blue}54.6}}\cdot100.0$ &   
   $30.7\cdot\bm{{\color{blue}59.4}}\cdot100.0$ &   
   $30.7\cdot\bm{{\color{blue}71.8}}\cdot100.0$ &   
   $30.7\cdot\bm{{\color{blue}76.0}}\cdot100.0$ \\
&    \textbf{EvNet} &  %- &     
$30.7\cdot\bm{{\color{blue}71.5}}\cdot100.0$ &  
$30.7\cdot\bm{{\color{blue}75.7}}\cdot100.0$ & 
$30.7\cdot\bm{{\color{blue}90.5}}\cdot100.0$ & 
$30.7\cdot\bm{{\color{black}54.7}}\cdot100.0$ &  
$30.9\cdot\bm{{\color{black}50.2}}\cdot100.0$ \\

			\bottomrule
		\end{tabular}}
	%\end{tiny}
\end{table*}


\begin{table*}[ht!]
	\centering
	\caption{Attack detection (FGSM label attacks) based on differential entropy. Smoothed DBU models on Segment. Column format: guaranteed lowest performance $\cdot$ empirical performance $\cdot$ guaranteed highest performance (blue: normally/adversarially trained smooth classifier is more robust than the base model)..}
	\label{tab:segment_smooth_attackdetection_fgsm}
	%\begin{tiny}
	\resizebox{\textwidth}{!}{
		\begin{tabular}{llcccccc}
			\toprule
			& \textbf{Att. Rad.} &   0.1 &  0.2 &  0.5 &  1.0 &  2.0 \\
			\midrule
			%& \multicolumn{6}{c}{\textbf{Smoothed models}} \\
               \multirow{4}{1.2cm}{Smoothed models} & 
   \textbf{PostNet} &  %$36.6\cdot\bm{{\color{blue}50.4}}\cdot71.1$ &  
   $30.8\cdot\bm{{\color{black}76.9}}\cdot100.0$ &  
   $30.8\cdot\bm{{\color{black}62.5}}\cdot100.0$ &  
   $30.8\cdot\bm{{\color{black}59.2}}\cdot100.0$ &  
   $30.8\cdot\bm{{\color{black}48.7}}\cdot100.0$ & 
   $49.7\cdot\bm{{\color{black}50.0}}\cdot\phantom{0}50.0$ \\
 & \textbf{PriorNet} &  %$41.8\cdot\bm{{\color{blue}50.2}}\cdot61.2$ &  
 $30.9\cdot\bm{{\color{black}81.3}}\cdot\phantom{0}99.9$ &    
 $30.8\cdot\bm{{\color{blue}85.0}}\cdot100.0$ &   
 $30.8\cdot\bm{{\color{blue}48.7}}\cdot100.0$ &  
 $30.8\cdot\bm{{\color{black}37.1}}\cdot100.0$ &
 $30.8\cdot\bm{{\color{blue}43.7}}\cdot100.0$ \\
  &  \textbf{DDNet} &  %$44.0\cdot\bm{{\color{blue}50.4}}\cdot58.0$ &  
  $31.7\cdot\bm{{\color{black}73.8}}\cdot\phantom{0}99.7$ & 
  $30.8\cdot\bm{{\color{black}80.5}}\cdot100.0$ &
  $30.8\cdot\bm{{\color{blue}80.4}}\cdot100.0$ &  
  $30.8\cdot\bm{{\color{blue}72.7}}\cdot100.0$ & 
  $30.8\cdot\bm{{\color{blue}70.6}}\cdot100.0$ \\
   & \textbf{EvNet} &  %$35.1\cdot\bm{{\color{blue}50.3}}\cdot76.1$ & 
   $30.8\cdot\bm{{\color{black}89.1}}\cdot100.0$ &    
   $30.8\cdot\bm{{\color{blue}89.5}}\cdot100.0$ &    
   $30.8\cdot\bm{{\color{blue}75.3}}\cdot100.0$ &    
   $30.8\cdot\bm{{\color{blue}73.1}}\cdot100.0$ &  
   $30.8\cdot\bm{{\color{blue}83.1}}\cdot100.0$ \\

			\midrule
			%& \multicolumn{6}{c}{\textbf{Smoothed models + adversarial training using label attacks}} \\
              \multirow{4}{1.2cm}{Smoothed + adv. w. label attacks} &  
  \textbf{PostNet} &  %- &  
  $30.8\cdot\bm{{\color{black}81.0}}\cdot100.0$ & 
  $30.8\cdot\bm{{\color{black}75.6}}\cdot100.0$ &  
  $30.8\cdot\bm{{\color{black}56.3}}\cdot100.0$ &  
  $30.8\cdot\bm{{\color{black}50.0}}\cdot100.0$ &  
  $50.0\cdot\bm{{\color{black}50.0}}\cdot\phantom{0}50.0$ \\
 & \textbf{PriorNet} &  %- &   
 $31.1\cdot\bm{{\color{black}77.9}}\cdot\phantom{0}99.4$ &  
 $30.8\cdot\bm{{\color{black}76.1}}\cdot100.0$ &  
 $30.8\cdot\bm{{\color{blue}62.4}}\cdot100.0$ &    
 $30.8\cdot\bm{{\color{blue}65.5}}\cdot100.0$ &   
 $30.8\cdot\bm{{\color{blue}53.3}}\cdot100.0$ \\
   & \textbf{DDNet} &  %- &   
   $31.9\cdot\bm{{\color{black}72.5}}\cdot\phantom{0}99.3$ &  
   $30.8\cdot\bm{{\color{black}82.0}}\cdot100.0$ &  
   $30.8\cdot\bm{{\color{blue}65.7}}\cdot100.0$ &    
   $30.8\cdot\bm{{\color{blue}53.0}}\cdot100.0$ &  
   $30.8\cdot\bm{{\color{blue}61.6}}\cdot100.0$ \\
&    \textbf{EvNet} &  %- &  
$30.8\cdot\bm{{\color{black}86.4}}\cdot100.0$ & 
$30.8\cdot\bm{{\color{blue}94.1}}\cdot100.0$ &    
$30.8\cdot\bm{{\color{blue}78.6}}\cdot100.0$ &   
$30.8\cdot\bm{{\color{blue}77.7}}\cdot100.0$ &   
$30.8\cdot\bm{{\color{blue}85.5}}\cdot100.0$ \\

			\midrule
			%& \multicolumn{6}{c}{\textbf{Adversarially training models using uncertainty attacks}} \\
              \multirow{4}{1.2cm}{Smoothed + adv. uncert. attacks} & 
  \textbf{PostNet} &  %- &  
  $30.8\cdot\bm{{\color{black}76.8}}\cdot100.0$ &  
  $30.8\cdot\bm{{\color{black}64.6}}\cdot100.0$ &  
  $30.8\cdot\bm{{\color{black}82.9}}\cdot100.0$ & 
  $32.2\cdot\bm{{\color{black}50.0}}\cdot100.0$ &  
  $50.0\cdot\bm{{\color{black}50.0}}\cdot\phantom{0}50.0$ \\
& \textbf{PriorNet} &  %- &   
$31.1\cdot\bm{{\color{black}77.6}}\cdot\phantom{0}99.7$ &  
$30.8\cdot\bm{{\color{black}76.7}}\cdot100.0$ &   
$30.8\cdot\bm{{\color{blue}69.0}}\cdot100.0$ &    
$30.8\cdot\bm{{\color{blue}53.1}}\cdot100.0$ &    
$30.8\cdot\bm{{\color{blue}61.4}}\cdot100.0$ \\
 &   \textbf{DDNet} &  %- &   
 $31.1\cdot\bm{{\color{black}74.3}}\cdot\phantom{0}99.8$ &  
 $30.8\cdot\bm{{\color{black}77.1}}\cdot100.0$ & 
 $30.8\cdot\bm{{\color{blue}76.0}}\cdot100.0$ &   
 $30.8\cdot\bm{{\color{blue}57.0}}\cdot100.0$ &    
 $30.8\cdot\bm{{\color{blue}43.5}}\cdot100.0$ \\
  &  \textbf{EvNet} &  %- &  
  $30.8\cdot\bm{{\color{black}88.8}}\cdot100.0$ &    
  $30.8\cdot\bm{{\color{blue}92.6}}\cdot100.0$ &     
  $30.8\cdot\bm{{\color{blue}70.2}}\cdot100.0$ &    
  $30.8\cdot\bm{{\color{blue}62.0}}\cdot100.0$ &   
  $30.8\cdot\bm{{\color{blue}96.2}}\cdot100.0$ \\

			\bottomrule
		\end{tabular}}
	%\end{tiny}
\end{table*}







%%%% OOD detection %%%%



\begin{table*}[ht!]
	\centering
	\caption{OOD detection based on differential entropy under PGD uncertainty attacks against differential entorpy on ID data and OOD data. Smoothed DBU models on CIFAR10. Column format: guaranteed lowest performance $\cdot$ empirical performance $\cdot$ guaranteed highest performance (blue: normally/adversarially trained smooth classifier is more robust than the base model).}
	\label{tab:cifar10_smooth_ooddetection}
	%\begin{tiny}
	\resizebox{\textwidth}{!}{
		\begin{tabular}{llcccccc}
			\toprule
			& \textbf{Att. Rad.} & 0.0 &   0.1 &  0.2 &  0.5 &  1.0 &  2.0 \\
			\midrule
			& & \multicolumn{6}{c}{\textbf{ID-Attack}} \\
              \multirow{4}{1.2cm}{Smoothed models} &  
  \textbf{PostNet} &     
  $72.1\cdot\bm{{\color{blue}82.7}}\cdot88.0$ &  
  $35.0\cdot\bm{{\color{black}56.6}}\cdot97.4$ &     
  $31.9\cdot\bm{{\color{blue}65.6}}\cdot99.8$ &  
  $30.7\cdot\bm{{\color{blue}50.6}}\cdot100.0$ &  
  $30.7\cdot\bm{{\color{blue}46.9}}\cdot100.0$ &  
  $30.7\cdot\bm{{\color{blue}51.6}}\cdot100.0$ \\
& \textbf{PriorNet} &  
$50.2\cdot\bm{{\color{black}53.1}}\cdot55.9$ &     
$33.5\cdot\bm{{\color{blue}43.3}}\cdot65.3$ &     
$31.3\cdot\bm{{\color{blue}39.7}}\cdot69.1$ &   
$31.3\cdot\bm{{\color{blue}48.3}}\cdot\phantom{0}98.2$ &   
$30.7\cdot\bm{{\color{blue}44.4}}\cdot\phantom{0}99.9$ &  
$30.7\cdot\bm{{\color{blue}45.4}}\cdot100.0$ \\
 &   \textbf{DDNet} &  
 $72.0\cdot\bm{{\color{black}75.8}}\cdot79.8$ &  
 $35.6\cdot\bm{{\color{black}46.2}}\cdot69.8$ &  
 $32.9\cdot\bm{{\color{black}50.3}}\cdot87.1$ &   
 $31.1\cdot\bm{{\color{blue}58.7}}\cdot\phantom{0}98.6$ & 
 $30.7\cdot\bm{{\color{blue}59.3}}\cdot100.0$ &  
 $30.7\cdot\bm{{\color{blue}44.5}}\cdot100.0$ \\
  &  \textbf{EvNet} &     
  $79.5\cdot\bm{{\color{blue}87.1}}\cdot92.8$ &  
  $34.1\cdot\bm{{\color{black}58.6}}\cdot95.1$ &     
  $32.5\cdot\bm{{\color{blue}61.2}}\cdot96.9$ &   
  $31.7\cdot\bm{{\color{blue}60.6}}\cdot\phantom{0}98.7$ &  
  $30.7\cdot\bm{{\color{blue}62.4}}\cdot100.0$ &  
  $30.7\cdot\bm{{\color{blue}57.3}}\cdot100.0$ \\

			\midrule
			%& &\multicolumn{6}{c}{\textbf{ID-Attack}} \\
            \multirow{4}{1.3cm}{Smoothed + adv.\ w. label attacks} &  
\textbf{PostNet} &  - & 
$35.0\cdot\bm{{\color{black}58.5}}\cdot97.7$ &  
$31.2\cdot\bm{{\color{black}46.6}}\cdot97.4$ &   
$30.8\cdot\bm{{\color{blue}57.7}}\cdot\phantom{0}99.7$ &  
$30.7\cdot\bm{{\color{blue}49.8}}\cdot100.0$ &  
$30.7\cdot\bm{{\color{blue}50.9}}\cdot100.0$ \\
& \textbf{PriorNet} &  - &  
$31.5\cdot\bm{{\color{black}36.7}}\cdot57.2$ &     
$33.1\cdot\bm{{\color{blue}51.8}}\cdot84.8$ &   
$30.7\cdot\bm{{\color{blue}57.7}}\cdot\phantom{0}98.7$ &   
$30.7\cdot\bm{{\color{blue}40.0}}\cdot\phantom{0}99.9$ &   
$30.9\cdot\bm{{\color{blue}53.6}}\cdot\phantom{0}96.7$ \\
 &   \textbf{DDNet} &  - &  
 $36.2\cdot\bm{{\color{black}50.0}}\cdot78.6$ &  
 $32.1\cdot\bm{{\color{black}41.3}}\cdot70.2$ &  
 $30.8\cdot\bm{{\color{blue}56.4}}\cdot100.0$ &  
 $30.7\cdot\bm{{\color{blue}49.4}}\cdot100.0$ &  
 $30.7\cdot\bm{{\color{blue}54.8}}\cdot100.0$ \\
  &  \textbf{EvNet} &  - &  
  $46.8\cdot\bm{{\color{black}61.0}}\cdot79.7$ &     
  $32.3\cdot\bm{{\color{blue}58.9}}\cdot99.1$ &  
  $30.7\cdot\bm{{\color{blue}45.0}}\cdot100.0$ &  
  $30.7\cdot\bm{{\color{blue}63.3}}\cdot100.0$ &  
  $30.8\cdot\bm{{\color{blue}38.1}}\cdot100.0$ \\

			\midrule
			%& & \multicolumn{6}{c}{\textbf{ID-Attack}} \\
            \multirow{4}{1.3cm}{Smoothed + adv. uncert. attacks} &  
\textbf{PostNet} &  - &  
$35.2\cdot\bm{{\color{black}55.9}}\cdot96.0$ &     
$34.5\cdot\bm{{\color{blue}59.2}}\cdot94.9$ &  
$30.7\cdot\bm{{\color{blue}47.0}}\cdot100.0$ &  
$30.7\cdot\bm{{\color{blue}58.2}}\cdot100.0$ &  
$30.7\cdot\bm{{\color{blue}42.9}}\cdot100.0$ \\
 & \textbf{PriorNet} &  - &  
 $31.8\cdot\bm{{\color{black}38.9}}\cdot64.1$ &     
 $31.0\cdot\bm{{\color{blue}41.8}}\cdot87.9$ &  
 $30.7\cdot\bm{{\color{blue}42.9}}\cdot\phantom{0}99.2$ & 
 $30.7\cdot\bm{{\color{blue}48.6}}\cdot100.0$ & 
 $30.7\cdot\bm{{\color{blue}46.6}}\cdot100.0$ \\
   & \textbf{DDNet} &  - & 
   $39.7\cdot\bm{{\color{black}52.1}}\cdot75.7$ &  
   $36.4\cdot\bm{{\color{black}56.8}}\cdot83.8$ &   
   $31.0\cdot\bm{{\color{blue}51.5}}\cdot\phantom{0}97.4$ &  
   $31.0\cdot\bm{{\color{blue}56.8}}\cdot\phantom{0}97.8$ &  
   $30.7\cdot\bm{{\color{blue}49.1}}\cdot100.0$ \\
&    \textbf{EvNet} &  - &     
$34.8\cdot\bm{{\color{blue}64.9}}\cdot99.6$ &     
$30.8\cdot\bm{{\color{blue}48.9}}\cdot99.8$ &  
$30.7\cdot\bm{{\color{blue}66.8}}\cdot100.0$ &  
$30.9\cdot\bm{{\color{blue}41.5}}\cdot\phantom{0}93.8$ &  
$31.1\cdot\bm{{\color{blue}55.1}}\cdot100.0$ \\

			\midrule
			\midrule
			& & \multicolumn{6}{c}{\textbf{OOD-Attack}} \\
            \begin{tabular}{lccccccc}
\toprule
\textbf{Att. Rad.} &                                           0.0 &                                           0.1 &                                           0.2 &                                           0.5 &                                            1.0 &                                            2.0 \\
\midrule
  \textbf{PostNet} &                 $72.0\cdot\bm{82.7}\cdot88.0$ &                 $35.1\cdot\bm{56.8}\cdot97.3$ &  $32.0\cdot{\color{blue} \bm{65.8}}\cdot99.8$ &                $30.7\cdot\bm{50.7}\cdot100.0$ &                 $30.7\cdot\bm{46.5}\cdot100.0$ &                 $30.7\cdot\bm{51.7}\cdot100.0$ \\
 \textbf{PriorNet} &                 $50.3\cdot\bm{53.1}\cdot55.9$ &                 $33.6\cdot\bm{43.7}\cdot65.9$ &                 $31.3\cdot\bm{39.8}\cdot69.4$ &                 $31.3\cdot\bm{48.3}\cdot98.2$ &                  $30.7\cdot\bm{44.5}\cdot99.9$ &                 $30.7\cdot\bm{46.4}\cdot100.0$ \\
    \textbf{DDNet} &                 $72.0\cdot\bm{75.8}\cdot79.8$ &                 $35.6\cdot\bm{46.2}\cdot70.0$ &                 $32.9\cdot\bm{50.1}\cdot86.7$ &                 $31.1\cdot\bm{58.8}\cdot98.6$ &                 $30.7\cdot\bm{59.3}\cdot100.0$ &                 $30.7\cdot\bm{44.6}\cdot100.0$ \\
    \textbf{EvNet} &  $79.5\cdot{\color{blue} \bm{87.1}}\cdot92.8$ &  $34.1\cdot{\color{blue} \bm{58.8}}\cdot95.2$ &                 $32.6\cdot\bm{61.2}\cdot96.9$ &  $31.7\cdot{\color{blue} \bm{60.5}}\cdot98.7$ &  $30.7\cdot{\color{blue} \bm{62.4}}\cdot100.0$ &  $30.7\cdot{\color{blue} \bm{57.6}}\cdot100.0$ \\
\bottomrule
\end{tabular}

			\midrule
			%& & \multicolumn{6}{c}{\textbf{OOD-Attack}} \\
            \multirow{4}{1.3cm}{Smoothed + adv. label attacks} &  
\textbf{PostNet} &  - &  
$35.0\cdot\bm{{\color{black}58.5}}\cdot97.8$ &     
$31.2\cdot\bm{{\color{blue}46.6}}\cdot97.2$ &   
$30.8\cdot\bm{{\color{blue}57.7}}\cdot\phantom{0}99.7$ &  
$30.7\cdot\bm{{\color{blue}50.2}}\cdot100.0$ & 
$30.7\cdot\bm{{\color{blue}51.5}}\cdot100.0$ \\
 & \textbf{PriorNet} &  - & 
 $31.6\cdot\bm{{\color{black}37.3}}\cdot59.3$ &    
 $33.2\cdot\bm{{\color{blue}52.7}}\cdot85.8$ & 
 $30.7\cdot\bm{{\color{blue}57.8}}\cdot\phantom{0}98.7$ & 
 $30.7\cdot\bm{{\color{blue}40.1}}\cdot\phantom{0}99.9$ & 
 $30.9\cdot\bm{{\color{blue}53.8}}\cdot\phantom{0}96.8$ \\
   & \textbf{DDNet} &  - & 
   $36.4\cdot\bm{{\color{black}50.2}}\cdot78.9$ &  
   $32.1\cdot\bm{{\color{black}41.5}}\cdot70.4$ & 
   $30.9\cdot\bm{{\color{blue}56.2}}\cdot100.0$ & 
   $30.7\cdot\bm{{\color{blue}49.3}}\cdot100.0$ &
   $30.7\cdot\bm{{\color{blue}55.1}}\cdot100.0$ \\
&    \textbf{EvNet} &  - &    
$47.2\cdot\bm{{\color{blue}61.1}}\cdot80.0$ &   
$32.4\cdot\bm{{\color{blue}59.1}}\cdot99.1$ &  
$30.7\cdot\bm{{\color{blue}45.0}}\cdot100.0$ &  
$30.7\cdot\bm{{\color{blue}63.2}}\cdot100.0$ & 
$30.8\cdot\bm{{\color{blue}38.0}}\cdot100.0$ \\

			\midrule
			%& & \multicolumn{6}{c}{\textbf{OOD-Attack}} \\
            \multirow{4}{1.3cm}{Smoothed + adv. uncert. attacks} &
\textbf{PostNet} &  - &  
$35.3\cdot\bm{{\color{black}56.4}}\cdot96.1$ &     
$34.5\cdot\bm{{\color{blue}59.0}}\cdot94.9$ &  
$30.7\cdot\bm{{\color{blue}46.8}}\cdot100.0$ &  
$30.7\cdot\bm{{\color{blue}57.8}}\cdot100.0$ &  
$30.7\cdot\bm{{\color{blue}43.2}}\cdot100.0$ \\
& \textbf{PriorNet} &  - &  
$31.9\cdot\bm{{\color{black}39.4}}\cdot65.5$ &     
$31.0\cdot\bm{{\color{blue}42.0}}\cdot88.6$ &  
$30.7\cdot\bm{{\color{blue}42.9}}\cdot\phantom{0}99.2$ & 
$30.7\cdot\bm{{\color{blue}48.4}}\cdot100.0$ & 
$30.7\cdot\bm{{\color{blue}47.1}}\cdot100.0$ \\
 &   \textbf{DDNet} &  - & 
 $40.2\cdot\bm{{\color{black}52.9}}\cdot76.5$ & 
 $36.4\cdot\bm{{\color{black}56.9}}\cdot83.9$ & 
 $31.1\cdot\bm{{\color{blue}51.5}}\cdot\phantom{0}97.3$ &  
 $31.0\cdot\bm{{\color{blue}57.0}}\cdot\phantom{0}97.8$ & 
 $30.7\cdot\bm{{\color{blue}49.1}}\cdot100.0$ \\
  &  \textbf{EvNet} &  - &     
  $34.9\cdot\bm{{\color{blue}64.8}}\cdot99.6$ &   
  $30.8\cdot\bm{{\color{blue}48.8}}\cdot99.8$ & 
  $30.7\cdot\bm{{\color{blue}66.1}}\cdot100.0$ & 
  $30.9\cdot\bm{{\color{blue}41.6}}\cdot\phantom{0}93.6$ &
  $31.1\cdot\bm{{\color{blue}54.7}}\cdot100.0$ \\

			\bottomrule
		\end{tabular}}
	%\end{tiny}
\end{table*}


\begin{table*}[ht!]
	\centering
	\caption{OOD detection based on differential entropy under PGD uncertainty attacks against differential entropy on ID data and OOD data. Smoothed DBU models  on MNIST. Column format: guaranteed lowest performance $\cdot$ empirical performance $\cdot$ guaranteed highest performance (blue: normally/adversarially trained smooth classifier is more robust than the base model).}
	\label{tab:mnist_smooth_ooddetection}
	%\begin{tiny}
	\resizebox{\textwidth}{!}{
		\begin{tabular}{llccccccc}
			\toprule
			&\textbf{Att. Rad.} & 0.0 &   0.1 &  0.2 &  0.5 &  1.0 &  2.0 \\
			\midrule
			& & \multicolumn{6}{c}{\textbf{ID-Attack}} \\
               \multirow{4}{1.2cm}{Smoothed models} & 
   \textbf{PostNet} & 
   $59.9\cdot\bm{{\color{black}91.1}}\cdot98.6$ &    
   $61.2\cdot\bm{{\color{blue}97.7}}\cdot99.6$ &   
   $64.8\cdot\bm{{\color{blue}94.7}}\cdot99.7$ & 
   $31.6\cdot\bm{{\color{black}64.9}}\cdot\phantom{0}99.7$ & 
   $30.7\cdot\bm{{\color{black}63.2}}\cdot100.0$ & 
   $30.7\cdot\bm{{\color{blue}70.5}}\cdot100.0$ \\
& \textbf{PriorNet} &     
$99.8\cdot\bm{{\color{blue}99.8}}\cdot99.8$ &    
$99.4\cdot\bm{{\color{blue}99.8}}\cdot99.9$ &  
$98.3\cdot\bm{{\color{blue}99.6}}\cdot99.9$ & 
$48.5\cdot\bm{{\color{black}91.9}}\cdot\phantom{0}99.9$ & 
$31.1\cdot\bm{{\color{black}74.6}}\cdot\phantom{0}99.8$ & 
$30.7\cdot\bm{{\color{black}67.3}}\cdot100.0$ \\
 &   \textbf{DDNet} & 
 $98.5\cdot\bm{{\color{black}98.6}}\cdot98.7$ & 
 $95.0\cdot\bm{{\color{black}97.6}}\cdot98.9$ & 
 $74.7\cdot\bm{{\color{black}92.0}}\cdot98.2$ & 
 $31.4\cdot\bm{{\color{black}52.0}}\cdot\phantom{0}98.5$ &  
 $30.7\cdot\bm{{\color{black}52.0}}\cdot100.0$ & 
 $30.7\cdot\bm{{\color{black}41.1}}\cdot100.0$ \\
  &  \textbf{EvNet} &     
  $85.7\cdot\bm{{\color{blue}87.5}}\cdot89.2$ &   
  $68.9\cdot\bm{{\color{blue}90.4}}\cdot97.7$ &   
  $42.5\cdot\bm{{\color{blue}90.2}}\cdot99.6$ &
  $30.7\cdot\bm{{\color{blue}69.8}}\cdot100.0$ &  
  $30.7\cdot\bm{{\color{black}50.3}}\cdot100.0$ & 
  $30.7\cdot\bm{{\color{black}45.6}}\cdot100.0$ \\

			\midrule
			%& & \multicolumn{6}{c}{\textbf{ID-Attack}} \\
              \multirow{4}{1.2cm}{Smoothed + adv. w. label attacks} &  
  \textbf{PostNet} &  - &     
  $84.3\cdot\bm{{\color{blue}96.2}}\cdot\phantom{0}99.3$ &  
  $50.4\cdot\bm{{\color{black}89.2}}\cdot\phantom{0}99.5$ &  
  $30.9\cdot\bm{{\color{black}46.2}}\cdot\phantom{0}99.4$ & 
  $30.7\cdot\bm{{\color{black}46.9}}\cdot100.0$ &  
  $30.7\cdot\bm{{\color{blue}62.2}}\cdot100.0$ \\
 & \textbf{PriorNet} &  - &   
 $99.7\cdot\bm{{\color{blue}99.9}}\cdot100.0$ &   
 $98.7\cdot\bm{{\color{blue}99.8}}\cdot100.0$ &   
 $83.3\cdot\bm{{\color{blue}99.1}}\cdot100.0$ & 
 $30.7\cdot\bm{{\color{black}82.6}}\cdot100.0$ & 
 $30.7\cdot\bm{{\color{black}64.8}}\cdot100.0$ \\
   & \textbf{DDNet} &  - &  
   $93.6\cdot\bm{{\color{black}96.9}}\cdot\phantom{0}98.5$ & 
   $71.2\cdot\bm{{\color{black}89.1}}\cdot\phantom{0}96.9$ &  
   $32.3\cdot\bm{{\color{black}50.3}}\cdot\phantom{0}99.0$ & 
   $30.7\cdot\bm{{\color{black}50.7}}\cdot100.0$ &  
   $30.7\cdot\bm{{\color{black}55.7}}\cdot100.0$ \\
&    \textbf{EvNet} &  - &   
$58.2\cdot\bm{{\color{blue}84.4}}\cdot\phantom{0}94.3$ &   
$40.9\cdot\bm{{\color{blue}87.4}}\cdot\phantom{0}99.2$ &  
$30.7\cdot\bm{{\color{black}59.4}}\cdot100.0$ & 
$30.7\cdot\bm{{\color{black}40.3}}\cdot100.0$ &    
$30.7\cdot\bm{{\color{blue}53.2}}\cdot100.0$ \\

			\midrule
			%& & \multicolumn{6}{c}{\textbf{ID-Attack}} \\
              \multirow{4}{1.2cm}{Smoothed + adv. w. uncert. attacks} &  
  \textbf{PostNet} &  - &    
  $58.9\cdot\bm{{\color{blue}96.1}}\cdot99.3$ &   
  $59.7\cdot\bm{{\color{blue}96.1}}\cdot99.9$ &   
  $31.2\cdot\bm{{\color{black}48.2}}\cdot\phantom{0}95.7$ & 
  $30.7\cdot\bm{{\color{black}42.0}}\cdot100.0$ & 
  $30.7\cdot\bm{{\color{blue}56.9}}\cdot100.0$ \\
 & \textbf{PriorNet} &  - &  
 $99.9\cdot\bm{{\color{blue}100.0}}\cdot100.0$ &  
 $96.5\cdot\bm{{\color{blue}99.2}}\cdot99.9$ & 
 $49.2\cdot\bm{{\color{black}96.9}}\cdot100.0$ &
 $31.3\cdot\bm{{\color{black}88.1}}\cdot100.0$ &  
 $30.7\cdot\bm{{\color{blue}77.8}}\cdot100.0$ \\
   & \textbf{DDNet} &  - &
   $95.0\cdot\bm{{\color{black}97.5}}\cdot98.8$ & 
   $80.6\cdot\bm{{\color{black}94.1}}\cdot98.7$ &
   $31.7\cdot\bm{{\color{black}55.6}}\cdot\phantom{0}98.6$ & 
   $30.7\cdot\bm{{\color{black}52.0}}\cdot100.0$ & 
   $30.7\cdot\bm{{\color{black}47.6}}\cdot100.0$ \\
&    \textbf{EvNet} &  - &  
$66.5\cdot\bm{{\color{blue}91.3}}\cdot98.1$ & 
$48.1\cdot\bm{{\color{blue}84.1}}\cdot97.6$ & 
$30.8\cdot\bm{{\color{black}49.7}}\cdot\phantom{0}99.9$ & 
$30.7\cdot\bm{{\color{black}37.9}}\cdot100.0$ &     
$30.8\cdot\bm{{\color{blue}63.5}}\cdot100.0$ \\

			\midrule
			\midrule
			& & \multicolumn{6}{c}{\textbf{OOD-Attack}} \\
              \multirow{4}{1.2cm}{Smoothed models} &  
  \textbf{PostNet} &  
  $59.0\cdot\bm{{\color{black}91.2}}\cdot97.7$ &    
  $57.8\cdot\bm{{\color{blue}97.2}}\cdot99.6$ &     
  $61.4\cdot\bm{{\color{blue}93.8}}\cdot\phantom{0}99.6$ &   
  $31.5\cdot\bm{{\color{black}58.9}}\cdot\phantom{0}99.5$ &  
  $30.7\cdot\bm{{\color{black}51.5}}\cdot100.0$ &  
  $30.7\cdot\bm{{\color{blue}53.5}}\cdot100.0$ \\
 & \textbf{PriorNet} &    
 $99.7\cdot\bm{{\color{blue}99.8}}\cdot99.8$ &   
 $99.4\cdot\bm{{\color{blue}99.8}}\cdot99.9$ & 
 $98.4\cdot\bm{{\color{blue}99.7}}\cdot100.0$ &
 $60.7\cdot\bm{{\color{black}96.8}}\cdot100.0$ & 
 $33.0\cdot\bm{{\color{black}88.9}}\cdot100.0$ &   
 $30.7\cdot\bm{{\color{blue}87.7}}\cdot100.0$ \\
   & \textbf{DDNet} &
   $98.4\cdot\bm{{\color{black}98.5}}\cdot98.7$ & 
   $94.2\cdot\bm{{\color{black}97.2}}\cdot98.7$ & 
   $72.1\cdot\bm{{\color{black}90.5}}\cdot\phantom{0}97.8$ &  
   $31.6\cdot\bm{{\color{black}52.3}}\cdot\phantom{0}98.1$ &
   $30.7\cdot\bm{{\color{black}51.7}}\cdot100.0$ &  
   $30.7\cdot\bm{{\color{black}37.7}}\cdot100.0$ \\
&    \textbf{EvNet} &   
$83.9\cdot\bm{{\color{blue}85.7}}\cdot88.0$ &  
$63.5\cdot\bm{{\color{blue}88.6}}\cdot97.9$ &  
$40.1\cdot\bm{{\color{blue}87.7}}\cdot\phantom{0}99.6$ &  
$30.8\cdot\bm{{\color{blue}68.9}}\cdot100.0$ & 
$30.7\cdot\bm{{\color{blue}43.3}}\cdot100.0$ &   
$30.7\cdot\bm{{\color{blue}36.8}}\cdot100.0$ \\

			\midrule
			%& & \multicolumn{6}{c}{\textbf{OOD-Attack}} \\
            \multirow{4}{1.2cm}{Smoothed + adv. w. label attacks} &  
\textbf{PostNet} &  - &    
$84.7\cdot\bm{{\color{blue}96.1}}\cdot\phantom{0}99.4$ & 
$49.7\cdot\bm{{\color{black}89.1}}\cdot\phantom{0}99.5$ &
$30.9\cdot\bm{{\color{black}45.6}}\cdot\phantom{0}99.3$ &
$30.7\cdot\bm{{\color{black}45.8}}\cdot100.0$ &   
$30.7\cdot\bm{{\color{blue}69.1}}\cdot100.0$ \\
& \textbf{PriorNet} &  - &    
$99.7\cdot\bm{{\color{blue}99.9}}\cdot100.0$ &   
$98.7\cdot\bm{{\color{blue}99.8}}\cdot100.0$ &   
$86.8\cdot\bm{{\color{blue}99.5}}\cdot100.0$ &  
$30.9\cdot\bm{{\color{black}93.2}}\cdot100.0$ & 
$30.7\cdot\bm{{\color{black}81.4}}\cdot100.0$ \\
 &   \textbf{DDNet} &  - &  
 $93.9\cdot\bm{{\color{black}97.0}}\cdot\phantom{0}98.6$ & 
 $72.0\cdot\bm{{\color{black}89.4}}\cdot\phantom{0}97.0$ & 
 $33.0\cdot\bm{{\color{black}52.4}}\cdot\phantom{0}98.8$ & 
 $30.7\cdot\bm{{\color{black}51.5}}\cdot100.0$ &  
 $30.7\cdot\bm{{\color{black}60.1}}\cdot100.0$ \\
  &  \textbf{EvNet} &  - &    
  $59.5\cdot\bm{{\color{blue}85.3}}\cdot\phantom{0}94.6$ &   
  $40.7\cdot\bm{{\color{blue}86.9}}\cdot\phantom{0}99.2$ &   
  $30.7\cdot\bm{{\color{blue}57.4}}\cdot100.0$ &   
  $30.7\cdot\bm{{\color{blue}39.2}}\cdot100.0$ &     
  $30.7\cdot\bm{{\color{blue}49.0}}\cdot100.0$ \\

			\midrule
			%& & \multicolumn{6}{c}{\textbf{OOD-Attack}} \\
              \multirow{4}{1.2cm}{Smoothed + adv. uncert. attacks} &  
  \textbf{PostNet} &  - &    
  $55.7\cdot\bm{{\color{blue}96.1}}\cdot\phantom{0}99.3$ &  
  $58.4\cdot\bm{{\color{blue}95.7}}\cdot99.8$ & 
  $31.1\cdot\bm{{\color{black}44.2}}\cdot\phantom{0}93.1$ & 
  $30.7\cdot\bm{{\color{black}41.2}}\cdot100.0$ & 
  $30.7\cdot\bm{{\color{black}48.8}}\cdot100.0$ \\
& \textbf{PriorNet} &  - & 
$99.9\cdot\bm{{\color{blue}100.0}}\cdot100.0$ & 
$97.0\cdot\bm{{\color{blue}99.3}}\cdot99.9$ &
$61.0\cdot\bm{{\color{blue}98.4}}\cdot100.0$ &
$33.2\cdot\bm{{\color{black}94.4}}\cdot100.0$ &
$30.7\cdot\bm{{\color{blue}90.2}}\cdot100.0$ \\
 &   \textbf{DDNet} &  - & 
 $95.3\cdot\bm{{\color{black}97.6}}\cdot\phantom{0}98.9$ &
 $82.2\cdot\bm{{\color{black}94.5}}\cdot98.7$ &  
 $32.1\cdot\bm{{\color{black}56.6}}\cdot\phantom{0}98.5$ & 
 $30.7\cdot\bm{{\color{black}48.6}}\cdot100.0$ & 
 $30.7\cdot\bm{{\color{black}42.9}}\cdot100.0$ \\
  &  \textbf{EvNet} &  - &   
  $65.2\cdot\bm{{\color{blue}90.4}}\cdot\phantom{0}98.0$ &  
  $46.8\cdot\bm{{\color{blue}83.4}}\cdot97.3$ &  
  $30.8\cdot\bm{{\color{blue}48.8}}\cdot\phantom{0}99.9$ &   
  $30.7\cdot\bm{{\color{blue}36.3}}\cdot100.0$ &    
  $30.8\cdot\bm{{\color{blue}60.1}}\cdot100.0$ \\

    \bottomrule
		\end{tabular}}
	%\end{tiny}
\end{table*}







\begin{table*}[ht!]
	\centering
	\caption{OOD detection based on differential entropy under PGD uncertainty attacks against differential entropy on ID data and OOD data. Smoothed DBU models  on Sensorless. Column format: guaranteed lowest performance $\cdot$ empirical performance $\cdot$ guaranteed highest performance (blue: normally/adversarially trained smooth classifier is more robust than the base model).}
	\label{tab:sensorless_smooth_ooddetection}
	%\begin{tiny}
	\resizebox{\textwidth}{!}{
		\begin{tabular}{llccccccc}
			\toprule
			& \textbf{Att. Rad.} & 0.0 &   0.1 &  0.2 &  0.5 &  1.0 &  2.0 \\
			\midrule
			& & \multicolumn{6}{c}{\textbf{ID-Attack}} \\
               \multirow{4}{1.2cm}{Smoothed models} & 
   \textbf{PostNet} &    
   $49.3\cdot\bm{{\color{blue}90.4}}\cdot99.8$ & 
   $30.7\cdot\bm{{\color{blue}49.2}}\cdot100.0$ & 
   $30.7\cdot\bm{{\color{black}36.0}}\cdot100.0$ &  
   $49.2\cdot\bm{{\color{blue}50.0}}\cdot\phantom{0}74.9$ & 
   $50.0\cdot\bm{{\color{blue}50.0}}\cdot\phantom{0}50.0$ &   
   $50.0\cdot\bm{{\color{blue}50.0}}\cdot\phantom{0}50.0$ \\
 & \textbf{PriorNet} &     
 $31.2\cdot\bm{{\color{blue}39.0}}\cdot66.9$ & 
 $30.7\cdot\bm{{\color{blue}35.5}}\cdot100.0$ &    
 $30.7\cdot\bm{{\color{blue}38.9}}\cdot100.0$ &
 $30.7\cdot\bm{{\color{blue}46.2}}\cdot100.0$ &
 $30.7\cdot\bm{{\color{blue}62.7}}\cdot100.0$ &
 $30.7\cdot\bm{{\color{blue}51.3}}\cdot100.0$ \\
   & \textbf{DDNet} & 
   $31.0\cdot\bm{{\color{black}31.5}}\cdot32.7$ & 
   $30.7\cdot\bm{{\color{blue}30.8}}\cdot100.0$ & 
   $30.7\cdot\bm{{\color{blue}31.8}}\cdot100.0$ &  
   $30.7\cdot\bm{{\color{blue}53.6}}\cdot100.0$ &
   $30.7\cdot\bm{{\color{blue}43.9}}\cdot100.0$ &
   $30.7\cdot\bm{{\color{blue}40.5}}\cdot100.0$ \\
&    \textbf{EvNet} &
$33.6\cdot\bm{{\color{black}55.2}}\cdot91.3$ &
$30.7\cdot\bm{{\color{blue}44.2}}\cdot100.0$ &   
$30.7\cdot\bm{{\color{blue}43.8}}\cdot100.0$ & 
$30.7\cdot\bm{{\color{blue}39.3}}\cdot100.0$ &  
$30.8\cdot\bm{{\color{blue}51.6}}\cdot100.0$ & 
$32.4\cdot\bm{{\color{blue}50.0}}\cdot100.0$ \\

			\midrule
			%& & \multicolumn{6}{c}{\textbf{ID-Attack}} \\
               \multirow{4}{1.2cm}{Smoothed + adv. label attacks} &
   \textbf{PostNet} &  - &
   $30.7\cdot\bm{{\color{blue}62.4}}\cdot100.0$ & 
   $30.7\cdot\bm{{\color{blue}39.2}}\cdot100.0$ & 
   $50.0\cdot\bm{{\color{blue}50.0}}\cdot\phantom{0}50.0$ &  
   $50.0\cdot\bm{{\color{blue}50.0}}\cdot\phantom{0}50.0$ & 
   $50.0\cdot\bm{{\color{blue}50.0}}\cdot\phantom{0}50.0$ \\
 & \textbf{PriorNet} &  - &  
 $30.7\cdot\bm{{\color{blue}30.9}}\cdot100.0$ &
 $30.7\cdot\bm{{\color{blue}32.4}}\cdot100.0$ &
 $30.7\cdot\bm{{\color{blue}31.0}}\cdot100.0$ &
 $30.8\cdot\bm{{\color{blue}30.7}}\cdot100.0$ & 
 $38.2\cdot\bm{{\color{blue}48.9}}\cdot100.0$ \\
  &  \textbf{DDNet} &  - &
  $30.7\cdot\bm{{\color{blue}32.2}}\cdot100.0$ & 
  $30.7\cdot\bm{{\color{blue}30.9}}\cdot100.0$ & 
  $30.7\cdot\bm{{\color{blue}37.1}}\cdot100.0$ &  
  $30.7\cdot\bm{{\color{blue}42.1}}\cdot100.0$ 
  &  $30.7\cdot\bm{{\color{blue}37.7}}\cdot100.0$ \\
   & \textbf{EvNet} &  - &  
   $30.7\cdot\bm{{\color{blue}48.9}}\cdot100.0$ &
   $30.7\cdot\bm{{\color{blue}34.0}}\cdot100.0$ & 
   $30.7\cdot\bm{{\color{blue}35.6}}\cdot100.0$ & 
   $30.7\cdot\bm{{\color{blue}33.6}}\cdot100.0$ & 
   $30.7\cdot\bm{{\color{blue}50.0}}\cdot100.0$ \\

			\midrule
			%& & \multicolumn{6}{c}{\textbf{ID-Attack}} \\
               \multirow{4}{1.2cm}{Smoothed + adv. w. uncert. attacks} & 
   \textbf{PostNet} &  - & 
   $30.7\cdot\bm{{\color{blue}46.0}}\cdot100.0$ &  
   $30.7\cdot\bm{{\color{blue}46.6}}\cdot100.0$ &  
   $50.0\cdot\bm{{\color{blue}50.0}}\cdot\phantom{0}50.0$ &  
   $50.0\cdot\bm{{\color{blue}50.0}}\cdot\phantom{0}50.0$ &  
   $50.0\cdot\bm{{\color{blue}50.0}}\cdot\phantom{0}50.0$ \\
 & \textbf{PriorNet} &  - & 
 $30.7\cdot\bm{{\color{blue}35.8}}\cdot100.0$ &
 $30.7\cdot\bm{{\color{blue}32.1}}\cdot100.0$ & 
 $30.7\cdot\bm{{\color{blue}81.6}}\cdot100.0$ &
 $30.8\cdot\bm{{\color{blue}41.7}}\cdot100.0$ &  
 $30.7\cdot\bm{{\color{blue}61.9}}\cdot100.0$ \\
   & \textbf{DDNet} &  - & 
   $30.7\cdot\bm{{\color{blue}32.8}}\cdot100.0$ &  
   $30.7\cdot\bm{{\color{blue}31.0}}\cdot100.0$ & 
   $30.7\cdot\bm{{\color{blue}31.8}}\cdot100.0$ & 
   $30.7\cdot\bm{{\color{blue}43.7}}\cdot100.0$ &
   $30.7\cdot\bm{{\color{blue}34.7}}\cdot100.0$ \\
&    \textbf{EvNet} &  - &
$30.7\cdot\bm{{\color{blue}31.0}}\cdot100.0$ & 
$30.7\cdot\bm{{\color{blue}49.6}}\cdot100.0$ & 
$30.7\cdot\bm{{\color{blue}47.7}}\cdot100.0$ & 
$30.7\cdot\bm{{\color{blue}42.6}}\cdot100.0$ & 
$30.7\cdot\bm{{\color{blue}50.0}}\cdot100.0$ \\

			\midrule
			\midrule
			& &\multicolumn{6}{c}{\textbf{OOD-Attack}} \\
              \multirow{4}{1.2cm}{Smoothed models} &  
  \textbf{PostNet} &     
  $49.3\cdot\bm{{\color{blue}90.4}}\cdot99.8$ & 
  $30.8\cdot\bm{{\color{blue}76.4}}\cdot100.0$ &  
  $30.7\cdot\bm{{\color{blue}61.3}}\cdot100.0$ &  
  $47.7\cdot\bm{{\color{blue}50.0}}\cdot\phantom{0}75.1$ &  
  $50.0\cdot\bm{{\color{blue}50.0}}\cdot\phantom{0}50.0$ &  
  $50.0\cdot\bm{{\color{black}50.0}}\cdot\phantom{0}50.0$ \\
& \textbf{PriorNet} &    
$31.2\cdot\bm{{\color{blue}39.0}}\cdot66.9$ & 
$30.7\cdot\bm{{\color{blue}33.9}}\cdot100.0$ &
$30.7\cdot\bm{{\color{blue}34.3}}\cdot100.0$ & 
$30.7\cdot\bm{{\color{blue}37.0}}\cdot100.0$ & 
$30.7\cdot\bm{{\color{blue}74.0}}\cdot100.0$ &  
$30.9\cdot\bm{{\color{blue}78.1}}\cdot100.0$ \\
  &  \textbf{DDNet} &  
  $31.0\cdot\bm{{\color{black}31.5}}\cdot32.7$ & 
  $30.7\cdot\bm{{\color{blue}30.7}}\cdot100.0$ & 
  $30.7\cdot\bm{{\color{blue}31.8}}\cdot100.0$ & 
  $30.7\cdot\bm{{\color{blue}47.7}}\cdot100.0$ &  
  $30.7\cdot\bm{{\color{blue}43.8}}\cdot100.0$ &   
  $30.7\cdot\bm{{\color{blue}52.5}}\cdot100.0$ \\
   & \textbf{EvNet} & 
   $33.6\cdot\bm{{\color{black}55.2}}\cdot91.2$ & 
   $30.7\cdot\bm{{\color{blue}54.7}}\cdot100.0$ & 
   $30.7\cdot\bm{{\color{blue}54.0}}\cdot100.0$ & 
   $30.7\cdot\bm{{\color{blue}51.0}}\cdot100.0$ & 
   $30.7\cdot\bm{{\color{blue}45.2}}\cdot100.0$ & 
   $31.7\cdot\bm{{\color{blue}50.0}}\cdot100.0$ \\

			\midrule
			%& & \multicolumn{6}{c}{\textbf{OOD-Attack}} \\
              \multirow{4}{1.2cm}{Smoothed + adv. w. label attacks} & 
  \textbf{PostNet} &  - & 
  $30.7\cdot\bm{{\color{blue}82.2}}\cdot100.0$ &  
  $30.7\cdot\bm{{\color{blue}61.4}}\cdot100.0$ &  
  $50.0\cdot\bm{{\color{blue}50.0}}\cdot\phantom{0}50.0$ &  
  $50.0\cdot\bm{{\color{blue}50.0}}\cdot\phantom{0}50.0$ & 
  $50.0\cdot\bm{{\color{black}50.0}}\cdot\phantom{0}50.0$ \\
& \textbf{PriorNet} &  - & 
$30.7\cdot\bm{{\color{blue}31.2}}\cdot100.0$ &  
$30.7\cdot\bm{{\color{blue}31.4}}\cdot\phantom{0}99.9$ &  
$30.7\cdot\bm{{\color{blue}30.8}}\cdot100.0$ & 
$30.8\cdot\bm{{\color{blue}30.7}}\cdot100.0$ &
$33.8\cdot\bm{{\color{blue}34.0}}\cdot100.0$ \\
 &   \textbf{DDNet} &  - & 
 $30.7\cdot\bm{{\color{blue}32.2}}\cdot100.0$ &   
 $30.7\cdot\bm{{\color{blue}30.8}}\cdot100.0$ &  
 $30.7\cdot\bm{{\color{blue}33.6}}\cdot100.0$ & 
 $30.7\cdot\bm{{\color{blue}46.9}}\cdot100.0$ &  
 $30.7\cdot\bm{{\color{blue}40.3}}\cdot100.0$ \\
  &  \textbf{EvNet} &  - &
  $30.8\cdot\bm{{\color{blue}75.3}}\cdot100.0$ &  
  $30.7\cdot\bm{{\color{black}31.6}}\cdot100.0$ &
  $30.7\cdot\bm{{\color{blue}42.1}}\cdot100.0$ &  
  $30.7\cdot\bm{{\color{blue}38.7}}\cdot100.0$ &  
  $30.7\cdot\bm{{\color{blue}50.0}}\cdot100.0$ \\

			\midrule
			%& & \multicolumn{6}{c}{\textbf{OOD-Attack}} \\
              \multirow{4}{1.2cm}{Smoothed + adv. uncert. attacks} &  
  \textbf{PostNet} &  - &     
  $30.7\cdot\bm{{\color{blue}73.7}}\cdot100.0$ & 
  $30.7\cdot\bm{{\color{blue}61.6}}\cdot100.0$ &  
  $50.0\cdot\bm{{\color{blue}50.0}}\cdot\phantom{0}50.0$ &  
  $50.0\cdot\bm{{\color{blue}50.0}}\cdot\phantom{0}50.0$ & 
  $50.0\cdot\bm{{\color{black}50.0}}\cdot\phantom{0}50.0$ \\
& \textbf{PriorNet} &  - &    
$30.7\cdot\bm{{\color{blue}35.9}}\cdot100.0$ &
$30.7\cdot\bm{{\color{blue}30.7}}\cdot100.0$ &  
$30.7\cdot\bm{{\color{blue}39.4}}\cdot100.0$ & 
$30.7\cdot\bm{{\color{blue}36.6}}\cdot100.0$ &   
$30.7\cdot\bm{{\color{blue}97.6}}\cdot100.0$ \\
 &   \textbf{DDNet} &  - & 
 $30.7\cdot\bm{{\color{blue}32.1}}\cdot100.0$ &  
 $30.7\cdot\bm{{\color{blue}30.8}}\cdot100.0$ & 
 $30.7\cdot\bm{{\color{blue}32.2}}\cdot100.0$ &  
 $30.7\cdot\bm{{\color{blue}50.7}}\cdot100.0$ &  
 $30.7\cdot\bm{{\color{blue}39.8}}\cdot100.0$ \\
  &  \textbf{EvNet} &  - &  
  $30.7\cdot\bm{{\color{black}31.3}}\cdot100.0$ &
  $30.8\cdot\bm{{\color{blue}39.7}}\cdot100.0$ &
  $30.7\cdot\bm{{\color{blue}52.2}}\cdot100.0$ &
  $30.7\cdot\bm{{\color{blue}42.3}}\cdot100.0$ & 
  $30.7\cdot\bm{{\color{blue}50.0}}\cdot100.0$ \\
			
        \bottomrule
		\end{tabular}}
	%\end{tiny}
\end{table*}



\begin{table*}[ht!]
	\centering
	\caption{OOD detection based on differential entropy under PGD uncertainty attacks against differential entropy on ID data and OOD data. Smoothed DBU models  on Segment. Column format: guaranteed lowest performance $\cdot$ empirical performance $\cdot$ guaranteed highest performance (blue: normally/adversarially trained smooth classifier is more robust than the base model).}
	\label{tab:segment_smooth_ooddetection}
	%\begin{tiny}
	\resizebox{\textwidth}{!}{
		\begin{tabular}{llccccccc}
			\toprule
			& \textbf{Att. Rad.} & 0.0 &   0.1 &  0.2 &  0.5 &  1.0 &  2.0 \\
			\midrule
			& & \multicolumn{6}{c}{\textbf{ID-Attack}} \\
              \multirow{4}{1.2cm}{Smoothed models} &  
  \textbf{PostNet} &   
  $99.6\cdot\bm{{\color{blue}99.9}}\cdot99.9$ &   
  $33.0\cdot\bm{{\color{blue}83.0}}\cdot100.0$ & 
  $30.8\cdot\bm{{\color{black}43.8}}\cdot100.0$ & 
  $30.8\cdot\bm{{\color{black}31.7}}\cdot100.0$ &  
  $30.8\cdot\bm{{\color{blue}40.8}}\cdot100.0$ &   
  $41.4\cdot\bm{{\color{blue}50.0}}\cdot\phantom{0}50.2$ \\
 & \textbf{PriorNet} & 
 $30.8\cdot\bm{{\color{black}31.0}}\cdot31.4$ & 
 $30.8\cdot\bm{{\color{black}30.8}}\cdot\phantom{0}42.6$ &  
 $30.8\cdot\bm{{\color{black}30.8}}\cdot\phantom{0}95.5$ &  
 $30.8\cdot\bm{{\color{blue}33.1}}\cdot100.0$ &  
 $30.8\cdot\bm{{\color{blue}76.4}}\cdot100.0$ &  
 $30.8\cdot\bm{{\color{blue}78.7}}\cdot100.0$ \\
   & \textbf{DDNet} & 
   $30.8\cdot\bm{{\color{black}30.8}}\cdot30.8$ & 
   $30.8\cdot\bm{{\color{black}30.8}}\cdot\phantom{0}32.1$ & 
   $30.8\cdot\bm{30.8}\cdot\phantom{0}69.4$ &                
   $30.8\cdot\bm{30.8}\cdot100.0$ & 
   $30.8\cdot\bm{{\color{blue}31.0}}\cdot100.0$ & 
   $30.8\cdot\bm{{\color{blue}33.4}}\cdot100.0$ \\
&    \textbf{EvNet} &    
$94.9\cdot\bm{{\color{blue}97.2}}\cdot98.3$ &   
$31.1\cdot\bm{{\color{blue}75.8}}\cdot\phantom{0}99.9$ &  
$30.8\cdot\bm{{\color{blue}74.2}}\cdot100.0$ & 
$30.8\cdot\bm{{\color{blue}62.9}}\cdot100.0$ & 
$30.8\cdot\bm{{\color{blue}58.1}}\cdot100.0$ & 
$30.8\cdot\bm{{\color{blue}43.4}}\cdot100.0$ \\

			\midrule
			%& & \multicolumn{6}{c}{\textbf{ID-Attack}} \\
              \multirow{4}{1.2cm}{Smoothed + adv. w. label attacks} &  
  \textbf{PostNet} &  - & 
  $31.0\cdot\bm{{\color{black}70.9}}\cdot100.0$ &  
  $30.8\cdot\bm{{\color{black}47.1}}\cdot100.0$ &  
  $30.8\cdot\bm{{\color{blue}85.0}}\cdot100.0$ &  
  $30.8\cdot\bm{{\color{blue}50.0}}\cdot100.0$ & 
  $50.0\cdot\bm{{\color{blue}50.0}}\cdot\phantom{0}50.0$ \\
& \textbf{PriorNet} &  - &  
$30.8\cdot\bm{{\color{black}30.8}}\cdot\phantom{0}46.0$ &  
$30.8\cdot\bm{{\color{black}30.8}}\cdot\phantom{0}32.7$ &  
$30.8\cdot\bm{30.8}\cdot100.0$ &               
$30.8\cdot\bm{30.8}\cdot100.0$ &             
$30.9\cdot\bm{30.8}\cdot100.0$ \\
 &   \textbf{DDNet} &  - &   
 $30.8\cdot\bm{{\color{black}30.8}}\cdot\phantom{0}30.8$ &       
 $30.8\cdot\bm{30.8}\cdot\phantom{0}79.5$ &           
 $30.8\cdot\bm{30.8}\cdot100.0$ &              
 $30.8\cdot\bm{30.8}\cdot100.0$ & 
 $30.8\cdot\bm{{\color{blue}57.3}}\cdot100.0$ \\
  &  \textbf{EvNet} &  - &  
  $36.3\cdot\bm{{\color{blue}94.3}}\cdot100.0$ &  
  $30.8\cdot\bm{{\color{black}32.2}}\cdot100.0$ & 
  $30.8\cdot\bm{{\color{blue}50.2}}\cdot100.0$ & 
  $30.8\cdot\bm{{\color{blue}93.9}}\cdot100.0$ &  
  $30.8\cdot\bm{{\color{blue}56.3}}\cdot100.0$ \\

			\midrule
			%& & \multicolumn{6}{c}{\textbf{ID-Attack}} \\
              \multirow{4}{1.2cm}{Smoothed + adv. uncert. attacks} &  
  \textbf{PostNet} &  - & 
  $30.8\cdot\bm{{\color{black}49.5}}\cdot100.0$ & 
  $30.8\cdot\bm{{\color{black}34.5}}\cdot100.0$ &  
  $30.8\cdot\bm{{\color{blue}96.1}}\cdot100.0$ & 
  $41.2\cdot\bm{{\color{blue}50.0}}\cdot\phantom{0}82.7$ &  
  $50.0\cdot\bm{{\color{blue}50.0}}\cdot\phantom{0}50.0$ \\
 & \textbf{PriorNet} &  - & 
 $30.8\cdot\bm{{\color{black}31.2}}\cdot\phantom{0}62.6$ &  
 $30.8\cdot\bm{{\color{black}30.8}}\cdot\phantom{0}32.9$ &   
 $30.8\cdot\bm{30.8}\cdot\phantom{0}88.9$ &               
 $30.8\cdot\bm{30.8}\cdot100.0$ &          
 $30.8\cdot\bm{30.8}\cdot100.0$ \\
   & \textbf{DDNet} &  - &  
   $30.8\cdot\bm{{\color{black}30.8}}\cdot\phantom{0}31.2$ &       
   $30.8\cdot\bm{30.8}\cdot\phantom{0}68.9$ &               
   $30.8\cdot\bm{30.8}\cdot100.0$ & 
   $30.8\cdot\bm{{\color{blue}30.9}}\cdot100.0$ & 
   $30.8\cdot\bm{{\color{blue}38.6}}\cdot100.0$ \\
&    \textbf{EvNet} &  - &    
$30.9\cdot\bm{{\color{blue}83.5}}\cdot100.0$ &  
$30.8\cdot\bm{{\color{blue}84.0}}\cdot100.0$ &
$30.8\cdot\bm{{\color{blue}98.6}}\cdot100.0$ &
$30.8\cdot\bm{{\color{blue}92.8}}\cdot100.0$ & 
$30.8\cdot\bm{{\color{blue}45.6}}\cdot100.0$ \\

			\midrule
			\midrule
			& & \multicolumn{6}{c}{\textbf{OOD-Attack}} \\
              \multirow{4}{1.2cm}{Smoothed models} &  
  \textbf{PostNet} &     
  $99.6\cdot\bm{{\color{blue}99.9}}\cdot99.9$ & 
  $31.3\cdot\bm{{\color{blue}95.2}}\cdot100.0$ &  
  $30.8\cdot\bm{{\color{black}48.7}}\cdot100.0$ & 
  $30.8\cdot\bm{{\color{black}34.0}}\cdot100.0$ &
  $30.8\cdot\bm{{\color{blue}41.0}}\cdot100.0$ &
  $41.8\cdot\bm{{\color{blue}50.0}}\cdot\phantom{0}50.2$ \\
 & \textbf{PriorNet} & 
 $30.8\cdot\bm{{\color{black}31.0}}\cdot31.4$ & 
 $30.8\cdot\bm{{\color{black}30.8}}\cdot\phantom{0}44.7$ & 
 $30.8\cdot\bm{{\color{black}30.8}}\cdot\phantom{0}86.3$ & 
 $30.8\cdot\bm{{\color{blue}30.9}}\cdot100.0$ & 
 $30.8\cdot\bm{{\color{blue}35.7}}\cdot100.0$ &
 $30.8\cdot\bm{{\color{blue}57.4}}\cdot100.0$ \\
   & \textbf{DDNet} &  
   $30.8\cdot\bm{{\color{black}30.8}}\cdot30.8$ &
   $30.8\cdot\bm{{\color{black}30.8}}\cdot\phantom{0}31.9$ & 
   $30.8\cdot\bm{30.8}\cdot\phantom{0}58.3$ &               
   $30.8\cdot\bm{30.8}\cdot100.0$ &             
   $30.8\cdot\bm{30.8}\cdot100.0$ &          
   $30.8\cdot\bm{30.8}\cdot100.0$ \\
&    \textbf{EvNet} &    
$94.9\cdot\bm{{\color{blue}97.2}}\cdot98.3$ & 
$31.4\cdot\bm{{\color{blue}92.5}}\cdot100.0$ &  
$30.8\cdot\bm{{\color{blue}94.2}}\cdot100.0$ &  
$30.8\cdot\bm{{\color{blue}80.4}}\cdot100.0$ & 
$30.8\cdot\bm{{\color{blue}70.2}}\cdot100.0$ & 
$30.8\cdot\bm{{\color{blue}48.2}}\cdot100.0$ \\

			\midrule
			%& & \multicolumn{6}{c}{\textbf{OOD-Attack}} \\
              \multirow{4}{1.2cm}{Smoothed + adv. w. label attacks} & 
  \textbf{PostNet} &  - &   
  $30.8\cdot\bm{{\color{blue}88.7}}\cdot100.0$ &    
  $30.8\cdot\bm{{\color{blue}70.9}}\cdot100.0$ &   
  $30.8\cdot\bm{{\color{blue}97.2}}\cdot100.0$ &  
  $30.8\cdot\bm{{\color{blue}50.0}}\cdot100.0$ & 
  $50.0\cdot\bm{{\color{blue}50.0}}\cdot\phantom{0}50.0$ \\
& \textbf{PriorNet} &  - &
$30.8\cdot\bm{{\color{black}30.9}}\cdot\phantom{0}47.2$ &  
$30.8\cdot\bm{{\color{black}30.8}}\cdot\phantom{0}32.5$ &   
$30.8\cdot\bm{30.8}\cdot\phantom{0}96.2$ &               
$30.8\cdot\bm{30.8}\cdot100.0$ &              
$30.9\cdot\bm{30.8}\cdot100.0$ \\
 &   \textbf{DDNet} &  - & 
 $30.8\cdot\bm{{\color{black}30.8}}\cdot\phantom{0}30.8$ &             
 $30.8\cdot\bm{30.8}\cdot\phantom{0}73.5$ &                 
 $30.8\cdot\bm{30.8}\cdot100.0$ &               
 $30.8\cdot\bm{30.8}\cdot100.0$ &  
 $30.8\cdot\bm{{\color{blue}34.3}}\cdot100.0$ \\
  &  \textbf{EvNet} &  - &   
  $35.9\cdot\bm{{\color{blue}95.9}}\cdot100.0$ & 
  $30.8\cdot\bm{{\color{black}36.6}}\cdot100.0$ & 
  $30.8\cdot\bm{{\color{black}45.8}}\cdot100.0$ &
  $30.8\cdot\bm{{\color{blue}75.2}}\cdot100.0$ &
  $30.8\cdot\bm{{\color{blue}93.8}}\cdot100.0$ \\

			\midrule
			%& & \multicolumn{6}{c}{\textbf{OOD-Attack}} \\
              \multirow{4}{1.2cm}{Smoothed + adv. w. uncert. attacks} &  
  \textbf{PostNet} &  - &  
  $30.8\cdot\bm{{\color{black}64.6}}\cdot100.0$ &
  $30.8\cdot\bm{{\color{black}31.9}}\cdot100.0$ & 
  $30.8\cdot\bm{{\color{blue}99.1}}\cdot100.0$ & 
  $37.2\cdot\bm{{\color{blue}50.0}}\cdot100.0$ & 
  $49.8\cdot\bm{{\color{blue}50.0}}\cdot\phantom{0}50.0$ \\
 & \textbf{PriorNet} &  - &  
 $30.8\cdot\bm{{\color{black}31.3}}\cdot\phantom{0}60.6$ & 
 $30.8\cdot\bm{{\color{black}30.8}}\cdot\phantom{0}34.8$ &        
 $30.8\cdot\bm{30.8}\cdot\phantom{0}73.8$ &               
 $30.8\cdot\bm{30.8}\cdot100.0$ &               
 $30.8\cdot\bm{30.8}\cdot100.0$ \\
  &  \textbf{DDNet} &  - &  
  $30.8\cdot\bm{{\color{black}30.8}}\cdot\phantom{0}31.7$ &    
  $30.8\cdot\bm{30.8}\cdot\phantom{0}64.6$ &              
  $30.8\cdot\bm{30.8}\cdot100.0$ &           
  $30.8\cdot\bm{30.8}\cdot100.0$ &     
  $30.8\cdot\bm{30.8}\cdot100.0$ \\
   & \textbf{EvNet} &  - &    
   $31.1\cdot\bm{{\color{blue}90.7}}\cdot100.0$ &   
   $30.8\cdot\bm{{\color{blue}96.6}}\cdot100.0$ &
   $30.8\cdot\bm{{\color{blue}98.9}}\cdot100.0$ &
   $30.8\cdot\bm{{\color{blue}97.5}}\cdot100.0$ & 
   $30.8\cdot\bm{{\color{blue}34.2}}\cdot100.0$ \\

		\bottomrule
		\end{tabular}}
	%\end{tiny}
\end{table*}









\begin{table*}[ht!]
	\centering
	\caption{OOD detection based on differential entropy under FGSM uncertainty attacks against differential entropy on ID data and OOD data. Smoothed DBU models  on CIFAR10. Column format: guaranteed lowest performance $\cdot$ empirical performance $\cdot$ guaranteed highest performance (blue: normally/adversarially trained smooth classifier is more robust than the base model).}
	\label{tab:cifar10_smooth_ooddetection_fgsm}
	%\begin{tiny}
	\resizebox{\textwidth}{!}{
		\begin{tabular}{llccccccc}
			\toprule
			& \textbf{Att. Rad.} & 0.0 &   0.1 &  0.2 &  0.5 &  1.0 &  2.0 \\
			\midrule
			& & \multicolumn{6}{c}{\textbf{ID-Attack}} \\
              \multirow{4}{1.2cm}{Smoothed models} &  
  \textbf{PostNet} &    
  $72.2\cdot\bm{{\color{blue}82.7}}\cdot88.0$ & 
  $35.0\cdot\bm{{\color{black}56.5}}\cdot97.5$ & 
  $31.9\cdot\bm{{\color{blue}65.5}}\cdot99.8$ &  
  $30.7\cdot\bm{{\color{black}50.6}}\cdot100.0$ & 
  $30.7\cdot\bm{{\color{black}46.9}}\cdot100.0$ &
  $30.7\cdot\bm{{\color{black}51.4}}\cdot100.0$ \\
 & \textbf{PriorNet} &  
 $50.3\cdot\bm{{\color{black}53.1}}\cdot55.9$ &  
 $33.5\cdot\bm{{\color{blue}43.2}}\cdot65.0$ &   
 $31.3\cdot\bm{{\color{blue}39.7}}\cdot69.1$ &   
 $31.3\cdot\bm{{\color{blue}48.3}}\cdot\phantom{0}98.2$ &   
 $30.7\cdot\bm{{\color{blue}44.2}}\cdot\phantom{0}99.9$ &   
 $30.7\cdot\bm{{\color{blue}44.9}}\cdot100.0$ \\
   & \textbf{DDNet} &  
   $72.0\cdot\bm{{\color{black}75.8}}\cdot79.8$ &  
   $35.5\cdot\bm{{\color{black}46.2}}\cdot69.7$ &  
   $32.9\cdot\bm{{\color{black}50.3}}\cdot87.0$ &  
   $31.1\cdot\bm{{\color{blue}58.6}}\cdot\phantom{0}98.6$ &    
   $30.7\cdot\bm{{\color{blue}59.4}}\cdot100.0$ &  
   $30.7\cdot\bm{{\color{black}44.5}}\cdot100.0$ \\
&    \textbf{EvNet} &     
$79.5\cdot\bm{{\color{blue}87.1}}\cdot92.8$ & 
$34.1\cdot\bm{{\color{black}58.6}}\cdot95.2$ &
$32.5\cdot\bm{{\color{black}61.1}}\cdot96.9$ &
$31.7\cdot\bm{{\color{black}60.6}}\cdot\phantom{0}98.8$ & 
$30.7\cdot\bm{{\color{blue}62.6}}\cdot100.0$ & 
$30.7\cdot\bm{{\color{black}57.3}}\cdot100.0$ \\

			\midrule
			%& & \multicolumn{6}{c}{\textbf{ID-Attack}} \\
               \multirow{4}{1.2cm}{Smoothed + adv. w. label attacks} & 
   \textbf{PostNet} &  - & 
   $35.0\cdot\bm{{\color{black}58.5}}\cdot97.7$ & 
   $31.2\cdot\bm{{\color{black}46.6}}\cdot97.4$ &  
   $30.8\cdot\bm{{\color{black}57.7}}\cdot\phantom{0}99.7$ &  
   $30.7\cdot\bm{{\color{black}50.1}}\cdot100.0$ & 
   $30.7\cdot\bm{{\color{black}50.6}}\cdot100.0$ \\
 & \textbf{PriorNet} &  - & 
 $31.5\cdot\bm{{\color{black}36.6}}\cdot56.7$ &    
 $33.1\cdot\bm{{\color{blue}51.7}}\cdot84.4$ &     
 $30.7\cdot\bm{{\color{blue}57.5}}\cdot\phantom{0}98.7$ &    
 $30.7\cdot\bm{{\color{blue}40.1}}\cdot\phantom{0}99.9$ &    
 $30.9\cdot\bm{{\color{blue}53.5}}\cdot\phantom{0}96.7$ \\
   & \textbf{DDNet} &  - &  
   $36.2\cdot\bm{{\color{black}50.0}}\cdot78.5$ & 
   $32.1\cdot\bm{{\color{black}41.3}}\cdot70.1$ &   
   $30.9\cdot\bm{{\color{blue}56.3}}\cdot100.0$ &  
   $30.7\cdot\bm{{\color{black}49.5}}\cdot100.0$ &   
   $30.7\cdot\bm{{\color{blue}54.9}}\cdot100.0$ \\
&    \textbf{EvNet} &  - &  
$46.8\cdot\bm{{\color{black}60.9}}\cdot79.6$ & 
$32.3\cdot\bm{{\color{black}58.9}}\cdot99.1$ &  
$30.7\cdot\bm{{\color{black}45.1}}\cdot100.0$ &  
$30.7\cdot\bm{{\color{blue}63.1}}\cdot100.0$ &  
$30.8\cdot\bm{{\color{black}38.1}}\cdot100.0$ \\
			\midrule
			%& & \multicolumn{6}{c}{\textbf{ID-Attack}} \\
              \multirow{4}{1.2cm}{Smoothed + adv. uncert. attacks} & 
  \textbf{PostNet} &  - & 
  $35.2\cdot\bm{{\color{black}56.0}}\cdot95.9$ & 
  $34.5\cdot\bm{{\color{black}59.0}}\cdot94.8$ &  
  $30.7\cdot\bm{{\color{black}47.0}}\cdot100.0$ &
  $30.7\cdot\bm{{\color{black}57.2}}\cdot100.0$ & 
  $30.7\cdot\bm{{\color{black}42.7}}\cdot100.0$ \\
& \textbf{PriorNet} &  - &  
$31.8\cdot\bm{{\color{black}38.8}}\cdot64.0$ &  
$31.0\cdot\bm{{\color{blue}41.7}}\cdot87.4$ &  
$30.7\cdot\bm{{\color{blue}42.9}}\cdot\phantom{0}99.3$ &  
$30.7\cdot\bm{{\color{blue}48.5}}\cdot100.0$ & 
$30.7\cdot\bm{{\color{blue}46.8}}\cdot100.0$ \\
 &   \textbf{DDNet} &  - &
 $39.6\cdot\bm{{\color{black}52.0}}\cdot75.6$ & 
 $36.4\cdot\bm{{\color{black}56.8}}\cdot83.8$ &
 $31.0\cdot\bm{{\color{black}51.4}}\cdot\phantom{0}97.3$ & 
 $31.0\cdot\bm{{\color{blue}56.9}}\cdot\phantom{0}97.7$ & 
 $30.7\cdot\bm{{\color{blue}49.2}}\cdot100.0$ \\
  &  \textbf{EvNet} &  - & 
  $34.8\cdot\bm{{\color{black}64.9}}\cdot99.7$ &
  $30.8\cdot\bm{{\color{black}48.9}}\cdot99.8$ & 
  $30.7\cdot\bm{{\color{blue}66.4}}\cdot100.0$ & 
  $30.9\cdot\bm{{\color{black}41.6}}\cdot\phantom{0}93.6$ &
  $31.1\cdot\bm{{\color{black}55.7}}\cdot100.0$ \\

			\midrule
			\midrule
			& & \multicolumn{6}{c}{\textbf{OOD-Attack}} \\
              \multirow{4}{1.2cm}{Smoothed models} &  
  \textbf{PostNet} &     
  $72.1\cdot\bm{{\color{blue}82.7}}\cdot88.0$ & 
  $35.1\cdot\bm{{\color{black}56.8}}\cdot97.3$ &  
  $31.9\cdot\bm{{\color{blue}65.8}}\cdot99.8$ &  
  $30.7\cdot\bm{{\color{blue}50.8}}\cdot100.0$ &
  $30.7\cdot\bm{{\color{black}46.5}}\cdot100.0$ & 
  $30.7\cdot\bm{{\color{black}51.5}}\cdot100.0$ \\
 & \textbf{PriorNet} &  
 $50.3\cdot\bm{{\color{black}53.1}}\cdot55.9$ &  
 $33.6\cdot\bm{{\color{blue}43.7}}\cdot65.9$ &    
 $31.3\cdot\bm{{\color{blue}39.8}}\cdot69.4$ &  
 $31.3\cdot\bm{{\color{blue}48.3}}\cdot\phantom{0}98.2$ &  
 $30.7\cdot\bm{{\color{blue}44.4}}\cdot\phantom{0}99.9$ &   
 $30.7\cdot\bm{{\color{blue}45.9}}\cdot100.0$ \\
   & \textbf{DDNet} & 
   $72.0\cdot\bm{{\color{black}75.8}}\cdot79.8$ & 
   $35.6\cdot\bm{{\color{black}46.1}}\cdot70.0$ & 
   $32.9\cdot\bm{{\color{black}50.1}}\cdot86.7$ & 
   $31.1\cdot\bm{{\color{blue}58.7}}\cdot\phantom{0}98.6$ &   
   $30.7\cdot\bm{{\color{blue}59.3}}\cdot100.0$ &  
   $30.7\cdot\bm{{\color{blue}44.6}}\cdot100.0$ \\
&    \textbf{EvNet} &  
$79.5\cdot\bm{{\color{blue}87.1}}\cdot92.8$ &  
$34.1\cdot\bm{{\color{black}58.8}}\cdot95.2$ &  
$32.6\cdot\bm{{\color{blue}61.3}}\cdot96.9$ &   
$31.7\cdot\bm{{\color{blue}60.5}}\cdot\phantom{0}98.8$ &  
$30.7\cdot\bm{{\color{blue}62.2}}\cdot100.0$ &  
$30.7\cdot\bm{{\color{blue}57.7}}\cdot100.0$ \\

			\midrule
			%& & \multicolumn{6}{c}{\textbf{OOD-Attack}} \\
              \multirow{4}{1.2cm}{Smoothed + adv. w. label attacks} &  
  \textbf{PostNet} &  - &  
  $35.0\cdot\bm{{\color{black}58.4}}\cdot97.9$ &  
  $31.2\cdot\bm{{\color{black}46.6}}\cdot97.3$ &    
  $30.8\cdot\bm{{\color{blue}57.7}}\cdot\phantom{0}99.7$ &  
  $30.7\cdot\bm{{\color{blue}50.1}}\cdot100.0$ & 
  $30.7\cdot\bm{{\color{black}51.4}}\cdot100.0$ \\
 & \textbf{PriorNet} &  - &  
 $31.6\cdot\bm{{\color{black}37.3}}\cdot59.2$ &   
 $33.2\cdot\bm{{\color{blue}52.6}}\cdot85.8$ &   
 $30.7\cdot\bm{{\color{blue}57.8}}\cdot\phantom{0}98.7$ &  
 $30.7\cdot\bm{{\color{blue}39.8}}\cdot\phantom{0}99.9$ &   
 $30.9\cdot\bm{{\color{blue}53.7}}\cdot\phantom{0}96.8$ \\
  &  \textbf{DDNet} &  - & 
  $36.4\cdot\bm{{\color{black}50.2}}\cdot78.8$ &  
  $32.1\cdot\bm{{\color{black}41.5}}\cdot70.5$ & 
  $30.8\cdot\bm{{\color{blue}56.2}}\cdot100.0$ & 
  $30.7\cdot\bm{{\color{blue}49.2}}\cdot100.0$ &     
  $30.7\cdot\bm{{\color{blue}55.0}}\cdot100.0$ \\
   & \textbf{EvNet} &  - &    
   $47.2\cdot\bm{{\color{blue}61.0}}\cdot79.9$ &   
   $32.4\cdot\bm{{\color{blue}59.1}}\cdot99.1$ &  
   $30.7\cdot\bm{{\color{black}45.1}}\cdot100.0$ &  
   $30.7\cdot\bm{{\color{blue}63.1}}\cdot100.0$ & 
   $30.8\cdot\bm{{\color{black}38.0}}\cdot100.0$ \\

			\midrule
			%& & \multicolumn{6}{c}{\textbf{OOD-Attack}} \\
              \multirow{4}{1.2cm}{Smoothed + adv. w. uncert. attacks} &  
  \textbf{PostNet} &  - &  
  $35.3\cdot\bm{{\color{black}56.3}}\cdot96.1$ &     
  $34.5\cdot\bm{{\color{blue}59.1}}\cdot94.9$ &  
  $30.7\cdot\bm{{\color{blue}46.9}}\cdot100.0$ & 
  $30.7\cdot\bm{{\color{blue}57.8}}\cdot100.0$ &
  $30.7\cdot\bm{{\color{black}43.1}}\cdot100.0$ \\
 & \textbf{PriorNet} &  - &  
 $31.9\cdot\bm{{\color{black}39.4}}\cdot65.4$ &  
 $31.0\cdot\bm{{\color{blue}42.0}}\cdot88.7$ &   
 $30.7\cdot\bm{{\color{blue}42.9}}\cdot\phantom{0}99.2$ &   
 $30.7\cdot\bm{{\color{blue}48.3}}\cdot100.0$ & 
 $30.7\cdot\bm{{\color{blue}47.2}}\cdot100.0$ \\
   & \textbf{DDNet} &  - &  
   $40.1\cdot\bm{{\color{black}52.8}}\cdot76.5$ &  
   $36.5\cdot\bm{{\color{black}56.9}}\cdot83.9$ &  
   $31.1\cdot\bm{{\color{blue}51.5}}\cdot\phantom{0}97.3$ &   
   $31.0\cdot\bm{{\color{blue}57.0}}\cdot\phantom{0}97.8$ &  
   $30.7\cdot\bm{{\color{blue}48.7}}\cdot100.0$ \\
&    \textbf{EvNet} &  - &     
$34.9\cdot\bm{{\color{blue}65.0}}\cdot99.6$ & 
$30.8\cdot\bm{{\color{black}48.8}}\cdot99.8$ & 
$30.7\cdot\bm{{\color{blue}66.6}}\cdot100.0$ & 
$30.9\cdot\bm{{\color{black}41.1}}\cdot\phantom{0}93.4$ &
$31.1\cdot\bm{{\color{black}55.3}}\cdot100.0$ \\
			\bottomrule
		\end{tabular}}
	%\end{tiny}
\end{table*}




\begin{table*}[ht!]
	\centering
	\caption{OOD detection based on differential entropy under FGSM uncertainty attacks against differential entropy on ID data and OOD data. Smoothed DBU models  on MNIST. Column format: guaranteed lowest performance $\cdot$ empirical performance $\cdot$ guaranteed highest performance (blue: normally/adversarially trained smooth classifier is more robust than the base model).}
	\label{tab:mnist_smooth_ooddetection_fgsm}
	%\begin{tiny}
	\resizebox{\textwidth}{!}{
		\begin{tabular}{llccccccc}
			\toprule
			& \textbf{Att. Rad.} & 0.0 &   0.1 &  0.2 &  0.5 &  1.0 &  2.0 \\
			\midrule
			& & \multicolumn{6}{c}{\textbf{ID-Attack}} \\
              \multirow{4}{1.2cm}{Smoothed models} &  
  \textbf{PostNet} & 
  $59.9\cdot\bm{{\color{black}91.3}}\cdot98.6$ &  
  $61.1\cdot\bm{{\color{blue}97.7}}\cdot99.7$ & 
  $65.1\cdot\bm{{\color{blue}94.8}}\cdot99.7$ &  
  $31.6\cdot\bm{{\color{black}64.8}}\cdot\phantom{0}99.7$ & 
  $30.7\cdot\bm{{\color{black}62.4}}\cdot100.0$ & 
  $30.7\cdot\bm{{\color{black}68.6}}\cdot100.0$ \\
 & \textbf{PriorNet} & 
 $99.8\cdot\bm{{\color{blue}99.8}}\cdot99.8$ &  
 $99.4\cdot\bm{{\color{blue}99.8}}\cdot99.9$ &   
 $98.4\cdot\bm{{\color{blue}99.7}}\cdot99.9$ &  
 $49.8\cdot\bm{{\color{black}92.7}}\cdot\phantom{0}99.9$ &  
 $31.3\cdot\bm{{\color{black}76.6}}\cdot\phantom{0}99.8$ &
 $30.7\cdot\bm{{\color{black}71.8}}\cdot100.0$ \\
   & \textbf{DDNet} & 
   $98.5\cdot\bm{{\color{black}98.6}}\cdot98.7$ &  
   $95.0\cdot\bm{{\color{black}97.6}}\cdot98.9$ &
   $74.4\cdot\bm{{\color{black}91.9}}\cdot98.2$ & 
   $31.4\cdot\bm{{\color{black}52.0}}\cdot\phantom{0}98.5$ & 
   $30.7\cdot\bm{{\color{black}51.8}}\cdot100.0$ & 
   $30.7\cdot\bm{{\color{black}40.2}}\cdot100.0$ \\
&    \textbf{EvNet} &     
$85.7\cdot\bm{{\color{blue}87.5}}\cdot89.2$ &   
$69.0\cdot\bm{{\color{blue}90.4}}\cdot97.7$ &   
$42.5\cdot\bm{{\color{blue}90.2}}\cdot99.6$ &  
$30.7\cdot\bm{{\color{blue}70.1}}\cdot100.0$ & 
$30.7\cdot\bm{{\color{black}50.0}}\cdot100.0$ & 
$30.7\cdot\bm{{\color{black}43.9}}\cdot100.0$ \\

			\midrule
			%& & \multicolumn{6}{c}{\textbf{ID-Attack}} \\
              \multirow{4}{1.2cm}{Smoothed + adv. label attacks} &  
  \textbf{PostNet} &  - &     
  $84.4\cdot\bm{{\color{blue}96.3}}\cdot\phantom{0}99.4$ & 
  $50.6\cdot\bm{{\color{black}89.3}}\cdot\phantom{0}99.5$ &  
  $30.9\cdot\bm{{\color{black}46.3}}\cdot\phantom{0}99.4$ &  
  $30.7\cdot\bm{{\color{black}46.3}}\cdot100.0$ &  
  $30.7\cdot\bm{{\color{black}63.3}}\cdot100.0$ \\
 & \textbf{PriorNet} &  - &  
 $99.7\cdot\bm{{\color{blue}99.9}}\cdot100.0$ &   
 $98.7\cdot\bm{{\color{blue}99.8}}\cdot100.0$ &   
 $84.1\cdot\bm{{\color{blue}99.2}}\cdot100.0$ &  
 $30.7\cdot\bm{{\color{black}84.6}}\cdot100.0$ & 
 $30.7\cdot\bm{{\color{black}68.1}}\cdot100.0$ \\
   & \textbf{DDNet} &  - &  
   $93.6\cdot\bm{{\color{black}96.9}}\cdot\phantom{0}98.5$ & 
   $71.0\cdot\bm{{\color{black}89.0}}\cdot\phantom{0}96.9$ &
   $32.3\cdot\bm{{\color{black}50.4}}\cdot\phantom{0}99.0$ &  
   $30.7\cdot\bm{{\color{black}51.1}}\cdot100.0$ & 
   $30.7\cdot\bm{{\color{black}54.1}}\cdot100.0$ \\
&    \textbf{EvNet} &  - &     
$58.2\cdot\bm{{\color{blue}84.5}}\cdot\phantom{0}94.3$ &     
$40.9\cdot\bm{{\color{blue}87.2}}\cdot\phantom{0}99.2$ &  
$30.7\cdot\bm{{\color{black}59.3}}\cdot100.0$ & 
$30.7\cdot\bm{{\color{black}39.7}}\cdot100.0$ &  
$30.7\cdot\bm{{\color{black}52.7}}\cdot100.0$ \\

			\midrule
			%& & \multicolumn{6}{c}{\textbf{ID-Attack}} \\
              \multirow{4}{1.2cm}{Smoothed + adv. w. uncert. attacks} &  
  \textbf{PostNet} &  - &    
  $58.6\cdot\bm{{\color{blue}96.1}}\cdot\phantom{0}99.3$ &   
  $59.9\cdot\bm{{\color{blue}96.2}}\cdot99.9$ &  
  $31.2\cdot\bm{{\color{black}47.6}}\cdot\phantom{0}95.5$ &
  $30.7\cdot\bm{{\color{black}41.8}}\cdot100.0$ &
  $30.7\cdot\bm{{\color{black}55.4}}\cdot100.0$ \\
& \textbf{PriorNet} &  - &  
$99.9\cdot\bm{{\color{blue}100.0}}\cdot100.0$ &     
$96.6\cdot\bm{99.2}\cdot99.9$ &  
$50.3\cdot\bm{{\color{black}97.1}}\cdot100.0$ &  
$31.7\cdot\bm{{\color{black}89.7}}\cdot100.0$ & 
$30.7\cdot\bm{{\color{black}81.8}}\cdot100.0$ \\
 &   \textbf{DDNet} &  - & 
 $95.0\cdot\bm{{\color{black}97.5}}\cdot\phantom{0}98.8$ & 
 $80.5\cdot\bm{{\color{black}94.0}}\cdot98.6$ & 
 $31.7\cdot\bm{{\color{black}55.6}}\cdot\phantom{0}98.6$ &  
 $30.7\cdot\bm{{\color{black}52.0}}\cdot100.0$ & 
 $30.7\cdot\bm{{\color{black}49.5}}\cdot100.0$ \\
  &  \textbf{EvNet} &  - &    
  $66.5\cdot\bm{{\color{blue}91.4}}\cdot\phantom{0}98.1$ &     
  $48.5\cdot\bm{{\color{blue}84.5}}\cdot97.6$ &  
  $30.8\cdot\bm{{\color{black}49.3}}\cdot\phantom{0}99.9$ &
  $30.7\cdot\bm{{\color{black}37.3}}\cdot100.0$ &   
  $30.8\cdot\bm{{\color{blue}62.0}}\cdot100.0$ \\

			\midrule
			\midrule
			& & \multicolumn{6}{c}{\textbf{OOD-Attack}} \\
              \multirow{4}{1.2cm}{Smoothed models} & 
  \textbf{PostNet} &  
  $59.2\cdot\bm{{\color{black}91.3}}\cdot97.7$ &  
  $57.9\cdot\bm{{\color{blue}97.2}}\cdot99.6$ &  
  $61.4\cdot\bm{{\color{blue}93.8}}\cdot\phantom{0}99.6$ &  
  $31.5\cdot\bm{{\color{black}59.1}}\cdot\phantom{0}99.5$ &
  $30.7\cdot\bm{{\color{black}52.4}}\cdot100.0$ &  
  $30.7\cdot\bm{{\color{black}53.9}}\cdot100.0$ \\
 & \textbf{PriorNet} &   
 $99.7\cdot\bm{{\color{blue}99.8}}\cdot99.8$ & 
 $99.4\cdot\bm{{\color{blue}99.8}}\cdot99.9$ & 
 $98.3\cdot\bm{{\color{blue}99.7}}\cdot100.0$ & 
 $60.4\cdot\bm{{\color{black}96.6}}\cdot100.0$ &
 $32.8\cdot\bm{{\color{black}88.2}}\cdot\phantom{0}99.9$ & 
 $30.7\cdot\bm{{\color{black}86.1}}\cdot100.0$ \\
  &  \textbf{DDNet} & 
  $98.4\cdot\bm{{\color{black}98.5}}\cdot98.7$ & 
  $94.3\cdot\bm{{\color{black}97.2}}\cdot98.7$ &  
  $72.2\cdot\bm{{\color{black}90.6}}\cdot\phantom{0}97.8$ &  
  $31.6\cdot\bm{{\color{black}52.2}}\cdot\phantom{0}98.1$ &  
  $30.7\cdot\bm{{\color{black}51.8}}\cdot100.0$ & 
  $30.7\cdot\bm{{\color{black}38.5}}\cdot100.0$ \\
   & \textbf{EvNet} &   
   $83.9\cdot\bm{{\color{blue}85.7}}\cdot88.0$ &  
   $63.6\cdot\bm{{\color{blue}88.6}}\cdot97.9$ &  
   $40.1\cdot\bm{{\color{blue}87.6}}\cdot\phantom{0}99.6$ &  
   $30.8\cdot\bm{{\color{blue}69.2}}\cdot100.0$ &  
   $30.7\cdot\bm{{\color{black}43.5}}\cdot100.0$ & 
   $30.7\cdot\bm{{\color{black}37.4}}\cdot100.0$ \\

			\midrule
			%& & \multicolumn{6}{c}{\textbf{OOD-Attack}} \\
              \multirow{4}{1.2cm}{Smoothed + adv. label attacks} &  
  \textbf{PostNet} &  - &    
  $84.4\cdot\bm{{\color{blue}96.2}}\cdot\phantom{0}99.4$ &  
  $49.7\cdot\bm{{\color{black}89.1}}\cdot\phantom{0}99.5$ & 
  $30.9\cdot\bm{{\color{black}45.6}}\cdot\phantom{0}99.3$ &  
  $30.7\cdot\bm{{\color{black}46.2}}\cdot100.0$ &  
  $30.7\cdot\bm{{\color{black}68.1}}\cdot100.0$ \\
& \textbf{PriorNet} &  - &  
$99.7\cdot\bm{{\color{blue}99.9}}\cdot100.0$ &  
$98.7\cdot\bm{{\color{blue}99.8}}\cdot100.0$ &   
$86.3\cdot\bm{{\color{blue}99.4}}\cdot100.0$ &  
$30.9\cdot\bm{{\color{black}91.9}}\cdot100.0$ &  
$30.7\cdot\bm{{\color{black}77.5}}\cdot100.0$ \\
 &   \textbf{DDNet} &  - &  
 $93.9\cdot\bm{{\color{black}97.0}}\cdot\phantom{0}98.6$ &  
 $72.1\cdot\bm{{\color{black}89.5}}\cdot\phantom{0}97.0$ &  
 $33.0\cdot\bm{{\color{black}52.3}}\cdot\phantom{0}98.8$ &  
 $30.7\cdot\bm{{\color{black}51.5}}\cdot100.0$ &
 $30.7\cdot\bm{{\color{black}60.4}}\cdot100.0$ \\
  &  \textbf{EvNet} &  - &   
  $59.4\cdot\bm{{\color{blue}85.6}}\cdot\phantom{0}94.6$ &  
  $40.7\cdot\bm{{\color{blue}86.7}}\cdot\phantom{0}99.2$ &  
  $30.7\cdot\bm{{\color{blue}57.3}}\cdot100.0$ &  
  $30.7\cdot\bm{{\color{black}39.4}}\cdot100.0$ & 
  $30.7\cdot\bm{{\color{black}49.0}}\cdot100.0$ \\
			\midrule
			%& & \multicolumn{6}{c}{\textbf{OOD-Attack}} \\
              \multirow{4}{1.2cm}{Smoothed + adv. w. uncert. attacks} & 
  \textbf{PostNet} &  - &    
  $55.8\cdot\bm{{\color{blue}96.1}}\cdot99.3$ &
  $58.4\cdot\bm{{\color{blue}95.7}}\cdot99.8$ &   
  $31.1\cdot\bm{{\color{black}44.6}}\cdot\phantom{0}93.3$ & 
  $30.7\cdot\bm{{\color{black}41.4}}\cdot100.0$ & 
  $30.7\cdot\bm{{\color{black}50.1}}\cdot100.0$ \\
& \textbf{PriorNet} &  - &  
$99.9\cdot\bm{{\color{blue}100.0}}\cdot100.0$ &  
$96.9\cdot\bm{{\color{blue}99.3}}\cdot99.9$ &  
$60.3\cdot\bm{{\color{black}98.2}}\cdot100.0$ &  
$33.0\cdot\bm{{\color{black}93.5}}\cdot100.0$ & 
$30.7\cdot\bm{{\color{black}87.8}}\cdot100.0$ \\
  &  \textbf{DDNet} &  - &  
  $95.3\cdot\bm{{\color{black}97.6}}\cdot98.9$ & 
  $82.3\cdot\bm{{\color{black}94.5}}\cdot98.7$ &
  $32.1\cdot\bm{{\color{black}56.3}}\cdot\phantom{0}98.5$ &  
  $30.7\cdot\bm{{\color{black}48.9}}\cdot100.0$ & 
  $30.7\cdot\bm{{\color{black}43.4}}\cdot100.0$ \\
   & \textbf{EvNet} &  - &    
   $65.3\cdot\bm{{\color{blue}90.3}}\cdot97.9$ &   
   $46.9\cdot\bm{{\color{blue}83.1}}\cdot97.3$ &   
   $30.8\cdot\bm{{\color{black}48.8}}\cdot\phantom{0}99.9$ & 
   $30.7\cdot\bm{{\color{black}36.6}}\cdot100.0$ &  
   $30.8\cdot\bm{{\color{blue}60.7}}\cdot100.0$ \\
		\bottomrule
		\end{tabular}}
	%\end{tiny}
\end{table*}





\begin{table*}[ht!]
	\centering
	\caption{OOD detection based on differential entropy under FGSM uncertainty attacks against differential entropy on ID data and OOD data. Smoothed DBU models  on Sensorless. Column format: guaranteed lowest performance $\cdot$ empirical performance $\cdot$ guaranteed highest performance (blue: normally/adversarially trained smooth classifier is more robust than the base model)..}
	\label{tab:sensorless_smooth_ooddetection_fgsm}
	%\begin{tiny}
	\resizebox{\textwidth}{!}{
		\begin{tabular}{llccccccc}
			\toprule
			& \textbf{Att. Rad.} & 0.0 &   0.1 &  0.2 &  0.5 &  1.0 &  2.0 \\
			\midrule
			& & \multicolumn{6}{c}{\textbf{ID-Attack}} \\
              \multirow{4}{1.2cm}{Smoothed models} & 
  \textbf{PostNet} &   
  $49.3\cdot\bm{{\color{blue}90.4}}\cdot99.8$ &  
  $30.7\cdot\bm{{\color{blue}50.3}}\cdot100.0$ & 
  $30.7\cdot\bm{{\color{black}36.6}}\cdot100.0$ &  
  $49.1\cdot\bm{{\color{blue}50.0}}\cdot\phantom{0}74.9$ &   
  $50.0\cdot\bm{{\color{blue}50.0}}\cdot\phantom{0}50.0$ &     
  $50.0\cdot\bm{{\color{blue}50.0}}\cdot\phantom{0}50.0$ \\
 & \textbf{PriorNet} &    
 $31.2\cdot\bm{{\color{blue}39.0}}\cdot66.9$ &   
 $30.7\cdot\bm{{\color{blue}40.1}}\cdot100.0$ &
 $30.7\cdot\bm{{\color{black}48.2}}\cdot100.0$ & 
 $30.7\cdot\bm{{\color{black}54.2}}\cdot100.0$ & 
 $30.7\cdot\bm{{\color{black}46.3}}\cdot100.0$ & 
 $30.7\cdot\bm{{\color{black}47.6}}\cdot100.0$ \\
   & \textbf{DDNet} &
   $31.0\cdot\bm{{\color{black}31.5}}\cdot32.7$ & 
   $30.7\cdot\bm{{\color{black}31.2}}\cdot100.0$ & 
   $30.7\cdot\bm{{\color{black}35.3}}\cdot100.0$ & 
   $30.7\cdot\bm{{\color{black}55.7}}\cdot100.0$ &  
   $30.7\cdot\bm{{\color{black}42.4}}\cdot100.0$ &  
   $30.7\cdot\bm{{\color{black}40.4}}\cdot100.0$ \\
&    \textbf{EvNet} &  
$33.6\cdot\bm{{\color{black}55.1}}\cdot91.3$ & 
$30.7\cdot\bm{{\color{black}39.1}}\cdot100.0$ & 
$30.7\cdot\bm{{\color{black}37.1}}\cdot100.0$ & 
$30.7\cdot\bm{{\color{black}35.4}}\cdot100.0$ &    
$30.8\cdot\bm{{\color{blue}52.1}}\cdot100.0$ &     
$32.5\cdot\bm{{\color{blue}50.0}}\cdot100.0$ \\

			\midrule
			%& & \multicolumn{6}{c}{\textbf{ID-Attack}} \\
               \multirow{4}{1.2cm}{Smoothed + adv. label attacks} &
   \textbf{PostNet} &  - &   
   $30.7\cdot\bm{{\color{blue}60.8}}\cdot100.0$ &  
   $30.7\cdot\bm{{\color{blue}40.7}}\cdot100.0$ &   
   $50.0\cdot\bm{{\color{blue}50.0}}\cdot\phantom{0}50.0$ &      
   $50.0\cdot\bm{{\color{blue}50.0}}\cdot\phantom{0}50.0$ &     
   $50.0\cdot\bm{{\color{blue}50.0}}\cdot\phantom{0}50.0$ \\
 & \textbf{PriorNet} &  - & 
 $30.7\cdot\bm{{\color{black}31.3}}\cdot100.0$ &  
 $30.7\cdot\bm{{\color{black}32.9}}\cdot100.0$ & 
 $30.7\cdot\bm{{\color{black}40.1}}\cdot100.0$ & 
 $30.8\cdot\bm{{\color{black}31.1}}\cdot100.0$ &  
 $38.1\cdot\bm{{\color{blue}91.0}}\cdot100.0$ \\
   & \textbf{DDNet} &  - &     
   $30.7\cdot\bm{{\color{blue}34.3}}\cdot100.0$ &  
   $30.7\cdot\bm{{\color{black}33.9}}\cdot100.0$ &  
   $30.7\cdot\bm{{\color{black}38.2}}\cdot100.0$ & 
   $30.7\cdot\bm{{\color{black}63.6}}\cdot100.0$ & 
   $30.7\cdot\bm{{\color{black}41.8}}\cdot100.0$ \\
&    \textbf{EvNet} &  - & 
$30.8\cdot\bm{{\color{black}41.0}}\cdot100.0$ & 
$30.7\cdot\bm{{\color{black}34.2}}\cdot100.0$ &
$30.7\cdot\bm{{\color{black}38.0}}\cdot100.0$ & 
$30.7\cdot\bm{{\color{black}39.0}}\cdot100.0$ & 
$30.7\cdot\bm{{\color{blue}50.0}}\cdot100.0$ \\

			\midrule
			%& & \multicolumn{6}{c}{\textbf{ID-Attack}} \\
              \multirow{4}{1.2cm}{Smoothed + adv. w. uncert. attacks} & 
  \textbf{PostNet} &  - &  
  $30.7\cdot\bm{{\color{blue}46.1}}\cdot100.0$ &    
  $30.7\cdot\bm{{\color{blue}46.8}}\cdot100.0$ &   
  $50.0\cdot\bm{{\color{blue}50.0}}\cdot\phantom{0}50.0$ &   
  $50.0\cdot\bm{{\color{blue}50.0}}\cdot\phantom{0}\phantom{0}50.0$ &  
  $50.0\cdot\bm{{\color{blue}50.0}}\cdot\phantom{0}\phantom{0}\phantom{0}50.0$ \\
 & \textbf{PriorNet} &  - &    
 $30.7\cdot\bm{{\color{blue}36.5}}\cdot100.0$ & 
 $30.7\cdot\bm{{\color{black}34.4}}\cdot100.0$ & 
 $30.7\cdot\bm{{\color{black}77.8}}\cdot100.0$ & 
 $30.8\cdot\bm{{\color{black}53.0}}\cdot100.0$ & 
 $30.7\cdot\bm{{\color{black}39.2}}\cdot100.0$ \\
  &  \textbf{DDNet} &  - &    
  $30.7\cdot\bm{{\color{blue}36.0}}\cdot100.0$ & 
  $30.7\cdot\bm{{\color{black}37.7}}\cdot100.0$ &
  $30.7\cdot\bm{{\color{black}41.0}}\cdot100.0$ & 
  $30.7\cdot\bm{{\color{black}42.3}}\cdot100.0$ & 
  $30.7\cdot\bm{{\color{black}39.0}}\cdot100.0$ \\
  &  \textbf{EvNet} &  - &  
  $30.7\cdot\bm{{\color{black}31.3}}\cdot100.0$ &  
  $30.7\cdot\bm{{\color{black}43.3}}\cdot100.0$ &  
  $30.7\cdot\bm{{\color{black}36.3}}\cdot100.0$ &  
  $30.7\cdot\bm{{\color{blue}43.2}}\cdot100.0$ &   
  $30.7\cdot\bm{{\color{blue}50.0}}\cdot100.0$ \\

			\midrule
			\midrule
			& & \multicolumn{6}{c}{\textbf{OOD-Attack}} \\
              \multirow{4}{1.2cm}{Smoothed models} &  
  \textbf{PostNet} &     
  $49.3\cdot\bm{{\color{blue}90.4}}\cdot99.8$ & 
  $30.8\cdot\bm{{\color{black}75.3}}\cdot100.0$ & 
  $30.7\cdot\bm{{\color{black}68.5}}\cdot100.0$ & 
  $46.1\cdot\bm{{\color{black}50.0}}\cdot\phantom{0}74.8$ &  
  $50.0\cdot\bm{{\color{black}50.0}}\cdot\phantom{0}50.0$ &   
  $50.0\cdot\bm{{\color{black}50.0}}\cdot\phantom{0}50.0$ \\
 & \textbf{PriorNet} &   
 $31.2\cdot\bm{{\color{blue}38.9}}\cdot67.0$ &    
 $30.7\cdot\bm{{\color{blue}34.1}}\cdot100.0$ &   
 $30.7\cdot\bm{{\color{blue}35.7}}\cdot100.0$ &  
 $30.7\cdot\bm{{\color{blue}35.0}}\cdot100.0$ &   
 $30.7\cdot\bm{{\color{blue}77.6}}\cdot100.0$ &  
 $30.8\cdot\bm{{\color{blue}95.3}}\cdot100.0$ \\
   & \textbf{DDNet} & 
   $31.0\cdot\bm{{\color{black}31.5}}\cdot32.7$ &  
   $30.7\cdot\bm{{\color{blue}30.8}}\cdot100.0$ &  
   $30.7\cdot\bm{{\color{blue}33.1}}\cdot100.0$ &  
   $30.7\cdot\bm{{\color{blue}65.7}}\cdot100.0$ &   
   $30.7\cdot\bm{{\color{blue}71.8}}\cdot100.0$ &  
   $30.7\cdot\bm{{\color{blue}71.5}}\cdot100.0$ \\
&    \textbf{EvNet} & 
$33.6\cdot\bm{{\color{black}55.2}}\cdot91.4$ &   
$30.7\cdot\bm{{\color{blue}64.7}}\cdot100.0$ &   
$30.7\cdot\bm{{\color{blue}69.6}}\cdot100.0$ &  
$30.7\cdot\bm{{\color{blue}78.9}}\cdot100.0$ & 
$30.7\cdot\bm{{\color{black}67.2}}\cdot100.0$ & 
$32.9\cdot\bm{{\color{black}50.0}}\cdot100.0$ \\

			\midrule
			%& & \multicolumn{6}{c}{\textbf{OOD-Attack}} \\
              \multirow{4}{1.2cm}{Smoothed + adv. w. label attacks} &  
  \textbf{PostNet} &  - &  
  $30.7\cdot\bm{{\color{black}86.0}}\cdot100.0$ & 
  $30.7\cdot\bm{{\color{black}86.6}}\cdot100.0$ & 
  $50.0\cdot\bm{{\color{black}50.0}}\cdot\phantom{0}50.0$ &  
  $50.0\cdot\bm{{\color{black}50.0}}\cdot\phantom{0}50.0$ &  
  $50.0\cdot\bm{{\color{black}50.0}}\cdot\phantom{0}50.0$ \\
 & \textbf{PriorNet} &  - &      
 $30.7\cdot\bm{{\color{blue}31.0}}\cdot\phantom{0}99.9$ &    
 $30.7\cdot\bm{{\color{blue}31.2}}\cdot\phantom{0}98.9$ &  
 $30.7\cdot\bm{{\color{blue}30.7}}\cdot100.0$ & 
 $30.8\cdot\bm{{\color{blue}30.7}}\cdot100.0$ &
 $36.1\cdot\bm{{\color{blue}35.3}}\cdot100.0$ \\
  &  \textbf{DDNet} &  - &     
  $30.7\cdot\bm{{\color{blue}37.2}}\cdot100.0$ &   
  $30.7\cdot\bm{{\color{blue}31.1}}\cdot100.0$ &  
  $30.7\cdot\bm{{\color{blue}37.1}}\cdot100.0$ &     
  $30.7\cdot\bm{{\color{blue}50.5}}\cdot100.0$ &
  $30.7\cdot\bm{{\color{blue}84.6}}\cdot100.0$ \\
  &  \textbf{EvNet} &  - &     
  $30.8\cdot\bm{{\color{blue}82.5}}\cdot100.0$ & 
  $30.7\cdot\bm{{\color{black}51.7}}\cdot100.0$ & 
  $30.7\cdot\bm{{\color{blue}91.5}}\cdot100.0$ &  
  $30.7\cdot\bm{{\color{black}70.0}}\cdot100.0$ &
  $30.9\cdot\bm{{\color{black}50.0}}\cdot100.0$ \\

			\midrule
			%& & \multicolumn{6}{c}{\textbf{OOD-Attack}} \\
              \multirow{4}{1.2cm}{Smoothed + adv. uncert. attacks} &  
  \textbf{PostNet} &  - & 
  $30.7\cdot\bm{{\color{black}78.5}}\cdot100.0$ & 
  $30.7\cdot\bm{{\color{black}67.1}}\cdot100.0$ & 
  $50.0\cdot\bm{{\color{black}50.0}}\cdot\phantom{0}50.0$ & 
  $50.0\cdot\bm{{\color{black}50.0}}\cdot\phantom{0}50.0$ & 
  $50.0\cdot\bm{{\color{black}50.0}}\cdot\phantom{0}50.0$ \\
& \textbf{PriorNet} &  - &   
$30.7\cdot\bm{{\color{blue}35.8}}\cdot100.0$ &  
$30.7\cdot\bm{{\color{blue}30.7}}\cdot100.0$ & 
$30.7\cdot\bm{{\color{blue}39.0}}\cdot100.0$ &  
$30.7\cdot\bm{{\color{blue}58.5}}\cdot100.0$ &  
$30.7\cdot\bm{{\color{blue}100.0}}\cdot100.0$ \\
 &   \textbf{DDNet} &  - &   
 $30.7\cdot\bm{{\color{blue}40.8}}\cdot100.0$ &   
 $30.7\cdot\bm{{\color{blue}33.1}}\cdot100.0$ &   
 $30.7\cdot\bm{{\color{blue}30.8}}\cdot100.0$ &   
 $30.7\cdot\bm{{\color{blue}34.3}}\cdot100.0$ & 
 $30.7\cdot\bm{{\color{blue}35.2}}\cdot100.0$ \\
  &  \textbf{EvNet} &  - &  
  $30.7\cdot\bm{{\color{black}32.7}}\cdot100.0$ & 
  $30.8\cdot\bm{{\color{black}50.2}}\cdot100.0$ & 
  $30.7\cdot\bm{{\color{blue}99.6}}\cdot100.0$ & 
  $30.7\cdot\bm{{\color{black}58.7}}\cdot100.0$ & 
  $30.7\cdot\bm{{\color{black}50.0}}\cdot100.0$ \\

		\bottomrule
		\end{tabular}}
	%\end{tiny}
\end{table*}




\begin{table*}[ht!]
	\centering
	\caption{OOD detection based on differential entropy under FGSM uncertainty attacks against differential entropy on ID data and OOD data. Smoothed DBU models  on Segment. Column format: guaranteed lowest performance $\cdot$ empirical performance $\cdot$ guaranteed highest performance (blue: normally/adversarially trained smooth classifier is more robust than the base model).}
	\label{tab:segment_smooth_ooddetection_fgsm}
	%\begin{tiny}
	\resizebox{\textwidth}{!}{
		\begin{tabular}{llccccccc}
			\toprule
			& \textbf{Att. Rad.} & 0.0 &   0.1 &  0.2 &  0.5 &  1.0 &  2.0 \\
			\midrule
			& &\multicolumn{6}{c}{\textbf{ID-Attack}} \\
               \multirow{4}{1.2cm}{Smoothed models} &
   \textbf{PostNet} &  
   $99.6\cdot\bm{{\color{blue}99.9}}\cdot99.9$ & 
   $33.1\cdot\bm{{\color{black}78.8}}\cdot100.0$ &
   $30.8\cdot\bm{{\color{black}46.2}}\cdot100.0$ &  
   $30.8\cdot\bm{{\color{black}34.2}}\cdot100.0$ & 
   $30.8\cdot\bm{{\color{blue}41.4}}\cdot100.0$ &   
   $41.5\cdot\bm{{\color{blue}50.0}}\cdot\phantom{0}50.2$ \\
 & \textbf{PriorNet} &
 $30.9\cdot\bm{{\color{black}31.0}}\cdot31.4$ &  
 $30.8\cdot\bm{{\color{black}30.8}}\cdot\phantom{0}39.3$ & 
 $30.8\cdot\bm{{\color{black}30.8}}\cdot\phantom{0}94.7$ &  
 $30.8\cdot\bm{{\color{black}41.2}}\cdot100.0$ &
 $30.8\cdot\bm{{\color{blue}92.7}}\cdot100.0$ & 
 $30.8\cdot\bm{{\color{black}79.9}}\cdot100.0$ \\
   & \textbf{DDNet} &
   $30.8\cdot\bm{{\color{black}30.8}}\cdot30.8$ & 
   $30.8\cdot\bm{30.8}\cdot\phantom{0}31.8$ &                
   $30.8\cdot\bm{30.8}\cdot\phantom{0}66.8$ &
   $30.8\cdot\bm{{\color{black}30.8}}\cdot100.0$ &
   $30.8\cdot\bm{{\color{black}32.6}}\cdot100.0$ & 
   $30.8\cdot\bm{{\color{black}38.2}}\cdot100.0$ \\
&    \textbf{EvNet} &  
$94.9\cdot\bm{{\color{blue}97.2}}\cdot98.2$ &  
$31.0\cdot\bm{{\color{blue}73.1}}\cdot100.0$ & 
$30.8\cdot\bm{{\color{blue}72.3}}\cdot100.0$ &  
$30.8\cdot\bm{{\color{black}57.1}}\cdot100.0$ & 
$30.8\cdot\bm{{\color{black}63.3}}\cdot100.0$ & 
$30.8\cdot\bm{{\color{black}49.6}}\cdot100.0$ \\

			\midrule
			%& &\multicolumn{6}{c}{\textbf{ID-Attack}} \\
              \multirow{4}{1.2cm}{Smoothed + adv. w. label attacks} &  
  \textbf{PostNet} &  - &
  $31.0\cdot\bm{{\color{black}62.9}}\cdot100.0$ &
  $30.8\cdot\bm{{\color{black}47.1}}\cdot100.0$ &  
  $30.8\cdot\bm{{\color{blue}90.0}}\cdot100.0$ &    
  $30.8\cdot\bm{{\color{blue}50.0}}\cdot100.0$ &    
  $50.0\cdot\bm{{\color{blue}50.0}}\cdot\phantom{0}50.0$ \\
 & \textbf{PriorNet} &  - &  
 $30.8\cdot\bm{{\color{black}30.8}}\cdot\phantom{0}43.5$ & 
 $30.8\cdot\bm{{\color{black}30.8}}\cdot\phantom{0}32.5$ & 
 $30.8\cdot\bm{{\color{black}30.9}}\cdot100.0$ & 
 $30.8\cdot\bm{{\color{black}30.9}}\cdot100.0$ &
 $30.8\cdot\bm{{\color{black}30.8}}\cdot100.0$ \\
  &  \textbf{DDNet} &  - &            
  $30.8\cdot\bm{30.8}\cdot\phantom{0}30.8$ &                 
  $30.8\cdot\bm{30.8}\cdot\phantom{0}76.1$ &             
  $30.8\cdot\bm{30.9}\cdot100.0$ & 
  $30.8\cdot\bm{{\color{black}34.8}}\cdot100.0$ &  
  $30.8\cdot\bm{{\color{black}53.0}}\cdot100.0$ \\
   & \textbf{EvNet} &  - &  
   $35.5\cdot\bm{{\color{blue}93.5}}\cdot100.0$ & 
   $30.8\cdot\bm{{\color{black}31.8}}\cdot100.0$ &
   $30.8\cdot\bm{{\color{black}48.7}}\cdot100.0$ &  
   $30.8\cdot\bm{{\color{blue}93.8}}\cdot100.0$ &  
   $30.8\cdot\bm{{\color{black}63.7}}\cdot100.0$ \\

			\midrule
			%& & \multicolumn{6}{c}{\textbf{ID-Attack,}} \\
               \multirow{4}{1.2cm}{Smoothed + adv. w. uncert. attacks} & 
   \textbf{PostNet} &  - & 
   $30.8\cdot\bm{{\color{black}47.5}}\cdot100.0$ &  
   $30.8\cdot\bm{{\color{black}37.5}}\cdot100.0$ &    
   $30.8\cdot\bm{{\color{blue}92.9}}\cdot100.0$ &    
   $41.1\cdot\bm{{\color{blue}50.0}}\cdot\phantom{0}97.3$ &    
   $50.0\cdot\bm{{\color{blue}50.0}}\cdot\phantom{0}50.0$ \\
& \textbf{PriorNet} &  - &
$30.8\cdot\bm{{\color{black}31.1}}\cdot\phantom{0}60.8$ &  
$30.8\cdot\bm{{\color{black}30.8}}\cdot\phantom{0}32.3$ & 
$30.8\cdot\bm{{\color{black}30.8}}\cdot\phantom{0}90.3$ & 
$30.8\cdot\bm{{\color{black}30.8}}\cdot100.0$ & 
$30.8\cdot\bm{{\color{black}36.3}}\cdot100.0$ \\
  &  \textbf{DDNet} &  - &   
  $30.8\cdot\bm{30.8}\cdot\phantom{0}31.0$ &                
  $30.8\cdot\bm{30.8}\cdot\phantom{0}66.8$ & 
  $30.8\cdot\bm{{\color{black}30.8}}\cdot100.0$ & 
  $30.8\cdot\bm{{\color{black}31.2}}\cdot100.0$ &
  $30.8\cdot\bm{{\color{blue}57.2}}\cdot100.0$ \\
   & \textbf{EvNet} &  - &  
   $30.9\cdot\bm{{\color{blue}80.3}}\cdot100.0$ &  
   $30.8\cdot\bm{{\color{blue}78.1}}\cdot100.0$ & 
   $30.8\cdot\bm{{\color{blue}99.4}}\cdot100.0$ &   
   $30.8\cdot\bm{{\color{blue}97.7}}\cdot100.0$ & 
   $30.8\cdot\bm{{\color{black}41.5}}\cdot100.0$ \\
		
    \midrule
			\midrule
			& &\multicolumn{6}{c}{\textbf{OOD-Attack}} \\
               \multirow{4}{1.2cm}{Smoothed models} & 
   \textbf{PostNet} &     
   $99.6\cdot\bm{{\color{blue}99.9}}\cdot99.9$ & 
   $31.2\cdot\bm{{\color{black}94.3}}\cdot100.0$ & 
   $30.8\cdot\bm{{\color{black}44.8}}\cdot100.0$ &  
   $30.8\cdot\bm{{\color{black}36.8}}\cdot100.0$ & 
   $30.8\cdot\bm{{\color{black}39.9}}\cdot100.0$ &  
   $44.3\cdot\bm{{\color{black}50.0}}\cdot\phantom{0}50.0$ \\
 & \textbf{PriorNet} & 
 $30.9\cdot\bm{{\color{black}31.0}}\cdot31.4$ &  
 $30.8\cdot\bm{{\color{black}30.8}}\cdot\phantom{0}42.0$ &  
 $30.8\cdot\bm{{\color{black}30.8}}\cdot\phantom{0}80.4$ &  
 $30.8\cdot\bm{{\color{black}30.8}}\cdot100.0$ & 
 $30.8\cdot\bm{{\color{blue}37.5}}\cdot100.0$ & 
 $30.8\cdot\bm{{\color{blue}94.9}}\cdot100.0$ \\
  &  \textbf{DDNet} & 
  $30.8\cdot\bm{{\color{black}30.8}}\cdot30.8$ &  
  $30.8\cdot\bm{{\color{black}30.8}}\cdot\phantom{0}31.5$ &    
  $30.8\cdot\bm{30.8}\cdot\phantom{0}48.0$ &                  
  $30.8\cdot\bm{30.8}\cdot100.0$ &                 
  $30.8\cdot\bm{30.8}\cdot100.0$ &            
  $30.8\cdot\bm{30.8}\cdot100.0$ \\
   & \textbf{EvNet} &    
   $94.9\cdot\bm{{\color{blue}97.2}}\cdot98.3$ &   
   $31.3\cdot\bm{{\color{blue}92.1}}\cdot100.0$ &   
   $30.8\cdot\bm{{\color{blue}90.8}}\cdot100.0$ &  
   $30.8\cdot\bm{{\color{blue}89.6}}\cdot100.0$ &   
   $30.8\cdot\bm{{\color{blue}89.8}}\cdot100.0$ & 
   $30.8\cdot\bm{{\color{blue}87.3}}\cdot100.0$ \\

			\midrule
			%& &\multicolumn{6}{c}{\textbf{OOD-Attack}} \\
              \multirow{4}{1.2cm}{Smoothed + adv. label attacks} & 
  \textbf{PostNet} &  - & 
  $30.8\cdot\bm{{\color{black}85.3}}\cdot100.0$ & 
  $30.8\cdot\bm{{\color{black}85.9}}\cdot100.0$ & 
  $30.8\cdot\bm{{\color{black}78.8}}\cdot100.0$ & 
  $30.9\cdot\bm{{\color{black}50.0}}\cdot100.0$ &  
  $50.0\cdot\bm{{\color{black}50.0}}\cdot\phantom{0}50.0$ \\
 & \textbf{PriorNet} &  - &   
 $30.8\cdot\bm{{\color{black}30.8}}\cdot\phantom{0}45.0$ & 
 $30.8\cdot\bm{{\color{black}30.8}}\cdot\phantom{0}32.1$ & 
 $30.8\cdot\bm{{\color{black}30.8}}\cdot\phantom{0}90.3$ & 
 $30.8\cdot\bm{{\color{black}30.8}}\cdot100.0$ &               
 $31.0\cdot\bm{30.8}\cdot100.0$ \\
  &  \textbf{DDNet} &  - &  
  $30.8\cdot\bm{{\color{black}30.8}}\cdot\phantom{0}30.8$ &              
  $30.8\cdot\bm{30.8}\cdot\phantom{0}64.9$ &                  
  $30.8\cdot\bm{30.8}\cdot100.0$ &            
  $30.8\cdot\bm{30.8}\cdot100.0$ &   
  $30.8\cdot\bm{{\color{blue}79.4}}\cdot100.0$ \\
  &  \textbf{EvNet} &  - &    
  $35.4\cdot\bm{{\color{blue}95.0}}\cdot100.0$ & 
  $30.8\cdot\bm{{\color{black}35.2}}\cdot100.0$ &
  $30.8\cdot\bm{{\color{black}51.9}}\cdot100.0$ & 
  $30.8\cdot\bm{{\color{blue}80.0}}\cdot100.0$ &   
  $30.8\cdot\bm{{\color{blue}99.9}}\cdot100.0$ \\
			\midrule
			%& &\multicolumn{6}{c}{\textbf{OOD-Attack}} \\
              \multirow{4}{1.2cm}{Smoothed + adv. uncert. attacks} &  
  \textbf{PostNet} &  - & 
  $30.8\cdot\bm{{\color{black}63.4}}\cdot100.0$ & 
  $30.8\cdot\bm{{\color{black}31.7}}\cdot100.0$ & 
  $30.8\cdot\bm{{\color{black}98.4}}\cdot100.0$ & 
  $33.2\cdot\bm{{\color{black}50.0}}\cdot100.0$ & 
  $50.0\cdot\bm{{\color{black}50.0}}\cdot\phantom{0}50.0$ \\
 & \textbf{PriorNet} &  - &  
 $30.8\cdot\bm{{\color{black}31.1}}\cdot\phantom{0}58.0$ &  
 $30.8\cdot\bm{{\color{black}30.8}}\cdot\phantom{0}34.1$ &  
 $30.8\cdot\bm{{\color{black}30.8}}\cdot\phantom{0}66.8$ &  
 $30.8\cdot\bm{{\color{black}30.8}}\cdot100.0$ &    
 $30.8\cdot\bm{30.8}\cdot100.0$ \\
  &  \textbf{DDNet} &  - &  
  $30.8\cdot\bm{{\color{black}30.8}}\cdot\phantom{0}31.2$ &         
  $30.8\cdot\bm{30.8}\cdot\phantom{0}61.5$ &                
  $30.8\cdot\bm{30.8}\cdot100.0$ &               
  $30.8\cdot\bm{30.8}\cdot100.0$ &             
  $30.8\cdot\bm{30.8}\cdot100.0$ \\
  &  \textbf{EvNet} &  - &   
  $31.0\cdot\bm{{\color{blue}89.0}}\cdot100.0$ &  
  $30.8\cdot\bm{{\color{blue}96.2}}\cdot100.0$ & 
  $30.8\cdot\bm{{\color{blue}99.6}}\cdot100.0$ &  
  $30.8\cdot\bm{{\color{blue}99.6}}\cdot100.0$ &  
  $30.8\cdot\bm{{\color{blue}69.7}}\cdot100.0$ \\
		\bottomrule
		\end{tabular}}
	%\end{tiny}
\end{table*}






\clearpage
\subsection{Visualization of differential entropy distributions on ID data and OOD data}

The following Figures visualize the differential entropy distribution for ID data and OOD data for all models with standard training. We used label attacks and uncertainty attacks for CIFAR10 and MNIST. Thus, they show how well the DBU models separate on clean and perturbed ID data and OOD data. 

Figures~\ref{fig:attaked_samples_idood_label_attacks_2} and \ref{fig:attaked_samples_idood_label_attacks_3} visualizes the differential entropy distribution of ID data and OOD data under label attacks. On CIFAR10, \PriorNet and \DDNet can barely distinguish between clean ID and OOD data. We observe a better ID/OOD distinction for \PostNet and \EvNet for clean data. However, we do not observe for any model an increase of the uncertainty estimates on label attacked data. Even worse, \PostNet, \PriorNet and \DDNet seem to assign higher confidence on class label attacks. On MNIST, models show a slightly better behavior. They are capable to assign a higher uncertainty to label attacks up to some attack radius.

Figures~\ref{fig:attaked_samples_idood_2}, \ref{fig:attaked_samples_idood_3}, \ref{fig:attaked_samples_idood_mnist} and \ref{fig:attaked_samples_idood_mnist_2} visualizes the differential entropy distribution of ID data and OOD data under uncertainty attacks. For both CIFAR10 and MNIST data sets, we observed that uncertainty estimations of all models can be manipulated. That is, OOD uncertainty attacks can shift the OOD uncertainty distribution to more certain predictions, and ID uncertainty attacks can shift the ID uncertainty distribution to less certain predictions.


\begin{figure*}[ht!]
    \centering
        \begin{subfigure}[t]{1.0\textwidth}
        \centering
        \includegraphics[width=0.99 \textwidth]{sections/008_icml2021/eval/unc_dist_label_id_cifar10_c.png}
    \end{subfigure}%
    \caption{Visualization of the differential entropy distribution of ID data (CIFAR10) and OOD data (SVHN) under label attack. The first row corresponds to no attack. The other rows correspond do increasingly stronger attack strength.}
    \label{fig:attaked_samples_idood_label_attacks_2}
	\vspace{-.5cm}
\end{figure*}
\newpage

\begin{figure*}[ht!]
    \centering
        \begin{subfigure}[t]{1.0\textwidth}
        \centering
        \includegraphics[width=0.99 \textwidth]{sections/008_icml2021/eval/unc_dist_label_id_mnist_c.png}
    \end{subfigure}%
    \caption{Visualization of the differential entropy distribution of ID data (MNIST) and OOD data (KMNIST) under label attack. The first row corresponds to no attack. The other rows correspond do increasingly stronger attack strength.}
    \label{fig:attaked_samples_idood_label_attacks_3}
	\vspace{-.5cm}
\end{figure*}
\newpage

\begin{figure*}[ht!]
    \centering
        \begin{subfigure}[t]{1.0\textwidth}
        \centering
        \includegraphics[width=0.99 \textwidth]{sections/008_icml2021/eval/unc_dist_unc_ood_cifar10_c.png}
    \end{subfigure}%
    \caption{Visualization of the differential entropy distribution of ID data (CIFAR10) and OOD data (SVHN) under OOD uncertainty attack. The first row corresponds to no attack. The other rows correspond do increasingly stronger attack strength.}
    \label{fig:attaked_samples_idood_2}
	\vspace{-.5cm}
\end{figure*}


\begin{figure*}[ht!]
    \centering
        \begin{subfigure}[t]{1.0\textwidth}
        \centering
        \includegraphics[width=0.99 \textwidth]{sections/008_icml2021/eval/unc_dist_unc_id_cifar10_c.png}
    \end{subfigure}%
    \caption{Visualization of the differential entropy distribution of ID data (CIFAR10) and OOD data (SVHN) under ID uncertainty attack. The first row corresponds to no attack. The other rows correspond do increasingly stronger attack strength.}
    \label{fig:attaked_samples_idood_3}
	\vspace{-.5cm}
\end{figure*}

\begin{figure*}[ht!]
    \centering
        \begin{subfigure}[t]{1.0\textwidth}
        \centering
        \includegraphics[width=0.99 \textwidth]{sections/008_icml2021/eval/unc_dist_unc_ood_mnist_c.png}
    \end{subfigure}%
    \caption{Visualization of the differential entropy distribution of ID data (MNIST) and OOD data (KMNIST) under OOD uncertainty attack. The first row corresponds to no attack. The other rows correspond do increasingly stronger attack strength.}
    \label{fig:attaked_samples_idood_mnist}
	\vspace{-.5cm}
\end{figure*}


\begin{figure*}[ht!]
    \centering
        \begin{subfigure}[t]{1.0\textwidth}
        \centering
        \includegraphics[width=0.99 \textwidth]{sections/008_icml2021/eval/unc_dist_unc_id_mnist_c.png}
    \end{subfigure}%
    \caption{Visualization of the differential entropy distribution of ID data (MNIST) and OOD data (KMNIST) under ID uncertainty attack. The first row corresponds to no attack. The other rows correspond do increasingly stronger attack strength.}
    \label{fig:attaked_samples_idood_mnist_2}
	\vspace{-.5cm}
\end{figure*}








\end{document}
