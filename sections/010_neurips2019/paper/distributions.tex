\section*{Supplementary Materials: Uncertainty on Asynchronous Time Event Prediction}
\section{Distributions}\label{distributions}

For reference, we give here the definition of the Dirichlet and Logistic-normal distribution.

\subsection{Dirichlet distribution}

The Dirichlet distribution with concentration parameters $\bm \alpha = ( \alpha_1, \dots, \alpha_K )$, where $\alpha_i > 0$, has the probability density function:
\begin{equation}
    f(\bm x; \bm \alpha) =
    \frac{\prod_{i=1}^K \Gamma(\alpha_i)}{\Gamma\left( \sum_{i=1}^K \alpha_i \right)}
    \prod_{i=1}^K x_i^{\alpha_i - 1}
\end{equation}
where $\Gamma$ is a gamma function:
\begin{align*}
    \Gamma(\alpha) = \int_0^\infty \alpha^{z-1} e^{-\alpha} dz
\end{align*}

\subsection{Logistic-normal distribution (LN)}

The logistic normal distribution is a generalization of the logit-normal distribution for the multidimensional case. If $\bm y \in \mathbb{R}^{\NbClasses}$ follows a normal distribution, $\bm y \sim \mathcal{N}(\bm \mu, \bm \Sigma)$, then
\begin{align*}
    \bm x = \left[
        \frac{e^{y_1}}{\sum_{i=1}^{C} e^{y_i}},
        \dots,
        \frac{e^{y_{\NbClasses}}}{\sum_{i=1}^{\NbClasses} e^{y_i}}
    \right]
\end{align*}
follows a logistic-normal distribution.
