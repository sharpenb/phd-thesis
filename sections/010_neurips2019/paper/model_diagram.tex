%%%%% Horizontal %%%%%%
%\begin{figure}[H]
%\centering
%\scalebox{0.9}{\begin{tikzpicture}[
%rectanglenode/.style={rectangle, draw=black!100, very thick, minimum size=10mm},
%roundnode/.style={circle, draw=black!100, very thick, minimum size=10mm},
%nonenode/.style={rectangle, draw=none, minimum size=6mm},
%]
%\tikzset{edge/.style = {->,> = latex'}}
%\node[nonenode] (H)         at (-5.9, 10.25)       {$\History$};
%\node[nonenode] (e)      at (-5.9, 9.75)       {$e$};
%\node[rectanglenode] (RNN)       at (-4.5, 10)       {RNN};
%\node[nonenode] (pi)        at (-3, 9.5)       {$w_\IndexPoint^{(\IndexClass)}$};
%\node[nonenode] (mu)       at (-3, 10)       {$\DeltaTime_\IndexPoint^{(\IndexClass)}$};
%\node[nonenode] (sigma)        at (-3,10.5)       {$y_\IndexPoint^{(\IndexClass)}$};
%\node[nonenode] (alpha)        at (-1.5, 10)       {$\theta(\DeltaTime)$};
%\node[nonenode] (p)        at (0.75, 10)       {$\bm{p}(\DeltaTime) \sim P(\theta(\DeltaTime))$};
%\draw[edge] (H) -> (RNN) {};
%\draw[edge] (e) -> (RNN) {};
%\draw[edge] (RNN) -> (pi) {};
%\draw[edge] (RNN) -> (mu) {};
%\draw[edge] (RNN) -> (sigma) {};
%\draw[edge] (pi) -> (alpha) {};
%\draw[edge] (mu) -> (alpha) {};
%\draw[edge] (sigma) -> (alpha) {};
%\draw[edge] (alpha) -> (p) {};
%\end{tikzpicture}}
%\caption{Model diagram. Both models output pseudo points given history and new event. In \GPModel they generate parameters $\theta(\DeltaTime) = \{\bm \mu (\DeltaTime), \bm \Sigma(\DeltaTime)\}$ that model $P(\theta(\DeltaTime))$ as a logistic-normal distribution. In \DirModel these points generate $\theta(\DeltaTime) = \bm \alpha(\DeltaTime)$ to model $P(\theta(\DeltaTime))$ as a Dirichlet distribution.}
%\label{fig:model_diagram}
%\end{figure}
%%%%% Vertical %%%%%%
% \InsertBoxR{3}{\parbox{0.3\linewidth}{
\begin{wrapfigure}[12]{r}{3.5cm}
    \vspace*{-0.6cm}
    \centering
    \scalebox{.75}{\begin{tikzpicture}[
    rectanglenode/.style={rectangle, draw=black!100, very thick, minimum size=10mm},
    roundnode/.style={circle, draw=black!100, very thick, minimum size=5mm},
    nonenode/.style={rectangle, draw=none, minimum size=6mm},
    ]
    \tikzset{edge/.style = {->,> = latex'}}
    \node[nonenode] (H)         at (8.7,4.5)       {$\History_{\IndexEvent-1}$};
    \node[nonenode] (e)      at (10,5.6)       {$e_{\IndexEvent-1}$};
    \node[rectanglenode] (RNN)       at (10,4.5)       {RNN};
    \node[nonenode] (pi)        at (9.25,3.4)       {$[w_\IndexPoint^{(\IndexClass)},$};
    \node[nonenode] (mu)       at (10,3.4)       {$\DeltaTime_\IndexPoint^{(\IndexClass)},$};
    \node[nonenode] (sigma)        at (11,3.4)       {$y_\IndexPoint^{(\IndexClass)}]_{\IndexPoint=1}^\NbPoints$};
    \node[roundnode] (regression)        at (10,2.4)       {GP};
    \node[nonenode] (alpha)        at (10,1.4)       {$\DeltaTime_{\IndexClass}(\DeltaTime), \sigma_{\IndexClass}^2(\DeltaTime)$};
    \node[nonenode] (p)        at (10,0.5)       {$\bm{p}(\DeltaTime) \sim P(\theta(\DeltaTime))$};
    \draw[edge] (H) -> (RNN) {};
    \draw[edge] (e) -> (RNN) {};
    \draw[edge] (RNN) -> (pi) {};
    \draw[edge] (RNN) -> (mu) {};
    \draw[edge] (RNN) -> (sigma) {};
    \draw[edge] (pi) -> (regression) {};
    \draw[edge] (mu) -> (regression) {};
    \draw[edge] (sigma) -> (regression) {};
    \draw[edge] (regression) -> (alpha) {};
    \draw[edge] (alpha) -> (p) {};
    \end{tikzpicture}}
    \caption{Model diagram}\label{fig:model_diagram}
% }}[6]
    %\vspace*{-1cm}
\end{wrapfigure}
