\section{Behavior of the min kernel}\label{gp_min_kernel}

The desired behavior of the min kernel function can easily be illustrated by considering the gram matrix $\bm{K}$  and vector  $\bm{k}$, which are required to estimate $\mu$ and $\sigma^2$ for a new time point $\DeltaTime$.
W.l.o.g.\ consider $\NbPoints$ pseudo points $\DeltaTime_1, \dots, \DeltaTime_\NbPoints$ such that $w_1 < \dots < w_\NbPoints$. Since the new query point is observed we assign it weight $1$. It follows:
\begin{equation}\label{eq:min-kernel-example}
\bm{k} = \begin{bmatrix}
   w_1 \\
   w_2 \\
   \vdots \\
   w_\NbPoints
\end{bmatrix}
\odot
\begin{bmatrix}
   k(\DeltaTime_1, \DeltaTime) \\
   k(\DeltaTime_2, \DeltaTime) \\
   \vdots \\
   k(\DeltaTime_\NbPoints, \DeltaTime) \\
\end{bmatrix},
\;
\bm{K} =
\begin{bmatrix}
   w_1 & w_1 & \dots & w_1 \\
   w_1 & w_2 & \dots & w_2 \\
   \vdots & \vdots & \ddots & \vdots \\
   w_1 & w_2 & \dots & w_\NbPoints
\end{bmatrix}
\odot
\begin{bmatrix}
   k(\DeltaTime_1, \DeltaTime_1) & \dots & k(\DeltaTime_1, \DeltaTime_\NbPoints) \\
   k(\DeltaTime_2, \DeltaTime_1) & \dots & k(\DeltaTime_2, \DeltaTime_\NbPoints) \\
   \vdots & \ddots & \vdots \\
   k(\DeltaTime_\NbPoints, \DeltaTime_1) & \dots & k(\DeltaTime_\NbPoints, \DeltaTime_\NbPoints)
\end{bmatrix}
\end{equation}
Assuming $w_1=0$ returns $\bm{k}$ without the first row and $\bm{K}$ without the first row and column. Plugging them back into \cref{eq:gp_prediction} we can see that the point $\DeltaTime_1$ is discarded, as desired.
In practice, the weights have values from interval $[0, 1]$ which in turn gives us the ability to \textit{softly discard} points. This is shown in \cref{fig:weighted_gaussian_process} we can see that the mean line does not have to cross through the points with weights $<1$ and the variance can remain higher around them.