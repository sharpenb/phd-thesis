\section{Conclusion}

%In this work we tackled the problem of event prediction in asynchronous sequences. The main problem is the ability to model rich multimodal functions that describe the evolution of the probability of the next event over time while having a way to evaluate the uncertainty in the prediction. To solve this we model the evolution of a distribution on the probability simplex; specifically, a function that outputs parameters of this distribution at any given time. In the two models we propose this function is defined via a function decomposition or a Gaussian process  using pseudo points generated by a neural network.
%We can evaluate the probability of the class at any time point together with the certainty in the prediction.
%We test our models against models based on point processes on the event and time prediction and on anomaly detection using real world and synthetic datasets. We outperform other models on all tasks across all datasets.

We proposed two new methods to predict the evolution of the probability of the next event in asynchronous sequences, including the distributions' uncertainty. Both methods follow a common framework consisting in generating pseudo points able to describe rich multimodal time-dependent parameters for the distribution over the probability simplex. The complex evolution is captured via a Gaussian Process or a function decomposition, respectively; still enabling easy training. We also provided an extension and interpretation within a point process framework. In the experiments, \GPModel and \DirModel have clearly outperformed state-of-the-art models based on point processes; for event and time prediction as well as for anomaly detection.

\subsection*{Acknowledgement}
\com{This research was supported by the German Federal Ministry of Education and Research (BMBF), grant no. 01IS18036B, and by the BMW AG. The authors would like to thank Bernhard Schlegel for helpful discussion and comments. The authors of this work take full responsibilities for its content.}
