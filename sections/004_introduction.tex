\chapter{Introduction}
\label{chap:introduction}

Artificical Intelligence (AI) consists of the intelligence demonstrated by machines in contrast with intelligence demonstrated by humans or animals.
AI is a very important technology with high potential economical and ethical impact. 

Economically, AI has a high momentum. Indeed, 37\% of companies use AI \bc{CITE} and 83\% companies have AI is their priority \bc{CITE}. Overall, the market value of AI is \$136 billion in 2022 \bc{CITE} and had a Compound Annual Growth Rate (CAGR) of 38.5\% between 2022-2030 \bc{CITE}, thus indicating a very fast growth.
Hence, AI has found many applications in science and industry. 
Industry applications include e.g. agriculture with \bc{X}, manufactoring with Exotec \bc{X}, automotive industry with \bc{X}, medicine with radiology \bc{X}, construction with \bc{X}, finance with \bc{X}, education with duolingo \bc{X}, or even for art with \bc{X}.
Scientific applications include e.g. physics with \bc{X}, chimie and biology with \bc{X}, or even math with \bc{X}.
Hence, AI have shown to be applicable in various data types (e.g. images, tabular, text) and at various scales from very small (e.g. molecules) to very large systems (e.g. social networks).

Ethically, AI systems demonstrate multiple important concerns.
It does not guarantee a safe behavior and can create accidents.
It does not provide clear explanations for its predictions or provide unreliable explanations.
It might be racist and discriminate minorities.
It can partially replace human jobs by stealing human creations.
Finally, it might have goals which are not aligned with human goals. This is particularly problematic when AI systems achieve a super-intelligence \bc{CITE Nick Bostrom}. In particular, this is realistic since some AI sustems are already better than human for many tasks including \bc{X} or human exams.

The fast AI economical growth and the multiple AI ethical concerins urge the development of reliable AI models.

\subsection{Why Do We Need Uncertainty ?}

Uncertainty estimation (a.k.a. uncertainty quantification) consists in evaluating the confidence of model in its prediction. This task is crucial for both practical and theoretical reasons:

\paragraph*{Practical reasons:}

\paragraph*{Theoretical reasons:}

\begin{itemize}
\item Practical motivation: trust/security (Dunning-Kruger effect), maintenance/exploration/exploitation (Life-long learing), fairness
\item Theoretical motivation: non deterministic world ,Micro/Macro, nothing is fully observable, uncertainty principle
\end{itemize}

\subsection{Why Do We Have I.I.D. and Non-I.I.D. Data ?}

\begin{itemize}
    \item IID: definition, images/tabular data, variety in outputs
    \item Non-IID: definition, sequential/graph, 
\end{itemize}

\section{Contributions}

\begin{itemize}
    \item Desiderata
    \item Models
    \item Evaluation
\end{itemize}

\begin{itemize}
    \item IID: different output
    \item Non-IID: different input
\end{itemize}

\section{Own Publications}

The publications devides in three topics:
\begin{itemize}
\item Uncertainty estimation (incl. sparse NN and Energy-based models)
\item Structure learning (incl. hierarhical, scikit-network and DAG learning)
\item Efficient models (incl. pruning)
\end{itemize}