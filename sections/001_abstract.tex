\chapter*{Abstract}
\addcontentsline{toc}{chapter}{\protect Abstract}%

Both \emph{practical} and \emph{theoretical} reasons justify why we need uncertainty estimation to build reliable machine learning models. While uncertainty estimation is expected to bring \emph{trust}, \emph{safety}, \emph{fairness} and facilitate \emph{maintenance} in real-world applications, uncertainty estimation is also highly required to represent the real physical world which is inherently \emph{non-deterministic} and \emph{partially observable}. 

Furthermore, machine learning models are confronted with various types of input data (e.g. tabular, images, graph data, sequential data) and output data (classes, real values, counts, time events) which can be assumed either \emph{independent} or \emph{non-independent}. While the independence assumption is convenient to represent many data types, the non-independence assumption is particularly useful to represent complex data types with network effects or time effects.

In this thesis, we look at uncertainty estimation for both independent and non-independent data. To this end, we differentiate between three main points: \textbf{(1)} We propose \emph{desiderata} capturing the desired behavior of uncertainty estimation. These desiderata cover both aleatoric and epistemic uncertainty in the presence of (adversarial) perturbations, network effects, or time effects. Further, we analyze  the desired behavior for uncertainty estimates at both training and testing time. \textbf{(2)} We propose a large family of new Bayesian \emph{models} which provide uncertainty estimates at a low practical cost. These models demonstrate strong empirical performance and have theoretical guarantees for different data types. \textbf{(3)} We propose extensive \emph{experimental setups} to evaluate uncertainty estimates for practical tasks. These experimental setups cover correct/wrong prediction detection, Out-Of-Distribution (OOD) detection, dataset shifts, and calibration metrics in the presence of (adversarial) perturbations, network effects, or time effects. Finally, we analyze the benefit of using uncertainty estimates to achieve good exploration/exploitation trade-off with high sample efficiency.
\\
